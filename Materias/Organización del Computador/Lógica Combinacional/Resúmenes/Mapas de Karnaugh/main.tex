\documentclass[20pt,margin=1in,innermargin=-4.5in,blockverticalspace=-0.25in]{tikzposter}
\geometry{paperwidth=42in,paperheight=30in}
\usepackage[utf8]{inputenc}
\usepackage{amsmath}
\usepackage{amsfonts}
\usepackage{amsthm}
\usepackage{amssymb}
\usepackage{mathrsfs}
\usepackage{graphicx}
\usepackage{adjustbox}
\usepackage{enumitem}
\usepackage[backend=biber,style=numeric]{biblatex}
\usepackage{emory-theme}
\usepackage{emory-theme}
\usepackage{mwe} % for placeholder images
\usepackage{booktabs}
\usepackage{karnaugh-map}
% set theme parameters
\tikzposterlatexaffectionproofoff
\usetheme{EmoryTheme}
\usecolorstyle{EmoryStyle}

\title{Sistema IEEE 754 - Precisión Simple}
\author{Pedro Villar}
\institute{Organización del Computador - Primer Cuatrimestre 2024}	
\titlegraphic{\adjustbox{left=0.35\textwidth}{\includegraphics[width=0.3\textwidth]{famaf-logo.jpg}}}

% begin document
\begin{document}
\maketitle
\centering
\block{¿Qué es el sistema IEEE 754?}{
        El sistema IEEE 754 es un estándar para la representación de números en punto flotante. Este sistema establece las caracteristicas de las tres partes de un número en punto flotante: el \textbf{signo}, \textbf{exponente} y \textbf{fracción (mantisa)}. Se enfoca en el método de precisión simple que utiliza 32 bits.
    }
\begin{columns}
    \column{0.5}
    \block{Mapa}{

    }


\end{columns}

\end{document}