\subsection{Operaciones de Lectura y Escritura}
Las dos operaciones que efectúa una memoria de acceso aleatorio son escritura y lectura. La señal de escritura especifica una operación de transferencia hacia adentro, y la de lectura, una de transferencia hacia afuera. Al aceptar una de estas señales de control, los circuitos internos de la memoria efectúan la operación deseada.
Los pasos que deben seguirse para transferir una nueva palabra a la memoria son:
\begin{enumerate}
    \item Aplique la dirección binaria de la localidad deseada a las líneas de dirección.
    \item Aplique a las líneas de entrada de datos los bits de datos que se guardarán en la memoria.
    \item Active la entrada \texttt{escribir}.
\end{enumerate}

La unidad de memoria tomará entonces los bits de las líneas de datos de entrada y los almacenará en la localidad especificada por las líneas de dirección.
Los pasos que deben seguirse para sacar de la memoria una palabra almacenada son:
\begin{enumerate}
    \item Aplique a las líneas de dirección la dirección binaria de la localidad deseada.
    \item Active la entrada \texttt{leer}.
\end{enumerate}

La unidad de memoria tomará entonces los bits de la localidad seleccionada por la dirección
y los aplicará a las líneas de datos de salida. El contenido de la localidad seleccionada no cambia después de la lectura.