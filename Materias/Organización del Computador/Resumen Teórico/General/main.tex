\documentclass{scrreprt}
\usepackage[utf8]{inputenc}
\usepackage{xcolor}
\usepackage{amsmath,amsthm,amssymb,amsfonts, fancyhdr, color, comment, graphicx, environ}
\usepackage[explicit]{titlesec}
\usepackage{soul}
\usepackage[top=1.8cm, bottom=1.5cm, left=1.5cm, right=1.5cm]{geometry}
\usepackage{karnaugh-map}
\usepackage{booktabs}
\usepackage{mdframed}
\usepackage{fix-cm}
\usepackage{tabularx}
\usepackage{tikz, fouriernc}
\usetikzlibrary{calc}
\usepackage[hidelinks]{hyperref}

\definecolor{lime}{RGB}{0,0,128}
\definecolor{titleblue}{HTML}{4a7aa4}
\setlength{\parindent}{0pt}

\newbox\TitleUnderlineTestBox
\newcommand*\TitleUnderline[1]
  {%
    \bgroup
    \setbox\TitleUnderlineTestBox\hbox{\colorbox{titleblue}\strut}%
    \setul{\dimexpr\dp\TitleUnderlineTestBox-.3ex\relax}{.3ex}%
    \ul{#1}%
    \egroup
  }
\newcommand*\SectionNumberBox[1]
  {%
    \colorbox{titleblue}
      {%
        \makebox[2.5em][c]
          {%
            \color{white}%
            \strut
            \csname the#1\endcsname
          }%
      }%
    \TitleUnderline{\ \ \ }%
  }
\titleformat{\section}
  {\Large\bfseries\sffamily\color{titleblue}}
  {\SectionNumberBox{section}}
  {0pt}
  {\TitleUnderline{#1}}
\titleformat{\subsection}
  {\large\bfseries\sffamily\color{titleblue}}
  {\SectionNumberBox{subsection}}
  {0pt}
  {\TitleUnderline{#1}}
\titleformat{\subsubsection}
  {\normalsize\bfseries\sffamily\color{titleblue}}
  {\SectionNumberBox{subsubsection}}
  {0pt}
  {\TitleUnderline{#1}}

  \definecolor{numbercolor}{gray}{0.7}

  \makeatletter
  \renewcommand\chapterheadstartvskip{\vskip50pt}
  
  \newcommand\chaptitlefont{%
    \fontfamily{pbk}\fontseries{db}%
    \fontshape{n}\fontsize{25}{35}\selectfont\raggedleft}
  
  \newcommand\chapnumfont{%
    \fontfamily{pbk}\fontseries{m}\fontshape{n}%
    \fontsize{1in}{0in}\selectfont\color{titleblue}}
  
  \renewcommand\chapterheadendvskip{\par\vskip2mm\hrule\vskip40pt}
  
  \renewcommand*{\@@makechapterhead}[1]{\chapterheadstartvskip
    {%
      \setlength{\parindent}{\z@}\setlength{\parfillskip}{\fill}%
      \normalfont\sectfont\nobreak\size@chapter{}%
      \if@chapterprefix
        \let\@tempa\raggedsection
      \else
        \let\@tempa\@hangfrom
      \fi
      \@tempa{\ifnum \c@secnumdepth >\m@ne%
            \if@chapterprefix
              \expandafter\size@chapterprefix
            \else
              \expandafter\size@chapter
            \fi
            \if@chapterprefix
              \size@chapterprefix{}\endgraf\nobreak\vskip.5\baselineskip
            \fi
        \fi
      }%
  \begin{tabularx}{\textwidth}{Xl}
  {\parbox[b]{\linewidth}{\chaptitlefont #1}}
  & \raisebox{-15pt}{\chapnumfont\thechapter}%
  \end{tabularx}%
   \nobreak\chapterheadendvskip
  }%
  }
  \renewcommand*{\@@makeschapterhead}[1]{%
    \chapterheadstartvskip%
    {\normalfont\sectfont\nobreak\size@chapter{}%
      \setlength{\parindent}{\z@}\setlength{\parfillskip}{\fill}%
      \raggedsection \interlinepenalty \@M 
  \begin{tabularx}{\textwidth}{X}%
  {\parbox[b]{\linewidth}{\chaptitlefont #1}%
  \vphantom{\raisebox{-15pt}{\chapnumfont 1}}}
  \end{tabularx}%
  \par}%
    \nobreak\chapterheadendvskip%
  }
  \makeatother

\pagestyle{fancy}
\fancyhf{}
\renewcommand{\headrulewidth}{0.4pt}
\renewcommand{\footrulewidth}{0pt}
\fancyhead[L]{\textbf{Villar Pedro}}
\fancyhead[R]{\textbf{FaMAF, Org. del Computador}}
\fancyhead[C]{\thepage}

%% Cambiar nombre de la tabla de contenidos

\begin{document}

%-------------------------------------------%
% Portada del documento
\begin{titlepage}
  \begin{tikzpicture}[overlay,remember picture]
      \fill[lime!10] (current page.south east) rectangle (current page.north west);
      \fill[lime,even odd rule] (current page.south west) rectangle ([xshift=1.5cm]current page.north west) ($(current page.north west)!.5!(current page.south west)$) arc(180:270:3) ($(current page.north west)!.5!(current page.south west)$) arc(270:450:3.5) ([yshift=7cm]$(current page.north west)!.5!(current page.south west)$)  arc(180:90:3);
%           \draw[lime!10,double=lime!10,double distance=2mm] (current page.south west) --+ (2,2) --+ (-2,6) --+ (2,8) --+ (-2,9) --+ (2,13) --+ (-2,18) --+ (2,30);
          \draw[lime,thin] (13,-1) node[lime,above left] {\Huge Org. del Computador} --+ (0,3) node[lime,below right] {\Huge 1C-2024};
      \node[yshift=-5cm,xslant=1,yscale=1.6,lime!40,opacity=.2] at (9.5,-6.5) {\scalebox{6}{Apunte}};
      \node[yshift=-5cm,lime] at (8,-7) {\scalebox{6}{Apunte}};
          \draw[lime,fill=titleblue!20] ([xshift=-1cm]current page.north east) -- ([yshift=-1cm]current page.north east) -- ([yshift=-2cm]current page.north east) -- ([xshift=-2cm]current page.north east);
      \path ([xshift=-1cm]current page.north east) -- ([yshift=-1cm]current page.north east) node[midway,sloped,below=.1cm,lime] {Villar Pedro};
  \end{tikzpicture}
\end{titlepage}
%-------------------------------------------%

\tableofcontents
\newpage

\listoffigures
\newpage
%-------------------------------------------%
\chapter{Primer Parcial}

\textbf{Fecha:} 24/04/2024
\vspace{1cm}

\textbf{Prácticos:} 1, 2, 3, 4 y 5.
\vspace{1cm}

\textbf{Temas:}
\begin{itemize}
    \item \hyperref[sec:num]{Sistemas de Numeración.}
    \item \hyperref[sec:bool]{Álgebra de Boole.}
    \item \hyperref[sec:logica]{Lógica Combinacional.}
    \item \hyperref[sec:mem]{Direccionamiento y Lógica de Decodificación de Memorias.}
    \item \hyperref[sec:sec]{Circuitos secuenciales.}
\end{itemize}

\newpage  
%-------------- Sistemas de Numeración --------------%
\section{Sistemas de Numeración}\label{sec:num}
\documentclass[20pt,margin=1in,innermargin=-4.5in,blockverticalspace=-0.25in]{tikzposter}
\geometry{paperwidth=42in,paperheight=30in}
\usepackage[utf8]{inputenc}
\usepackage{amsmath}
\usepackage{amsfonts}
\usepackage{amsthm}
\usepackage{amssymb}
\usepackage{mathrsfs}
\usepackage{graphicx}
\usepackage{adjustbox}
\usepackage{enumitem}
\usepackage[backend=biber,style=numeric]{biblatex}
\usepackage{emory-theme}
\usepackage{emory-theme}
\usepackage{mwe} % for placeholder images
\usepackage{booktabs}
% set theme parameters
\tikzposterlatexaffectionproofoff
\usetheme{EmoryTheme}
\usecolorstyle{EmoryStyle}

\title{Sistemas de Numeración}
\author{Pedro Villar}
\institute{Organización del Computador - Primer Cuatrimestre 2024}	
\titlegraphic{\adjustbox{left=0.35\textwidth}{\includegraphics[width=0.3\textwidth]{famaf-logo.jpg}}}

% begin document
\begin{document}
\maketitle
\centering

\begin{columns}
    \column{0.33}
    \block{Pasar de Hexadecimal a Binario}{
        El sistema hexadecimal es una base 16, por lo que cada dígito puede representar 4 bits. Para pasar de hexadecimal a binario, simplemente se reemplaza cada dígito por su representación en 4 bits. La tabla de conversión es la siguiente:
        
    \begin{center}
     \begin{tabular}{cccccc} % Eliminar las líneas verticales
        \toprule % Línea superior más gruesa
        Hexadecimal & Binario & Hexadecimal & Binario & Hexadecimal & Binario\\
        \midrule % Línea intermedia más fina
        0 & 0000 & 5 & 0101 & A & 1010 \\
        1 & 0001 & 6 & 0110 & B & 1011 \\
        2 & 0010 & 7 & 0111 & C & 1100 \\
        3 & 0011 & 8 & 1000 & D & 1101 \\
        4 & 0100 & 9 & 1001 & E & 1110 \\
          &      &   &      & F & 1111 \\
        \bottomrule % Línea inferior más gruesa
    \end{tabular}
    \end{center}
    Por ejemplo, para pasar de \texttt{0x123456} a binario, simplemente se reemplaza cada dígito por su representación en 4 bits:
    \begin{center}
        \texttt{0x123456} $\rightarrow$ \texttt{0001 0010 0011 0100 0101 0110}
    \end{center}
    }

    \block{Pasar de Binario a Hexadecimal}{
        Para pasar de binario a hexadecimal, simplemente se agrupan los bits de a 4 y se reemplaza cada grupo por su representación en hexadecimal, teniendo en cuenta las siguientes consideraciones:
        \begin{itemize}
            \item Si el número tiene parte decimal, se agrupan los bits de a 4 a partir del punto decimal y luego se hace la parte entera.
            \item Si el número de bits no es múltiplo de 4, se agrega un 0 a la izquierda para completar el último grupo.
            \item En el caso de que en la parte fraccionaria se necesite agregar ceros se agregan a la derecha.
        \end{itemize}
        Por ejemplo, para pasar de \texttt{10 1100 1101 1011.1100 0010 000} a hexadecimal:
        \begin{equation*}
            \underbrace{0010}_{2}\underbrace{1100}_{C}\underbrace{1101}_{D}\underbrace{1011}_{B} \ \ \ \ \underbrace{1100}_{C}\underbrace{0010}_{2}\underbrace{0000}_{0}
        \end{equation*}
        \begin{equation*}
            (10110011011011.11000010000)_2 = 0x2CDB.C20 
        \end{equation*}
    }

    \block{Pasar de Hexadecimal a Decimal}{
        Para pasar de hexadecimal a decimal, simplemente se reemplaza cada dígito por su representación en decimal y se multiplica por la potencia de 16 correspondiente. Por ejemplo, para pasar de \texttt{0x123456} a decimal:
        \begin{equation*}
            0x123456 = 1 \cdot 16^5 + 2 \cdot 16^4 + 3 \cdot 16^3 + 4 \cdot 16^2 + 5 \cdot 16^1 + 6 \cdot 16^0 = 1193046
        \end{equation*}
    }
    
    \block{Pasar de decimal a binario}{
        Para pasar de decimal a binario, simplemente se divide el número por 2 y se toma el resto. Luego se divide el cociente por 2 y se toma el resto, y así sucesivamente hasta que el cociente sea 0. Luego se toman los restos en orden inverso. Por ejemplo, para pasar de 59 a binario:
        \begin{equation*}
            59 = 1 \cdot 2^5 + 1 \cdot 2^4 + 0 \cdot 2^3 + 1 \cdot 2^2 + 1 \cdot 2^1 + 1 \cdot 2^0 = 111011
        \end{equation*}
    }

    \column{0.66}
    \block{Complemento a 2 para números negativos}{
        El complemento a 2 es una forma de representar números negativos en binario. Para obtener el complemento a 2 de un número negativo, primero se agregan ceros a la izquierda para \textbf{completar la cantidad de bits} que de el registro, luego se \textbf{invierten todos los bits} y \textbf{se le suma 1 al resultado}.
        Por ejemplo, para obtener el complemento a 2 de -121 en 8 bits:
        \begin{align*}
            121/2 &= 60 \quad \text{residuo } 1\\
            60/2 &= 30 \quad \text{residuo } 0\\
            30/2 &= 15 \quad \text{residuo } 0\\
            15/2 &= 7 \quad \text{residuo } 1 \quad \quad \Rightarrow 01111001 \rightarrow 10000110 + 1 = 10000111 \\
            7/2 &= 3 \quad \text{residuo } 1\\
            3/2 &= 1 \quad \text{residuo } 1\\
            1/2 &= 0 \quad \text{residuo } 1
        \end{align*}
    }
    %\column{0.33}
    \block{Decimal negativo a binario con complemento a 2}{
        \begin{enumerate}
            \item Calcular el valor absoluto del número en binario.
            \item Completar con ceros a la izquierda para que el número tenga la cantidad de bits que se desea.
            \item Aplicar el complemento a 2.
        \end{enumerate}
    }
    %\column{0.33}
    \block{Binario con complemento a 2 a decimal}{
        \begin{enumerate}
            \item Si el bit más significativo es 1, el número es negativo. Se aplica el complemento a 2 para obtener el valor absoluto.
            \item Se multiplica cada bit por la potencia de 2 correspondiente y se suma.
        \end{enumerate}
    }
    \block{Suma binaria}{
        Para sumar dos números en binario, simplemente se suman los bits de a pares, teniendo en cuenta que:
        \begin{itemize}
            \item 0 + 0 = 0
            \item 0 + 1 = 1
            \item 1 + 0 = 1
            \item 1 + 1 = 10
        \end{itemize}
        Si la suma de dos bits es 10, se escribe 0 y se lleva 1 al siguiente par de bits.
    }
    \block{Resta binaria}{
        Para restar dos números en binario, se aplica el complemento a 2 al sustraendo y se suma al minuendo. 
        \newline Es decir si se busca hacer $A - B$, se hace $A + (-B)$. Por ejemplo para restar $10000000 - 11010000$:
        \begin{align*}
            10000000 - 11010000 &= 10000000 + 00110000 = 10110000
        \end{align*}
        El resultado es $0xB0$ en hexadecimal.
    }
\end{columns}

\end{document}
%-------------------------------------------------%

\newpage
%-------------- Álgebra de Boole --------------%
\section{Álgebra de Boole}\label{sec:bool}
\documentclass[20pt,margin=1in,innermargin=-4.5in,blockverticalspace=-0.25in]{tikzposter}
\geometry{paperwidth=42in,paperheight=30in}
\usepackage[utf8]{inputenc}
\usepackage{amsmath}
\usepackage{amsfonts}
\usepackage{amsthm}
\usepackage{amssymb}
\usepackage{mathrsfs}
\usepackage{graphicx}
\usepackage{adjustbox}
\usepackage{enumitem}
\usepackage[backend=biber,style=numeric]{biblatex}
\usepackage{emory-theme}
\usepackage{emory-theme}
\usepackage{mwe} % for placeholder images
\usepackage{booktabs}
% set theme parameters
\tikzposterlatexaffectionproofoff
\usetheme{EmoryTheme}
\usecolorstyle{EmoryStyle}

\title{Sistemas de Numeración}
\author{Pedro Villar}
\institute{Organización del Computador - Primer Cuatrimestre 2024}	
\titlegraphic{\adjustbox{left=0.35\textwidth}{\includegraphics[width=0.3\textwidth]{famaf-logo.jpg}}}

% begin document
\begin{document}
\maketitle
\centering

\begin{columns}
    \column{0.33}
    \block{Pasar de Hexadecimal a Binario}{
        El sistema hexadecimal es una base 16, por lo que cada dígito puede representar 4 bits. Para pasar de hexadecimal a binario, simplemente se reemplaza cada dígito por su representación en 4 bits. La tabla de conversión es la siguiente:
        
    \begin{center}
     \begin{tabular}{cccccc} % Eliminar las líneas verticales
        \toprule % Línea superior más gruesa
        Hexadecimal & Binario & Hexadecimal & Binario & Hexadecimal & Binario\\
        \midrule % Línea intermedia más fina
        0 & 0000 & 5 & 0101 & A & 1010 \\
        1 & 0001 & 6 & 0110 & B & 1011 \\
        2 & 0010 & 7 & 0111 & C & 1100 \\
        3 & 0011 & 8 & 1000 & D & 1101 \\
        4 & 0100 & 9 & 1001 & E & 1110 \\
          &      &   &      & F & 1111 \\
        \bottomrule % Línea inferior más gruesa
    \end{tabular}
    \end{center}
    Por ejemplo, para pasar de \texttt{0x123456} a binario, simplemente se reemplaza cada dígito por su representación en 4 bits:
    \begin{center}
        \texttt{0x123456} $\rightarrow$ \texttt{0001 0010 0011 0100 0101 0110}
    \end{center}
    }

    \block{Pasar de Binario a Hexadecimal}{
        Para pasar de binario a hexadecimal, simplemente se agrupan los bits de a 4 y se reemplaza cada grupo por su representación en hexadecimal, teniendo en cuenta las siguientes consideraciones:
        \begin{itemize}
            \item Si el número tiene parte decimal, se agrupan los bits de a 4 a partir del punto decimal y luego se hace la parte entera.
            \item Si el número de bits no es múltiplo de 4, se agrega un 0 a la izquierda para completar el último grupo.
            \item En el caso de que en la parte fraccionaria se necesite agregar ceros se agregan a la derecha.
        \end{itemize}
        Por ejemplo, para pasar de \texttt{10 1100 1101 1011.1100 0010 000} a hexadecimal:
        \begin{equation*}
            \underbrace{0010}_{2}\underbrace{1100}_{C}\underbrace{1101}_{D}\underbrace{1011}_{B} \ \ \ \ \underbrace{1100}_{C}\underbrace{0010}_{2}\underbrace{0000}_{0}
        \end{equation*}
        \begin{equation*}
            (10110011011011.11000010000)_2 = 0x2CDB.C20 
        \end{equation*}
    }

    \block{Pasar de Hexadecimal a Decimal}{
        Para pasar de hexadecimal a decimal, simplemente se reemplaza cada dígito por su representación en decimal y se multiplica por la potencia de 16 correspondiente. Por ejemplo, para pasar de \texttt{0x123456} a decimal:
        \begin{equation*}
            0x123456 = 1 \cdot 16^5 + 2 \cdot 16^4 + 3 \cdot 16^3 + 4 \cdot 16^2 + 5 \cdot 16^1 + 6 \cdot 16^0 = 1193046
        \end{equation*}
    }
    
    \block{Pasar de decimal a binario}{
        Para pasar de decimal a binario, simplemente se divide el número por 2 y se toma el resto. Luego se divide el cociente por 2 y se toma el resto, y así sucesivamente hasta que el cociente sea 0. Luego se toman los restos en orden inverso. Por ejemplo, para pasar de 59 a binario:
        \begin{equation*}
            59 = 1 \cdot 2^5 + 1 \cdot 2^4 + 0 \cdot 2^3 + 1 \cdot 2^2 + 1 \cdot 2^1 + 1 \cdot 2^0 = 111011
        \end{equation*}
    }

    \column{0.66}
    \block{Complemento a 2 para números negativos}{
        El complemento a 2 es una forma de representar números negativos en binario. Para obtener el complemento a 2 de un número negativo, primero se agregan ceros a la izquierda para \textbf{completar la cantidad de bits} que de el registro, luego se \textbf{invierten todos los bits} y \textbf{se le suma 1 al resultado}.
        Por ejemplo, para obtener el complemento a 2 de -121 en 8 bits:
        \begin{align*}
            121/2 &= 60 \quad \text{residuo } 1\\
            60/2 &= 30 \quad \text{residuo } 0\\
            30/2 &= 15 \quad \text{residuo } 0\\
            15/2 &= 7 \quad \text{residuo } 1 \quad \quad \Rightarrow 01111001 \rightarrow 10000110 + 1 = 10000111 \\
            7/2 &= 3 \quad \text{residuo } 1\\
            3/2 &= 1 \quad \text{residuo } 1\\
            1/2 &= 0 \quad \text{residuo } 1
        \end{align*}
    }
    %\column{0.33}
    \block{Decimal negativo a binario con complemento a 2}{
        \begin{enumerate}
            \item Calcular el valor absoluto del número en binario.
            \item Completar con ceros a la izquierda para que el número tenga la cantidad de bits que se desea.
            \item Aplicar el complemento a 2.
        \end{enumerate}
    }
    %\column{0.33}
    \block{Binario con complemento a 2 a decimal}{
        \begin{enumerate}
            \item Si el bit más significativo es 1, el número es negativo. Se aplica el complemento a 2 para obtener el valor absoluto.
            \item Se multiplica cada bit por la potencia de 2 correspondiente y se suma.
        \end{enumerate}
    }
    \block{Suma binaria}{
        Para sumar dos números en binario, simplemente se suman los bits de a pares, teniendo en cuenta que:
        \begin{itemize}
            \item 0 + 0 = 0
            \item 0 + 1 = 1
            \item 1 + 0 = 1
            \item 1 + 1 = 10
        \end{itemize}
        Si la suma de dos bits es 10, se escribe 0 y se lleva 1 al siguiente par de bits.
    }
    \block{Resta binaria}{
        Para restar dos números en binario, se aplica el complemento a 2 al sustraendo y se suma al minuendo. 
        \newline Es decir si se busca hacer $A - B$, se hace $A + (-B)$. Por ejemplo para restar $10000000 - 11010000$:
        \begin{align*}
            10000000 - 11010000 &= 10000000 + 00110000 = 10110000
        \end{align*}
        El resultado es $0xB0$ en hexadecimal.
    }
\end{columns}

\end{document}
%-------------------------------------------------%

\newpage
%-------------- Lógica Combinacional --------------%
\section{Lógica Combinacional}\label{sec:logica}
\subsection{Producto de Sumas y Suma de Productos (Maxitérminos y Mintérminos)}
Las funciones lógicas se pueden expresar de dos formas diferentes, como una suma de productos o como un producto de sumas. La forma de obtener dicha expresión se basa en identificar que valores de salida tomarán el valor de $1$ y el valor de $0$.
\newline Supongamos que se tiene una función lógica de tres variables $x$, $y$ y $z$, con la siguiente tabla de verdad:

\begin{table}[h]
    \centering
    \begin{tabular}{ccc|c}
        \toprule
        \textbf{x} & \textbf{y} & \textbf{z} & \textbf{S}\\
        \midrule
        0 & 0 & 0 & 0\\
        0 & 0 & 1 & 1\\
        0 & 1 & 0 & 1\\
        0 & 1 & 1 & 0\\
        1 & 0 & 0 & 1\\
        1 & 0 & 1 & 0\\
        1 & 1 & 0 & 0\\
        1 & 1 & 1 & 1\\
        \bottomrule
    \end{tabular}
\end{table}

\begin{enumerate}
    \item Para expresar como \textbf{suma de productos}, se toman los minitérminos que hacen que la función devuelva $1$, luego se suman los minitérminos.
    \item Para expresar como \textbf{producto de sumas}, se toman los maxitérminos que hacen que la función devuelva $0$, luego se multiplican los maxitérminos.
\end{enumerate}

\begin{table}[h]
    \centering
    \begin{tabular}{cccccc}
        \toprule
        \textbf{x} & \textbf{y} & \textbf{z} & \textbf{S} & \textbf{Minterminos} & \textbf{Maxiterminos}\\
        \midrule
        0 & 0 & 0 & 0 & $m_0 = x'y'z'$ & $M_0 = x+y+z$\\
        0 & 0 & 1 & 1 & $m_1 = x'y'z$ & $M_1 = x+y+z'$\\
        0 & 1 & 0 & 1 & $m_2 = x'yz'$ & $M_2 = x+y'+z$\\
        0 & 1 & 1 & 0 & $m_3 = x'yz$ & $M_3 = x+y'+z'$\\
        1 & 0 & 0 & 1 & $m_4 = xy'z'$ & $M_4 = x'+y+z$\\
        1 & 0 & 1 & 0 & $m_5 = xy'z$ & $M_5 = x'+y+z'$\\
        1 & 1 & 0 & 0 & $m_6 = xyz'$ & $M_6 = x'+y'+z$\\
        1 & 1 & 1 & 1 & $m_7 = xyz$ & $M_7 = x'+y'+z'$\\
        \bottomrule
    \end{tabular}
\end{table}

Con esto se tienen las formas canónicas de la función lógica, las cuales se pueden simplificar utilizando mapas de Karnaugh.
\begin{itemize}
    \item \textbf{Suma de productos:} $S = m_1 + m_2 + m_4 + m_7 = x'y'z + x'yz' + xy'z' + xyz$,
    \item \textbf{Producto de sumas:} $S = M_0 \cdot M_3 \cdot M_5 \cdot M_6 = (x+y+z)(x+y'+z')(x'+y+z')(x'+y'+z)$.
\end{itemize}


\subsection{Operaciones de Lectura y Escritura}
Las dos operaciones que efectúa una memoria de acceso aleatorio son escritura y lectura. La señal de escritura especifica una operación de transferencia hacia adentro, y la de lectura, una de transferencia hacia afuera. Al aceptar una de estas señales de control, los circuitos internos de la memoria efectúan la operación deseada.
Los pasos que deben seguirse para transferir una nueva palabra a la memoria son:
\begin{enumerate}
    \item Aplique la dirección binaria de la localidad deseada a las líneas de dirección.
    \item Aplique a las líneas de entrada de datos los bits de datos que se guardarán en la memoria.
    \item Active la entrada \texttt{escribir}.
\end{enumerate}

La unidad de memoria tomará entonces los bits de las líneas de datos de entrada y los almacenará en la localidad especificada por las líneas de dirección.
Los pasos que deben seguirse para sacar de la memoria una palabra almacenada son:
\begin{enumerate}
    \item Aplique a las líneas de dirección la dirección binaria de la localidad deseada.
    \item Active la entrada \texttt{leer}.
\end{enumerate}

La unidad de memoria tomará entonces los bits de la localidad seleccionada por la dirección
y los aplicará a las líneas de datos de salida. El contenido de la localidad seleccionada no cambia después de la lectura.

\subsection{PLA (Programmable Logic Array)}
Programmable Logic Array (PLA) es un dispositivo lógico de arquitectura fija con puertas AND programables seguidas de puertas OR programables. PLA es básicamente un tipo de dispositivo lógico programable que se utiliza para construir un circuito digital reconfigurable. La estructura de un PLA es la siguiente:

\begin{figure}[h]
    \centering
    \includegraphics[scale=0.2]{img/pla.png}
    \caption{Estructura de un PLA}
\end{figure}

El funcionamiento se podria resumir en tres pasos:

\begin{enumerate}
    \item \textbf{Programación:} el usuario define la función lógica que se desea implementar.
    \item \textbf{Generación de términos del producto:} las entradas se aplican a la matriz de puertas AND para producir un conjunto de términos de producto.
    \item \textbf{Generación de suma de términos:} los términos de producto se aplican a la matriz de puertas OR para producir la salida.
\end{enumerate}

\newpage
\subsubsection{Método para implementar una función lógica en un PLA}
\begin{mdframed}[backgroundcolor=gray!10,linewidth=0]
    \begin{enumerate}
        \item Una vez se tenga la función lógica simplificada, se deben identificar los términos de la función.
        \item Cada término de la función se convierte en una fila de la matriz de puertas AND.
        \item Luego se suman los términos de la función y se convierten en una fila de la matriz de puertas OR.
        \item La salida de la matriz de puertas OR es la función lógica implementada en el PLA.
    \end{enumerate}
\end{mdframed}


\subsection{Construcción de memorias}
Cuando surge la pregunta: ¿Cuantos chips de memoria se necesitan para almacenar $1K$ palabras de $16$ bits cada una? La respuesta es: $1K \times 16 = 16K$ bits. Si cada chip de memoria tiene $1K$ bits, entonces se necesitan $16$ chips.

\begin{mdframed}[backgroundcolor=gray!10,linewidth=0]
    Para generalizar, si se tienen $m$ palabras de $n$ bits cada una, se necesitan $m \times n$ bits. Si cada chip de memoria tiene $k$ bits, entonces se necesitan $\frac{m \times n}{k}$ chips.
\end{mdframed}

\newpage
\subsubsection{Tamaño de la memoria y dirección}
La cantidad de bits de una memoria se calcula como $2^{\text{dirección}} \times \text{tamaño de palabra}$. Por ejemplo, si se tiene una memoria de $16$ palabras de $8$ bits cada una, la cantidad de bits de la memoria es $2^4 \times 8 = 128$ bits. 

Generalmente el gráfico de una memoria se representa de la siguiente manera:

\begin{figure}[h]
    \centering
    \includegraphics[scale=1]{img/diagramamem.pdf}
    \caption{Diagrama de una memoria}
\end{figure}
Donde de la cantidad de líneas de dirección se obtiene la cantidad de palabras y de la cantidad de líneas de datos se obtiene el tamaño de la palabra.


\newpage
\subsection{Decodificadores}

Es un circuito combinacional cuya función es detectar la presencia de una determinada combinación de bits en sus entradas y señalar la presencia de este código mediante un cierto nivel de salida.

\begin{mdframed}[backgroundcolor=gray!10,linewidth=0]
Un decodificador posee $N$ líneas de entrada para gestionar $N$ bits y en una de las $2^N$ líneas de salida indica la presencia de una o mas combinaciones de $n$ bits. Es decir, para cualquier código dado en las entradas solo se activa una de las $N$ posibles salidas.
\end{mdframed}

Por ejemplo un decodificador de 3 entradas y 8 salidas, se activa una de las 8 salidas según la combinación de los 3 bits de entrada.

\begin{figure}[h]
\centering
\includegraphics[scale=0.5]{img/3a8.png}
\caption{Decodificador de 3 entradas y 8 salidas}
\end{figure}

La tabla de verdad de un decodificador de 3 entradas y 8 salidas es la siguiente:

\begin{table}[h]
    \centering
    \begin{tabular}{ccc|cccccccc}
        \toprule
        \textbf{X} & \textbf{Y} & \textbf{Z} & \textbf{Y0} & \textbf{Y1} & \textbf{Y2} & \textbf{Y3} & \textbf{Y4} & \textbf{Y5} & \textbf{Y6} & \textbf{Y7}\\
        \midrule
        0 & 0 & 0 & 1 & 0 & 0 & 0 & 0 & 0 & 0 & 0\\
        0 & 0 & 1 & 0 & 1 & 0 & 0 & 0 & 0 & 0 & 0\\
        0 & 1 & 0 & 0 & 0 & 1 & 0 & 0 & 0 & 0 & 0\\
        0 & 1 & 1 & 0 & 0 & 0 & 1 & 0 & 0 & 0 & 0\\
        1 & 0 & 0 & 0 & 0 & 0 & 0 & 1 & 0 & 0 & 0\\
        1 & 0 & 1 & 0 & 0 & 0 & 0 & 0 & 1 & 0 & 0\\
        1 & 1 & 0 & 0 & 0 & 0 & 0 & 0 & 0 & 1 & 0\\
        1 & 1 & 1 & 0 & 0 & 0 & 0 & 0 & 0 & 0 & 1\\
        \bottomrule
    \end{tabular}
\end{table}

\subsubsection{Implementación de lógica con decodificadores}
Un decodificador produce los $2^n$ minitérminos de $n$ variables de entrada. Puesto que cualquier función booleana es susceptible de expresarse como suma de minitérminos, es posible utilizar un decodificador para generar los minitérminos y una compuerta \texttt{OR} externa para formar la suma lógica. Así, cualquier circuito combinacional con $n$ entradas y $m$ salidas se puede implementar con un decodificador de $n$ a $2^n$ líneas y $m$ compuertas \texttt{OR}.

El procedimiento para implementar un circuito combinacional con un decodificador y compuertas OR requiere expresar la función booleana del circuito como suma de minitérminos. Entonces se escoge un decodificador que genere todos los minitérminos de las variables de entrada. Las entradas a cada compuerta \texttt{OR} se escogen de entre las salidas del decodificador, de acuerdo con la lista de minitérminos de cada función.

\newpage
\subsubsection{Ejemplo de Implementación}
Se tiene que implementar la lógica de un sumador completo, la tabla es la siguiente:
\begin{table}[h]
    \centering
    \begin{tabular}{ccc|cc}
        \toprule
        \textbf{A} & \textbf{B} & \textbf{$C_{in}$} & \textbf{S} & \textbf{$C_{out}$}\\
        \midrule
        0 & 0 & 0 & 0 & 0\\
        0 & 0 & 1 & 1 & 0\\
        0 & 1 & 0 & 1 & 0\\
        0 & 1 & 1 & 0 & 1\\
        1 & 0 & 0 & 1 & 0\\
        1 & 0 & 1 & 0 & 1\\
        1 & 1 & 0 & 0 & 1\\
        1 & 1 & 1 & 1 & 1\\
        \bottomrule
    \end{tabular}
\end{table}

De la tabla obtenemos las funciones para el circuito combinacional en forma de suma de minitérminos:

\begin{align*}
    S(x,y,z) &= \sum (1, 2, 4, 7) \\
    C_{out}(x,y,z) &= \sum (3, 5, 6, 7) 
\end{align*}

Puesto que hay tres entradas y un total de ocho minitérminos, se necesita un decodificador de 3 a 8 líneas. El decodificador genera los ocho minitérminos para $x$, $y$, $z$. La compuerta OR de la salida S forma la suma lógica de los minitérminos 1, 2, 4 y 7. La compuerta \texttt{OR} de la salida $C_{out}$ forma la suma lógica de los minitérminos 3, 5, 6 y 7.

\begin{figure}[h]
\centering
\includegraphics[scale=0.8]{img/sumador.png}
\caption{Implementación de un sumador completo con decodificador}
\end{figure}

\subsection{Ampliación de capacidad}
Para ampliar la capacidad de una memoria, se deben agregar chips de memoria en serie y se deben conectar las líneas de dirección de manera que se seleccione el chip adecuado.

\subsubsection{Ejemplo}
En el ejemplo se debe conformar un banco de
4K direcciones, cada una de 4 bits, y para hacerlo se cuenta con chips de 1K x 4 bits. En primer lugar dividimos la capacidad total de memoria necesaria por la capacidad de cada chip, a los efectos de obtener la cantidad de chips que debemos emplear. En este caso se necesitarán 4 chips.

Se pretende aumentar la cantidad de direcciones disponibles. Cada chip posee 10 líneas de dirección, pero el banco es de 4K, o sea que necesita 12 líneas. Las líneas A0, hasta A9 (10 bits) van a todos los chips simultáneamente. Las líneas restantes (A10 y A11) entran a un decodificador cuya función es la de ir habilitando uno a uno, a los diferentes chips de memorias del banco, con el objeto de evitar un conflicto en el Bus de Datos. De este modo, sólo un chip por vez estará activo, quedando el resto, en alta impedancia, con lo que cada chip pondrá, o recibirá, datos del bus de datos en forma individual.

Lo vamos a representar con una tabla, primero tendremos el chip, cada chip tiene las direcciones que tomará, y luego las líneas de dirección que se le asignan. Podemos notar que como se busca tener 4K direcciones, se necesitan 12 líneas de dirección.

\begin{figure}[h]
    \centering
    \includegraphics[scale=0.8]{img/tabla.png}
\end{figure}
Se puede observar que el banco funcionará siempre que A12 sea 0, ya que en caso contrario se generarían espejos. Luego la parte verde son las líneas de dirección que comparten todos los chips, y la parte azul son las líneas de dirección que se le asignan a cada chip. Es decir las que usaremos para seleccionar el chip que queremos leer o escribir. Para ello podemos notar que usar un decodificador de 2x4 nos permitirá seleccionar el chip que queremos leer o escribir.

\begin{figure}[h]
    \centering
    \includegraphics[scale=0.8]{img/ejmemorias.pdf}
\end{figure}
%-------------------------------------------------%


\newpage
%-- Direccionamiento y lógica de Decodificación de Memorias --%
\section{Direccionamiento y Lógica de Decodificación de Memorias}\label{sec:mem}

\subsection{Producto de Sumas y Suma de Productos (Maxitérminos y Mintérminos)}
Las funciones lógicas se pueden expresar de dos formas diferentes, como una suma de productos o como un producto de sumas. La forma de obtener dicha expresión se basa en identificar que valores de salida tomarán el valor de $1$ y el valor de $0$.
\newline Supongamos que se tiene una función lógica de tres variables $x$, $y$ y $z$, con la siguiente tabla de verdad:

\begin{table}[h]
    \centering
    \begin{tabular}{ccc|c}
        \toprule
        \textbf{x} & \textbf{y} & \textbf{z} & \textbf{S}\\
        \midrule
        0 & 0 & 0 & 0\\
        0 & 0 & 1 & 1\\
        0 & 1 & 0 & 1\\
        0 & 1 & 1 & 0\\
        1 & 0 & 0 & 1\\
        1 & 0 & 1 & 0\\
        1 & 1 & 0 & 0\\
        1 & 1 & 1 & 1\\
        \bottomrule
    \end{tabular}
\end{table}

\begin{enumerate}
    \item Para expresar como \textbf{suma de productos}, se toman los minitérminos que hacen que la función devuelva $1$, luego se suman los minitérminos.
    \item Para expresar como \textbf{producto de sumas}, se toman los maxitérminos que hacen que la función devuelva $0$, luego se multiplican los maxitérminos.
\end{enumerate}

\begin{table}[h]
    \centering
    \begin{tabular}{cccccc}
        \toprule
        \textbf{x} & \textbf{y} & \textbf{z} & \textbf{S} & \textbf{Minterminos} & \textbf{Maxiterminos}\\
        \midrule
        0 & 0 & 0 & 0 & $m_0 = x'y'z'$ & $M_0 = x+y+z$\\
        0 & 0 & 1 & 1 & $m_1 = x'y'z$ & $M_1 = x+y+z'$\\
        0 & 1 & 0 & 1 & $m_2 = x'yz'$ & $M_2 = x+y'+z$\\
        0 & 1 & 1 & 0 & $m_3 = x'yz$ & $M_3 = x+y'+z'$\\
        1 & 0 & 0 & 1 & $m_4 = xy'z'$ & $M_4 = x'+y+z$\\
        1 & 0 & 1 & 0 & $m_5 = xy'z$ & $M_5 = x'+y+z'$\\
        1 & 1 & 0 & 0 & $m_6 = xyz'$ & $M_6 = x'+y'+z$\\
        1 & 1 & 1 & 1 & $m_7 = xyz$ & $M_7 = x'+y'+z'$\\
        \bottomrule
    \end{tabular}
\end{table}

Con esto se tienen las formas canónicas de la función lógica, las cuales se pueden simplificar utilizando mapas de Karnaugh.
\begin{itemize}
    \item \textbf{Suma de productos:} $S = m_1 + m_2 + m_4 + m_7 = x'y'z + x'yz' + xy'z' + xyz$,
    \item \textbf{Producto de sumas:} $S = M_0 \cdot M_3 \cdot M_5 \cdot M_6 = (x+y+z)(x+y'+z')(x'+y+z')(x'+y'+z)$.
\end{itemize}


\subsection{Operaciones de Lectura y Escritura}
Las dos operaciones que efectúa una memoria de acceso aleatorio son escritura y lectura. La señal de escritura especifica una operación de transferencia hacia adentro, y la de lectura, una de transferencia hacia afuera. Al aceptar una de estas señales de control, los circuitos internos de la memoria efectúan la operación deseada.
Los pasos que deben seguirse para transferir una nueva palabra a la memoria son:
\begin{enumerate}
    \item Aplique la dirección binaria de la localidad deseada a las líneas de dirección.
    \item Aplique a las líneas de entrada de datos los bits de datos que se guardarán en la memoria.
    \item Active la entrada \texttt{escribir}.
\end{enumerate}

La unidad de memoria tomará entonces los bits de las líneas de datos de entrada y los almacenará en la localidad especificada por las líneas de dirección.
Los pasos que deben seguirse para sacar de la memoria una palabra almacenada son:
\begin{enumerate}
    \item Aplique a las líneas de dirección la dirección binaria de la localidad deseada.
    \item Active la entrada \texttt{leer}.
\end{enumerate}

La unidad de memoria tomará entonces los bits de la localidad seleccionada por la dirección
y los aplicará a las líneas de datos de salida. El contenido de la localidad seleccionada no cambia después de la lectura.

\subsection{PLA (Programmable Logic Array)}
Programmable Logic Array (PLA) es un dispositivo lógico de arquitectura fija con puertas AND programables seguidas de puertas OR programables. PLA es básicamente un tipo de dispositivo lógico programable que se utiliza para construir un circuito digital reconfigurable. La estructura de un PLA es la siguiente:

\begin{figure}[h]
    \centering
    \includegraphics[scale=0.2]{img/pla.png}
    \caption{Estructura de un PLA}
\end{figure}

El funcionamiento se podria resumir en tres pasos:

\begin{enumerate}
    \item \textbf{Programación:} el usuario define la función lógica que se desea implementar.
    \item \textbf{Generación de términos del producto:} las entradas se aplican a la matriz de puertas AND para producir un conjunto de términos de producto.
    \item \textbf{Generación de suma de términos:} los términos de producto se aplican a la matriz de puertas OR para producir la salida.
\end{enumerate}

\newpage
\subsubsection{Método para implementar una función lógica en un PLA}
\begin{mdframed}[backgroundcolor=gray!10,linewidth=0]
    \begin{enumerate}
        \item Una vez se tenga la función lógica simplificada, se deben identificar los términos de la función.
        \item Cada término de la función se convierte en una fila de la matriz de puertas AND.
        \item Luego se suman los términos de la función y se convierten en una fila de la matriz de puertas OR.
        \item La salida de la matriz de puertas OR es la función lógica implementada en el PLA.
    \end{enumerate}
\end{mdframed}


\subsection{Construcción de memorias}
Cuando surge la pregunta: ¿Cuantos chips de memoria se necesitan para almacenar $1K$ palabras de $16$ bits cada una? La respuesta es: $1K \times 16 = 16K$ bits. Si cada chip de memoria tiene $1K$ bits, entonces se necesitan $16$ chips.

\begin{mdframed}[backgroundcolor=gray!10,linewidth=0]
    Para generalizar, si se tienen $m$ palabras de $n$ bits cada una, se necesitan $m \times n$ bits. Si cada chip de memoria tiene $k$ bits, entonces se necesitan $\frac{m \times n}{k}$ chips.
\end{mdframed}

\newpage
\subsubsection{Tamaño de la memoria y dirección}
La cantidad de bits de una memoria se calcula como $2^{\text{dirección}} \times \text{tamaño de palabra}$. Por ejemplo, si se tiene una memoria de $16$ palabras de $8$ bits cada una, la cantidad de bits de la memoria es $2^4 \times 8 = 128$ bits. 

Generalmente el gráfico de una memoria se representa de la siguiente manera:

\begin{figure}[h]
    \centering
    \includegraphics[scale=1]{img/diagramamem.pdf}
    \caption{Diagrama de una memoria}
\end{figure}
Donde de la cantidad de líneas de dirección se obtiene la cantidad de palabras y de la cantidad de líneas de datos se obtiene el tamaño de la palabra.


\newpage
\subsection{Decodificadores}

Es un circuito combinacional cuya función es detectar la presencia de una determinada combinación de bits en sus entradas y señalar la presencia de este código mediante un cierto nivel de salida.

\begin{mdframed}[backgroundcolor=gray!10,linewidth=0]
Un decodificador posee $N$ líneas de entrada para gestionar $N$ bits y en una de las $2^N$ líneas de salida indica la presencia de una o mas combinaciones de $n$ bits. Es decir, para cualquier código dado en las entradas solo se activa una de las $N$ posibles salidas.
\end{mdframed}

Por ejemplo un decodificador de 3 entradas y 8 salidas, se activa una de las 8 salidas según la combinación de los 3 bits de entrada.

\begin{figure}[h]
\centering
\includegraphics[scale=0.5]{img/3a8.png}
\caption{Decodificador de 3 entradas y 8 salidas}
\end{figure}

La tabla de verdad de un decodificador de 3 entradas y 8 salidas es la siguiente:

\begin{table}[h]
    \centering
    \begin{tabular}{ccc|cccccccc}
        \toprule
        \textbf{X} & \textbf{Y} & \textbf{Z} & \textbf{Y0} & \textbf{Y1} & \textbf{Y2} & \textbf{Y3} & \textbf{Y4} & \textbf{Y5} & \textbf{Y6} & \textbf{Y7}\\
        \midrule
        0 & 0 & 0 & 1 & 0 & 0 & 0 & 0 & 0 & 0 & 0\\
        0 & 0 & 1 & 0 & 1 & 0 & 0 & 0 & 0 & 0 & 0\\
        0 & 1 & 0 & 0 & 0 & 1 & 0 & 0 & 0 & 0 & 0\\
        0 & 1 & 1 & 0 & 0 & 0 & 1 & 0 & 0 & 0 & 0\\
        1 & 0 & 0 & 0 & 0 & 0 & 0 & 1 & 0 & 0 & 0\\
        1 & 0 & 1 & 0 & 0 & 0 & 0 & 0 & 1 & 0 & 0\\
        1 & 1 & 0 & 0 & 0 & 0 & 0 & 0 & 0 & 1 & 0\\
        1 & 1 & 1 & 0 & 0 & 0 & 0 & 0 & 0 & 0 & 1\\
        \bottomrule
    \end{tabular}
\end{table}

\subsubsection{Implementación de lógica con decodificadores}
Un decodificador produce los $2^n$ minitérminos de $n$ variables de entrada. Puesto que cualquier función booleana es susceptible de expresarse como suma de minitérminos, es posible utilizar un decodificador para generar los minitérminos y una compuerta \texttt{OR} externa para formar la suma lógica. Así, cualquier circuito combinacional con $n$ entradas y $m$ salidas se puede implementar con un decodificador de $n$ a $2^n$ líneas y $m$ compuertas \texttt{OR}.

El procedimiento para implementar un circuito combinacional con un decodificador y compuertas OR requiere expresar la función booleana del circuito como suma de minitérminos. Entonces se escoge un decodificador que genere todos los minitérminos de las variables de entrada. Las entradas a cada compuerta \texttt{OR} se escogen de entre las salidas del decodificador, de acuerdo con la lista de minitérminos de cada función.

\newpage
\subsubsection{Ejemplo de Implementación}
Se tiene que implementar la lógica de un sumador completo, la tabla es la siguiente:
\begin{table}[h]
    \centering
    \begin{tabular}{ccc|cc}
        \toprule
        \textbf{A} & \textbf{B} & \textbf{$C_{in}$} & \textbf{S} & \textbf{$C_{out}$}\\
        \midrule
        0 & 0 & 0 & 0 & 0\\
        0 & 0 & 1 & 1 & 0\\
        0 & 1 & 0 & 1 & 0\\
        0 & 1 & 1 & 0 & 1\\
        1 & 0 & 0 & 1 & 0\\
        1 & 0 & 1 & 0 & 1\\
        1 & 1 & 0 & 0 & 1\\
        1 & 1 & 1 & 1 & 1\\
        \bottomrule
    \end{tabular}
\end{table}

De la tabla obtenemos las funciones para el circuito combinacional en forma de suma de minitérminos:

\begin{align*}
    S(x,y,z) &= \sum (1, 2, 4, 7) \\
    C_{out}(x,y,z) &= \sum (3, 5, 6, 7) 
\end{align*}

Puesto que hay tres entradas y un total de ocho minitérminos, se necesita un decodificador de 3 a 8 líneas. El decodificador genera los ocho minitérminos para $x$, $y$, $z$. La compuerta OR de la salida S forma la suma lógica de los minitérminos 1, 2, 4 y 7. La compuerta \texttt{OR} de la salida $C_{out}$ forma la suma lógica de los minitérminos 3, 5, 6 y 7.

\begin{figure}[h]
\centering
\includegraphics[scale=0.8]{img/sumador.png}
\caption{Implementación de un sumador completo con decodificador}
\end{figure}

\subsection{Ampliación de capacidad}
Para ampliar la capacidad de una memoria, se deben agregar chips de memoria en serie y se deben conectar las líneas de dirección de manera que se seleccione el chip adecuado.

\subsubsection{Ejemplo}
En el ejemplo se debe conformar un banco de
4K direcciones, cada una de 4 bits, y para hacerlo se cuenta con chips de 1K x 4 bits. En primer lugar dividimos la capacidad total de memoria necesaria por la capacidad de cada chip, a los efectos de obtener la cantidad de chips que debemos emplear. En este caso se necesitarán 4 chips.

Se pretende aumentar la cantidad de direcciones disponibles. Cada chip posee 10 líneas de dirección, pero el banco es de 4K, o sea que necesita 12 líneas. Las líneas A0, hasta A9 (10 bits) van a todos los chips simultáneamente. Las líneas restantes (A10 y A11) entran a un decodificador cuya función es la de ir habilitando uno a uno, a los diferentes chips de memorias del banco, con el objeto de evitar un conflicto en el Bus de Datos. De este modo, sólo un chip por vez estará activo, quedando el resto, en alta impedancia, con lo que cada chip pondrá, o recibirá, datos del bus de datos en forma individual.

Lo vamos a representar con una tabla, primero tendremos el chip, cada chip tiene las direcciones que tomará, y luego las líneas de dirección que se le asignan. Podemos notar que como se busca tener 4K direcciones, se necesitan 12 líneas de dirección.

\begin{figure}[h]
    \centering
    \includegraphics[scale=0.8]{img/tabla.png}
\end{figure}
Se puede observar que el banco funcionará siempre que A12 sea 0, ya que en caso contrario se generarían espejos. Luego la parte verde son las líneas de dirección que comparten todos los chips, y la parte azul son las líneas de dirección que se le asignan a cada chip. Es decir las que usaremos para seleccionar el chip que queremos leer o escribir. Para ello podemos notar que usar un decodificador de 2x4 nos permitirá seleccionar el chip que queremos leer o escribir.

\begin{figure}[h]
    \centering
    \includegraphics[scale=0.8]{img/ejmemorias.pdf}
\end{figure}
%-------------------------------------------------%

\newpage
%-------------- Circuitos Secuenciales --------------%
\section{Circuitos Secuenciales}\label{sec:sec}

\documentclass[20pt,margin=1in,innermargin=-4.5in,blockverticalspace=-0.25in]{tikzposter}
\geometry{paperwidth=42in,paperheight=30in}
\usepackage[utf8]{inputenc}
\usepackage{amsmath}
\usepackage{amsfonts}
\usepackage{amsthm}
\usepackage{amssymb}
\usepackage{mathrsfs}
\usepackage{graphicx}
\usepackage{adjustbox}
\usepackage{enumitem}
\usepackage[backend=biber,style=numeric]{biblatex}
\usepackage{emory-theme}
\usepackage{emory-theme}
\usepackage{mwe} % for placeholder images
\usepackage{booktabs}
% set theme parameters
\tikzposterlatexaffectionproofoff
\usetheme{EmoryTheme}
\usecolorstyle{EmoryStyle}

\title{Sistemas de Numeración}
\author{Pedro Villar}
\institute{Organización del Computador - Primer Cuatrimestre 2024}	
\titlegraphic{\adjustbox{left=0.35\textwidth}{\includegraphics[width=0.3\textwidth]{famaf-logo.jpg}}}

% begin document
\begin{document}
\maketitle
\centering

\begin{columns}
    \column{0.33}
    \block{Pasar de Hexadecimal a Binario}{
        El sistema hexadecimal es una base 16, por lo que cada dígito puede representar 4 bits. Para pasar de hexadecimal a binario, simplemente se reemplaza cada dígito por su representación en 4 bits. La tabla de conversión es la siguiente:
        
    \begin{center}
     \begin{tabular}{cccccc} % Eliminar las líneas verticales
        \toprule % Línea superior más gruesa
        Hexadecimal & Binario & Hexadecimal & Binario & Hexadecimal & Binario\\
        \midrule % Línea intermedia más fina
        0 & 0000 & 5 & 0101 & A & 1010 \\
        1 & 0001 & 6 & 0110 & B & 1011 \\
        2 & 0010 & 7 & 0111 & C & 1100 \\
        3 & 0011 & 8 & 1000 & D & 1101 \\
        4 & 0100 & 9 & 1001 & E & 1110 \\
          &      &   &      & F & 1111 \\
        \bottomrule % Línea inferior más gruesa
    \end{tabular}
    \end{center}
    Por ejemplo, para pasar de \texttt{0x123456} a binario, simplemente se reemplaza cada dígito por su representación en 4 bits:
    \begin{center}
        \texttt{0x123456} $\rightarrow$ \texttt{0001 0010 0011 0100 0101 0110}
    \end{center}
    }

    \block{Pasar de Binario a Hexadecimal}{
        Para pasar de binario a hexadecimal, simplemente se agrupan los bits de a 4 y se reemplaza cada grupo por su representación en hexadecimal, teniendo en cuenta las siguientes consideraciones:
        \begin{itemize}
            \item Si el número tiene parte decimal, se agrupan los bits de a 4 a partir del punto decimal y luego se hace la parte entera.
            \item Si el número de bits no es múltiplo de 4, se agrega un 0 a la izquierda para completar el último grupo.
            \item En el caso de que en la parte fraccionaria se necesite agregar ceros se agregan a la derecha.
        \end{itemize}
        Por ejemplo, para pasar de \texttt{10 1100 1101 1011.1100 0010 000} a hexadecimal:
        \begin{equation*}
            \underbrace{0010}_{2}\underbrace{1100}_{C}\underbrace{1101}_{D}\underbrace{1011}_{B} \ \ \ \ \underbrace{1100}_{C}\underbrace{0010}_{2}\underbrace{0000}_{0}
        \end{equation*}
        \begin{equation*}
            (10110011011011.11000010000)_2 = 0x2CDB.C20 
        \end{equation*}
    }

    \block{Pasar de Hexadecimal a Decimal}{
        Para pasar de hexadecimal a decimal, simplemente se reemplaza cada dígito por su representación en decimal y se multiplica por la potencia de 16 correspondiente. Por ejemplo, para pasar de \texttt{0x123456} a decimal:
        \begin{equation*}
            0x123456 = 1 \cdot 16^5 + 2 \cdot 16^4 + 3 \cdot 16^3 + 4 \cdot 16^2 + 5 \cdot 16^1 + 6 \cdot 16^0 = 1193046
        \end{equation*}
    }
    
    \block{Pasar de decimal a binario}{
        Para pasar de decimal a binario, simplemente se divide el número por 2 y se toma el resto. Luego se divide el cociente por 2 y se toma el resto, y así sucesivamente hasta que el cociente sea 0. Luego se toman los restos en orden inverso. Por ejemplo, para pasar de 59 a binario:
        \begin{equation*}
            59 = 1 \cdot 2^5 + 1 \cdot 2^4 + 0 \cdot 2^3 + 1 \cdot 2^2 + 1 \cdot 2^1 + 1 \cdot 2^0 = 111011
        \end{equation*}
    }

    \column{0.66}
    \block{Complemento a 2 para números negativos}{
        El complemento a 2 es una forma de representar números negativos en binario. Para obtener el complemento a 2 de un número negativo, primero se agregan ceros a la izquierda para \textbf{completar la cantidad de bits} que de el registro, luego se \textbf{invierten todos los bits} y \textbf{se le suma 1 al resultado}.
        Por ejemplo, para obtener el complemento a 2 de -121 en 8 bits:
        \begin{align*}
            121/2 &= 60 \quad \text{residuo } 1\\
            60/2 &= 30 \quad \text{residuo } 0\\
            30/2 &= 15 \quad \text{residuo } 0\\
            15/2 &= 7 \quad \text{residuo } 1 \quad \quad \Rightarrow 01111001 \rightarrow 10000110 + 1 = 10000111 \\
            7/2 &= 3 \quad \text{residuo } 1\\
            3/2 &= 1 \quad \text{residuo } 1\\
            1/2 &= 0 \quad \text{residuo } 1
        \end{align*}
    }
    %\column{0.33}
    \block{Decimal negativo a binario con complemento a 2}{
        \begin{enumerate}
            \item Calcular el valor absoluto del número en binario.
            \item Completar con ceros a la izquierda para que el número tenga la cantidad de bits que se desea.
            \item Aplicar el complemento a 2.
        \end{enumerate}
    }
    %\column{0.33}
    \block{Binario con complemento a 2 a decimal}{
        \begin{enumerate}
            \item Si el bit más significativo es 1, el número es negativo. Se aplica el complemento a 2 para obtener el valor absoluto.
            \item Se multiplica cada bit por la potencia de 2 correspondiente y se suma.
        \end{enumerate}
    }
    \block{Suma binaria}{
        Para sumar dos números en binario, simplemente se suman los bits de a pares, teniendo en cuenta que:
        \begin{itemize}
            \item 0 + 0 = 0
            \item 0 + 1 = 1
            \item 1 + 0 = 1
            \item 1 + 1 = 10
        \end{itemize}
        Si la suma de dos bits es 10, se escribe 0 y se lleva 1 al siguiente par de bits.
    }
    \block{Resta binaria}{
        Para restar dos números en binario, se aplica el complemento a 2 al sustraendo y se suma al minuendo. 
        \newline Es decir si se busca hacer $A - B$, se hace $A + (-B)$. Por ejemplo para restar $10000000 - 11010000$:
        \begin{align*}
            10000000 - 11010000 &= 10000000 + 00110000 = 10110000
        \end{align*}
        El resultado es $0xB0$ en hexadecimal.
    }
\end{columns}

\end{document}
%-------------------------------------------------%

\end{document}