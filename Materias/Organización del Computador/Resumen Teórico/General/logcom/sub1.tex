\subsection{Producto de Sumas y Suma de Productos (Maxitérminos y Mintérminos)}
Las funciones lógicas se pueden expresar de dos formas diferentes, como una suma de productos o como un producto de sumas. La forma de obtener dicha expresión se basa en identificar que valores de salida tomarán el valor de $1$ y el valor de $0$.
\newline Supongamos que se tiene una función lógica de tres variables $x$, $y$ y $z$, con la siguiente tabla de verdad:

\begin{table}[h]
    \centering
    \begin{tabular}{ccc|c}
        \toprule
        \textbf{x} & \textbf{y} & \textbf{z} & \textbf{S}\\
        \midrule
        0 & 0 & 0 & 0\\
        0 & 0 & 1 & 1\\
        0 & 1 & 0 & 1\\
        0 & 1 & 1 & 0\\
        1 & 0 & 0 & 1\\
        1 & 0 & 1 & 0\\
        1 & 1 & 0 & 0\\
        1 & 1 & 1 & 1\\
        \bottomrule
    \end{tabular}
\end{table}

\begin{enumerate}
    \item Para expresar como \textbf{suma de productos}, se toman los minitérminos que hacen que la función devuelva $1$, luego se suman los minitérminos.
    \item Para expresar como \textbf{producto de sumas}, se toman los maxitérminos que hacen que la función devuelva $0$, luego se multiplican los maxitérminos.
\end{enumerate}

\begin{table}[h]
    \centering
    \begin{tabular}{cccccc}
        \toprule
        \textbf{x} & \textbf{y} & \textbf{z} & \textbf{S} & \textbf{Minterminos} & \textbf{Maxiterminos}\\
        \midrule
        0 & 0 & 0 & 0 & $m_0 = x'y'z'$ & $M_0 = x+y+z$\\
        0 & 0 & 1 & 1 & $m_1 = x'y'z$ & $M_1 = x+y+z'$\\
        0 & 1 & 0 & 1 & $m_2 = x'yz'$ & $M_2 = x+y'+z$\\
        0 & 1 & 1 & 0 & $m_3 = x'yz$ & $M_3 = x+y'+z'$\\
        1 & 0 & 0 & 1 & $m_4 = xy'z'$ & $M_4 = x'+y+z$\\
        1 & 0 & 1 & 0 & $m_5 = xy'z$ & $M_5 = x'+y+z'$\\
        1 & 1 & 0 & 0 & $m_6 = xyz'$ & $M_6 = x'+y'+z$\\
        1 & 1 & 1 & 1 & $m_7 = xyz$ & $M_7 = x'+y'+z'$\\
        \bottomrule
    \end{tabular}
\end{table}

Con esto se tienen las formas canónicas de la función lógica, las cuales se pueden simplificar utilizando mapas de Karnaugh.
\begin{itemize}
    \item \textbf{Suma de productos:} $S = m_1 + m_2 + m_4 + m_7 = x'y'z + x'yz' + xy'z' + xyz$,
    \item \textbf{Producto de sumas:} $S = M_0 \cdot M_3 \cdot M_5 \cdot M_6 = (x+y+z)(x+y'+z')(x'+y+z')(x'+y'+z)$.
\end{itemize}
