\documentclass[20pt,margin=1in,innermargin=-4.5in,blockverticalspace=-0.25in]{tikzposter}
\geometry{paperwidth=42in,paperheight=30in}
\usepackage[utf8]{inputenc}
\usepackage{amsmath}
\usepackage{amsfonts}
\usepackage{amsthm}
\usepackage{amssymb}
\usepackage{mathrsfs}
\usepackage{graphicx}
\usepackage{adjustbox}
\usepackage{enumitem}
\usepackage[backend=biber,style=numeric]{biblatex}
\usepackage{emory-theme}

\usepackage{mwe} % for placeholder images

% set theme parameters
\tikzposterlatexaffectionproofoff
\usetheme{EmoryTheme}
\usecolorstyle{EmoryStyle}

\title{Notación O grande y o pequeña}
\author{Pedro Villar}
\institute{Análisis Numérico - Primer Cuatrimestre 2024}	
\titlegraphic{\adjustbox{left=0.35\textwidth}{\includegraphics[width=0.3\textwidth]{famaf-logo.jpg}}}

% begin document
\begin{document}
\maketitle
\centering
\begin{columns}

    % ----------- Columna 1 - Teoría ------------------- %
    \column{0.40}
    \block{Sucesión convergente}{
        Una sucesión $\{x_n\}$ es convergente si existe un número $L$ tal que para todo $\varepsilon > 0$ existe un $N \in \mathbb{N}$ tal que si $n \ge N$ entonces $|x_n - L| < \varepsilon$.
    }

    \block{Convergencia lineal, superlineal y cuadrática}{
        Sea $\{x_n\}$ una sucesión convergente a $x_{ast}$.
        \begin{itemize}
            \item Se dice que la sucesión $\{x_n\}$ tiene tasa de convergencia (al menos) \textbf{lineal} si existe una constante $c$ tal que $0 < c < 1$ y un $N\in \mathbb{N}$ tal que
            \begin{equation*}
                |x_{n+1} - x_{\ast}|\leq c|x_n - x_{\ast}|, \quad \forall n \ge N.
            \end{equation*}
            \item Se dice que la tasa de convergencia es (al menos) \textbf{superlineal} si existe una sucesión $\{ \epsilon_n \}$ que converge a $0$ y un $N\in \mathbb{N}$ tal que
            \begin{equation*}
                |x_{n+1} - x_{\ast}| \leq \epsilon_n|x_n - x_{\ast}|, \quad \forall n \ge N.
            \end{equation*}
            \item Se dice que la tasa de convergencia es (al menos) \textbf{cuadrática} si existe una constante positiva $c$ y un $N\in \mathbb{N}$ tal que
            \begin{equation*}
                |x_{n+1} - x_{\ast}| \leq c|x_n - x_{\ast}|^2, \quad \forall n \ge N.
            \end{equation*}
        \end{itemize}
    }
    \block{Notación $\mathcal{O}$ grande y $O$ chica}{
    Introducimos una notación para comparar sucesiones y funciones. Sean $\{ x_n \}$ y $\{ \alpha_n \}$ dos sucesiones. 
    \begin{itemize}
        \item Decimos que $$\{ x_n \}= \mathcal{O}(\alpha_n)$$ si existe una constante $C > 0$ y un $r \in \mathbb{N}$ tal que $$|x_n| \leq C|\alpha_n|, \quad \forall n \geq r.$$
        \item Decimos que $$\{ x_n \}= O(\alpha_n)$$ si existe una sucesión $\{ \varepsilon_n \}$ que converge a $0$, con $\varepsilon_n \ge 0$ y un $r \in \mathbb{N}$ tal que $|x_n| \leq \varepsilon_n|\alpha_n|, \quad \forall n \geq r.$
    \end{itemize}
    Esta notación también se puede extender a funciones. Se dice que
    \begin{equation*}
        f(x) = \mathcal{O}(g(x)) \quad \text{cuando} \quad x\to \infty
    \end{equation*}
    si existe una constante $C > 0$ y un $r \in \mathbb{R}$ tal que $|f(x)| \leq C|g(x)|, \quad \forall x \geq r$.
    Análogamente, se dice que
    \begin{equation*}
        f(x) = O(g(x)) \quad \text{cuando} \quad x\to \infty
    \end{equation*}
    si $\lim_{x\to \infty} \frac{f(x)}{g(x)} = 0$.
    }
    \block{Ejemplo de notación $O$ con sucesiones}{
        \begin{equation*}
            \frac{1}{n \cdot ln(n)} = O \left( \frac{1}{n} \right).
        \end{equation*} 
        Si
        \begin{equation*}
            \frac{1}{n \cdot ln(n)} \leq \varepsilon_n \left( \frac{1}{n} \right).
        \end{equation*}
        basta tomar $\varepsilon_n = \frac{1}{ln(n)}$.
    }

    \column{0.60}
\end{columns}

\end{document}