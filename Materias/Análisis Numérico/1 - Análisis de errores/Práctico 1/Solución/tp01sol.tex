%%%%%%%%%%%%%%%%%%%%%%%%%%%%%%%%%%%%%%%%%%%%%%%%%%%%%%%%%%%%%%%%%%%%%%%%%%%%%%%%%%%%
% Configuración de Paquetes
\documentclass{article}
\usepackage[framemethod=TikZ]{mdframed}
\usepackage{booktabs}
\usepackage{float}
\usepackage{scrextend}
\usepackage{titletoc}
\usepackage[margin=1in]{geometry} 
\usepackage{amsmath,amsthm,amssymb,amsfonts, fancyhdr, color, comment, graphicx, environ}
\usepackage{xcolor}
\usepackage{mdframed}
\usepackage[shortlabels]{enumitem}
\usepackage{indentfirst}
\usepackage{hyperref}
\usepackage{listings}
\usepackage{tikz}
\usepackage[framemethod=TikZ]{mdframed}
\usepackage{mathptmx}
\usepackage{cfr-lm}
\hypersetup{
    colorlinks=true,
    linkcolor=blue,
    filecolor=magenta,      
    urlcolor=blue,
}
\setlength{\headheight}{1.5cm}
\renewcommand{\qed}{\quad\qedsymbol}
\usetikzlibrary{calc}
\renewcommand{\familydefault}{\sfdefault}
%%%%%%%%%%%%%%%%%%%%%%%%%%%%%%%%%%%%%%%%%%%%%%%%%%%%%%%%%%%%%%%%%%%%%%%%%%%%%%%%%%%%

\fancypagestyle{mipagina}{
    \fancyhf{} % Limpiar encabezado y pie de página
    \fancyhead[L]{Pedro Villar} % Nombre a la izquierda
    \fancyhead[C]{\rightmark} % Texto al centro 
    \fancyhead[R]{Análisis Numérico 1} % Texto a la derecha
    \fancyfoot[C]{\thepage} % Número de página al centro
    \renewcommand{\headrulewidth}{0.4pt} % Grosor de la línea horizontal en el encabezado
}

\pagestyle{mipagina}
\mdfsetup{skipabove=\topskip,skipbelow=\topskip}

% Box de Definición
\newcounter{def}[section]

\NewDocumentEnvironment{defi}{o}{%
    \stepcounter{def}%
    \begin{mdframed}[
        frametitle={%
            \begin{tikzpicture}[baseline=(current bounding box.east),outer sep=0pt]
                \node[anchor=east,rectangle,fill=blue!20,inner xsep=5pt] at (0,0) {\strut\IfValueTF{#1}{Definición~\thedef:~#1}{Definición~\thedef}};
            \end{tikzpicture}%
        },
        innertopmargin=5pt,
        linecolor=blue!20,
        linewidth=2pt,
        topline=true,
        frametitleaboveskip=-\ht\strutbox, % Ajuste de espacio entre título y contenido
        frametitlealignment={\hspace{5pt}}, % Ajuste de espacio entre el borde del frame y el título
    ]
}{%
    \end{mdframed}%
}

% Box de ejemplos
\newcounter{ejemplo}[section]

\NewDocumentEnvironment{ejemplo}{o}{%
    \stepcounter{ejemplo}%
    \begin{mdframed}[
        frametitle={%
            \begin{tikzpicture}[baseline=(current bounding box.east),outer sep=0pt]
                \node[anchor=east,rectangle,fill=brown!30,inner xsep=5pt] at (0,0) {\strut\IfValueTF{#1}{Ejemplo~\theejemplo:~#1}{Ejemplo~\theejemplo}};
            \end{tikzpicture}%
        },
        innertopmargin=5pt,
        linecolor=brown!30,
        linewidth=2pt,
        topline=true,
        frametitleaboveskip=-\ht\strutbox, % Ajuste de espacio entre título y contenido
        frametitlealignment={\hspace{5pt}}, % Ajuste de espacio entre el borde del frame y el título
    ]
}{%
    \end{mdframed}%
}

% Entorno para Métodos (color verde)
\newcounter{metodo}[section]
\NewDocumentEnvironment{metodo}{o}{%
    \stepcounter{metodo}%
    \begin{mdframed}[
        frametitle={%
            \begin{tikzpicture}[baseline=(current bounding box.east),outer sep=0pt]
                \node[anchor=east,rectangle,fill=green!30,inner xsep=5pt] at (0,0) {\strut\IfValueTF{#1}{Método~\themetodo:~#1}{Método~\themetodo}};
            \end{tikzpicture}%
        },
        innertopmargin=5pt,
        linecolor=green!30,
        linewidth=2pt,
        topline=true,
        frametitleaboveskip=-\ht\strutbox, % Ajuste de espacio entre título y contenido
        frametitlealignment={\hspace{5pt}}, % Ajuste de espacio entre el borde del frame y el título
    ]
}{%
    \end{mdframed}%
}

% Entorno para Observaciones (color amarillo más oscuro)
\newcounter{observacion}[section]
\NewDocumentEnvironment{observacion}{o}{%
    \stepcounter{observacion}%
    \begin{mdframed}[
        frametitle={%
            \begin{tikzpicture}[baseline=(current bounding box.east),outer sep=0pt]
                \node[anchor=east,rectangle,fill=yellow!50,inner xsep=5pt] at (0,0) {\strut\IfValueTF{#1}{Observación~\theobservacion:~#1}{Observación~\theobservacion}};
            \end{tikzpicture}%
        },
        innertopmargin=5pt,
        linecolor=yellow!50,
        linewidth=2pt,
        topline=true,
        frametitleaboveskip=-\ht\strutbox, % Ajuste de espacio entre título y contenido
        frametitlealignment={\hspace{5pt}}, % Ajuste de espacio entre el borde del frame y el título
    ]
}{%
    \end{mdframed}%
}

% Entorno para Ejercicio
\newcounter{ejer}[section]
\NewDocumentEnvironment{ejer}{o}{%
    \stepcounter{ejer}%
    \begin{mdframed}[
        frametitle={%
            \begin{tikzpicture}[baseline=(current bounding box.east),outer sep=0pt]
                \node[anchor=east,rectangle,fill=black!30,inner xsep=5pt] at (0,0) {\strut\IfValueTF{#1}{Ejercicio~\theejer:~#1}{Ejercicio~\theejer}};
            \end{tikzpicture}%
        },
        innertopmargin=5pt,
        linecolor=black!30,
        linewidth=2pt,
        topline=true,
        frametitleaboveskip=-\ht\strutbox, % Ajuste de espacio entre título y contenido
        frametitlealignment={\hspace{5pt}}, % Ajuste de espacio entre el borde del frame y el título
    ]
}{%
    \end{mdframed}%
}

\newenvironment{solution}
    {\textit{Solución:}}
    {}

%Configuraciones adicionales
\binoppenalty=\maxdimen 
\relpenalty=\maxdimen 
\setlength{\parindent}{0pt}

\begin{document}

\section*{Ejercicio 1}
\begin{itemize}
    \item[a)] Obtener la serie de Taylor centrada en 0 para la función $f(x) = ln(x + 1)$. Escribir la serie usando la notación de sumatorias. Dar una expresión para el resto cuando la serie es truncada en $k$ términos.
    \item[b)] Estimar el número de términos que deberán incluirse en la serie para aproximar $ln(1.5)$ con un margen de error no mayor que $10^{-10}$.
\end{itemize}

\begin{solution}
    \textbf{Inciso a)}
    \begin{itemize}
        \item Primero, necesitamos obtener las derivadas sucesivas de la función $f(x) = \ln(x + 1)$ evaluadas en $x = 0$.
        $$f(x) = \ln(x + 1)$$
        $$f'(x) = \frac{1}{x + 1}$$
        $$f'(0) = 1$$
        $$f''(x) = -\frac{1}{(x + 1)^2}$$
        $$f''(0) = -1$$
        $$f'''(x) = \frac{2}{(x + 1)^3}$$
        $$f'''(0) = 2$$
        $$f^{(n)}(x) = \frac{(-1)^{n-1}(n-1)!}{(x + 1)^n}$$
        $$f^{(n)}(0) = (-1)^{n-1}(n-1)!$$
        \item La serie de Taylor centrada en $x = 0$ para $f(x)$ viene dada por:
        $$f(x) = \sum_{n=0}^\infty \frac{f^{(n)}(0)}{n!}x^n$$
        Sustituyendo los valores encontrados anteriormente, tenemos:
        $$\ln(x + 1) = \sum_{n=0}^\infty \frac{(-1)^{n-1}(n-1)!}{n!}x^n$$
        \item Usando la notación de sumatorias, la serie de Taylor se puede escribir como:
        $$\ln(x + 1) = x - \frac{x^2}{2} + \frac{x^3}{3} - \frac{x^4}{4} + \cdots + \frac{(-1)^{n-1}x^n}{n} + \cdots$$
        $$\ln(x + 1) = \sum_{n=1}^\infty \frac{(-1)^{n-1}x^n}{n}$$
        \item Para obtener una expresión para el resto cuando la serie es truncada en $k$ términos, uso la fórmula del resto de Taylor:
        $$R_k(x) = \frac{f^{(k+1)}(\xi)}{(k+1)!}x^{k+1}$$
        donde $\xi$ es un punto entre 0 y $x$.
        En este caso,
        $$f^{(k+1)}(x) = \frac{(-1)^k k!}{(x + 1)^{k+1}}$$
        Entonces, el resto $R_k(x)$ cuando la serie es truncada en $k$ términos viene dado por:
        $$R_k(x) = \frac{(-1)^k k! x^{k+1}}{(k+1)!(x + 1)^{k+1}}$$
    \end{itemize}
\end{solution}
\newpage

\begin{solution}
    \textbf{Inciso b)}
    \begin{itemize}
        \item Primero, evaluo el resto $R_k(x)$ cuando $x = 0.5$ (ya que $\ln(1.5) = \ln(1 + 0.5)$):
        $$R_k(0.5) = \frac{(-1)^k k! 0.5^{k+1}}{(k+1)!(0.5)^{k+1}} = \frac{(-1)^k k!}{(k+1)!2^{k+1}}$$
        \item Se busca que el valor absoluto de $R_k(0.5)$ sea menor que $10^{-10}$, es decir:
        $$\left|\frac{(-1)^k k!}{(k+1)!2^{k+1}}\right| < 10^{-10}$$
        \item Dividiendo ambos lados por $k!$, obtengo:
        $$\left|\frac{(-1)^k}{(k+1)2^{k+1}}\right| < \frac{10^{-10}}{k!}$$
        \item Tomando el valor absoluto de ambos lados, se tiene:
        $$\frac{1}{(k+1)2^{k+1}} < \frac{10^{-10}}{k!}$$
        Reemplazando $k=1, 2, 3, \ldots$ hasta que la desigualdad se cumpla, se obtiene el número de términos necesarios para aproximar $\ln(1.5)$ con un margen de error no mayor que $10^{-10}$.
    \end{itemize}
\end{solution}

\newpage
\section*{Ejercicio 2}
Si la serie para $ln(x)$ centrada en $x = 1$ se corta después del término que comprende a $(x-1)^{1000}$ y después se utiliza para calcular $ln(2)$ ¿Qué cota se puede imponer al error?

\begin{solution}
    Para resolver este problema, primero debo encontrar la serie de Taylor de $\ln(x)$ centrada en $x = 1$. Luego, evaluaremos el resto de la serie truncada después del término que contiene $(x - 1)^{1000}$ cuando $x = 2$, lo que nos dará una cota para el error al calcular $\ln(2)$.
    \begin{itemize}
        \item Serie de Taylor de $\ln(x)$ centrada en $x = 1$:
        $$\ln(x) = \sum_{n=0}^\infty \frac{(-1)^{n+1}(x-1)^n}{n}$$
        \item Truncando la serie después del término que contiene $(x - 1)^{1000}$, obtenemos:
        $$\ln(x) \approx \sum_{n=0}^{1000} \frac{(-1)^{n+1}(x-1)^n}{n}$$
        \item El resto de la serie truncada está dado por:
        $$R_{1000}(x) = \frac{\ln^{(1001)}(\xi)}{1001!}(x-1)^{1001}$$
        donde $\xi$ es un punto entre 1 y $x$.
        \item Calculando la derivada $(1001)$-ésima de $\ln(x)$,
        $$\ln^{(1001)}(x) = \frac{(-1)^{1001}(1001!)}{{x}^{1001}}$$
        \item Sustituyendo esta expresión en la fórmula del resto, se obtiene
        $$R_{1000}(x) = \frac{(-1)^{1001}(x-1)^{1001}}{{x}^{1001}}$$
        \item Evaluando el resto en $x = 2$, se tiene:
        $$R_{1000}(2) = \frac{(-1)^{1001}}{2^{1001}}$$
        \item Tomando el valor absoluto de $R_{1000}(2)$, se obtiene:
        $$|R_{1000}(2)| = \frac{1}{2^{1001}}$$
        Por lo tanto, al utilizar la serie de Taylor truncada después del término que contiene $(x - 1)^{1000}$ para calcular $\ln(2)$, el error estará acotado por:
        $$\left|\ln(2) - \sum_{n=0}^{1000} \frac{(-1)^{n+1}(2-1)^n}{n}\right| \leq \frac{1}{2^{1001}}$$
        \textit{Se puede ver que la cota es muy chica, lo que indica que la aproximación de $\ln(2)$ utilizando los primeros 1001 términos de la serie de Taylor centrada en $x = 1$ es muy precisa.}
    \end{itemize}
\end{solution}

\newpage
\section*{Ejercicio 3}
Verificar la siguiente igualdad y mostrar que la serie converge en el intervalo $-e < x \leq e$
\begin{equation*}
    \ln(e+x) = 1 + \sum_{k=1}^\infty \frac{(-1)^{k-1}}{k}\left( \frac{x}{e} \right)^k
\end{equation*}

\begin{solution}
El problema equivale a mostrar que la serie de Taylor de $\ln(e+x)$ centrada en $x = 0$ es igual a la serie dada y que converge en el intervalo $-e < x \leq e$.
    \begin{itemize}
        \item Para sacar la serie de Taylor de $\ln(e+x)$ centrada en $x = 0$, primero necesito obtener las derivadas sucesivas de la función $f(x) = \ln(e+x)$.
        \item Derivando $f(x) = \ln(e+x)$, se obtiene:
        $$f'(x) = \frac{1}{e+x}$$
        $$f''(x) = -\frac{1}{{(e+x)}^2}$$
        $$f'''(x) = \frac{2}{{(e+x)}^3}$$
        $$f^{(n)}(x) = \frac{(-1)^{n-1}(n-1)!}{{(e+x)}^n}$$
        \item La serie de Taylor centrada en $x = 0$ para $f(x)$ viene dada por:
        \begin{equation*}
            \ln(e+x) = \sum_{n=0}^\infty \frac{f^{(n)}(0)}{n!}x^n
        \end{equation*}
        Sustituyendo los valores encontrados anteriormente, se tiene:
        \begin{equation*}
            \ln(e+x) = \sum_{n=0}^\infty \frac{(-1)^{n-1}(n-1)!}{n!}x^n
        \end{equation*}
        \item Usando la notación de sumatorias, la serie de Taylor se puede escribir como:
        \begin{equation*}
            \ln(e+x) = 1 + \sum_{k=1}^\infty \frac{(-1)^{k-1}}{k}\left( \frac{x}{e} \right)^k
        \end{equation*}
        El número $1$ proviene de la evaluación de la función en $x = 0$, ya que $\ln(e+0) = \ln(e) = 1$.
        \item Para calcular el radio de convergencia de la serie, uso el criterio de la razón:
        $$\lim_{k \to \infty} \left| \frac{a_{k+1}}{a_k} \right| = \lim_{k \to \infty} \left| \frac{\frac{(-1)^{k}}{k+1}\left( \frac{x}{e} \right)^{k+1}}{\frac{(-1)^{k-1}}{k}\left( \frac{x}{e} \right)^k} \right| = \lim_{k \to \infty} \left| \frac{-x}{e(k+1)} \right| = \frac{|x|}{e}$$
        \item La serie converge si el radio de convergencia es mayor que 1, es decir, si $|x| < e$. Por lo tanto, la serie converge en el intervalo $-e < x \leq e$.
    \end{itemize}
\end{solution}

\newpage
\section*{Ejercicio 4}
Desarrollar la función $\sqrt{x}$ en serie de potencias centrada en $x = 1$ y verificar que utilizando la aproximación lineal de dicha función se puede aproximar $\sqrt{0.9999999995}$ con un error no mayor que $10^{-10}$.

\begin{solution}
    Para desarrollar la función $\sqrt{x}$ en serie de potencias centrada en $x = 1$, primero necesito obtener las derivadas sucesivas de la función $f(x) = \sqrt{x}$.
    \begin{itemize}
        \item Derivando $f(x) = \sqrt{x}$, se obtiene:
        $$f'(x) = \frac{1}{2\sqrt{x}}$$
        $$f''(x) = -\frac{1}{4x^{3/2}}$$
        $$f'''(x) = \frac{3}{8x^{5/2}}$$
        $$f^{(n)}(x) = \frac{(-1)^{n-1}(2n-3)!!}{2^n x^{(2n-1)/2}}$$
        \item La serie de Taylor centrada en $x = 1$ para $f(x)$ viene dada por:
        $$\sqrt{x} = \sum_{n=0}^\infty \frac{f^{(n)}(1)}{n!}(x-1)^n$$
        Cada término de la serie es:
        $$\frac{f^{(n)}(1)}{n!}(x-1)^n = \frac{(-1)^{n-1}(2n-3)!!}{2^n n!}(x-1)^n$$
        \item La serie de Taylor centrada en $x = 1$ para $f(x)$ es:
    \end{itemize}
\end{solution}

\end{document}