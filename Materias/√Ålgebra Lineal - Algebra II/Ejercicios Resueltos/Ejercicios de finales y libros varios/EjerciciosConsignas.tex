%%%%%%%%%%%%%%%%%%%%%%%%%%%%%%%%%%%%%%%%%%%%%%%%%%%%%%%%%%%%%%%%%%%%%%%%%%%%%%%%%%%%
% Do not alter this block (unless you're familiar with LaTeX
\documentclass{article}
\usepackage[margin=1in]{geometry} 
\usepackage{amsmath,amsthm,amssymb,amsfonts, fancyhdr, color, comment, graphicx, environ}
\usepackage{xcolor}
\usepackage{mdframed}
\usepackage[shortlabels]{enumitem}
\usepackage{indentfirst}
\usepackage{hyperref}
\usepackage{background}
\hypersetup{
    colorlinks=true,
    linkcolor=blue,
    filecolor=magenta,      
    urlcolor=blue,
}
\setlength{\headheight}{1.5cm}


\pagestyle{fancy}

\SetBgContents{https://github.com/PedroMVillar}
\SetBgScale{3} % Escala de la marca de agua
\SetBgColor{gray!80} % Color de la marca de agua
\SetBgAngle{45} % Ángulo de inclinación de la marca de agua
\SetBgOpacity{0.2} % Opacidad de la marca de agua

\newenvironment{problem}[2][Ejercicio]
    { \begin{mdframed}[backgroundcolor=gray!20] \textbf{#1 #2} \\}
    {  \end{mdframed}}

% Define solution environment
\newenvironment{solution}
    {\textit{Solución:}}
    {}

\renewcommand{\qed}{\quad\qedsymbol}

% prevent line break in inline mode
\binoppenalty=\maxdimen
\relpenalty=\maxdimen

%%%%%%%%%%%%%%%%%%%%%%%%%%%%%%%%%%%%%%%%%%%%%
%Fill in the appropriate information below
\lhead{Pedro Villar}
\rhead{Álgebra Lineal} 
\chead{\textbf{Guía - Álgebra Lineal}}
%%%%%%%%%%%%%%%%%%%%%%%%%%%%%%%%%%%%%%%%%%%%%

\begin{document}


\begin{mdframed}[backgroundcolor=blue!20]
Consignas
\end{mdframed}

%%%%%%%%%%%%%%%%%%%%%%%%%%%%%%%%%%%%%%%%%%%%%
%Rectas y planos
\section{Rectas y Planos}\label{sec:rectas-y-planos}

\begin{problem}{1.1}
    \emph{Halla las ecuaciones de la recta $L$ que pasa por los puntos $P = (1,0,-1)$ y $Q = (2,1-3)$:}
    \end{problem}
    \begin{problem}{1.2}
    \emph{Hallar las ecuaciones del plano $P$ que pasa por los puntos $A=(0,1,-1),$ $B=(2,3,-5)$ y $C=(1,4,3)$.}    
\end{problem}
\begin{problem}{1.3}
    \emph{Comprueba si existe alguna recta que pase por los puntos $P=(3,1,0)$, $Q=(0,-5,1)$, $R=(6,-5,1)$}.
    \end{problem}
    \begin{problem}{1.4}
    \emph{Halla todas las ecuaciones del plano $\Pi$ Determinado por el punto $A=(1,-3,2)$ y por los vectores $u=(2,1,0)$, $v=(-1,0,3)$}.
    \end{problem}

%%%%%%%%%%%%%%%%%%%%%%%%%%%%%%%%%%%%%%%%%%%%%

%%%%%%%%%%%%%%%%%%%%%%%%%%%%%%%%%%%%%%%%%%%%%
%Sistemas lineales
\section{Sistemas lineales}\label{sec:sist-lineales}
\begin{problem}{2.1}

Para que valores del parámetro $k$, el siguiente sistema de ecuaciones lineales:
\begin{itemize}
\item 
i) tiene solución única.
\item 
ii) no tiene solución.
\item 
iii) tiene soluciones infinitas.
\end{itemize}
\[
\begin{cases}
kx + y + z = 1 \\
x + ky + z = 1 \\
x + y + kz = 1
\end{cases}
\]
\end{problem}

\begin{problem}{2.2}
Encuentre el determinante de las siguientes matrices, usando únicamente las propiedades de los determinantes:
\[
A = \begin{bmatrix}
1 & a & b+c \\
1 & b & c+a \\
1 & c & a+b
\end{bmatrix} \ \ \ \ \ \ \ \ \
B = \begin{bmatrix}
a_1^2 & a_1 & 1 \\
a_2^2 & a_2 & 1 \\
a_3^2 & a_3 & 1
\end{bmatrix}
\]
\end{problem}
\newpage
\begin{problem}{2.3}	
    Considere el valor del siguiente determinante
\[
\begin{vmatrix}
a_{11} & a_{12} & a_{13} \\
a_{21} & a_{22} & a_{23} \\
a_{31} & a_{32} & a_{33} 
\end{vmatrix} = 8
\]
en base a este resultado, encuentre el valor de
\[
\begin{vmatrix}
-3a_{11} & -3a_{12} & -3a_{13} \\
2a_{31} & 2a_{32} & 2a_{33} \\
5a_{21} & 5a_{22} & 5a_{23}
\end{vmatrix}
\]
\end{problem}
\begin{problem}{2.4}
    
    Demuestra que si $a\neq b$ entonces el siguiente determinante de $n\times n$
    \[
    \begin{vmatrix}
    a+ b & ab & 0 & \dots & 0 & 0 \\
    1 & a+b & ab & \dots & 0 & 0 \\
    0 & 1 & a+b & \dots & 0 & 0 \\
    \vdots & \ddots & \ddots & \dots & \vdots & \vdots \\
    0 & 0 & 0 & \dots & a+b & ab \\
    0 & 0 & 0 & \dots & 1 & a+b
    \end{vmatrix} =
    \frac{a^{n+1}-b^{n+1}}{a-b}
    \]
\end{problem}

\begin{problem}{2.5}
    Encuentre la forma general de las matrices $A \in M_{2\times 2}$ tales que conmuten con la matriz $B$, esto es, $AB = BA$ donde
\[
B = \begin{bmatrix}
2 & -1 \\
1 & 1
\end{bmatrix}
\]
\end{problem}

\begin{problem}{2.6}
    Muestre que si $A$ es una matriz de tamaño $n\times n$, entonces el determinante de su adjunta es igual al determinante de $A$ elevado a la $(n - 1)$, esto es
\[
|adj(A)| = (|A|)^{n-1}
\]
\end{problem}
\begin{problem}{2.7}
    Sea $A_n = \begin{pmatrix} 2 & -1 & 0 & \dots & 0 & 0 \\ -1 & 2 & -1 & \dots & 0 & 0 \\ 0 & -1 & 2 & \dots & 0 & 0 \\ \vdots & \vdots & \vdots & \ddots & \vdots & \vdots \\ 0 & 0 & 0 & \dots & 2 & -1 \\ 0 & 0 & 0 & \dots & -1 & 2 \end{pmatrix} \in M_{n\times n}(\mathbb{R})$ probar que $det(A_n)=n+1$ para todo $n\in \mathbb{N}$.
\end{problem}
\newpage

\begin{problem}{2.8}
    Determinar para que valores de $x\in \mathbb{R}$ la siguiente matriz es invertible
\[
A = \begin{pmatrix}
1 & 0 & 9 & 0 \\
3 & 1 & 2 & 3 \\
3 & 0 & -1 & 0 \\
3 & 1 & 2 & x
\end{pmatrix}
\]
\end{problem}

\begin{problem}{2.9}
    Sea $A =  \begin{pmatrix} x & a & b & c & d \\ x & x & a & b & c \\ x & x & x & a & b \\  x & x & x & x & a \\ x & x & x & x & x \end{pmatrix}$ calcular $det(A).$
\end{problem}
%%%%%%%%%%%%%%%%%%%%%%%%%%%%%%%%%%%%%%%%%%%%%

\section{Espacios Vectoriales}\label{sec:esp-vectoriales}

\begin{problem}{3.1}
    Sea $M$ el conjunto de todas las matrices invertibles de tamaño $3 \times 3$, muestre que este conjunto no es un espacio vectorial.
\end{problem}

\begin{problem}{3.2}
    Cuales de los siguientes subconjuntos de $R_3[x]$ (espacio vectorial de polinomios de grado < 3) son subespacios vectoriales. Con las operaciones de suma y producto normales que conocemos.
\begin{itemize}
\item 
\textbf{a.} $V = \{ a_0 + a_1t + a_2t^2 \ \ | \ \ a_0 = 0 \}$
\item 
\textbf{b.} $V = \{ a_0 + a_1t + a_2t^2 \ \ | \ \ a_2 = a_1 + 1 \}$
\end{itemize}
\end{problem}

\begin{problem}{3.3}
    Determine si los siguientes conjuntos $W$ son subespacios vectoriales o no del espacio vectorial $M_n$ (espacio vectorial de las matrices de tamaño $n \times n$)
\begin{itemize}
\item 
\textbf{a.} $\{ W = A \in M_n \ \ | \  \ A^t = A \}$
\item 
\textbf{b.} $\{ A \in M_n \ \ | \ \ A \text{ es triangular superior} \}$
\end{itemize}
\end{problem}

\begin{problem}{3.4}
    Muestre que el conjunto de todos los puntos del plano $ax +by +cz = 0$ es un subespacio vectorial de $\mathbb{R}^3$.
\end{problem}
\begin{problem}{3.5}
    Determine si $\{ 1 - 2x, x - x^2\}$ forma una base para $R_3[x]$.
\end{problem}
\newpage
\begin{problem}{3.6}
    Encuentre una base para el espacio solución del sistema de ecuaciones lineales homogéneo siguiente,
\[
\begin{cases}
2x-6y+4z = 0 \\
-x+3y-2z = 0 \\
-3x+9y-6z = 0
\end{cases}
\]
\end{problem}

\section{Transformaciones Lineales}\label{sec:trans-lineales}

\begin{problem}{4.1}
    Sea $T : \mathbb{R}^3 \to \mathbb{R}_3[x]$ definida por
\[
T(a,b,c) = (a+b)x^2+(2a+4c)x+a-b+4c
\]\begin{itemize}
\item 
(a) Dar la dimensión del núcleo de $T$. Justifique apropiadamente.
\item 
(b) Calcular la matriz de la transformación $T$ con respecto a la base canónica ordenada $\mathcal{C}= \{ e_1,e_2,e_3\}$ de $\mathbb{R}^3$ y la base ordenada $\mathcal{B} = \{1,1+x,1+x+x^2\}$.
\item 
(c) Calcular los autovalores reales y complejos de la matriz $[T]_{\mathcal{CB}}$.
\end{itemize}
\end{problem}
\begin{problem}{4.2}
    Definir una transformación lineal $T:\mathbb{R}^3 \to M_2(\mathbb{R})$ que verifique que
\[
Nu(T) = \{ (x,y,z) : z = x = 3y \}
\]\[
Im(T) = \left \{  \begin{pmatrix} a & b \\ c & d \end{pmatrix}  \in M_2(\mathbb{R}) \ | \ b=a-c, \ b-d = c \right \}
\]
Escribir explícitamente $T(x,y,z)$ para cualquier $(x,y,z) \in \mathbb{R}^3$. Justifique cada paso.
\end{problem}
\begin{problem}{4.3}
    Definir una transformación lineal $T : R^2 \to \mathbb{R}^2$ tal que: $T(1, 1) = (1, -2)$ y $T(-1, 1) = (2, 3)$.
\end{problem}

\begin{problem}{4.4}
    Dada la siguiente transformación lineal $T : M_2 \to \mathbb{R}_3[x]$ definida como
\[
T  \begin{pmatrix} a & b \\ c & d \end{pmatrix} = 
a+b+c+(b+c+d)x+(a-d)x^2+(a+2b+2c+d)x^3
\]
determinar
\begin{itemize}
\item 
El núcleo de la transformación y decir si es inyectiva,
\item 
la imagen de la transformación y decir si es sobreyectiva.
\end{itemize}
\end{problem}
\newpage
\begin{problem}{4.5}
    Sea $T:\mathcal{C}^4 \to \mathcal{C}^3$ una transformación lineal cuyo núcleo esta generado por los vectores
\[
(i,0,-1,i) \ \ \ \ (2,1,-1,0) \ \ \ \ (1,1,-1-i,1)
\]
Determinar la dimensión de la imagen de $T$.
\end{problem}

\begin{problem}{4.6}
    Considere la siguiente transformación lineal $T : \mathbb{R}^3 \to \mathbb{R}^2$ dada por
\[
T(x,y,z) = (x+y,3x-z)
\]
Sean $\mathcal{B}_1 \{ (1, 1, 2), (-3, 0, 1), (2, 4, 3) \}$ y $\mathcal{B}_2 = \{ (4, 1) , (3, 1) \}$ Encuentre la matriz de $T$ asociada a las bases dadas.
\end{problem}
\begin{problem}{4.7}
    Sea $T : M_2 \to M_2$ una transformación lineal definida como
\[
T(A) = AM - MA
\]
donde $M=\begin{pmatrix} 1 & 2 \\ 0 & 3 \end{pmatrix}.$
\begin{itemize}
\item 
Determinar el núcleo y la imagen de $T$.
\item 
Encontrar la matriz asociada a $T$.
\end{itemize}
\end{problem}

\begin{problem}{4.8}
    Sea $T: \mathbb{R}^3 \to \mathbb{R}^3$ la transformación lineal definida por
\[
T(x,y,z) = (x+y,x-z,-y+z)
\]\begin{itemize}
\item 
Calcular los autovalores de $T$.
\item 
¿$T$ es diagonizable? Justificar.
\end{itemize}
\end{problem}
\begin{problem}{4.9}
    Hallar una transformación lineal $T : R_3[x] \to \mathbb{R}^4$ tal que $Nu(T) = \{ a_1x + a_1x^2 \ | \ a_1 \in \mathbb{R} \}$.
\end{problem}
\begin{problem}{4.10}
    Sea $T:V\to W$ una transformación lineal, demostrar que el núcleo es un subespacio de $V$ y la imagen un subespacio de $W$.
\end{problem}

\end{document}