\subsection{Ejercicio 1}
\begin{itemize}
    \item[(a)] Dado el siguiente algoritmo, planteá la recurrencia que indica la cantidad de asignaciones realizadas en función de la entrada $n$:
\begin{codebox}
\begin{pascallike}
fun f(n: nat) ret m: nat
    if n $\leq$ 2 then
        m := n
    else 
        m := 3*f(n/2) + n
    fi
end fun
\end{pascallike}
\end{codebox}
    \item[(b)] Resolvé la siguiente recurrencia: $t(n) = \begin{cases} 1 & \text{si } n \leq 2 \\ 4t(n/2) + n^2 & \text{si } n > 2 \end{cases}$.
    \item[(c)] La recurrencia que obtuviste en el item a) ¿es la misma que resolviste en el item b)?. En caso contrario, resolvela también.
\end{itemize}

\subsubsection{Punto a}
\begin{itemize}
    \item Recibe una entrada del tamaño n.
    \item La operación a contar es la asignación de la variable m.
\end{itemize}

\begin{equation*}
    t(n) = \begin{cases}
        1 & \text{si } n \leq 2 \\
        3t(n/2) & \text{si } n > 2
    \end{cases}
\end{equation*}

\begin{itemize}
    \item $a = 1$,
    \item $b = 2$,
    \item $k = 0$.
\end{itemize}

Entonces $a = 1 = b^k = 2^0 \rightarrow t(n) = \log_2 n$

\subsubsection{Punto b}
\begin{equation*}
    t(n) = \begin{cases}
        1 & \text{si } n \leq 2 \\
        4t(n/2) + n^2 & \text{si } n > 2
    \end{cases}
\end{equation*}

\begin{itemize}
    \item $a = 4$,
    \item $b = 2$,
    \item $k = 2$.
\end{itemize}

Entonces $a = 4 = b^k = 2^2 = 4 \rightarrow t(n) = n^2 \log_2 n$.