\subsection{Ejercicio 1}
Completar la implementación de listas dada en el teórico usando punteros.

\begin{codebox}
\begin{pascallike}
implement List of T where

type Node of T = tuple
                    elem : T
                    next : pointer to (Node of T)
                 end tuple

type List of T = pointer to (Node of T)

fun empty() ret l : List of T
    l := null
end fun

proc addl(in e : T, in/out l : List of T)
    var p : pointer to (Node of T)
    alloc(p)
    p->elem := e
    p->next := l
    l := p
end proc

fun is_empty(l: List of T) ret b: bool
    b := l = null
end fun

{-PRE: not is_empty-}
fun head(l: List of T) ret e : T
    e := l->elem
end fun
\end{pascallike}
\end{codebox}

\begin{codebox}
\begin{pascallike}
{-PRE: not is_empty-}
proc tail(in/out l: List of T)
    var p: pointer to (Node of T)
    p := l
    l := l->next
    free(p)
end proc

proc addr(in/out l: List of T, in e: T)
    var p,q: pointer to (Node of T)
    alloc(q)
    q->elem := e
    q->next := null 
    if (not is_empty(l)) then
        p := l
        do p->next $\neq$ null 
            p := p->next
        od
        p->next := q
    else
        l := q
    fi
end proc

fun length(l: List of T) ret n: nat
    var p: pointer to (Node of T)
    p := l
    n := 0
    do p $\neq$ null 
        p := p->next
        n := n+1
    od
end fun

proc concat(in/out l: List of T, in lO List of T)
    var p: pointer to (Node of T)
    p := l
    do p->next $\neq$ null
        p := p->next
    od
    p->next := l0
end proc

fun index (l : List of T, n : nat) ret e : T
    var p : pointer to (Node of T)
    p := l
    for i := 1 to n do
        p := p->next
    end for
    e := p->elem
end fun
\end{pascallike}
\end{codebox}

\begin{codebox}
\begin{pascallike}
proc take(in/out l : List of T, in n : nat)
    var p,q : pointer to (Node of T)
    p := l
    for i := 1 to n do
        p := p->next
    end for
    q := p->next
    p->next := null
    do q != null 
        var r : pointer to (Node of T)
        r := q
        q := q->next
        free(r)
    do
end proc

proc drop(in/out l : List of T, in n : nat)
    var p : pointer to (Node of T)
    p := l
    for i := 1 to n do
        p := p->next
    end for
    l := p
end proc

fun copy_list(l1 : List of T) ret l2 : List of T
    var p : pointer to (Node of T)
    var q : pointer to (Node of T)
    if l1 = null then
        l2 := null
    else
        alloc(q)
        q->elem := l1->elem
        q->next := null
        l2 := q
        p := l1->next
        do p != null 
            alloc(q->next)
            q := q->next
            q->elem := p->elem
            q->next := null
            p := p->next
        end while
    end if
end fun

proc destroy(in/out l : List of T)
    var p : pointer to (Node of T)
    do l != null 
        p := l
        l := l->next
        free(p)
    end while
end proc
\end{pascallike}
\end{codebox}

\subsection{Ejercicio 2}
Un \textbf{multiconjunto} M con elementos en S, es un subconjunto de S donde cada elemento tiene asociado un número natural que indica cuantas veces el elemento ocurre. En otras palabras, un multiconjunto puede pensarse como un conjunto en el cual los elementos pueden estar incluidos más de una vez, indicando cuantas veces lo hace.
\begin{itemize}
    \item[(a)] Especificar el TAD \texttt{Multiset of T} de multiconjuntos con elementos de tipo T incluyendo constructores para crear el multiconjunto vacío y otro para agregar un elemento a un multiconjunto. Y operaciones para saber si un multiconjunto es vacío o no, para saber si un elemento dado existe en el multiconjunto, para obtener cuántas veces un elemento dado ocurre en un multiconjunto, y para eliminar una ocurrencia de un elemento en un multiconjunto.
    \item[(b)] Implementar el TAD \texttt{Multiset of T} especifiado en el punto anterior, utilizando como representación interna una lista donde cada elemento es un par consistente de un elemento de tipo T y un número natural que indica la cantidad de ocurrencias.
    \item[(c)] Implementar una función que reciba un arreglo de N naturales y devuelva el multiconjunto que contiene todos los elementos pares que ocurren en el arreglo. Todo tipo de datos utilizado en la función debe utilizarse de manera abstracta. 
\end{itemize}

\subsubsection{Punto a}
\begin{codebox}
\begin{pascallike}
spec Multiset of T where

constructors
    fun empty_m() ret m : Multiset of T 
    {-crea un multiconjunto vacio-}

    proc add_elem(in e: T, in/out m: Multiset of T)
    {-agrega un elemento al multiconjunto-}

destroy
    proc destroy_m(in/out m: Multiset of T)
    {-libera memoria en caso de que sea necesario-}

operations
    fun is_empty(m: Multiset of T) ret b : bool
    {-devuelve true si el multiconjunto es vacio-}

    fun exists(m: Multiset of T, e: T) ret b: bool
    {-devuelve true si el elemento existe en el multiconjunto-}

    fun hm_ocu(m: Multiset of T, e: T) ret k: nat
    {-devuelve la cantidad de ocurrencias de un elemento en el multiconjunto-}

    {-PRE: exists(m,e)-}
    proc del_ocu(in/out m: Multiset of T, e: T)
    {-elimina una ocurrencia de un elemento en un multiconjunto-}
\end{pascallike}
\end{codebox}

\newpage
\subsubsection{Punto b}
\begin{codebox}
\begin{pascallike}
implement Multiset of T where

type par of T = tuple
                  elem: T
                  ocu: nat
                end tuple

type Multiset of T = List of par

fun empty_m() ret m: Multiset of T
    m := empty()
end fun

proc add_elem(in e: T, in/out m: Multiset of T)
    var p: par of T
    var n: Multiset of T
    p.elem := e
    p.ocu := 1
    if is_empty(m) then
        addl(p, m)
    else
        n := copy_list(m)
        do not is_empty(n) -> 
            if (head(n).elem = e) then
                head(n).ocu := head(n)->ocu + 1
            fi
            tail(n)
        od
        addl(p, m)
    fi
end proc

proc destroy_m(in/out m: Multiset of T)
    destroy(m)
end proc

fun exits(m: Multiset of T, e: T) ret b: bool
    var n: Multiset of T
    var tmp: par of T
    b := false
    n := copy_list(m)
    do not is_empty(n) ->
        tmp := head(n)
        if (tmp.elem = e) then
            b := true
        fi
        tail(n)
    od
end fun
\end{pascallike}
\end{codebox}

\begin{codebox}
\begin{pascallike}
fun hm_ocu(m: Multiset of T, e: T) ret k: nat
    var n: Multiset of T
    var tmp: par of T
    k := 0
    n := copy_list(m)
    do not is_empty(n) ->
        tmp := head(n)
        if (tmp.elem = e) then
            k := k+1
        fi
        tail(n)
    od
end fun

fun del_ocu(in/out m: Multiset of T, e: T)
    var n,o: Multiset of T
    n := copy_list(m)
    o := copy_list(m)
    var index : nat
    var tmp: par of T
    index := 1
    do not is_empty(n) ->
        tmp := head(n)
        if(tmp.elem = e) 
            destroy(n)
        fi
        else index := index +1
        tail(n)
    od
    m := concat(take(o,index-1), drop(o,index+1))
end fun
\end{pascallike}
\end{codebox}

\subsubsection{Punto c}
\begin{codebox}
\begin{pascallike}
fun mul_pares(a : array[1..N] of nat) ret m: Multiset of T
    {-pongo los pares en el multiconjunto-}
    m := empty_m()
    for i := 1 to N do
        if a[i] mod 2 = 0 then
            add_elem(a[i], m)
        fi
    od
end fun
\end{pascallike}
\end{codebox}

\newpage
\subsection{Ejercicio 3}
Implementá el TAD Pila utilizando la siguiente representación
\begin{codebox}
\begin{pascallike}
implement Stack of T where

type Stack of T = List of T
\end{pascallike}
\end{codebox}
(la especificación del TAD Stack es la siguiente)

\begin{codebox}
\begin{pascallike}
spec Stack of T where

constructors
    fun empty_stack() ret s : Stack of T
    {-crea una pila vacia.-}

    proc push (in e : T,in/outs : Stack of T)
    {-agrega el elemento e al tope de la pilas. -}

operations
    fun is_empty_stack(s : Stack of T) ret b : Bool
    {-Devuelve True si la pila es vacia-}

    fun top(s : Stack of T) ret e : T
    {-Devuelve el elemento que se encuentra en el tope des. -}

    {-PRE: not is_empty_stack(s) -}
    proc pop (in/out s : Stack of T)
    {-Elimina el elemento que se encuentra en el tope des. -}

    {-PRE: not is_empty_stack(s) -}
    fun copy_stack (s1 : Stack of T) ret s2 : Stack of T
    {- copia el contenido de la pila s1 en la nueva pila s2 -}

destroy
    proc destroy_stack (in/out s: Stack of T)
    {- elimina la memoria usada por la pila s en caso de ser necesario -}
    
end spec
\end{pascallike}
\end{codebox}

\subsubsection{Solución}
\begin{codebox}
\begin{pascallike}
implement Stack of T where

type Stack of T = List of T

fun empty_stack() ret s : Stack of T
    s := empty()
end fun

proc push (in e : T,in/outs : Stack of T)
    addl(e,s)
end proc

fun is_empty_stack(s: Stack of T) ret b : bool
    b := is_empty(s)
end fun

fun top(s: Stack of T) ret e : T
    e := head(s)
end fun

proc pop(in/out s: Stack of T)
    tail(s)
end proc

fun copy_stack(s1: Stack of T) ret s2 : Stack of T
    s2 := copy_list(s1)
end fun

proc destroy_stack(in/out s: Stack of T)
    destroy(s)
end proc

end implement
\end{pascallike}
\end{codebox}

\newpage
\subsection{Ejercicio 4}
Implementar el TAD Pila utilizando la siguiente representación
\begin{codebox}
\begin{pascallike}
implement Stack of T where

type Node of T = tuple
                    elem : T
                    next : pointer to (Node of T)
                 end tuple

type Stack of T = pointer to (Node of T)      
\end{pascallike}
\end{codebox}
\subsubsection{Solución}

\begin{codebox}
\begin{pascallike}
implement Stack of T where

type Node of T = tuple
                    elem : T
                    next : pointer to (Node of T)
                 end tuple

type Stack of T = pointer to (Node of T)

fun empty_stack() ret s: Stack of T
    s := null
end fun

proc push(in e: T, in/out s: Stack of T)
    var p: pointer to (Node of T)
    alloc(p)
    p->elem := e
    if s = null then
        s := p
    else
        p->next := s
        s := p
    fi
end proc 

fun is_empty_stack(s : Stack of T) ret b : Bool
    b := s = null
end fun
\end{pascallike}
\end{codebox}

\begin{codebox}
\begin{pascallike}
fun top(s : Stack of T) ret e : T
    e := s -> elem
end fun

proc pop (in/out s : Stack of T)
    var p: pointer to (Node of T)
    p := s
    s := s->next
    free(p)
end proc

proc copy_stack(s1 : Stack of T) ret s2 : Stack of T
    var s10: pointer to (Node of T)
    var s20: pointer to (Node of T)
    var t : Node of T
    if is_empty_stack(s1) then
        s2 := empty_stack()
    else
        s10 := s1 
        s2->elem := s1->elem
        s2->next := null
        s20 := s2

    while s10 != null do
        alloc(t)
        t->elem := s10->elem
        t->next := null
        s10 := s10->next
        s20 := s20->next
    fi
end proc
\end{pascallike}
\end{codebox}

\newpage
\subsection{Ejercicio 5}
Implementá el TAD Cola utilizando un arreglo, pero asegurando que todas las operaciones estén implementadas en orden constante.

\subsubsection{Solución}
\begin{codebox}
\begin{pascallike}
implement Queue of T where

type Queue of T = tuple
                    elems : array [0 .. N-1] of T
                    size : nat
                    start : nat 
                  end tuple

fun empty_queue() ret q : Queue of T
    q.size := 0
    q.start := 0
end fun

proc enqueue (in/out q : Queue of T,ine : T)
    var pos: nat
    pos := (q.start + q.size) {-N-}
    q.elems[pos] := e
    q.size := q.size + 1
end proc

fun is_empty_queue(q : Queue of T) ret b : bool
    b := q.size = 0
end fun

fun first(q : Queue of T) ret e : T
    e := q.elems[0]
end fun

proc dequeue (in/out q : Queue of T)
    var start_pos: nat
    start_pos := (q.start + 1) {-N-}
    q.start := start_pos
    q.size := q.size - 1
end proc

proc destroy_queue (in/out q: Queue of T)
    skip
end proc

end implement
\end{pascallike}
\end{codebox}

\newpage
\subsection{Ejercicio 6}
