%%%%%%%%%%%%%%%%%%%%%%%%%%%%%%%%%%%%%%%%%%%%%%%%%%%%%%%%%%%%%%%%%%%%%%%%%%%%%%%%%%%%
% Configuración de Paquetes
\documentclass{article}
\usepackage[framemethod=TikZ]{mdframed}
\usepackage{booktabs}
\usepackage{float}
\usepackage{scrextend}
\usepackage{titletoc}
\usepackage[margin=1in]{geometry} 
\usepackage{amsmath,amsthm,amssymb,amsfonts, fancyhdr, color, comment, graphicx, environ}
\usepackage{xcolor}
\usepackage{mdframed}
\usepackage[shortlabels]{enumitem}
\usepackage{indentfirst}
\usepackage{hyperref}
\usepackage{listings}
\usepackage{tikz}
\usepackage[framemethod=TikZ]{mdframed}
\usepackage{mathptmx}
\usepackage{cfr-lm}
\usepackage{tabularx}
\hypersetup{
    colorlinks=true,
    linkcolor=blue,
    filecolor=magenta,      
    urlcolor=blue,
}
\setlength{\headheight}{1.5cm}
\renewcommand{\qed}{\quad\qedsymbol}
\usetikzlibrary{calc}
\renewcommand{\familydefault}{\sfdefault}
%%%%%%%%%%%%%%%%%%%%%%%%%%%%%%%%%%%%%%%%%%%%%%%%%%%%%%%%%%%%%%%%%%%%%%%%%%%%%%%%%%%%

\fancypagestyle{mipagina}{
    \fancyhf{} % Limpiar encabezado y pie de página
    \fancyhead[L]{Pedro Villar} % Nombre a la izquierda
    \fancyhead[C]{\rightmark} % Texto al centro 
    \fancyhead[R]{Alg. y E.D 2} % Texto a la derecha
    \fancyfoot[C]{\thepage} % Número de página al centro
    \renewcommand{\headrulewidth}{0.4pt} % Grosor de la línea horizontal en el encabezado
}

\pagestyle{mipagina}
\mdfsetup{skipabove=\topskip,skipbelow=\topskip}

% Box de Definición
\newcounter{def}[section]

\NewDocumentEnvironment{defi}{o}{%
    \stepcounter{def}%
    \begin{mdframed}[
        frametitle={%
            \begin{tikzpicture}[baseline=(current bounding box.east),outer sep=0pt]
                \node[anchor=east,rectangle,fill=blue!20,inner xsep=5pt] at (0,0) {\strut\IfValueTF{#1}{Definición~\thedef:~#1}{Definición~\thedef}};
            \end{tikzpicture}%
        },
        innertopmargin=5pt,
        linecolor=blue!20,
        linewidth=2pt,
        topline=true,
        frametitleaboveskip=-\ht\strutbox, % Ajuste de espacio entre título y contenido
        frametitlealignment={\hspace{5pt}}, % Ajuste de espacio entre el borde del frame y el título
    ]
}{%
    \end{mdframed}%
}

% Box de ejemplos
\newcounter{ejemplo}[section]

\NewDocumentEnvironment{ejemplo}{o}{%
    \stepcounter{ejemplo}%
    \begin{mdframed}[
        frametitle={%
            \begin{tikzpicture}[baseline=(current bounding box.east),outer sep=0pt]
                \node[anchor=east,rectangle,fill=brown!30,inner xsep=5pt] at (0,0) {\strut\IfValueTF{#1}{Ejemplo~\theejemplo:~#1}{Ejemplo~\theejemplo}};
            \end{tikzpicture}%
        },
        innertopmargin=5pt,
        linecolor=brown!30,
        linewidth=2pt,
        topline=true,
        frametitleaboveskip=-\ht\strutbox, % Ajuste de espacio entre título y contenido
        frametitlealignment={\hspace{5pt}}, % Ajuste de espacio entre el borde del frame y el título
    ]
}{%
    \end{mdframed}%
}

% Entorno para Métodos (color verde)
\newcounter{metodo}[section]
\NewDocumentEnvironment{metodo}{o}{%
    \stepcounter{metodo}%
    \begin{mdframed}[
        frametitle={%
            \begin{tikzpicture}[baseline=(current bounding box.east),outer sep=0pt]
                \node[anchor=east,rectangle,fill=green!30,inner xsep=5pt] at (0,0) {\strut\IfValueTF{#1}{Método~\themetodo:~#1}{Método~\themetodo}};
            \end{tikzpicture}%
        },
        innertopmargin=5pt,
        linecolor=green!30,
        linewidth=2pt,
        topline=true,
        frametitleaboveskip=-\ht\strutbox, % Ajuste de espacio entre título y contenido
        frametitlealignment={\hspace{5pt}}, % Ajuste de espacio entre el borde del frame y el título
    ]
}{%
    \end{mdframed}%
}

% Entorno para Observaciones (color amarillo más oscuro)
\newcounter{observacion}[section]
\NewDocumentEnvironment{observacion}{o}{%
    \stepcounter{observacion}%
    \begin{mdframed}[
        frametitle={%
            \begin{tikzpicture}[baseline=(current bounding box.east),outer sep=0pt]
                \node[anchor=east,rectangle,fill=yellow!50,inner xsep=5pt] at (0,0) {\strut\IfValueTF{#1}{Observación~\theobservacion:~#1}{Observación~\theobservacion}};
            \end{tikzpicture}%
        },
        innertopmargin=5pt,
        linecolor=yellow!50,
        linewidth=2pt,
        topline=true,
        frametitleaboveskip=-\ht\strutbox, % Ajuste de espacio entre título y contenido
        frametitlealignment={\hspace{5pt}}, % Ajuste de espacio entre el borde del frame y el título
    ]
}{%
    \end{mdframed}%
}

% Entorno para Ejercicio
\newcounter{ejer}[section]
\NewDocumentEnvironment{ejer}{o}{%
    \stepcounter{ejer}%
    \begin{mdframed}[
        frametitle={%
            \begin{tikzpicture}[baseline=(current bounding box.east),outer sep=0pt]
                \node[anchor=east,rectangle,fill=black!30,inner xsep=5pt] at (0,0) {\strut\IfValueTF{#1}{Ejercicio~\theejer:~#1}{Ejercicio~\theejer}};
            \end{tikzpicture}%
        },
        innertopmargin=5pt,
        linecolor=black!30,
        linewidth=2pt,
        topline=true,
        frametitleaboveskip=-\ht\strutbox, % Ajuste de espacio entre título y contenido
        frametitlealignment={\hspace{5pt}}, % Ajuste de espacio entre el borde del frame y el título
    ]
}{%
    \end{mdframed}%
}

\newenvironment{solution}
    {\textit{Solución:}}
    {}


\newcounter{pascalcode}[section]

\renewcommand{\thepascalcode}{\arabic{section}.\arabic{pascalcode}}

\lstdefinelanguage{pascal-like}{
  morekeywords={var, type, tuple, enumerate, fi, ret, downto, where, proc, fun, in, out, if, then, else, while, do, for, to, end, od},
  morecomment=[s]{\{-}{-\}},
  morestring=[b]",
  sensitive=true,
  keywordstyle=\color{purple},
  commentstyle=\color{green!70!black},
  stringstyle=\color{blue!70!black},
  numbers=left,
  numberstyle=\tiny\color{gray},
  alsoletter={:,=,+,-,*,>,<,(,)},
  morekeywords=[2]{:,=,+,-,*,>,<,(,),&,&&,||,!,!=,==},
  keywordstyle=[2]{\color{blue}},
  literate={:=}{{\textcolor{blue}{:=}}}2,
  morekeywords=[3]{nat, array, of, int, real, bool, char}, 
  keywordstyle=[3]{\color{green!40!black}},
  literate={\^}{{\textasciicircum}}1,
}

\lstnewenvironment{pascallike}[1][]{
  \refstepcounter{pascalcode}
  \lstset{
    language=pascal-like,
    numbers=left,
    numberstyle=\tiny\color{gray},
    captionpos=b,
    #1
  }
}{\phantomsection\addcontentsline{lol}{section}{Código \thepascalcode}}


%Configuraciones adicionales
\binoppenalty=\maxdimen 
\relpenalty=\maxdimen 
\setlength{\parindent}{0pt}

\begin{document}

\section*{Ejercicio 1}
\begin{enumerate}[a)]
  \item Ordená los arreglos del ejercicio 4 del práctico anterior utilizando el algoritmo de ordenación por intercalación.
  \item En el caso del inciso a) del ejercicio 4, dar la secuencia de llamadas al procedimiento merge\_sort\_rec con los valores correspondientes de sus argumentos.
\end{enumerate}

\textbf{Punto a:}

\begin{minipage}{0.5\textwidth}
  \begin{equation*}
    \large
    \begin{array}{|r|r|r|r|r|r|r|}
      \hline \textcolor{red}{7} & \textcolor{red}{1} & \textcolor{red}{10} & \textcolor{red}{3} & \textcolor{red}{4} & \textcolor{red}{9} & \textcolor{red}{5} \\ \hline
    \end{array}
  \end{equation*}
  Se divide en dos partes:
  \begin{equation*}
    \large
    \begin{array}{|r|r|r|r|}
      \hline \textcolor{red}{7} & \textcolor{red}{1} & \textcolor{red}{10} & \textcolor{red}{3} \\ \hline
    \end{array} \quad 
    \begin{array}{|r|r|r|}
      \hline \textcolor{blue}{4} & \textcolor{blue}{9} & \textcolor{blue}{5} \\ \hline
    \end{array}
  \end{equation*}
  Se toma el arreglo izquierdo y se divide en dos partes:
  \begin{equation*}
    \large
    \begin{array}{|r|r|}
      \hline \textcolor{red}{7} & \textcolor{red}{1} \\ \hline
    \end{array} \quad 
    \begin{array}{|r|r|}
      \hline 10 & 3 \\ \hline
    \end{array}\quad 
    \begin{array}{|r|r|r|}
      \hline \textcolor{blue}{4} & \textcolor{blue}{9} & \textcolor{blue}{5} \\ \hline
    \end{array}
  \end{equation*}
  Se toma el arreglo izquierdo y se divide en dos partes:
  \begin{equation*}
    \large
    \begin{array}{|r|}
      \hline \textcolor{red}{7} \\ \hline
    \end{array} \quad 
    \begin{array}{|r|}
      \hline 1 \\ \hline
    \end{array} \quad 
    \begin{array}{|r|r|}
      \hline 10 & 3 \\ \hline
    \end{array}\quad 
    \begin{array}{|r|r|r|}
      \hline \textcolor{blue}{4} & \textcolor{blue}{9} & \textcolor{blue}{5} \\ \hline
    \end{array}
  \end{equation*}
  El arreglo de la izquierda ya no se puede dividir, se procede a ordenar:
  \begin{equation*}
    \large
    \begin{array}{|r|}
      \hline \textcolor{violet}{7} \\ \hline
    \end{array} \quad 
    \begin{array}{|r|}
      \hline 1 \\ \hline
    \end{array} \quad 
    \begin{array}{|r|r|}
      \hline 10 & 3 \\ \hline
    \end{array}\quad 
    \begin{array}{|r|r|r|}
      \hline \textcolor{blue}{4} & \textcolor{blue}{9} & \textcolor{blue}{5} \\ \hline
    \end{array}
  \end{equation*}
  El arreglo de la derecha ya no se puede dividir, se procede a ordenar:
  \begin{equation*}
    \large
    \begin{array}{|r|}
      \hline 7 \\ \hline
    \end{array} \quad 
    \begin{array}{|r|}
      \hline \textcolor{violet}{1} \\ \hline
    \end{array} \quad 
    \begin{array}{|r|r|}
      \hline 10 & 3 \\ \hline
    \end{array}\quad 
    \begin{array}{|r|r|r|}
      \hline \textcolor{blue}{4} & \textcolor{blue}{9} & \textcolor{blue}{5} \\ \hline
    \end{array}
  \end{equation*}
  Se ordena el arreglo seleccionado:
  \begin{equation*}
    \large
    \begin{array}{|r|}
      \hline \textcolor{green}{7} \\ \hline
    \end{array} \quad 
    \begin{array}{|r|}
      \hline \textcolor{green}{1} \\ \hline
    \end{array} \quad 
    \begin{array}{|r|r|}
      \hline 10 & 3 \\ \hline
    \end{array}\quad 
    \begin{array}{|r|r|r|}
      \hline \textcolor{blue}{4} & \textcolor{blue}{9} & \textcolor{blue}{5} \\ \hline
    \end{array}
  \end{equation*}
  Seleccionar el mínimo de los dos:
  \begin{equation*}
    \large
    \begin{array}{|r|}
      \hline \textcolor{green}{7} \\ \hline
    \end{array} \quad 
    \begin{array}{|r|}
      \hline \textcolor{orange}{1} \\ \hline
    \end{array} \quad 
    \begin{array}{|r|r|}
      \hline 10 & 3 \\ \hline
    \end{array}\quad 
    \begin{array}{|r|r|r|}
      \hline \textcolor{blue}{4} & \textcolor{blue}{9} & \textcolor{blue}{5} \\ \hline
    \end{array}
  \end{equation*}
  Agrega el mínimo al arreglo ordenado
  \begin{equation*}
    \large
    \begin{array}{|r|r|}
      \hline \textcolor{red}{1} & \\ \hline
    \end{array} \quad
    \begin{array}{|r|r|}
      \hline 10 & 3 \\ \hline
    \end{array}\quad 
    \begin{array}{|r|r|r|}
      \hline \textcolor{blue}{4} & \textcolor{blue}{9} & \textcolor{blue}{5} \\ \hline
    \end{array}
  \end{equation*}
  \begin{equation*}
    \large
    \begin{array}{|r|}
      \hline \textcolor{green}{7} \\ \hline
    \end{array} \quad 
    \begin{array}{|r|}
      \hline \\ \hline
    \end{array} \quad 
    \begin{array}{|r|r|}
      \hline 10 & 3 \\ \hline
    \end{array}\quad 
    \begin{array}{|r|r|r|}
      \hline \textcolor{blue}{4} & \textcolor{blue}{9} & \textcolor{blue}{5} \\ \hline
    \end{array}
  \end{equation*}
  Cuando una lista es vacía, se agrega el resto de la otra lista
  \begin{equation*}
    \large
    \begin{array}{|r|r|}
      \hline \textcolor{red}{1} & \textcolor{red}{7}\\ \hline
    \end{array} \quad
    \begin{array}{|r|r|}
      \hline 10 & 3 \\ \hline
    \end{array}\quad 
    \begin{array}{|r|r|r|}
      \hline \textcolor{blue}{4} & \textcolor{blue}{9} & \textcolor{blue}{5} \\ \hline
    \end{array}
  \end{equation*}
\end{minipage}
\begin{minipage}{0.5\textwidth}
  Se selecciona el arreglo de la derecha
  \begin{equation*}
    \large
    \begin{array}{|r|r|}
      \hline 1 & 7\\ \hline
    \end{array} \quad
    \begin{array}{|r|r|}
      \hline \textcolor{red}{10} & \textcolor{red}{3} \\ \hline
    \end{array}\quad 
    \begin{array}{|r|r|r|}
      \hline \textcolor{blue}{4} & \textcolor{blue}{9} & \textcolor{blue}{5} \\ \hline
    \end{array}
  \end{equation*}
  Se divide en dos partes el arreglo de la derecha
  \begin{equation*}
    \large
    \begin{array}{|r|r|}
      \hline 1 & 7\\ \hline
    \end{array} \quad
    \begin{array}{|r|}
      \hline \textcolor{red}{10} \\ \hline
    \end{array} \quad
    \begin{array}{|r|}
      \hline 3 \\ \hline
    \end{array} \quad 
    \begin{array}{|r|r|r|}
      \hline \textcolor{blue}{4} & \textcolor{blue}{9} & \textcolor{blue}{5} \\ \hline
    \end{array}
  \end{equation*}
  El arreglo de la izquierda ya no se puede dividir, se procede a ordenar:
  \begin{equation*}
    \large
    \begin{array}{|r|r|}
      \hline 1 & 7\\ \hline
    \end{array} \quad
    \begin{array}{|r|}
      \hline 10 \\ \hline
    \end{array} \quad
    \begin{array}{|r|}
      \hline 3 \\ \hline
    \end{array} \quad 
    \begin{array}{|r|r|r|}
      \hline \textcolor{blue}{4} & \textcolor{blue}{9} & \textcolor{blue}{5} \\ \hline
    \end{array}
  \end{equation*}
  El arreglo de la derecha ya no se puede dividir, se procede a ordenar:
  \begin{equation*}
    \large
    \begin{array}{|r|r|}
      \hline 1 & 7\\ \hline
    \end{array} \quad
    \begin{array}{|r|}
      \hline 10 \\ \hline
    \end{array} \quad
    \begin{array}{|r|}
      \hline 3 \\ \hline
    \end{array} \quad 
    \begin{array}{|r|r|r|}
      \hline \textcolor{blue}{4} & \textcolor{blue}{9} & \textcolor{blue}{5} \\ \hline
    \end{array}
  \end{equation*}
  Se procede a ordenar el arreglo seleccionado:
  \begin{equation*}
    \large
    \begin{array}{|r|r|}
      \hline 1 & 7\\ \hline
    \end{array} \quad
    \begin{array}{|r|}
      \hline \textcolor{green}{10} \\ \hline
    \end{array} \quad
    \begin{array}{|r|}
      \hline \textcolor{green}{3} \\ \hline
    \end{array} \quad 
    \begin{array}{|r|r|r|}
      \hline \textcolor{blue}{4} & \textcolor{blue}{9} & \textcolor{blue}{5} \\ \hline
    \end{array}
  \end{equation*}
  Seleccionar el mínimo de los dos:
  \begin{equation*}
    \large
    \begin{array}{|r|r|}
      \hline 1 & 7\\ \hline
    \end{array} \quad
    \begin{array}{|r|}
      \hline \textcolor{green}{10} \\ \hline
    \end{array} \quad
    \begin{array}{|r|}
      \hline \textcolor{orange}{3} \\ \hline
    \end{array} \quad 
    \begin{array}{|r|r|r|}
      \hline \textcolor{blue}{4} & \textcolor{blue}{9} & \textcolor{blue}{5} \\ \hline
    \end{array}
  \end{equation*}
  Agregar el mínimo al arreglo ordenado
  \begin{equation*}
    \large
    \begin{array}{|r|r|}
      \hline 1 & 7\\ \hline
    \end{array} \quad
    \begin{array}{|r|r|}
      \hline \textcolor{orange}{3} & \\ \hline
    \end{array}\quad 
    \begin{array}{|r|r|r|}
      \hline \textcolor{blue}{4} & \textcolor{blue}{9} & \textcolor{blue}{5} \\ \hline
    \end{array}
  \end{equation*}
  \begin{equation*}
    \large
    \begin{array}{|r|r|}
      \hline 1 & 7\\ \hline
    \end{array} \quad
    \begin{array}{|r|}
      \hline \textcolor{green}{10} \\ \hline
    \end{array} \quad
    \begin{array}{|r|}
      \hline \\ \hline
    \end{array} \quad 
    \begin{array}{|r|r|r|}
      \hline \textcolor{blue}{4} & \textcolor{blue}{9} & \textcolor{blue}{5} \\ \hline
    \end{array}
  \end{equation*}
  Cuando una lista es vacía, se agrega el resto de la otra lista
  \begin{equation*}
    \large
    \begin{array}{|r|r|}
      \hline 1 & 7\\ \hline
    \end{array} \quad
    \begin{array}{|r|r|}
      \hline \textcolor{orange}{3} & \textcolor{orange}{10} \\ \hline
    \end{array}\quad 
    \begin{array}{|r|r|r|}
      \hline \textcolor{blue}{4} & \textcolor{blue}{9} & \textcolor{blue}{5} \\ \hline
    \end{array}
  \end{equation*}
  Se ordenan los arreglos seleccionados
  \begin{equation*}
    \large
    \begin{array}{|r|r|}
      \hline \textcolor{green}{1} & \textcolor{green}{7}\\ \hline
    \end{array} \quad
    \begin{array}{|r|r|}
      \hline \textcolor{green}{3} & \textcolor{green}{10} \\ \hline
    \end{array}\quad 
    \begin{array}{|r|r|r|}
      \hline \textcolor{blue}{4} & \textcolor{blue}{9} & \textcolor{blue}{5} \\ \hline
    \end{array}
  \end{equation*}
  Selecciona el mínimo elemento desde la cabeza de las dos listas
  \begin{equation*}
    \large
    \begin{array}{|r|r|}
      \hline \textcolor{orange}{1} & \textcolor{green}{7}\\ \hline
    \end{array} \quad
    \begin{array}{|r|r|}
      \hline \textcolor{green}{3} & \textcolor{green}{10} \\ \hline
    \end{array}\quad 
    \begin{array}{|r|r|r|}
      \hline \textcolor{blue}{4} & \textcolor{blue}{9} & \textcolor{blue}{5} \\ \hline
    \end{array}
  \end{equation*}
\end{minipage}

\newpage

\begin{minipage}{0.5\textwidth}
  Agregar el número mínimo al arreglo ordenado
  \begin{equation*}
    \large
    \begin{array}{|r|r|r|r|}
      \hline \textcolor{red}{1} & \textcolor{red}{} & \textcolor{red}{} & \textcolor{red}{} \\ \hline
    \end{array}\quad 
    \begin{array}{|r|r|r|}
      \hline \textcolor{blue}{4} & \textcolor{blue}{9} & \textcolor{blue}{5} \\ \hline
    \end{array}
  \end{equation*}
  \begin{equation*}
    \large
    \begin{array}{|r|r|}
      \hline & \textcolor{green}{7}\\ \hline
    \end{array} \quad
    \begin{array}{|r|r|}
      \hline \textcolor{green}{3} & \textcolor{green}{10} \\ \hline
    \end{array}\quad 
    \begin{array}{|r|r|r|}
      \hline \textcolor{blue}{4} & \textcolor{blue}{9} & \textcolor{blue}{5} \\ \hline
    \end{array}
  \end{equation*}
  Seleccionar el mínimo de las dos listas
  \begin{equation*}
    \large
    \begin{array}{|r|r|r|r|}
      \hline \textcolor{red}{1} & \textcolor{red}{} & \textcolor{red}{} & \textcolor{red}{} \\ \hline
    \end{array}\quad 
    \begin{array}{|r|r|r|}
      \hline \textcolor{blue}{4} & \textcolor{blue}{9} & \textcolor{blue}{5} \\ \hline
    \end{array}
  \end{equation*}
  \begin{equation*}
    \large
    \begin{array}{|r|r|}
      \hline & \textcolor{green}{7}\\ \hline
    \end{array} \quad
    \begin{array}{|r|r|}
      \hline \textcolor{orange}{3} & \textcolor{green}{10} \\ \hline
    \end{array}\quad 
    \begin{array}{|r|r|r|}
      \hline \textcolor{blue}{4} & \textcolor{blue}{9} & \textcolor{blue}{5} \\ \hline
    \end{array}
  \end{equation*}
  Agregar el mínimo al arreglo ordenado
  \begin{equation*}
    \large
    \begin{array}{|r|r|r|r|}
      \hline \textcolor{red}{1} & \textcolor{red}{3} & \textcolor{red}{} & \textcolor{red}{} \\ \hline
    \end{array}\quad 
    \begin{array}{|r|r|r|}
      \hline \textcolor{blue}{4} & \textcolor{blue}{9} & \textcolor{blue}{5} \\ \hline
    \end{array}
  \end{equation*}
  \begin{equation*}
    \large
    \begin{array}{|r|r|}
      \hline & \textcolor{green}{7}\\ \hline
    \end{array} \quad
    \begin{array}{|r|r|}
      \hline & \textcolor{green}{10} \\ \hline
    \end{array}\quad 
    \begin{array}{|r|r|r|}
      \hline \textcolor{blue}{4} & \textcolor{blue}{9} & \textcolor{blue}{5} \\ \hline
    \end{array}
  \end{equation*}
  Seleccionar el mínimo de los dos valores restantes
  \begin{equation*}
    \large
    \begin{array}{|r|r|r|r|}
      \hline \textcolor{red}{1} & \textcolor{red}{3} & \textcolor{red}{} & \textcolor{red}{} \\ \hline
    \end{array}\quad 
    \begin{array}{|r|r|r|}
      \hline \textcolor{blue}{4} & \textcolor{blue}{9} & \textcolor{blue}{5} \\ \hline
    \end{array}
  \end{equation*}
  \begin{equation*}
    \large
    \begin{array}{|r|r|}
      \hline & \textcolor{orange}{7}\\ \hline
    \end{array} \quad
    \begin{array}{|r|r|}
      \hline & \textcolor{green}{10} \\ \hline
    \end{array}\quad 
    \begin{array}{|r|r|r|}
      \hline \textcolor{blue}{4} & \textcolor{blue}{9} & \textcolor{blue}{5} \\ \hline
    \end{array}
  \end{equation*}
  Agregar el mínimo al arreglo ordenado
  \begin{equation*}
    \large
    \begin{array}{|r|r|r|r|}
      \hline \textcolor{red}{1} & \textcolor{red}{3} & \textcolor{red}{7} & \textcolor{red}{} \\ \hline
    \end{array}\quad 
    \begin{array}{|r|r|r|}
      \hline \textcolor{blue}{4} & \textcolor{blue}{9} & \textcolor{blue}{5} \\ \hline
    \end{array}
  \end{equation*}
  \begin{equation*}
    \large
    \begin{array}{|r|r|}
      \hline & \\ \hline
    \end{array} \quad
    \begin{array}{|r|r|}
      \hline & \textcolor{green}{10} \\ \hline
    \end{array}\quad 
    \begin{array}{|r|r|r|}
      \hline \textcolor{blue}{4} & \textcolor{blue}{9} & \textcolor{blue}{5} \\ \hline
    \end{array}
  \end{equation*}
  Cuando una lista es vacía, se agrega el resto de la otra lista
  \begin{equation*}
    \large
    \begin{array}{|r|r|r|r|}
      \hline \textcolor{red}{1} & \textcolor{red}{3} & \textcolor{red}{7} & \textcolor{red}{10} \\ \hline
    \end{array}\quad 
    \begin{array}{|r|r|r|}
      \hline \textcolor{blue}{4} & \textcolor{blue}{9} & \textcolor{blue}{5} \\ \hline
    \end{array}
  \end{equation*}
  Se selecciona el arreglo de la derecha
  \begin{equation*}
    \large
    \begin{array}{|r|r|r|r|}
      \hline 1 & 3 & 7 & 10 \\ \hline
    \end{array}\quad 
    \begin{array}{|r|r|r|}
      \hline \textcolor{red}{4} & \textcolor{red}{9} & \textcolor{red}{5} \\ \hline
    \end{array}
  \end{equation*}
  Se divide en dos partes el arreglo de la derecha
  \begin{equation*}
    \large
    \begin{array}{|r|r|r|r|}
      \hline 1 & 3 & 7 & 10 \\ \hline
    \end{array}\quad 
    \begin{array}{|r|r|}
      \hline \textcolor{red}{4} & \textcolor{red}{9} \\ \hline
    \end{array}\quad 
    \begin{array}{|r|}
      \hline 5 \\ \hline
    \end{array}
  \end{equation*}
  Se toma el arreglo de la izquierda y se divide en dos partes
  \begin{equation*}
    \large
    \begin{array}{|r|r|r|r|}
      \hline 1 & 3 & 7 & 10 \\ \hline
    \end{array}\quad 
    \begin{array}{|r|}
      \hline \textcolor{red}{4} \\ \hline
    \end{array} \quad
    \begin{array}{|r|}
      \hline 9 \\ \hline
    \end{array}\quad 
    \begin{array}{|r|}
      \hline 5 \\ \hline
    \end{array}
  \end{equation*}
  El arreglo de la izquierda ya no se puede dividir, se procede a ordenar:
  \begin{equation*}
    \large
    \begin{array}{|r|r|r|r|}
      \hline 1 & 3 & 7 & 10 \\ \hline
    \end{array}\quad 
    \begin{array}{|r|}
      \hline \textcolor{violet}{4} \\ \hline
    \end{array} \quad
    \begin{array}{|r|}
      \hline 9 \\ \hline
    \end{array}\quad 
    \begin{array}{|r|}
      \hline 5 \\ \hline
    \end{array}
  \end{equation*}
  El arreglo de la derecha ya no se puede dividir, se procede a ordenar:
  \begin{equation*}
    \large
    \begin{array}{|r|r|r|r|}
      \hline 1 & 3 & 7 & 10 \\ \hline
    \end{array}\quad 
    \begin{array}{|r|}
      \hline \textcolor{violet}{4} \\ \hline
    \end{array} \quad
    \begin{array}{|r|}
      \hline \textcolor{violet}{9} \\ \hline
    \end{array}\quad 
    \begin{array}{|r|}
      \hline 5 \\ \hline
    \end{array}
  \end{equation*}
\end{minipage}
\begin{minipage}{0.5\textwidth}
  Se ordena el arreglo seleccionado:
  \begin{equation*}
    \large
    \begin{array}{|r|r|r|r|}
      \hline 1 & 3 & 7 & 10 \\ \hline
    \end{array}\quad 
    \begin{array}{|r|}
      \hline \textcolor{green}{4} \\ \hline
    \end{array} \quad
    \begin{array}{|r|}
      \hline \textcolor{green}{9} \\ \hline
    \end{array}\quad 
    \begin{array}{|r|}
      \hline 5 \\ \hline
    \end{array}
  \end{equation*}
  Seleccionar el mínimo de los dos:
  \begin{equation*}
    \large
    \begin{array}{|r|r|r|r|}
      \hline 1 & 3 & 7 & 10 \\ \hline
    \end{array}\quad 
    \begin{array}{|r|}
      \hline \textcolor{orange}{4} \\ \hline
    \end{array} \quad
    \begin{array}{|r|}
      \hline \textcolor{green}{9} \\ \hline
    \end{array}\quad 
    \begin{array}{|r|}
      \hline 5 \\ \hline
    \end{array}
  \end{equation*}
  Agregar el mínimo al arreglo ordenado
  \begin{equation*}
    \large
    \begin{array}{|r|r|r|r|}
      \hline 1 & 3 & 7 & 10 \\ \hline
    \end{array}\quad 
    \begin{array}{|r|r|}
      \hline \textcolor{red}{4} & \textcolor{red}{} \\ \hline
    \end{array} \quad
    \begin{array}{|r|}
      \hline 5 \\ \hline
    \end{array}
  \end{equation*}
  \begin{equation*}
    \large
    \begin{array}{|r|r|r|r|}
      \hline 1 & 3 & 7 & 10 \\ \hline
    \end{array}\quad 
    \begin{array}{|r|}
      \hline \textcolor{orange}{} \\ \hline
    \end{array} \quad
    \begin{array}{|r|}
      \hline \textcolor{green}{9} \\ \hline
    \end{array}\quad 
    \begin{array}{|r|}
      \hline 5 \\ \hline
    \end{array}
  \end{equation*}
  Cuando una lista es vacía, se agrega el resto de la otra lista
  \begin{equation*}
    \large
    \begin{array}{|r|r|r|r|}
      \hline 1 & 3 & 7 & 10 \\ \hline
    \end{array}\quad 
    \begin{array}{|r|r|}
      \hline \textcolor{red}{4} & \textcolor{red}{9} \\ \hline
    \end{array} \quad
    \begin{array}{|r|}
      \hline 5 \\ \hline
    \end{array}
  \end{equation*}
  Se selecciona el arreglo de la derecha
  \begin{equation*}
    \large
    \begin{array}{|r|r|r|r|}
      \hline 1 & 3 & 7 & 10 \\ \hline
    \end{array}\quad 
    \begin{array}{|r|r|}
      \hline 4 & 9 \\ \hline
    \end{array} \quad
    \begin{array}{|r|}
      \hline \textcolor{red}{5} \\ \hline
    \end{array}
  \end{equation*}
  El arreglo de la derecha ya no se puede dividir, se procede a ordenar:
  \begin{equation*}
    \large
    \begin{array}{|r|r|r|r|}
      \hline 1 & 3 & 7 & 10 \\ \hline
    \end{array}\quad 
    \begin{array}{|r|r|}
      \hline 4 & 9 \\ \hline
    \end{array} \quad
    \begin{array}{|r|}
      \hline \textcolor{violet}{5} \\ \hline
    \end{array}
  \end{equation*}
  Se ordena el arreglo seleccionado:
  \begin{equation*}
    \large
    \begin{array}{|r|r|r|r|}
      \hline 1 & 3 & 7 & 10 \\ \hline
    \end{array}\quad 
    \begin{array}{|r|r|}
      \hline \textcolor{green}{4} & \textcolor{green}{9} \\ \hline
    \end{array} \quad
    \begin{array}{|r|}
      \hline \textcolor{green}{5} \\ \hline
    \end{array}
  \end{equation*}
  Selecciona el mínimo de las dos cabezas de las listas
  \begin{equation*}
    \large
    \begin{array}{|r|r|r|r|}
      \hline 1 & 3 & 7 & 10 \\ \hline
    \end{array}\quad 
    \begin{array}{|r|r|}
      \hline \textcolor{orange}{4} & \textcolor{green}{9} \\ \hline
    \end{array} \quad
    \begin{array}{|r|}
      \hline \textcolor{green}{5} \\ \hline
    \end{array}
  \end{equation*}
  Agregar el mínimo al arreglo ordenado
  \begin{equation*}
    \large
    \begin{array}{|r|r|r|r|}
      \hline 1 & 3 & 7 & 10 \\ \hline
    \end{array}\quad 
    \begin{array}{|r|r|r|}
      \hline \textcolor{red}{4} & \textcolor{red}{} & \textcolor{red}{} \\ \hline
    \end{array}
  \end{equation*}
  \begin{equation*}
    \large
    \begin{array}{|r|r|r|r|}
      \hline 1 & 3 & 7 & 10 \\ \hline
    \end{array}\quad 
    \begin{array}{|r|r|}
      \hline & \textcolor{green}{9} \\ \hline
    \end{array} \quad
    \begin{array}{|r|}
      \hline \textcolor{green}{5} \\ \hline
    \end{array}
  \end{equation*}
  Seleccionar el mínimo de las dos cabezas sin tomar en cuenta el mínimo anterior
  \begin{equation*}
    \large
    \begin{array}{|r|r|r|r|}
      \hline 1 & 3 & 7 & 10 \\ \hline
    \end{array}\quad 
    \begin{array}{|r|r|r|}
      \hline \textcolor{red}{4} & \textcolor{red}{} & \textcolor{red}{} \\ \hline
    \end{array}
  \end{equation*}
  \begin{equation*}
    \large
    \begin{array}{|r|r|r|r|}
      \hline 1 & 3 & 7 & 10 \\ \hline
    \end{array}\quad 
    \begin{array}{|r|r|}
      \hline & \textcolor{green}{9} \\ \hline
    \end{array} \quad
    \begin{array}{|r|}
      \hline \textcolor{orange}{5} \\ \hline
    \end{array}
  \end{equation*}
  Agregar el mínimo al arreglo ordenado
  \begin{equation*}
    \large
    \begin{array}{|r|r|r|r|}
      \hline 1 & 3 & 7 & 10 \\ \hline
    \end{array}\quad 
    \begin{array}{|r|r|r|}
      \hline \textcolor{red}{4} & \textcolor{red}{5} & \textcolor{red}{} \\ \hline
    \end{array}
  \end{equation*}
  \begin{equation*}
    \large
    \begin{array}{|r|r|r|r|}
      \hline 1 & 3 & 7 & 10 \\ \hline
    \end{array}\quad 
    \begin{array}{|r|r|}
      \hline & \textcolor{green}{9} \\ \hline
    \end{array} \quad
    \begin{array}{|r|}
      \hline \\ \hline
    \end{array}
  \end{equation*}
  Cuando una lista es vacía, se agrega el resto de la otra lista
  \begin{equation*}
    \large
    \begin{array}{|r|r|r|r|}
      \hline 1 & 3 & 7 & 10 \\ \hline
    \end{array}\quad 
    \begin{array}{|r|r|r|}
      \hline \textcolor{red}{4} & \textcolor{red}{5} & \textcolor{red}{9} \\ \hline
    \end{array}
  \end{equation*}
\end{minipage}

\newpage
\begin{minipage}{0.5\textwidth}
  Se ordena el arreglo seleccionado
  \begin{equation*}
    \large
    \begin{array}{|r|r|r|r|}
      \hline \textcolor{green}{1} & \textcolor{green}{3} & \textcolor{green}{7} & \textcolor{green}{10} \\ \hline
    \end{array}\quad 
    \begin{array}{|r|r|r|}
      \hline \textcolor{green}{4} & \textcolor{green}{5} & \textcolor{green}{9} \\ \hline
    \end{array}
  \end{equation*}
  Seleccionar el mínimo de las dos cabezas de las listas
  \begin{equation*}
    \large
    \begin{array}{|r|r|r|r|}
      \hline \textcolor{orange}{1} & \textcolor{green}{3} & \textcolor{green}{7} & \textcolor{green}{10} \\ \hline
    \end{array}\quad 
    \begin{array}{|r|r|r|}
      \hline \textcolor{green}{4} & \textcolor{green}{5} & \textcolor{green}{9} \\ \hline
    \end{array}
  \end{equation*}
  Agregar el mínimo al arreglo ordenado
  \begin{equation*}
    \large
    \begin{array}{|r|r|r|r|r|r|r|}
      \hline \textcolor{red}{1} & \textcolor{red}{} & \textcolor{red}{} & \textcolor{red}{} & \textcolor{red}{} & \textcolor{red}{} & \textcolor{red}{} \\ \hline
    \end{array}
  \end{equation*}
  \begin{equation*}
    \large
    \begin{array}{|r|r|r|r|}
      \hline & \textcolor{green}{3} & \textcolor{green}{7} & \textcolor{green}{10} \\ \hline
    \end{array}\quad 
    \begin{array}{|r|r|r|}
      \hline \textcolor{green}{4} & \textcolor{green}{5} & \textcolor{green}{9} \\ \hline
    \end{array}
  \end{equation*}
  Seleccionar el mínimo de las dos cabezas de las listas (sin tomar en cuenta el mínimo anterior)
  \begin{equation*}
    \large
    \begin{array}{|r|r|r|r|r|r|r|}
      \hline \textcolor{red}{1} & \textcolor{red}{} & \textcolor{red}{} & \textcolor{red}{} & \textcolor{red}{} & \textcolor{red}{} & \textcolor{red}{} \\ \hline
    \end{array}
  \end{equation*}
  \begin{equation*}
    \large
    \begin{array}{|r|r|r|r|}
      \hline & \textcolor{orange}{3} & \textcolor{green}{7} & \textcolor{green}{10} \\ \hline
    \end{array}\quad 
    \begin{array}{|r|r|r|}
      \hline \textcolor{green}{4} & \textcolor{green}{5} & \textcolor{green}{9} \\ \hline
    \end{array}
  \end{equation*}
  Agregar el mínimo al arreglo ordenado
  \begin{equation*}
    \large
    \begin{array}{|r|r|r|r|r|r|r|}
      \hline \textcolor{red}{1} & \textcolor{red}{3} & \textcolor{red}{} & \textcolor{red}{} & \textcolor{red}{} & \textcolor{red}{} & \textcolor{red}{} \\ \hline
    \end{array}
  \end{equation*}
  \begin{equation*}
    \large
    \begin{array}{|r|r|r|r|}
      \hline & & \textcolor{green}{7} & \textcolor{green}{10} \\ \hline
    \end{array}\quad 
    \begin{array}{|r|r|r|}
      \hline \textcolor{green}{4} & \textcolor{green}{5} & \textcolor{green}{9} \\ \hline
    \end{array}
  \end{equation*}
  Seleccionar el mínimo de las dos cabezas de las listas (sin tomar en cuenta el mínimo anterior)
  \begin{equation*}
    \large
    \begin{array}{|r|r|r|r|r|r|r|}
      \hline \textcolor{red}{1} & \textcolor{red}{3} & \textcolor{red}{} & \textcolor{red}{} & \textcolor{red}{} & \textcolor{red}{} & \textcolor{red}{} \\ \hline
    \end{array}
  \end{equation*}
  \begin{equation*}
    \large
    \begin{array}{|r|r|r|r|}
      \hline & & \textcolor{green}{7} & \textcolor{green}{10} \\ \hline
    \end{array}\quad 
    \begin{array}{|r|r|r|}
      \hline \textcolor{orange}{4} & \textcolor{green}{5} & \textcolor{green}{9} \\ \hline
    \end{array}
  \end{equation*}
  Agregar el mínimo al arreglo ordenado
  \begin{equation*}
    \large
    \begin{array}{|r|r|r|r|r|r|r|}
      \hline \textcolor{red}{1} & \textcolor{red}{3} & \textcolor{red}{4} & \textcolor{red}{} & \textcolor{red}{} & \textcolor{red}{} & \textcolor{red}{} \\ \hline
    \end{array}
  \end{equation*}
  \begin{equation*}
    \large
    \begin{array}{|r|r|r|r|}
      \hline & & \textcolor{green}{7} & \textcolor{green}{10} \\ \hline
    \end{array}\quad 
    \begin{array}{|r|r|r|}
      \hline & \textcolor{green}{5} & \textcolor{green}{9} \\ \hline
    \end{array}
  \end{equation*}
\end{minipage}
\begin{minipage}{0.5\textwidth}
  Seleccionar el mínimo de las dos cabezas de las listas (sin tomar en cuenta el mínimo anterior)
  \begin{equation*}
    \large
    \begin{array}{|r|r|r|r|r|r|r|}
      \hline \textcolor{red}{1} & \textcolor{red}{3} & \textcolor{red}{4} & \textcolor{red}{} & \textcolor{red}{} & \textcolor{red}{} & \textcolor{red}{} \\ \hline
    \end{array}
  \end{equation*}
  \begin{equation*}
    \large
    \begin{array}{|r|r|r|r|}
      \hline & & \textcolor{green}{7} & \textcolor{green}{10} \\ \hline
    \end{array}\quad 
    \begin{array}{|r|r|r|}
      \hline & \textcolor{orange}{5} & \textcolor{green}{9} \\ \hline
    \end{array}
  \end{equation*}
  Agregar el mínimo al arreglo ordenado
  \begin{equation*}
    \large
    \begin{array}{|r|r|r|r|r|r|r|}
      \hline \textcolor{red}{1} & \textcolor{red}{3} & \textcolor{red}{4} & \textcolor{red}{5} & \textcolor{red}{} & \textcolor{red}{} & \textcolor{red}{} \\ \hline
    \end{array}
  \end{equation*}
  \begin{equation*}
    \large
    \begin{array}{|r|r|r|r|}
      \hline & & \textcolor{green}{7} & \textcolor{green}{10} \\ \hline
    \end{array}\quad 
    \begin{array}{|r|r|r|}
      \hline &  & \textcolor{green}{9} \\ \hline
    \end{array}
  \end{equation*}
  Seleccionar el mínimo de las dos cabezas de las listas (sin tomar en cuenta el mínimo anterior)
  \begin{equation*}
    \large
    \begin{array}{|r|r|r|r|r|r|r|}
      \hline \textcolor{red}{1} & \textcolor{red}{3} & \textcolor{red}{4} & \textcolor{red}{5} & \textcolor{red}{} & \textcolor{red}{} & \textcolor{red}{} \\ \hline
    \end{array}
  \end{equation*}
  \begin{equation*}
    \large
    \begin{array}{|r|r|r|r|}
      \hline & & \textcolor{orange}{7} & \textcolor{green}{10} \\ \hline
    \end{array}\quad 
    \begin{array}{|r|r|r|}
      \hline &  & \textcolor{green}{9} \\ \hline
    \end{array}
  \end{equation*}
  Agregar el mínimo al arreglo ordenado
  \begin{equation*}
    \large
    \begin{array}{|r|r|r|r|r|r|r|}
      \hline \textcolor{red}{1} & \textcolor{red}{3} & \textcolor{red}{4} & \textcolor{red}{5} & \textcolor{red}{7} & \textcolor{red}{} & \textcolor{red}{} \\ \hline
    \end{array}
  \end{equation*}
  \begin{equation*}
    \large
    \begin{array}{|r|r|r|r|}
      \hline & & & \textcolor{green}{10} \\ \hline
    \end{array}\quad 
    \begin{array}{|r|r|r|}
      \hline &  & \textcolor{green}{9} \\ \hline
    \end{array}
  \end{equation*}
  Seleccional el mínimo de los últimos dos elementos
  \begin{equation*}
    \large
    \begin{array}{|r|r|r|r|r|r|r|}
      \hline \textcolor{red}{1} & \textcolor{red}{3} & \textcolor{red}{4} & \textcolor{red}{5} & \textcolor{red}{7} & \textcolor{red}{} & \textcolor{red}{} \\ \hline
    \end{array}
  \end{equation*}
  \begin{equation*}
    \large
    \begin{array}{|r|r|r|r|}
      \hline & & & \textcolor{green}{10} \\ \hline
    \end{array}\quad 
    \begin{array}{|r|r|r|}
      \hline &  & \textcolor{orange}{9} \\ \hline
    \end{array}
  \end{equation*}
  Agregar el mínimo al arreglo ordenado
  \begin{equation*}
    \large
    \begin{array}{|r|r|r|r|r|r|r|}
      \hline \textcolor{red}{1} & \textcolor{red}{3} & \textcolor{red}{4} & \textcolor{red}{5} & \textcolor{red}{7} & \textcolor{red}{9} & \textcolor{red}{} \\ \hline
    \end{array}
  \end{equation*}
  \begin{equation*}
    \large
    \begin{array}{|r|r|r|r|}
      \hline & & & \textcolor{green}{10} \\ \hline
    \end{array}\quad 
    \begin{array}{|r|r|r|}
      \hline &  &\\ \hline
    \end{array}
  \end{equation*}
  Cuando una lista es vacía, se agrega el resto de la otra lista
  \begin{equation*}
    \large
    \begin{array}{|r|r|r|r|r|r|r|}
      \hline \textcolor{red}{1} & \textcolor{red}{3} & \textcolor{red}{4} & \textcolor{red}{5} & \textcolor{red}{7} & \textcolor{red}{9} & \textcolor{red}{10} \\ \hline
    \end{array}
  \end{equation*}
\end{minipage}

\newpage
\textbf{Punto b:} Se muestra la tabla de llamadas a la función merge\_sort\_rec con los valores de los parámetros en cada iteración.\\


\begin{tabularx}{\textwidth}{|p{2cm}|X|X|p{2cm}|X|}
  \hline
  \textbf{Iteración} & \textbf{Llamada} & \textbf{Condición rgt > lft} & \textbf{mid} & \textbf{a[lft,rgt]} \\
  \hline
  0 & merge\_sort\_rec(a,1,7) & True & 4 & [7,1,10,3,4,9,5] \\
  1 & merge\_sort\_rec(a,1,4) & True & 2 & [7,1,10,3] \\
  2 & merge\_sort\_rec(a,1,2) & True & 1 & [7,1] \\
  3 & merge\_sort\_rec(a,1,1) & False & - & [7] \\
  4 & merge\_sort\_rec(a,2,2) & False & - & [1] \\
  \hline
  2 & merge(a,1,1,2) & - & 1 & [1,7] \\
  \hline
  2 & merge\_sort\_rec(a,3,4) & True & 3 & [10,3] \\
  3 & merge\_sort\_rec(a,3,3) & False & - & [10] \\
  4 & merge\_sort\_rec(a,4,4) & False & - & [3] \\
  \hline
  2 & merge(a,3,3,4) & - & 3 & [3,10] \\
  \hline
  0 & merge\_sort\_rec(a,5,7) & True & 6 & [4,9,5] \\
  1 & merge\_sort\_rec(a,5,6) & True & 5 & [4,9] \\
  2 & merge\_sort\_rec(a,5,5) & False & - & [4] \\
  3 & merge\_sort\_rec(a,6,6) & False & - & [9] \\
  \hline
  1 & merge(a,5,5,6) & - & 5 & [4,9] \\
  \hline
  1 & merge\_sort\_rec(a,7,7) & False & - & [5] \\
  \hline
  1 & merge(a,1,2,4) & - & 2 & [1,3,7,10] \\
  \hline
  0 & merge(a,5,6,7) & - & 6 & [4,5,9] \\
  \hline
  0 & merge(a,1,4,7) & - & 4 & [1,3,4,5,7,9,10] \\
  \hline
\end{tabularx}

\section*{Ejercicio 2}


\end{document}