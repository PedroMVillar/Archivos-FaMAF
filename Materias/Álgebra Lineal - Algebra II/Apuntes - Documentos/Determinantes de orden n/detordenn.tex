%%%%%%%%%%%%%%%%%%%%%%%%%%%%%%%%%%%%%%%%%%%%%%%%%%%%%%%%%%%%%%%%%%%%%%%%%%%%%%%%%%%%
% Do not alter this block (unless you're familiar with LaTeX
\documentclass{article}
\usepackage[margin=1in]{geometry} 
\usepackage{amsmath,amsthm,amssymb,amsfonts, fancyhdr, color, comment, graphicx, environ}
\usepackage{xcolor}
\usepackage{mdframed}
\usepackage[shortlabels]{enumitem}
\usepackage{indentfirst}
\usepackage{hyperref}
\usepackage{background}
\hypersetup{
    colorlinks=true,
    linkcolor=blue,
    filecolor=magenta,      
    urlcolor=blue,
}
\setlength{\headheight}{1.5cm}


\pagestyle{fancy}

\SetBgContents{https://github.com/PedroMVillar}
\SetBgScale{3} % Escala de la marca de agua
\SetBgColor{gray!80} % Color de la marca de agua
\SetBgAngle{45} % Ángulo de inclinación de la marca de agua
\SetBgOpacity{0.2} % Opacidad de la marca de agua

\newenvironment{problem}[2][Ejercicio]
    { \begin{mdframed}[backgroundcolor=gray!20] \textbf{#1 #2} \\}
    {  \end{mdframed}}

\newenvironment{definition}[2][Definición]
    { \begin{mdframed}[backgroundcolor=red!20] \textbf{#1 #2} \\}
    {  \end{mdframed}}

\newenvironment{theorem}[2][Teorema]
    { \begin{mdframed}[backgroundcolor=green!20] \textbf{#1 #2} \\}
    {  \end{mdframed}}

\newenvironment{lemma}[2][Lema]
    { \begin{mdframed}[backgroundcolor=green!20] \textbf{#1 #2} \\}
    {  \end{mdframed}}

\newenvironment{proposition}[2][Proposición]
    { \begin{mdframed}[backgroundcolor=green!20] \textbf{#1 #2} \\}
    {  \end{mdframed}}

\newenvironment{corollary}[2][Corolario]
    { \begin{mdframed}[backgroundcolor=green!20] \textbf{#1 #2} \\}
    {  \end{mdframed}}

\newenvironment{example}[2][Ejemplo]
    { \begin{mdframed}[backgroundcolor=yellow!20] \textbf{#1 #2} \\}
    {  \end{mdframed}}

\newenvironment{remark}[2][Observación]
    { \begin{mdframed}[backgroundcolor=yellow!20] \textbf{#1 #2} \\}
    {  \end{mdframed}}

\newenvironment{note}
    {\textit{Nota:}}
    {}

\newenvironment{conclusion}
    {\textit{Conclusión:}}
    {}

\newenvironment{notation}
    {\textit{Notación:}}
    {}

\newenvironment{question}
    {\textit{Pregunta:}}
    {}

\newenvironment{answer}
    {\textit{Respuesta:}}
    {}

% Define solution environment
\newenvironment{solution}
    {\textit{Solución:}}
    {}

\renewcommand{\qed}{\quad\qedsymbol}

% prevent line break in inline mode
\binoppenalty=\maxdimen
\relpenalty=\maxdimen

%%%%%%%%%%%%%%%%%%%%%%%%%%%%%%%%%%%%%%%%%%%%%
%Header Configuarción
\lhead{Pedro Villar}
\rhead{Álgebra Lineal} 
\chead{\textbf{Determinantes de orden $n$}}
%%%%%%%%%%%%%%%%%%%%%%%%%%%%%%%%%%%%%%%%%%%%%

\begin{document}

\begin{mdframed}[backgroundcolor=gray!40,shadow=true,roundcorner=8pt]
    El determinante puede ser pensado como una función que a cada matriz cuadrada $n \times n$ con coeficientes en $\mathbb{K}$,  le asocia un elemento de $\mathbb{K}$.
\end{mdframed}

%%%%%%%%%%%%%%%%%%%%%%%%%%%%%%%%%%%%%%%%%%%%%
%Definiciones y teoremas
\section*{Definiciones y Observaciones}
Sea $n \in \mathbb{N}$ y $A =[a_{ij}] \in M_n(\mathbb{K})$ , entonces el \textit{determinante de $A$} (en este caso expandido por la primera columna), denotado $\det(A)$ se define como:
        \begin{enumerate}
            \item[(1)] si $n=1$,  $\det([a]) =a$;
            \item[($n$)] si $n >1$, 
            \begin{align*}
            \det(A) &=  a_{11}\det A(1|1) - a_{21}\det A(2|1) + \cdots + (-1)^{1+n}  a_{n1}\det A(n|1) \\
            &= \sum_{i=1}^{n} (-1)^{1+i}  a_{i1}\det A(i|1).
            \end{align*}
        \end{enumerate}

\begin{remark}{- Propiedades de determinantes}
Sea $A$ una matriz cuadrada de orden $n$ y $k \in \mathbb{N}$, entonces:
\begin{enumerate}
    \item $\det(A^t) = \det(A)$.
    \item Si $A$ tiene dos filas iguales, entonces $\det(A) = 0$.
    \item Si $A$ tiene una fila de ceros, entonces $\det(A) = 0$.
    \item Si una fila es combinación lineal de otras filas, entonces $\det(A) = 0$.
    \item Si los elementos de una fila de $A$ son múltiplos de una constante $k$, entonces $\det(A) = k\det(A)$.
    \item Si en un determinante se intercambian dos filas, entonces $\det(A) = -\det(A)$.
    \item Si a los elementos de una fila se le suman los elementos de otra fila multiplicados por una constante, entonces $\det(A) = \det(A)$. 
\end{enumerate}
\end{remark}
%%%%%%%%%%%%%%%%%%%%%%%%%%%%%%%%%%%%%%%%%%%%%

%%%%%%%%%%%%%%%%%%%%%%%%%%%%%%%%%%%%%%%%%%%%%
%Triangulacion
\section*{Determinantes por triangulación}
Si $A$ es una matriz triangular superior (es decir que son iguales a cero los elementos que están debajo de la diagonal principal), la aplicación del desarrollo por la primera columna muestra que $det(A)$ se obtiene multiplicando los elementos de esta diagonal.
$$
A=\begin{bmatrix} 
a_{11} & a_{12} & \cdots & a_{1j} & \cdots & a_{1n} \\ 
0 & a_{22} & \cdots & a_{2j} & \cdots & a_{2n}\\ 
\vdots & 0 & \ddots & \vdots & \ddots & \vdots \\ 
\vdots & 0 & \cdots & a_{jj} & \cdots & a_{in} \\
\vdots & 0 & \ddots & 0 & \ddots & \vdots \\
0 & \cdots & \cdots & \cdots & \cdots & a_{nn}
\end{bmatrix}
\Rightarrow det(A) = \prod_{j=1}^{n}a_{jj}
$$
Por lo tanto si se tiene una matriz que no es triangular, un método útil para obtener el determinante es reducir la matriz a triangular superior y así llegar a hacer el producto de la diagonal.
\subsection*{Método para matrices $n \times n$}
La triangulación de una matriz $n\times n$ es un proceso algebraico que busca convertir la matriz original en una forma triangular superior o inferior mediante operaciones elementales de fila. Estas operaciones incluyen intercambio de filas, multiplicación de filas por un escalar y suma/resta de filas.
\subsection*{Ejemplos}
\begin{problem}{1}
    Calcular el determinante:
\[
\Delta_n = 
\begin{vmatrix}
1 & 2 & 2 & \dots & 2 \\
2 & 2 & 2 & \dots & 2 \\
2 & 2 & 3 & \dots & 2 \\
\vdots & \vdots & \vdots & \ddots & \vdots \\
2 & 2 & 2 & \dots & n
\end{vmatrix}.
\]
\end{problem}
\begin{solution}
    Efectuando las transformaciones $F_2-2F_1$, $F_3-F_2$, $F_4 - F_2$,...,$F_n-F_2$:
\[
\Delta_n = 
\begin{vmatrix}
1 & 2 & 2 & \dots & 2 \\
2 & 2 & 2 & \dots & 2 \\
2 & 2 & 3 & \dots & 2 \\
\vdots & \vdots & \vdots & \ddots & \vdots \\
2 & 2 & 2 & \dots & n
\end{vmatrix} = 
\begin{vmatrix}
1 & 2 & 2 & \dots & 2 \\
0 & -2 & -2 & \dots & -2 \\
0 & 0 & 1 & \dots & 0 \\
\vdots & \vdots & \vdots & \ddots & \vdots \\
0 & 0 & 0 & \dots & n-2
\end{vmatrix} = 1 \cdot (-2) \cdot 1 \cdot \dots \cdot (n-2) = (-2)(n-2)!
\]
Esto quiere decir que en la diagonal luego del $1$ aparecerán todos los escalares hasta $n-2$.
\end{solution}
\begin{problem}{2}
    Calcular el determinante:
\[
\Delta_n = \begin{vmatrix}
n & 1 & 1 & \dots & 1 \\
n & 2 & 1 & \dots & 1 \\
n & 1 & 3 & \dots & 1 \\
\vdots & \vdots & \vdots & \ddots & \vdots \\
n & 1 & 1 & \dots & n
\end{vmatrix}.
\]
\end{problem}
\begin{solution}
    Restando a cada fila, a partir de la segunda, la primera:
\[
\Delta_n = \begin{vmatrix}
n & 1 & 1 & \dots & 1 \\
n & 2 & 1 & \dots & 1 \\
n & 1 & 3 & \dots & 1 \\
\vdots & \vdots & \vdots & \ddots & \vdots \\
n & 1 & 1 & \dots & n
\end{vmatrix} =
\begin{vmatrix}
n & 1 & 1 & \dots & 1 \\
0 & 1 & 0 & \dots & 0 \\
0 & 0 & 2 & \dots & 0 \\
\vdots & \vdots & \vdots & \ddots & \vdots \\
0 & 0 & 0 & \dots & n-1
\end{vmatrix} = n \cdot (n-1)! = n!
\]
\end{solution}

\begin{problem}{3}
    Calcular el determinante:
\[
\Delta_n = \begin{vmatrix}
1 & 2 & 3 & \dots & n \\
-1 & 0 & 3 & \dots & n \\
-1 & -2 & 0 & \dots & n \\
\vdots & \vdots & \vdots & \ddots & \vdots \\
-1 & -2 & -3 & \dots & 0
\end{vmatrix}.
\]
\end{problem}
\begin{solution}
    Sumando a cada fila menos a la primera, la primera:
\[
\Delta_n = \begin{vmatrix}
1 & 2 & 3 & \dots & n \\
-1 & 0 & 3 & \dots & n \\
-1 & -2 & 0 & \dots & n \\
\vdots & \vdots & \vdots & \ddots & \vdots \\
-1 & -2 & -3 & \dots & 0
\end{vmatrix} = 
\begin{vmatrix}
1 & 2 & 3 & \dots & n \\
0 & 2 & 6 & \dots & 2n \\
0 & 0 & 3 & \dots & 2n \\
\vdots & \vdots & \vdots & \ddots & \vdots \\
0 & 0 & 0 & \dots & n
\end{vmatrix} = n!
\]
\end{solution}

%%%%%%%%%%%%%%%%%%%%%%%%%%%%%%%%%%%%%%%%%%%%%

%%%%%%%%%%%%%%%%%%%%%%%%%%%%%%%%%%%%%%%%%%%%%
%Induccion
\section*{Determinantes por Inducción}
Para resolver determinantes por inducción fuerte en $n$, se sigue un enfoque inductivo para matrices de orden $n$.
\begin{enumerate}
    \item \textbf{Caso base:} Si $n=1$, entonces $\det(A) = a_{11}$ (en algunos casos se toma $n=2$).
    \item \textbf{Hipótesis de inducción:} Supongamos que el determinante vale para $n<k$. Y se debe probar que vale para $n=k+1$.
    \item \textbf{Paso inductivo:} Se expande el determinante de la matriz de orden $n$ por la primera fila o columna, y se obtiene una expresión en términos de determinantes de matrices de orden $n$ o menor. Se busca aplicar la hipótesis de inducción para obtener una expresión en términos de determinantes de matrices de orden $n$ o menor.
\end{enumerate}
\subsection*{Ejemplos}
\begin{problem}{1}
    Demuestra que si $a\neq b$ entonces el siguiente determinante de $n\times n$
    \[
    \begin{vmatrix}
    a+ b & ab & 0 & \dots & 0 & 0 \\
    1 & a+b & ab & \dots & 0 & 0 \\
    0 & 1 & a+b & \dots & 0 & 0 \\
    \vdots & \ddots & \ddots & \dots & \vdots & \vdots \\
    0 & 0 & 0 & \dots & a+b & ab \\
    0 & 0 & 0 & \dots & 1 & a+b
    \end{vmatrix} =
    \frac{a^{n+1}-b^{n+1}}{a-b}
    \]
\end{problem}
\begin{solution}
    Para esto haremos inducción fuerte sobre $n$: \newline
    \textbf{Caso base:}
        Tomando $n=2$, probaremos que $\begin{vmatrix} a+b & ab \\ 1 &  a+b \end{vmatrix} =  \frac{a^3-b^3}{a-b}$:
        \[
    \begin{aligned}
    \begin{vmatrix} a+b & ab \\ 1 &  a+b \end{vmatrix} &= (a+b)^2 - ab = a^2 +2ab + b^2 - ab = a^2 + ab + b^2 \\
    &= \frac{(a-b)\cdot (a^2 + ab + b^2)}{(a-b)} = \frac{a^3-b^3}{a-b}
    \end{aligned}
        \]
    \textbf{Hipótesis de inducción:} 
    Ahora supongamos que el determinante vale para $n\leq k$ y con esto probar que vale para $n=k+1$ el determinante:
\[
\begin{vmatrix}
a+ b & ab & 0 & \dots & 0 & 0 \\
1 & a+b & ab & \dots & 0 & 0 \\
0 & 1 & a+b & \dots & 0 & 0 \\
\vdots & \ddots & \ddots & \dots & \vdots & \vdots \\
0 & 0 & 0 & \dots & a+b & ab \\
0 & 0 & 0 & \dots & 1 & a+b
\end{vmatrix}_{(k+1)\times(k+1)} =
\frac{a^{k+2}-b^{k+2}}{a-b}
\]
\textbf{Paso inductivo:} 
Calculo el determinante expandiendo por la fila 1:
\[
\begin{aligned}
& \begin{vmatrix}
a+ b & ab & 0 & \dots & 0 & 0 \\
1 & a+b & ab & \dots & 0 & 0 \\
0 & 1 & a+b & \dots & 0 & 0 \\
\vdots & \ddots & \ddots & \dots & \vdots & \vdots \\
0 & 0 & 0 & \dots & a+b & ab \\
0 & 0 & 0 & \dots & 1 & a+b
\end{vmatrix} \\ \\ \\
&= (a+b) \cdot
\begin{vmatrix}
a+ b & ab & 0 & \dots & 0 & 0 \\
1 & a+b & ab & \dots & 0 & 0 \\
0 & 1 & a+b & \dots & 0 & 0 \\
\vdots & \ddots & \ddots & \dots & \vdots & \vdots \\
0 & 0 & 0 & \dots & a+b & ab \\
0 & 0 & 0 & \dots & 1 & a+b
\end{vmatrix}_{k\times k} - ab \cdot
\begin{vmatrix}
1 & ab & 0 & \dots & 0 & 0 \\
0 & a+b & ab & \dots & 0 & 0 \\
0 & 1 & a+b & \dots & 0 & 0 \\
\vdots & \ddots & \ddots & \dots & \vdots & \vdots \\
0 & 0 & 0 & \dots & a+b & ab \\
0 & 0 & 0 & \dots & 1 & a+b
\end{vmatrix}_{k\times k} \\
\\
\\
&=(a+b) \cdot \frac{a^{k+1}-b^{k+1}}{a-b} - ab \cdot
\begin{vmatrix}
1 & ab & 0 & \dots & 0 & 0 \\
0 & a+b & ab & \dots & 0 & 0 \\
0 & 1 & a+b & \dots & 0 & 0 \\
\vdots & \ddots & \ddots & \dots & \vdots & \vdots \\
0 & 0 & 0 & \dots & a+b & ab \\
0 & 0 & 0 & \dots & 1 & a+b
\end{vmatrix}_{k\times k} \\
\\
\\
\end{aligned}
\]
\[
\begin{aligned}
&= (a+b) \cdot \frac{a^{k+1}-b^{k+1}}{a-b} - ab \cdot1
\begin{vmatrix}
a+b & ab & \dots & 0 & 0 \\
1 & a+b & \dots & 0 & 0 \\
\ddots & \ddots & \dots & \vdots & \vdots \\
0 & 0 & \dots & a+b & ab \\
0 & 0 & \dots & 1 & a+b
\end{vmatrix}_{(k-1)\times (k-1)} \\
\\
\\
&= (a+b) \cdot \frac{a^{k+1}-b^{k+1}}{a-b} - ab \cdot \frac{a^k-b^k}{a-b} = \frac{a^{k+2}-ab^{k+1}+ba^{k+1}-b^{k+2}}{a-b} - \frac{ba^{k+1}-ab^{k+1}}{a-b} \\
\\
\\
&= \frac{a^{k+2}-b^{k+2}}{a-b}
\end{aligned}
\]
Con esto queda probado que vale para todo $n$.
\end{solution}
\newpage
\begin{problem}{2}
    Sea $A_n = \begin{pmatrix} 2 & -1 & 0 & \dots & 0 & 0 \\ -1 & 2 & -1 & \dots & 0 & 0 \\ 0 & -1 & 2 & \dots & 0 & 0 \\ \vdots & \vdots & \vdots & \ddots & \vdots & \vdots \\ 0 & 0 & 0 & \dots & 2 & -1 \\ 0 & 0 & 0 & \dots & -1 & 2 \end{pmatrix} \in M_{n\times n}(\mathbb{R})$ probar que $det(A_n)=n+1$ para todo $n\in \mathbb{N}$.
\end{problem}
\begin{solution}
    \textbf{Caso base:} Para la prueba hacemos inducción fuerte sobre $n$, para el caso base, tomamos $n=1$ y $n=2$:
    \[
\begin{vmatrix}
2
\end{vmatrix} = 2 = 1+1, \ \ \ \ \ 
\begin{vmatrix}
2 & -1 \\
-1 & 2
\end{vmatrix} = 4-1 = 3 = 2+1
\]
\textbf{Hipótesis de inducción:} Ahora supongamos que el determinante vale para $n\leq k$ y con esto probar que vale para $n=k+1$ el determinante:
\[
\begin{vmatrix} 2 & -1 & 0 & \dots & 0 & 0 \\ -1 & 2 & -1 & \dots & 0 & 0 \\ 0 & -1 & 2 & \dots & 0 & 0 \\ \vdots & \vdots & \vdots & \ddots & \vdots & \vdots \\ 0 & 0 & 0 & \dots & 2 & -1 \\ 0 & 0 & 0 & \dots & -1 & 2 \end{vmatrix}_{(k+1)\times(k+1)} = k+2
\]
\textbf{Paso inductivo:} 
Calculo el determinante expandiendo por la fila 1:
\[
\begin{aligned}
\begin{vmatrix} 2 & -1 & 0 & \dots & 0 & 0 \\ -1 & 2 & -1 & \dots & 0 & 0 \\ 0 & -1 & 2 & \dots & 0 & 0 \\ \vdots & \vdots & \vdots & \ddots & \vdots & \vdots \\ 0 & 0 & 0 & \dots & 2 & -1 \\ 0 & 0 & 0 & \dots & -1 & 2 \end{vmatrix}_{(k+1)\times(k+1)} &= 2 \cdot 
\begin{vmatrix} 2 & -1 & 0 & \dots & 0 & 0 \\ -1 & 2 & -1 & \dots & 0 & 0 \\ 0 & -1 & 2 & \dots & 0 & 0 \\ \vdots & \vdots & \vdots & \ddots & \vdots & \vdots \\ 0 & 0 & 0 & \dots & 2 & -1 \\ 0 & 0 & 0 & \dots & -1 & 2 \end{vmatrix}_{k\times k} + 1 \cdot 
\begin{vmatrix} -1 & -1 & \dots & 0 & 0 \\ 0 & 2 & \dots & 0 & 0 \\ \vdots  & \vdots & \ddots & \vdots & \vdots \\ 0 & 0 & \dots & 2 & -1 \\ 0 & 0 & \dots & -1 & 2 \end{vmatrix}_{k\times k}  \\ \\
&= 2 (k+1) + (-1) 
\begin{vmatrix} 2 & -1 &  \dots & 0 & 0 \\  -1 & 2 & \dots & 0 & 0 \\ \vdots & \vdots & \ddots & \vdots & \vdots \\ 0 & 0 & \dots & 2 & -1 \\ 0 &0& \dots & -1 & 2 \end{vmatrix}_{(k-1)\times (k-1)}  \\ \\
&= 2 (k+1) + (-1)k = 2k+2-k = k+2
\end{aligned}
\]
Con esto queda probado que $det(A_n)=n+1$ vale para todo $n$ natural.z
\end{solution}

\begin{problem}{3}
    Calcular el determinante de orden $n$ de:
\[
A =
\begin{bmatrix}
1 + x^2 & x & 0 & \dots & 0 \\
x & 1 + x^2 & x & \dots & 0 \\
0 & x & 1 + x^2 & \dots & 0 \\
\vdots & \vdots & \vdots & \ddots & \vdots \\
0 & 0 & 0 & \dots & 1+x^2
\end{bmatrix}
\]
\end{problem}
\begin{solution}
    Como no tenemos una fórmula, todavía no podemos hacer inducción fuerte para calcular el determinante, analicemos los resultados cuando la matriz es de orden 1,2 y 3:
\[
\begin{vmatrix}
1 + x^2
\end{vmatrix} = 1+x^2, \ \ \
\begin{vmatrix}
1 + x^2 & x \\
x & 1+x^2
\end{vmatrix} = 1 + x^2 + x^4 \ \ \
\begin{vmatrix}
1 + x^2 & x & 0 \\
x & 1+x^2 & x \\
0 & x & 1+x^2
\end{vmatrix} = 1+x^2+x^4+x^6
\]
Se puede observar un patrón el determinante se va formando como $1+x^2+x^4+\dots +x^{2n}$, entonces la fórmula es
\[
det(A) = \sum_{k=0}^{n} x^{2k}
\]
\textbf{Caso base:} Tomando $n=2$, ya está probado en el paso anterior que
\[
\begin{vmatrix}
1 + x^2 & x \\
x & 1+x^2
\end{vmatrix} = 1 + x^2 + x^4 = \sum_{k = 0}^{2} x^{2k}
\]
\textbf{Hipótesis de inducción:} Ahora supongamos que el determinante vale para $n\leq k$ y con esto probar que vale para $n=k+1$ el determinante:
\[
\begin{vmatrix}
1 + x^2 & x & 0 & \dots & 0 \\
x & 1 + x^2 & x & \dots & 0 \\
0 & x & 1 + x^2 & \dots & 0 \\
\vdots & \vdots & \vdots & \ddots & \vdots \\
0 & 0 & 0 & \dots & 1+x^2
\end{vmatrix}_{(k+ 1)\times (k+1)} = \sum_{k = 0}^{n} x^{2k+2}
\]
\textbf{Paso inductivo:} Desarrollo el determinante por la primera columna=
\[
\begin{aligned}
\begin{vmatrix}
1 + x^2 & x & 0 & \dots & 0 \\
x & 1 + x^2 & x & \dots & 0 \\
0 & x & 1 + x^2 & \dots & 0 \\
\vdots & \vdots & \vdots & \ddots & \vdots \\
0 & 0 & 0 & \dots & 1+x^2
\end{vmatrix}_{(k+ 1)\times (k+1)} &= (1+x^2)\sum_{k=0}^{n} x^{2k} - x
\begin{vmatrix}
x & 0 & \dots & 0 \\
x & 1 + x^2 & \dots & 0 \\
\vdots & \vdots & \ddots & \vdots \\
0 & 0 & \dots & 1+x^2
\end{vmatrix}_{k\times k} \\
&= (1+x^2)\sum_{k=0}^{n} x^{2k} - x^2\sum_{k=0}^{n} x^{2k-2} = \sum_{k=0}^{n} x^{2k+2}
\end{aligned}
\]
Con esto queda probado que vale para todo $n$ vale el determinante.
\end{solution}
\newpage
\begin{problem}{4}
    Calcular el determinante de orden $n$:
\[
D_n(x) = 
\begin{vmatrix}
2x & x^2 & 0 & \dots & 0 \\
1 & 2x & x^2 & \dots & 0 \\
0 & 1 & 2x & \dots & 0 \\
\vdots & \vdots & \vdots & \ddots & \vdots \\
0 & 0 & 0 & \dots & 2x
\end{vmatrix}
\]
\end{problem}
\begin{solution}
    Como no tenemos una fórmula, todavía no podemos hacer inducción fuerte para calcular el determinante, analicemos los resultados cuando la matriz es de orden 1,2 y 3:
\[
D_1(x) = |2x| = 2x,
\]\[
D_2(x) = \begin{vmatrix} 2x & x^2 \\ 1 & 2x \end{vmatrix}= 3x^2,
\]\[
D_3(x)=\begin{vmatrix} 2x & x^2 & 0 \\ 1 & 2x & x^2 \\ 0 & 1 & 2x\end{vmatrix} = 4x^3
\]
Esto nos permite conjeturar una fórmula
\[
D_n(x)=(n+1)x^n
\]
Ahora hagamos inducción fuerte sobre $n$ para demostrar la fórmula:
\textbf{Caso base:} El caso base ya está demostrado con $n=1$ y $n=2$
\[
D_1(x) = |2x| = 2x, \ \ \ \ \ D_2(x) = \begin{vmatrix} 2x & x^2 \\ 1 & 2x \end{vmatrix}= 3x^2
\]
\textbf{Hipótesis de inducción:} Ahora supongamos que el determinante vale para $n\leq k$ y con esto probar que vale para $n=k+1$ el determinante:
\[
D_{k+1}(x) = 
\begin{vmatrix}
2x & x^2 & 0 & \dots & 0 \\
1 & 2x & x^2 & \dots & 0 \\
0 & 1 & 2x & \dots & 0 \\
\vdots & \vdots & \vdots & \ddots & \vdots \\
0 & 0 & 0 & \dots & 2x
\end{vmatrix}_{(k+1)\times(k+1)} = (k+2)x^{k+1}
\]
\textbf{Paso inductivo:} Desarrollando por la primera columna:
\[
\begin{aligned}
\begin{vmatrix}
2x & x^2 & 0 & \dots & 0 \\
1 & 2x & x^2 & \dots & 0 \\
0 & 1 & 2x & \dots & 0 \\
\vdots & \vdots & \vdots & \ddots & \vdots \\
0 & 0 & 0 & \dots & 2x
\end{vmatrix}_{(k+1)\times(k+1)} &= 2x 
\begin{vmatrix}
2x & x^2 & \dots & 0 \\
1 & 2x & \dots & 0 \\
\vdots & \vdots & \ddots & \vdots \\
0 & 0 & \dots & 2x
\end{vmatrix}_{k\times k} - 
\begin{vmatrix}
x^2 & 0 & \dots & 0 \\
1 & 2x & \dots & 0 \\
\vdots & \vdots & \ddots & \vdots \\
0 & 0 & \dots & 2x
\end{vmatrix}_{k\times k} \\ \\
&= 2xD_k(x)-x^2D_{k-1}(x) = 2x(k+1)x^k - x^2kx^{k-1} \\
&= (k+2)x^{k+1}
\end{aligned}
\]
Con esto queda probado que el determinante vale para todo $n$.
\end{solution}


%%%%%%%%%%%%%%%%%%%%%%%%%%%%%%%%%%%%%%%%%%%%%


\end{document}