%%%%%%%%%%%%%%%%%%%%%%%%%%%%%%%%%%%%%%%%%%%%%%%%%%%%%%%%%%%%%%%%%%%%%%%%%%%%%%%%%%%%
% Do not alter this block (unless you're familiar with LaTeX
\documentclass{article}
\usepackage[margin=1in]{geometry} 
\usepackage{amsmath,amsthm,amssymb,amsfonts, fancyhdr, color, comment, graphicx, environ}
\usepackage{xcolor}
\usepackage{mdframed}
\usepackage[shortlabels]{enumitem}
\usepackage{indentfirst}
\usepackage{hyperref}
\usepackage{background}
\hypersetup{
    colorlinks=true,
    linkcolor=blue,
    filecolor=magenta,      
    urlcolor=blue,
}
\setlength{\headheight}{1.5cm}


\pagestyle{fancy}

\SetBgContents{https://github.com/PedroMVillar}
\SetBgScale{3} % Escala de la marca de agua
\SetBgColor{gray!80} % Color de la marca de agua
\SetBgAngle{45} % Ángulo de inclinación de la marca de agua
\SetBgOpacity{0.2} % Opacidad de la marca de agua


\newenvironment{problem}[2][Ejercicio]
    { \begin{mdframed}[backgroundcolor=gray!20] \textbf{#1 #2} \\}
    {  \end{mdframed}}

\newenvironment{definition}[2][Definición]
    { \begin{mdframed}[backgroundcolor=red!20] \textbf{#1 #2} \\}
    {  \end{mdframed}}

\newenvironment{theorem}[2][Teorema]
    { \begin{mdframed}[backgroundcolor=green!20] \textbf{#1 #2} \\}
    {  \end{mdframed}}

\newenvironment{lemma}[2][Lema]
    { \begin{mdframed}[backgroundcolor=green!20] \textbf{#1 #2} \\}
    {  \end{mdframed}}

\newenvironment{proposition}[2][Proposición]
    { \begin{mdframed}[backgroundcolor=green!20] \textbf{#1 #2} \\}
    {  \end{mdframed}}

\newenvironment{corollary}[2][Corolario]
    { \begin{mdframed}[backgroundcolor=green!20] \textbf{#1 #2} \\}
    {  \end{mdframed}}

\newenvironment{example}[2][Ejemplo]
    { \begin{mdframed}[backgroundcolor=yellow!20] \textbf{#1 #2} \\}
    {  \end{mdframed}}

\newenvironment{remark}[2][Observación]
    { \begin{mdframed}[backgroundcolor=yellow!20] \textbf{#1 #2} \\}
    {  \end{mdframed}}

\newenvironment{note}
    {\textit{Nota:}}
    {}

\newenvironment{conclusion}
    {\textit{Conclusión:}}
    {}

\newenvironment{notation}
    {\textit{Notación:}}
    {}

\newenvironment{question}
    {\textit{Pregunta:}}
    {}

\newenvironment{answer}
    {\textit{Respuesta:}}
    {}

% Define solution environment
\newenvironment{solution}
    {\textit{Solución:}}
    {}

\renewcommand{\qed}{\quad\qedsymbol}

% prevent line break in inline mode
\binoppenalty=\maxdimen
\relpenalty=\maxdimen

%%%%%%%%%%%%%%%%%%%%%%%%%%%%%%%%%%%%%%%%%%%%%
%Header Configuarción
\lhead{Pedro Villar}
\rhead{Álgebra Lineal} 
\chead{\textbf{Diagonalización de matrices}}
%%%%%%%%%%%%%%%%%%%%%%%%%%%%%%%%%%%%%%%%%%%%%

\begin{document}

\section*{Definiciones y Teoremas}

%%%%%%%%%%%%%%%%%%%%%%%%%%%%%%%%%%%%%%%%%%%%%
%Definiciones y teoremas
\begin{definition}{1}
Sea $A$ una matriz cuadrada de orden $n$. Decimos que $A$ es diagonalizable si $A$ es semejante a una matriz diagonal, es decir, existe una matriz $P$ de orden $n$ inversible tal que:
$$
P^{-1}AP = D
$$
donde $D$ es una matriz diagonal.
Es un caso especial de semejanza. Una matriz es diagonalizable cuando es semejante a una matriz diagonal.
\end{definition}

\begin{theorem}{1}
    Una matriz $A$ de orden $n$ es \textit{diagonizable} si y sólo si tiene $n$ vectores linealmente independientes. En tal caso, la matriz diagonal $D$ semejante a $A$ está dada por
    $$
    \begin{pmatrix}
    \lambda_1 & 0 & \cdots & 0 \\
    0 & \lambda_2 & \cdots & 0 \\
    \vdots & \vdots & \ddots & \vdots \\
    0 & 0 & \cdots & \lambda_n
    \end{pmatrix}
    $$
    donde $\lambda_1, \lambda_2, \ldots, \lambda_n$ son los autovalores de $A$. Y si $P$ es una matriz cuyas columnas son los vectores característicos linealmente independientes de $A$, entonces $P^{-1}AP = D$.
\end{theorem}

\begin{corollary}{1}
    Si la matriz $A$ de orden $n$ tiene $n$ autovalores diferentes, entonces $A$ es diagonizable.
\end{corollary}

\begin{theorem}{2}
    Una transformación lineal $T:V \to V$ es diagonalizable si y sólo si existe una base de $V$ formada por vectores propios de $T$.
\end{theorem}
%%%%%%%%%%%%%%%%%%%%%%%%%%%%%%%%%%%%%%%%%%%%%

%%%%%%%%%%%%%%%%%%%%%%%%%%%%%%%%%%%%%%%%%%%%%
% Método para diagonalizar una matriz
\section*{Método para diagonalizar una matriz}
\begin{enumerate}
    \item Encontrar los autovalores de la matriz $A$ resolviendo la ecuación característica $|A - xId_n| = 0$.
    \item Para cada autovalor $\lambda_i$, encontrar los vectores característicos asociados resolviendo el sistema homogéneo $(A - \lambda_i Id_n)x = 0$.
    \item Si se encuentran $n$ vectores característicos linealmente independientes, entonces la matriz $A$ es diagonalizable.
    \item Si la matriz $A$ es diagonalizable, entonces la matriz diagonal $D$ semejante a $A$ está dada por
    $$
    \begin{pmatrix}
    \lambda_1 & 0 & \cdots & 0 \\
    0 & \lambda_2 & \cdots & 0 \\
    \vdots & \vdots & \ddots & \vdots \\
    0 & 0 & \cdots & \lambda_n
    \end{pmatrix}
    $$
    donde $\lambda_1, \lambda_2, \ldots, \lambda_n$ son los autovalores de $A$. Y si $P$ es una matriz cuyas columnas son los vectores característicos linealmente independientes de $A$, entonces $P^{-1}AP = D$.
\end{enumerate}
%%%%%%%%%%%%%%%%%%%%%%%%%%%%%%%%%%%%%%%%%%%%%

%%%%%%%%%%%%%%%%%%%%%%%%%%%%%%%%%%%%%%%%%%%%%
% Ejemplos
\section*{Ejemplos}
Sea $T:\mathbb{R}^3 \to \mathbb{R}^3$ la transformación lineal dada por
$$
T(x,y,z) = (x-y+4z, 3x+2y-z,2x+y-z)
$$
\begin{problem}{1}
    Encuentre la matriz asociada a la transformación lineal $T$.
\end{problem}
\begin{solution}
    La matriz asociada a la transformación lineal $T$ es
    $$
    A = \begin{pmatrix}
    1 & -1 & 4 \\
    3 & 2 & -1 \\
    2 & 1 & -1
    \end{pmatrix}
    $$
\end{solution}

\begin{problem}{2}
    Encuentre los autovalores de la matriz $A$.
\end{problem}
\begin{solution}
    Los autovalores de la matriz $A$ son las raíces del polinomio característico $|A - xId_3| = 0$. Entonces, resolvemos la ecuación
    $$
    |A - xId_3| = 0
    $$
    $$
    \begin{vmatrix}
    1-x & -1 & 4 \\
    3 & 2-x & -1 \\
    2 & 1 & -1-x
    \end{vmatrix} = 0
    $$
    Calculo el determinante expandiendo por la columna 1:
    $$
    (1-x)\begin{vmatrix}
    2-x & -1 \\
    1 & -1-x
    \end{vmatrix} - (-1)\begin{vmatrix}
    3 & -1 \\
    2 & -1-x
    \end{vmatrix} + 4\begin{vmatrix}
    3 & 2-x \\
    2 & 1
    \end{vmatrix} = 0
    $$
    $$
    (1-x)[(2-x)(-1-x) - (-1)1] - (-1)[(-3-x) - 2(-1)] + 4[(3)(1) - (2-x)(2)] = 0
    $$
    $$
    (1-x)[-2-2x+x+x^2] + 1[-3-x+2] + 4[3-4+2x] = 0
    $$
    $$
    -2-2x+x+x^2+2x+2x^2-x^2-x^3-3-x+2+12-16+8x = 0
    $$
    $$
    -x^3+2x^2+5x-6 = 0
    $$

A simple vista notamos que $x=1$ es una raíz del polinomio, entonces $x-1$ es un factor del polinomio. Dividimos el polinomio por $x-1$ para encontrar los otros dos factores: 
$$
-x^3+2x^2+5x-6 = (x-1)(-x^2+x+6)
$$
Resolvemos la ecuación cuadrática $-x^2+x+6 = 0$ para encontrar los otros dos autovalores: 
$$
x = \frac{-1 \pm \sqrt{1^2-4(-1)(6)}}{-2} = \frac{-1 \pm \sqrt{1+24}}{-2} = \frac{-1 \pm \sqrt{25}}{-2} = \frac{-1 \pm 5}{-2}
$$
Entonces, los autovalores de la matriz $A$ son $\lambda_1 = 1$, $\lambda_2 = -2$ y $\lambda_3 = 3$.
\end{solution}

\begin{problem}{3}
    Encuentre los vectores característicos asociados a cada autovalor.
\end{problem}
\begin{solution}
    Para el autovalor $\lambda_1 = 1$, resolvemos el sistema homogéneo $(A - \lambda_1 Id_3)x = 0$:
    $$
    (A - \lambda_1 Id_3)x = 0
    $$
    $$
    \begin{pmatrix}
    0 & -1 & 4 \\
    3 & 1 & -1 \\
    2 & 1 & -2
    \end{pmatrix}
    \begin{pmatrix}
    x \\
    y \\
    z
    \end{pmatrix}
    = 0
    $$
    Reduzco la matriz a su forma escalonada reducida:
    $$
    \begin{pmatrix}
        0 & -1 & 4 \\
        3 & 1 & -1 \\
        2 & 1 & -2
    \end{pmatrix} \xrightarrow{F_1 \leftrightarrow F_2}
    \begin{pmatrix}
        3 & 1 & -1 \\
        0 & -1 & 4 \\
        2 & 1 & -2
    \end{pmatrix} \xrightarrow{F_1/3}
    \begin{pmatrix}
        1 & 1/3 & -1/3 \\
        0 & -1 & 4 \\
        2 & 1 & -2
    \end{pmatrix} \xrightarrow{F_3-2F_1}
    \begin{pmatrix}
        1 & 1/3 & -1/3 \\
        0 & -1 & 4 \\
        0 & 1/3 & -4/3
    \end{pmatrix} \xrightarrow{F_2/(-1)}
    $$
    $$
    \begin{pmatrix}
        1 & 1/3 & -1/3 \\
        0 & 1 & -4 \\
        0 & 1/3 & -4/3
    \end{pmatrix} \xrightarrow{F_3 - F_2/3}
    \begin{pmatrix}
        1 & 1/3 & -1/3 \\
        0 & 1 & -4 \\
        0 & 0 & 0
    \end{pmatrix} \xrightarrow{F_1-F_2/3}
    \begin{pmatrix}
        1 & 0 & 1 \\
        0 & 1 & -4 \\
        0 & 0 & 0
    \end{pmatrix}
    $$
    De acá salen las ecuaciones: $x+z = 0$ y $y-4z = 0$. Entonces, $x=-z$ e $y=4z$. Por lo tanto las soluciones son
    de la forma $x = -z$, $y = 4z$ y $z = z$. El vector característico asociado al autovalor $\lambda_1 = 1$ es $v_1 = \begin{pmatrix} -1 \\ 4 \\ 1 \end{pmatrix}$.
    
    Para el autovalor $\lambda_2 = -2$, resolvemos el sistema homogéneo $(A - \lambda_2 Id_3)x = 0$:
    $$
    (A - \lambda_2 Id_3)x = 0
    $$
    $$
    \begin{pmatrix}
    3 & -1 & 4 \\
    3 & 4 & -1 \\
    2 & 1 & 1
    \end{pmatrix}
    \begin{pmatrix}
    x \\
    y \\
    z
    \end{pmatrix}
    = 0
    $$
    Reduzco la matriz a su forma escalonada reducida:
    $$
    \begin{pmatrix}
        3 & -1 & 4 \\
        3 & 4 & -1 \\
        2 & 1 & 1
    \end{pmatrix} \xrightarrow{F_1/3}
    \begin{pmatrix}
        1 & -1/3 & 4/3 \\
        3 & 4 & -1 \\
        2 & 1 & 1
    \end{pmatrix} \xrightarrow{F_2-3F_1}
    \begin{pmatrix}
        1 & -1/3 & 4/3 \\
        0 & 5 & -5 \\
        2 & 1 & 1
    \end{pmatrix} \xrightarrow{F_3-2F_1}
    \begin{pmatrix}
        1 & -1/3 & 4/3 \\
        0 & 5 & -5 \\
        0 & 5/3 & -5/3
    \end{pmatrix} \xrightarrow{F_2/5}
    $$
    $$
    \begin{pmatrix}
        1 & -1/3 & 4/3 \\
        0 & 1 & -1 \\
        0 & 5/3 & -5/3
    \end{pmatrix} \xrightarrow{F_3-5F_2/3}
    \begin{pmatrix}
        1 & -1/3 & 4/3 \\
        0 & 1 & -1 \\
        0 & 0 & 0
    \end{pmatrix} \xrightarrow{F_1+F_2/3}
    \begin{pmatrix}
        1 & 0 & 1 \\
        0 & 1 & -1 \\
        0 & 0 & 0
    \end{pmatrix}
    $$
    De acá salen las ecuaciones: $x+z = 0$ y $y-z = 0$. Entonces, $x=-z$ e $y=z$. Por lo tanto las soluciones son
    de la forma $x = -z$, $y = z$ y $z = z$. El vector característico asociado al autovalor $\lambda_2 = -2$ es $v_2 = \begin{pmatrix} -1 \\ 1 \\ 1 \end{pmatrix}$.

    Para el autovalor $\lambda_3 = 3$, resolvemos el sistema homogéneo $(A - \lambda_3 Id_3)x = 0$:
    $$
    (A - \lambda_3 Id_3)x = 0
    $$
    $$
    \begin{pmatrix}
    -2 & -1 & 4 \\
    3 & -1 & -1 \\
    2 & 1 & -4
    \end{pmatrix}
    \begin{pmatrix}
    x \\
    y \\
    z
    \end{pmatrix}
    = 0
    $$
    Reduzco la matriz a su forma escalonada reducida:
    $$
    \begin{pmatrix}
        -2 & -1 & 4 \\
        3 & -1 & -1 \\
        2 & 1 & -4
    \end{pmatrix} \xrightarrow{F_1/(-2)}
    \begin{pmatrix}
        1 & 1/2 & -2 \\
        3 & -1 & -1 \\
        2 & 1 & -4
    \end{pmatrix} \xrightarrow{F_2-3F_1}
    \begin{pmatrix}
        1 & 1/2 & -2 \\
        0 & -5/2 & 5 \\
        2 & 1 & -4
    \end{pmatrix} \xrightarrow{F_3-2F_1}
    \begin{pmatrix}
        1 & 1/2 & -2 \\
        0 & -5/2 & 5 \\
        0 & 0 & 0
    \end{pmatrix} \xrightarrow{F_2/(-5/2)}
    $$
    $$
    \begin{pmatrix}
        1 & 1/2 & -2 \\
        0 & 1 & -2 \\
        0 & 0 & 0
    \end{pmatrix} \xrightarrow{F_1-F_2/2}
    \begin{pmatrix}
        1 & 0 & -1 \\
        0 & 1 & -2 \\
        0 & 0 & 0
    \end{pmatrix}
    $$
    De acá salen las ecuaciones: $x-z = 0$ y $y-2z = 0$. Entonces, $x=z$ e $y=2z$. Por lo tanto las soluciones son
    de la forma $x = z$, $y = 2z$ y $z = z$. El vector característico asociado al autovalor $\lambda_3 = 3$ es $v_3 = \begin{pmatrix} 1 \\ 2 \\ 1 \end{pmatrix}$.
\end{solution}

\begin{problem}{4}
    Encuentre la matriz diagonal $D$ semejante a $A$ y la matriz $P$ cuyas columnas son los vectores característicos linealmente independientes de $A$.
\end{problem}
\begin{solution}
    La matriz diagonal $D$ semejante a $A$ está dada por
    $$
    \begin{pmatrix}
    1 & 0 & 0 \\
    0 & -2 & 0 \\
    0 & 0 & 3
    \end{pmatrix}
    $$
    donde $1$, $-2$ y $3$ son los autovalores de $A$. Y si $P$ es una matriz cuyas columnas son los vectores característicos linealmente independientes de $A$, entonces $P^{-1}AP = D$. Entonces, la matriz $P$ cuyas columnas son los vectores característicos linealmente independientes de $A$ es
    $$
    P = \begin{pmatrix}
    -1 & -1 & 1 \\
    4 & 1 & 2 \\
    1 & 1 & 1
    \end{pmatrix}
    $$
\end{solution}

    

%%%%%%%%%%%%%%%%%%%%%%%%%%%%%%%%%%%%%%%%%%%%%


\end{document}