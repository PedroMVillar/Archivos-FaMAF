%%%%%%%%%%%%%%%%%%%%%%%%%%%%%%%%%%%%%%%%%%%%%%%%%%%%%%%%%%%%%%%%%%%%%%%%%%%%%%%%%%%%
% Do not alter this block (unless you're familiar with LaTeX
\documentclass{article}
\usepackage[margin=1in]{geometry} 
\usepackage{amsmath,amsthm,amssymb,amsfonts, fancyhdr, color, comment, graphicx, environ}
\usepackage{xcolor}
\usepackage{mdframed}
\usepackage[shortlabels]{enumitem}
\usepackage{indentfirst}
\usepackage{hyperref}
\usepackage{background}
\hypersetup{
    colorlinks=true,
    linkcolor=blue,
    filecolor=magenta,      
    urlcolor=blue,
}
\setlength{\headheight}{1.5cm}


\pagestyle{fancy}

\SetBgContents{https://github.com/PedroMVillar}
\SetBgScale{3} % Escala de la marca de agua
\SetBgColor{gray!80} % Color de la marca de agua
\SetBgAngle{45} % Ángulo de inclinación de la marca de agua
\SetBgOpacity{0.2} % Opacidad de la marca de agua


\newenvironment{problem}[2][Ejercicio]
    { \begin{mdframed}[backgroundcolor=gray!20] \textbf{#1 #2} \\}
    {  \end{mdframed}}

\newenvironment{definition}[2][Definición]
    { \begin{mdframed}[backgroundcolor=red!20] \textbf{#1 #2} \\}
    {  \end{mdframed}}

\newenvironment{theorem}[2][Teorema]
    { \begin{mdframed}[backgroundcolor=green!20] \textbf{#1 #2} \\}
    {  \end{mdframed}}

\newenvironment{lemma}[2][Lema]
    { \begin{mdframed}[backgroundcolor=green!20] \textbf{#1 #2} \\}
    {  \end{mdframed}}

\newenvironment{proposition}[2][Proposición]
    { \begin{mdframed}[backgroundcolor=green!20] \textbf{#1 #2} \\}
    {  \end{mdframed}}

\newenvironment{corollary}[2][Corolario]
    { \begin{mdframed}[backgroundcolor=green!20] \textbf{#1 #2} \\}
    {  \end{mdframed}}

\newenvironment{example}[2][Ejemplo]
    { \begin{mdframed}[backgroundcolor=yellow!20] \textbf{#1 #2} \\}
    {  \end{mdframed}}

\newenvironment{remark}[2][Observación]
    { \begin{mdframed}[backgroundcolor=yellow!20] \textbf{#1 #2} \\}
    {  \end{mdframed}}

\newenvironment{note}
    {\textit{Nota:}}
    {}

\newenvironment{conclusion}
    {\textit{Conclusión:}}
    {}

\newenvironment{notation}
    {\textit{Notación:}}
    {}

\newenvironment{question}
    {\textit{Pregunta:}}
    {}

\newenvironment{answer}
    {\textit{Respuesta:}}
    {}

% Define solution environment
\newenvironment{solution}
    {\textit{Solución:}}
    {}

\renewcommand{\qed}{\quad\qedsymbol}

% prevent line break in inline mode
\binoppenalty=\maxdimen
\relpenalty=\maxdimen

%%%%%%%%%%%%%%%%%%%%%%%%%%%%%%%%%%%%%%%%%%%%%
%Header Configuarción
\lhead{Pedro Villar}
\rhead{Álgebra Lineal} 
\chead{\textbf{Cambio de base}}
%%%%%%%%%%%%%%%%%%%%%%%%%%%%%%%%%%%%%%%%%%%%%

\begin{document}

%%%%%%%%%%%%%%%%%%%%%%%%%%%%%%%%%%%%%%%%%%%%%
%Definiciones y teoremas
\section*{Definiciones y Teoremas}
\begin{corollary}{1}  
Sea $V$ un espacio vectorial de dimensión finita sobre el cuerpo $\mathbb{K}$, sean $\mathcal{B}$, $\mathcal{B}'$  bases ordenadas de $V$. Entonces
\[
[v]_{\mathcal{B}} = [Id]_{\mathcal{B}' \mathcal{B}}\, [v]_{\mathcal{B}'}, \quad \forall v \in V.
\]
\end{corollary}
\begin{definition}{1}
Sea $V$ un espacio vectorial de dimensión finita sobre el cuerpo $K$ y sean $\mathcal{B}$ y $\mathcal{B}'$ bases ordenadas de $V$. La matriz $P =[Id]_{\mathcal{B}' \mathcal{B}}$  es llamada la \emph{matriz de cambio de base} de la base $\mathcal{B}'$  a la base $\mathcal{B}$. Basicamente es una matriz que nos permite cambiar las coordenadas de un vector en una base por las coordenadas del mismo vector en otra base.
\end{definition}
%%%%%%%%%%%%%%%%%%%%%%%%%%%%%%%%%%%%%%%%%%%%%

%%%%%%%%%%%%%%%%%%%%%%%%%%%%%%%%%%%%%%%%%%%%%
%Métodos
\section*{Método para hallar la matriz de cambio de base}
Sea $V$ un espacio vectorial de dimensión finita sobre el cuerpo $\mathbb{K}$, sean $\mathcal{B}$, $\mathcal{B}'$  bases ordenadas de $V$, para hallar la matriz de cambio de base de $\mathcal{B}$ a $\mathcal{B}'$,
\begin{enumerate}
    \item Se construye la matriz $P = [Id]_{\mathcal{B} \mathcal{B}'}$ colocando como columnas cada uno de los vectores de la base $\mathcal{B}$ expresados en la base $\mathcal{B}'$. Es decir si $\mathcal{B} = \{ v_1,v_2,\dots, v_n \}$:
    $$
    P = [Id]_{\mathcal{B} \mathcal{B}'} = \begin{bmatrix}
        \vdots & \vdots & \vdots & \vdots \\ 
    [v_1]_{\mathcal{B}'} & [v_2]_{\mathcal{B}'} & \cdots & [v_n]_{\mathcal{B}'} \\
    \vdots & \vdots & \vdots & \vdots 
    \end{bmatrix}
    $$
    \item Para hacer uso de la matriz de cambio de base, se tiene que multiplicar la matriz $P$ por el vector $v$ expresado en la base $\mathcal{B}$, y dará como resultado el vector $v$ expresado en la base $\mathcal{B}'$. Es decir, si $v \in V$, entonces
    \[
    [v]_{\mathcal{B}'} = P\, [v]_{\mathcal{B}} = [Id]_{\mathcal{B} \mathcal{B}'}\, [v]_{\mathcal{B}}.
    \]
\end{enumerate}
Sea $V$ un espacio vectorial de dimensión finita sobre el cuerpo $\mathbb{K}$, sean $\mathcal{C}$ la base canónica de $V$, $\mathcal{B}$ una base ordenada de $V$, para hallar la matriz de cambio de base de $\mathcal{C}$ a $\mathcal{B}$,
\begin{enumerate}
    \item Se construye la matriz $P = [Id]_{\mathcal{B} \mathcal{C}}$ colocando como columnas cada uno de los vectores de la base $\mathcal{B}$ expresados en la base canónica. Es decir si $\mathcal{B} = \{ v_1,v_2,\dots, v_n \}$:
    $$
    P = [Id]_{\mathcal{B} \mathcal{C}} = \begin{bmatrix}
        \vdots & \vdots & \vdots & \vdots \\ 
    [v_1]_{\mathcal{C}} & [v_2]_{\mathcal{C}} & \cdots & [v_n]_{\mathcal{C}} \\
    \vdots & \vdots & \vdots & \vdots \end{bmatrix} =
    \begin{bmatrix}
        \vdots & \vdots & \vdots & \vdots \\ 
    v_1 & v_2 & \cdots & v_n \\
    \vdots & \vdots & \vdots & \vdots
    \end{bmatrix}
    $$
    \item Luego, como estamos trabajando con el operador lineal identidad, $P$ es invertible y la matriz de cambio de base de $\mathcal{C}$ a $\mathcal{B}$ es la matriz $P^{-1}$.
\end{enumerate}
%%%%%%%%%%%%%%%%%%%%%%%%%%%%%%%%%%%%%%%%%%%%%

%%%%%%%%%%%%%%%%%%%%%%%%%%%%%%%%%%%%%%%%%%%%%
%Ejemplos
\section*{Ejemplo}
\begin{problem}{1}
    Sea $\mathcal{B}= \{ (1,-2,1),(2,-3,3), (-2,2,-3) \}$ un subconjunto del $\mathbb{R}$-espacio vectorial $\mathbb{R}^3$, hallar la matriz de cambio de base de la base canónica $\mathcal{C}$ a $\mathcal{B}$ y con el resultado, obtener las coordenadas, respecto de $\mathcal{B}$, de los vectores $(1,0,1)$ y $(-1,2,1)$.
\end{problem}
\begin{solution}
    Para hallar la matriz de cambio de base de la base canónica $\mathcal{C}$ a $\mathcal{B}$ primero obtengo la matriz de cambio de base de $\mathcal{B}$ a $\mathcal{C}$:
    Se tiene que $[v]_{\mathcal{C}} = P_{\mathcal{B} \mathcal{C}} \cdot [v]_{\mathcal{B}}$. \\
    Primero debo expresar los vectores de $\mathcal{B}$ como combinacion lineal de los vectores de la base canónica para obtener las coordenadas:
    $$
    (1,-2,1) = 1(1,0,0)+(-2)(0,1,0)+1(0,0,1)
    $$
    $$
    (2,-3,3) = 2(1,0,0)+(-3)(0,1,0)+3(0,0,1)
    $$
    $$
    (-2,2,-3) = -2(1,0,0)+2(0,1,0)+(-3)(0,0,1)
    $$
    Entonces, la matriz de cambio de base de $\mathcal{B}$ a $\mathcal{C}$ es:
$$
P_{\mathcal{B} \mathcal{C}} = \begin{bmatrix}
    1 & 2 & -2 \\
    -2 & -3 & 2 \\
    1 & 3 & -3
\end{bmatrix}
$$
Ahora, para obtener la matriz de cambio de base de $\mathcal{C}$ a $\mathcal{B}$, debo invertir la matriz anterior, ya que lo que se busca es $[v]_{\mathcal{B}}$:
planteo la matriz ampliada:
$$
\left ( \left.\begin{matrix}
    1 & 2 & -2 \\ 
    -2 & -3 & 2 \\ 
    1 & 3 & -3
    \end{matrix}\right| \begin{matrix}
    1 & 0 & 0 \\ 
    0 & 1 & 0 \\ 
    0 & 0 & 1
    \end{matrix}\right )
\xrightarrow[]{f_2+2f_1}
\left ( \left.\begin{matrix}
    1 & 2 & -2 \\ 
    0 & 1 & -2 \\ 
    1 & 3 & -3
    \end{matrix}\right| \begin{matrix}
    1 & 0 & 0 \\ 
    2 & 1 & 0 \\ 
    0 & 0 & 1
    \end{matrix}\right )
\xrightarrow[]{f_3-f_1}
\left ( \left.\begin{matrix}
    1 & 2 & -2 \\ 
    0 & 1 & -2 \\ 
    0 & 1 & -1
    \end{matrix}\right| \begin{matrix}
    1 & 0 & 0 \\ 
    2 & 1 & 0 \\ 
    -1 & 0 & 1
    \end{matrix}\right )
$$
$$
\xrightarrow[]{f_3-f_2}
\left ( \left.\begin{matrix}
    1 & 2 & -2 \\ 
    0 & 1 & -2 \\ 
    0 & 0 & 1
    \end{matrix}\right| \begin{matrix}
    1 & 0 & 0 \\ 
    2 & 1 & 0 \\ 
    -3 & -1 & 1
    \end{matrix}\right )
\xrightarrow[]{f_2+2f_3}
\left ( \left.\begin{matrix}
    1 & 2 & -2 \\ 
    0 & 1 & 0 \\ 
    0 & 0 & 1
    \end{matrix}\right| \begin{matrix}
    1 & 0 & 0 \\ 
    -4 & -1 & 2 \\ 
    -3 & -1 & 1
    \end{matrix}\right )
$$
$$
\xrightarrow[]{f_1+2f_3}
\left ( \left.\begin{matrix}
    1 & 2 & 0 \\ 
    0 & 1 & 0 \\ 
    0 & 0 & 1
    \end{matrix}\right| \begin{matrix}
    -5 & -2 & 2 \\ 
    -4 & -1 & 2 \\ 
    -3 & -1 & 1
    \end{matrix}\right )
\xrightarrow[]{f_1-2f_2}
\left ( \left.\begin{matrix}
    1 & 0 & 0 \\ 
    0 & 1 & 0 \\ 
    0 & 0 & 1
    \end{matrix}\right| \begin{matrix}
    3 & 0 & -2 \\ 
    -4 & -1 & 2 \\ 
    -3 & -1 & 1
    \end{matrix}\right )
$$
Entonces, la matriz de cambio de base de $\mathcal{C}$ a $\mathcal{B}$ es:
$$
P_{\mathcal{C} \mathcal{B}} = \begin{bmatrix}
    3 & 0 & -2 \\ 
    -4 & -1 & 2 \\ 
    -3 & -1 & 1
\end{bmatrix}
$$
Para hallar las coordenadas de los vectores $(1,0,1)$ y $(-1,2,1)$ en la base $\mathcal{B}$, debo plantear la matriz de cambio de base de $\mathcal{C}$ a $\mathcal{B}$ y multiplicarla por el vector:
$$
[(1,0,1)]_{\mathcal{B}} = \begin{bmatrix}
    3 & 0 & -2 \\ 
    -4 & -1 & 2 \\ 
    -3 & -1 & 1
\end{bmatrix}
\begin{bmatrix}
    1 \\ 
    0 \\ 
    1
\end{bmatrix}
=
\begin{bmatrix}
    1 \\ 
    -2 \\ 
    -2
\end{bmatrix}
$$
$$
[(-1,2,1)]_{\mathcal{B}} = \begin{bmatrix}
    3 & 0 & -2 \\ 
    -4 & -1 & 2 \\ 
    -3 & -1 & 1
\end{bmatrix}
\begin{bmatrix}
    -1 \\ 
    2 \\ 
    1
\end{bmatrix}
=
\begin{bmatrix}
    -5 \\ 
    4 \\ 
    2
\end{bmatrix}
$$
\end{solution}
%%%%%%%%%%%%%%%%%%%%%%%%%%%%%%%%%%%%%%%%%%%%%


\end{document}