%%%%%%%%%%%%%%%%%%%%%%%%%%%%%%%%%%%%%%%%%%%%%%%%%%%%%%%%%%%%%%%%%%%%%%%%%%%%%%%%%%%%
% Do not alter this block (unless you're familiar with LaTeX
\documentclass{article}
\usepackage[margin=1in]{geometry} 
\usepackage{amsmath,amsthm,amssymb,amsfonts, fancyhdr, color, comment, graphicx, environ}
\usepackage{xcolor}
\usepackage{mdframed}
\usepackage[shortlabels]{enumitem}
\usepackage{indentfirst}
\usepackage{hyperref}
\usepackage{background}
\hypersetup{
    colorlinks=true,
    linkcolor=blue,
    filecolor=magenta,      
    urlcolor=blue,
}
\setlength{\headheight}{1.5cm}


\pagestyle{fancy}

\SetBgContents{https://github.com/PedroMVillar}
\SetBgScale{3} % Escala de la marca de agua
\SetBgColor{gray!80} % Color de la marca de agua
\SetBgAngle{45} % Ángulo de inclinación de la marca de agua
\SetBgOpacity{0.2} % Opacidad de la marca de agua

\newenvironment{problem}[2][Ejercicio]
    { \begin{mdframed}[backgroundcolor=gray!20] \textbf{#1 #2} \\}
    {  \end{mdframed}}

\newenvironment{definition}[2][Definición]
    { \begin{mdframed}[backgroundcolor=red!20] \textbf{#1 #2} \\}
    {  \end{mdframed}}

\newenvironment{theorem}[2][Teorema]
    { \begin{mdframed}[backgroundcolor=green!20] \textbf{#1 #2} \\}
    {  \end{mdframed}}

\newenvironment{lemma}[2][Lema]
    { \begin{mdframed}[backgroundcolor=green!20] \textbf{#1 #2} \\}
    {  \end{mdframed}}

\newenvironment{proposition}[2][Proposición]
    { \begin{mdframed}[backgroundcolor=green!20] \textbf{#1 #2} \\}
    {  \end{mdframed}}

\newenvironment{corollary}[2][Corolario]
    { \begin{mdframed}[backgroundcolor=green!20] \textbf{#1 #2} \\}
    {  \end{mdframed}}

\newenvironment{example}[2][Ejemplo]
    { \begin{mdframed}[backgroundcolor=yellow!20] \textbf{#1 #2} \\}
    {  \end{mdframed}}

\newenvironment{remark}[2][Observación]
    { \begin{mdframed}[backgroundcolor=yellow!20] \textbf{#1 #2} \\}
    {  \end{mdframed}}

\newenvironment{note}
    {\textit{Nota:}}
    {}

\newenvironment{conclusion}
    {\textit{Conclusión:}}
    {}

\newenvironment{notation}
    {\textit{Notación:}}
    {}

\newenvironment{question}
    {\textit{Pregunta:}}
    {}

\newenvironment{answer}
    {\textit{Respuesta:}}
    {}

% Define solution environment
\newenvironment{solution}
    {\textit{Solución:}}
    {}

\renewcommand{\qed}{\quad\qedsymbol}

% prevent line break in inline mode
\binoppenalty=\maxdimen
\relpenalty=\maxdimen

%%%%%%%%%%%%%%%%%%%%%%%%%%%%%%%%%%%%%%%%%%%%%
%Header Configuarción
\lhead{Pedro Villar}
\rhead{Álgebra Lineal} 
\chead{\textbf{Examen 1}}
%%%%%%%%%%%%%%%%%%%%%%%%%%%%%%%%%%%%%%%%%%%%%

\begin{document}



%%%%%%%%%%%%%%%%%%%%%%%%%%%%%%%%%%%%%%%%%%%%%
%Preguntas
\begin{problem}{1}
    (Consejos: usar contraejemplos para probar falsedades)
    \begin{enumerate}
        \item[(a)] (\textit{5pts.}) Demostrar la falsedad de la afirmación: Si $A$ es una matriz de $2\times 2$ tal que $A^2=0$, entonces $A=0$.
        \item[(b)] (\textit{5pts.}) Sea $A$ una matriz sobre el cuerpo $\mathbb{K}$. Demuestre que si $A$ es invertible, entonces $A^{-1}$ es invertible.
        \item[(c)] (\textit{5pts.}) Probar que la suma de matrices triangulares superiores es una matriz triangular superior.
        \item[(d)] (\textit{5pts.}) Dar la definición de matriz diagonal.
    \end{enumerate} 
\end{problem}

\begin{problem}{2}
    (\textit{10pts.}) Hallar todas las matrices $A$ de la forma $\begin{pmatrix}a & 1 & 0 \\ 0 & b & 1 \\ 0 & 0 & c \end{pmatrix}$ tales que $A^2=\begin{pmatrix} 1 & 0 & 1 \\ 0 & 1 & 0 \\ 0 & 0 & 1 \end{pmatrix}$
\end{problem}

\begin{problem}{3}
    Sean $a,b\in \mathbb{R}$
    \begin{enumerate}
        \item[(a)] (\textit{5pts.}) Describir de forma implícita al plano $P$ que pasa por $(1,-3,2)$ y es paralelo a los vectores $(2,1,0)$ y $(-1,0,3)$.
        \item[(b)] (\textit{5pts.}) Hallar la ecuación normal del plano que pasa por el origen y es perpendicular a $P$.
        \item[(c)] (\textit{5pts.}) Para que valores de $(a,b)$ el plano $P$ es paralelo al plano descripto de forma implícita como $x+ay+bz=1$.
    \end{enumerate}
\end{problem}

\begin{problem}{4}
    (\textit{20pts.}) Considere las siguientes bases ordenadas de $\mathbb{R}_3[x]$.
    \begin{align*}
        \mathcal{B}_1 &= \{ x^2,x,1 \} \\
        \mathcal{B}_2 &= \{ x^2+2x+3, \ 4x^2+9x+12, \ 7x^2+14x+20 \}
    \end{align*}
    \begin{enumerate}
        \item[(a)] Calcular la matriz de cambio de base de $\mathcal{B}_1$ a la base canónica de $\mathcal{B}_2$.
        \item[(b)] Calcular la matriz de cambio de base de $\mathcal{B}_2$ a la base canónica de $\mathcal{B}_1$.
        \item[(c)] Dar el polinomio $p(x)$ tal que $[p(x)]_{\mathcal{B}_2} = \begin{pmatrix} 4 \\ -1 \\ -1 \end{pmatrix}$.
    \end{enumerate}
\end{problem}

\begin{problem}{5}
    Sea $T:\mathbb{R}^3 \to \mathbb{R}^3$ la transformación lineal dada por
    $$
    T(x,y,z) = (x-y+2z, 3x+y+4z, 5x-y+8z)
    $$
    \begin{enumerate}
        \item[(a)] (\textit{10pts.}) Dar bases del núcleo y la imagen de $T$.
        \item[(b)] (\textit{5pts.}) Decidir si la $T$ es un isomorfismo.
        \item[(b)] (\textit{5pts.}) Hallar la matriz asociada a $T$ y decir si es diagonizable.
    \end{enumerate}
\end{problem}

\begin{problem}{6}
    (\textit{10pts.}) Considere el siguiente subespacio de $\mathbb{R}_3[x]$:
    $$
    W = \{ p(x) \in \mathbb{R}_3[x] \ | \ p(1) = p'(1) \}
    $$
    Dar la dimensión de $W$ y una base de $W$.
\end{problem}

\begin{problem}{7}
    (\textit{5pts.}) Sea $T:\mathbb{R}^2 \to \mathbb{R}^2$ una transformación lineal, y sea $A=\begin{bmatrix} 2 & 0 \\ 3 & 7 \end{bmatrix}$ la matriz asociada a $T$ con respecto de las bases $\mathcal{B}=\mathcal{B'}= \{(1,2),(0,1) \}$. Dar una definición de $T$.
\end{problem}

%%%%%%%%%%%%%%%%%%%%%%%%%%%%%%%%%%%%%%%%%%%%%


\end{document}