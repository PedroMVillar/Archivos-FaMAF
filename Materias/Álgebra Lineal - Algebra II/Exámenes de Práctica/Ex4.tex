%%%%%%%%%%%%%%%%%%%%%%%%%%%%%%%%%%%%%%%%%%%%%%%%%%%%%%%%%%%%%%%%%%%%%%%%%%%%%%%%%%%%
% Do not alter this block (unless you're familiar with LaTeX
\documentclass{article}
\usepackage[margin=1in]{geometry} 
\usepackage{amsmath,amsthm,amssymb,amsfonts, fancyhdr, color, comment, graphicx, environ}
\usepackage{xcolor}
\usepackage{mdframed}
\usepackage[shortlabels]{enumitem}
\usepackage{indentfirst}
\usepackage{hyperref}
\usepackage{background}
\hypersetup{
    colorlinks=true,
    linkcolor=blue,
    filecolor=magenta,      
    urlcolor=blue,
}
\setlength{\headheight}{1.5cm}


\pagestyle{fancy}

\SetBgContents{https://github.com/PedroMVillar}
\SetBgScale{3} % Escala de la marca de agua
\SetBgColor{gray!80} % Color de la marca de agua
\SetBgAngle{45} % Ángulo de inclinación de la marca de agua
\SetBgOpacity{0.2} % Opacidad de la marca de agua


\newenvironment{problem}[2][Ejercicio]
    { \begin{mdframed}[backgroundcolor=gray!20] \textbf{#1 #2} \\}
    {  \end{mdframed}}

\newenvironment{definition}[2][Definición]
    { \begin{mdframed}[backgroundcolor=red!20] \textbf{#1 #2} \\}
    {  \end{mdframed}}

\newenvironment{theorem}[2][Teorema]
    { \begin{mdframed}[backgroundcolor=green!20] \textbf{#1 #2} \\}
    {  \end{mdframed}}

\newenvironment{lemma}[2][Lema]
    { \begin{mdframed}[backgroundcolor=green!20] \textbf{#1 #2} \\}
    {  \end{mdframed}}

\newenvironment{proposition}[2][Proposición]
    { \begin{mdframed}[backgroundcolor=green!20] \textbf{#1 #2} \\}
    {  \end{mdframed}}

\newenvironment{corollary}[2][Corolario]
    { \begin{mdframed}[backgroundcolor=green!20] \textbf{#1 #2} \\}
    {  \end{mdframed}}

\newenvironment{example}[2][Ejemplo]
    { \begin{mdframed}[backgroundcolor=yellow!20] \textbf{#1 #2} \\}
    {  \end{mdframed}}

\newenvironment{remark}[2][Observación]
    { \begin{mdframed}[backgroundcolor=yellow!20] \textbf{#1 #2} \\}
    {  \end{mdframed}}

\newenvironment{note}
    {\textit{Nota:}}
    {}

\newenvironment{conclusion}
    {\textit{Conclusión:}}
    {}

\newenvironment{notation}
    {\textit{Notación:}}
    {}

\newenvironment{question}
    {\textit{Pregunta:}}
    {}

\newenvironment{answer}
    {\textit{Respuesta:}}
    {}

% Define solution environment
\newenvironment{solution}
    {\textit{Solución:}}
    {}

\renewcommand{\qed}{\quad\qedsymbol}

% prevent line break in inline mode
\binoppenalty=\maxdimen
\relpenalty=\maxdimen

%%%%%%%%%%%%%%%%%%%%%%%%%%%%%%%%%%%%%%%%%%%%%
%Header Configuarción
\lhead{Pedro Villar}
\rhead{Álgebra Lineal} 
\chead{\textbf{Examen 4}}
%%%%%%%%%%%%%%%%%%%%%%%%%%%%%%%%%%%%%%%%%%%%%

\begin{document}



%%%%%%%%%%%%%%%%%%%%%%%%%%%%%%%%%%%%%%%%%%%%%
%Preguntas

\begin{problem}{1}
    Justificar cada respuesta.
    \begin{itemize}
        \item[(a)] (\textit{5pts.}) ¿Qué es un monomorfismo?
        \item[(b)] (\textit{5pts.}) ¿Qué son las coordenadas de un vector con respecto a una base? 
        \item[(c)] (\textit{5pts.}) ¿Qué es una transformación lineal?
        \item[(d)] (\textit{10pts.}) ¿Qué es el núcleo de una transformación lineal?  
    \end{itemize}
\end{problem}

\begin{problem}{2}
    (\textit{10pts.}) Sea $A= \begin{pmatrix} 3 & 2 & 1 \\ -3 & -2 & t \\ 6 & 4 & 2 \end{pmatrix}$ con $t \in \mathbb{R}$, hallar el valor de $t$ para el cual $A$ posee un autovalor igual a $2$.
\end{problem}

\begin{problem}{3}
    (\textit{15pts.}) Hallar todas las ecuaciones del plano que pasa por $(-1,2,3)$ y $(2,1,1)$ y es perpendicular al plano $\pi_1$ descripto implícitamente por la ecuación $2x-y+3z=10$.
\end{problem}

\begin{problem}{4}
    Sea $T: M_{2 \times 2} \to M_{2 \times 2}$ la transformación lineal dada por $T(A)=A+A^t$.
    \begin{itemize}
        \item[(a)] (\textit{5pts.}) Hallar el polinomio característico de $T$.
        \item[(b)] (\textit{5pts.}) Hallar sus autovalores y autovectores.
        \item[(c)] (\textit{10pts.}) ¿Es $T$ diagonalizable? Justificar.   
    \end{itemize} 
\end{problem}

\begin{problem}{5}
    Sea $T: \mathbb{R}^3 \to \mathbb{R}^3$ la transformación lineal tal que
    $$
    T(e_1) = e_1+e_2+e_3, \quad T(e_2) = -2e_1+2e_3, \quad T(e_3) = 5e_2+10e_3
    $$
    donde $\mathcal{B} = \{e_1, e_2, e_3\}$ es la base canónica de $\mathbb{R}^3$.
    \begin{itemize}
        \item[(a)] (\textit{5pts.}) Dar una base del núcleo de $T$.
        \item[(b)] (\textit{5pts.}) Dar una base de la imagen de $T$.
        \item[(c)] (\textit{10pts.}) Dar la matriz de $T$ con respecto a las bases ordenadas $\mathcal{B}$ y $\mathcal{B}'=\{ 2e_3,-e_1,3e_2 \}$.  
    \end{itemize}
\end{problem}

\begin{problem}{6}
    Sea $T:\mathbb{R}^3 \to M_{3\times 3}$ la transformación lineal dada por $T(x,y,z) = \begin{pmatrix} 3y-z & 2z \\ x-y & y \end{pmatrix}$.
    \begin{itemize}
        \item[(a)] (\textit{10pts.}) Hallar una base de $Im(T)$ y decidir si $\begin{pmatrix} 1 & 2 \\ 3 & 4 \end{pmatrix}$ pertenece a $Im(T)$.
        \item[(b)] (\textit{15pts.}) ¿Es $T$ inyectiva? Justificar. 
    \end{itemize}
\end{problem}
%%%%%%%%%%%%%%%%%%%%%%%%%%%%%%%%%%%%%%%%%%%%%


\end{document}