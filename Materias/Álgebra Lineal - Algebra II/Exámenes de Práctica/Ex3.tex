%%%%%%%%%%%%%%%%%%%%%%%%%%%%%%%%%%%%%%%%%%%%%%%%%%%%%%%%%%%%%%%%%%%%%%%%%%%%%%%%%%%%
% Do not alter this block (unless you're familiar with LaTeX
\documentclass{article}
\usepackage[margin=1in]{geometry} 
\usepackage{amsmath,amsthm,amssymb,amsfonts, fancyhdr, color, comment, graphicx, environ}
\usepackage{xcolor}
\usepackage{mdframed}
\usepackage[shortlabels]{enumitem}
\usepackage{indentfirst}
\usepackage{hyperref}
\usepackage{background}
\hypersetup{
    colorlinks=true,
    linkcolor=blue,
    filecolor=magenta,      
    urlcolor=blue,
}
\setlength{\headheight}{1.5cm}


\pagestyle{fancy}

\SetBgContents{https://github.com/PedroMVillar}
\SetBgScale{3} % Escala de la marca de agua
\SetBgColor{gray!80} % Color de la marca de agua
\SetBgAngle{45} % Ángulo de inclinación de la marca de agua
\SetBgOpacity{0.2} % Opacidad de la marca de agua


\newenvironment{problem}[2][Ejercicio]
    { \begin{mdframed}[backgroundcolor=gray!20] \textbf{#1 #2} \\}
    {  \end{mdframed}}

\newenvironment{definition}[2][Definición]
    { \begin{mdframed}[backgroundcolor=red!20] \textbf{#1 #2} \\}
    {  \end{mdframed}}

\newenvironment{theorem}[2][Teorema]
    { \begin{mdframed}[backgroundcolor=green!20] \textbf{#1 #2} \\}
    {  \end{mdframed}}

\newenvironment{lemma}[2][Lema]
    { \begin{mdframed}[backgroundcolor=green!20] \textbf{#1 #2} \\}
    {  \end{mdframed}}

\newenvironment{proposition}[2][Proposición]
    { \begin{mdframed}[backgroundcolor=green!20] \textbf{#1 #2} \\}
    {  \end{mdframed}}

\newenvironment{corollary}[2][Corolario]
    { \begin{mdframed}[backgroundcolor=green!20] \textbf{#1 #2} \\}
    {  \end{mdframed}}

\newenvironment{example}[2][Ejemplo]
    { \begin{mdframed}[backgroundcolor=yellow!20] \textbf{#1 #2} \\}
    {  \end{mdframed}}

\newenvironment{remark}[2][Observación]
    { \begin{mdframed}[backgroundcolor=yellow!20] \textbf{#1 #2} \\}
    {  \end{mdframed}}

\newenvironment{note}
    {\textit{Nota:}}
    {}

\newenvironment{conclusion}
    {\textit{Conclusión:}}
    {}

\newenvironment{notation}
    {\textit{Notación:}}
    {}

\newenvironment{question}
    {\textit{Pregunta:}}
    {}

\newenvironment{answer}
    {\textit{Respuesta:}}
    {}

% Define solution environment
\newenvironment{solution}
    {\textit{Solución:}}
    {}

\renewcommand{\qed}{\quad\qedsymbol}

% prevent line break in inline mode
\binoppenalty=\maxdimen
\relpenalty=\maxdimen

%%%%%%%%%%%%%%%%%%%%%%%%%%%%%%%%%%%%%%%%%%%%%
%Header Configuarción
\lhead{Pedro Villar}
\rhead{Álgebra Lineal} 
\chead{\textbf{Examen 3}}
%%%%%%%%%%%%%%%%%%%%%%%%%%%%%%%%%%%%%%%%%%%%%

\begin{document}



%%%%%%%%%%%%%%%%%%%%%%%%%%%%%%%%%%%%%%%%%%%%%
%Preguntas

\begin{problem}{1}
    Justificar apropiadamente.
    \begin{itemize}
        \item[(a)] (\textit{5pts.}) Dar la definición de suma directa de subespacios vectoriales.
        \item[(b)] (\textit{5pts.}) Cuales son las condiciones necesarias para que un conjunto sea un subconjunto de un espacio vectorial. 
    \end{itemize}
\end{problem}

\begin{problem}{2}
    (\textit{15pts.}) Sean $A,B,C \in M:{5 \times 5}$ tales que $det(A)=2$, $det(B)=3$ y $det(C)=4$. Calcular $det(PQR)$ donde $P,Q,R$ son las matrices que se obtienen a partir de $A,B,C$ mediante operaciones elementales por filas de la siguiente manera
    \begin{enumerate}
        \item $P$ se obtiene de $A$ sumando a la fila 1 la fila 2 multiplicada por 3.
        \item $Q$ se obtiene de $B$ multiplicando la fila 4 por 2.
        \item $R$ se obtiene de $C$ intercambiando la fila 1 con la fila 3.
    \end{enumerate}
\end{problem}

\begin{problem}{3}
    (\textit{15pts.}) Hallar la representación paramétrica del plano que pasa por el origen y por el punto $(1,2,-1)$ y tal que uno de sus vectores direccionales es $(2,3,0)$.
\end{problem}

\begin{problem}{4}
    (\textit{20pts.}) Para la siguiente matriz $A$, determine sus valores propios y sus vectores propios y determine la matriz que diagonaliza a $A$.
    $$
    A = \begin{pmatrix}
        7 & -2 & -4 \\
        3 & 0 & -2 \\
        6 & -2 & -3
    \end{pmatrix}
    $$
\end{problem}

\begin{problem}{5}
    Decidir si las siguientes afirmaciones son verdaderas o falsas:
    \begin{enumerate}
        \item[(a)] (\textit{5pts.}) Todo conjunto que contenga al vector nulo, es linealmente dependiente.
        \item[(b)] (\textit{5pts.}) Si $T: \mathbb{R}^5 \to \mathbb{R}^3$ es una transformación lineal, entonces $dim(Nu(T)) \geq 3$.
        \item[(c)] (\textit{5pts.}) Existe una transformación lineal $T: \mathbb{R}^4 \to \mathbb{R}^4$ tal que $dim(Im(T))=3$ y $dim(Nu(T))=3$.
    \end{enumerate}
\end{problem}

\begin{problem}{6}
    Considere la transformaciónlineal $T: \mathbb{R}_4[x] \to M_{2\times 2}$ definida como
    $$
    T(a+bx+cx^2+dx^3) = \begin{pmatrix} a+c+2d & 2a+b+c+4d \\ 2a+b+c+3d & a+b+2d \end{pmatrix}
    $$
    \begin{itemize}
        \item[(a)] (\textit{10pts.}) Determinar el nuceo e imagen de $T$.
        \item[(b)] (\textit{5pts.}) Encontrar la matriz asociada a $T$.
        \item[(c)] (\textit{10pts.}) Decidir si $T$ es diagonizable, y si es así dar la matriz diagonal.
    \end{itemize}
\end{problem}
%%%%%%%%%%%%%%%%%%%%%%%%%%%%%%%%%%%%%%%%%%%%%


\end{document}