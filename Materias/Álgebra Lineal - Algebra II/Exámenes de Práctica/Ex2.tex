%%%%%%%%%%%%%%%%%%%%%%%%%%%%%%%%%%%%%%%%%%%%%%%%%%%%%%%%%%%%%%%%%%%%%%%%%%%%%%%%%%%%
% Do not alter this block (unless you're familiar with LaTeX
\documentclass{article}
\usepackage[margin=1in]{geometry} 
\usepackage{amsmath,amsthm,amssymb,amsfonts, fancyhdr, color, comment, graphicx, environ}
\usepackage{xcolor}
\usepackage{mdframed}
\usepackage[shortlabels]{enumitem}
\usepackage{indentfirst}
\usepackage{hyperref}
\usepackage{background}
\hypersetup{
    colorlinks=true,
    linkcolor=blue,
    filecolor=magenta,      
    urlcolor=blue,
}
\setlength{\headheight}{1.5cm}


\pagestyle{fancy}

\SetBgContents{https://github.com/PedroMVillar}
\SetBgScale{3} % Escala de la marca de agua
\SetBgColor{gray!80} % Color de la marca de agua
\SetBgAngle{45} % Ángulo de inclinación de la marca de agua
\SetBgOpacity{0.2} % Opacidad de la marca de agua


\newenvironment{problem}[2][Ejercicio]
    { \begin{mdframed}[backgroundcolor=gray!20] \textbf{#1 #2} \\}
    {  \end{mdframed}}

\newenvironment{definition}[2][Definición]
    { \begin{mdframed}[backgroundcolor=red!20] \textbf{#1 #2} \\}
    {  \end{mdframed}}

\newenvironment{theorem}[2][Teorema]
    { \begin{mdframed}[backgroundcolor=green!20] \textbf{#1 #2} \\}
    {  \end{mdframed}}

\newenvironment{lemma}[2][Lema]
    { \begin{mdframed}[backgroundcolor=green!20] \textbf{#1 #2} \\}
    {  \end{mdframed}}

\newenvironment{proposition}[2][Proposición]
    { \begin{mdframed}[backgroundcolor=green!20] \textbf{#1 #2} \\}
    {  \end{mdframed}}

\newenvironment{corollary}[2][Corolario]
    { \begin{mdframed}[backgroundcolor=green!20] \textbf{#1 #2} \\}
    {  \end{mdframed}}

\newenvironment{example}[2][Ejemplo]
    { \begin{mdframed}[backgroundcolor=yellow!20] \textbf{#1 #2} \\}
    {  \end{mdframed}}

\newenvironment{remark}[2][Observación]
    { \begin{mdframed}[backgroundcolor=yellow!20] \textbf{#1 #2} \\}
    {  \end{mdframed}}

\newenvironment{note}
    {\textit{Nota:}}
    {}

\newenvironment{conclusion}
    {\textit{Conclusión:}}
    {}

\newenvironment{notation}
    {\textit{Notación:}}
    {}

\newenvironment{question}
    {\textit{Pregunta:}}
    {}

\newenvironment{answer}
    {\textit{Respuesta:}}
    {}

% Define solution environment
\newenvironment{solution}
    {\textit{Solución:}}
    {}

\renewcommand{\qed}{\quad\qedsymbol}

% prevent line break in inline mode
\binoppenalty=\maxdimen
\relpenalty=\maxdimen

%%%%%%%%%%%%%%%%%%%%%%%%%%%%%%%%%%%%%%%%%%%%%
%Header Configuarción
\lhead{Pedro Villar}
\rhead{Álgebra Lineal} 
\chead{\textbf{Examen 2}}
%%%%%%%%%%%%%%%%%%%%%%%%%%%%%%%%%%%%%%%%%%%%%

\begin{document}



%%%%%%%%%%%%%%%%%%%%%%%%%%%%%%%%%%%%%%%%%%%%%
%Preguntas
\begin{problem}{1}
    Justificar apropiadamente.
    \begin{enumerate}
        \item[(a)] (\textit{5pts.}) Sea $A$ una matriz $n \times n$. Probar que si $A$ tiene dos filas iguales o una nula, entonces $det(A)=0$.
        \item[(b)] (\textit{5pts.}) Probar la falsedad o no de la siguiente afirmación: si $A$ y $B$ son matrices invertibles, entonces $A+B$ es invertible.
        \item[(c)] (\textit{5pts.}) Sean $A,B$ y $C$ matrices $n\times n $ tales que $det(A)=-1$, $det(B)=2$ y $det(C)=3$. Calcular $det(A^2BC^tB^{-1})$.  
        \item[(d)] (\textit{5pts.}) Dar la Definición del núcleo de una transformación lineal.
    \end{enumerate}
\end{problem}

\begin{problem}{2}
    (\textit{20pts.}) Sean $a,b,c \in \mathbb{R}$. Consideremos las siguientes matrices:
    $$
    A = 
    \begin{pmatrix}
        1 & -1 & 0 & 0 & 0 \\
        1 & 3 & 0 & 0 & 0 \\
        0 & 0 & a & b & c \\
        0 & 0 & -2 & 0 & 1 \\
        0 & 0 & 1 & 1 & 1 
    \end{pmatrix} \ \ \ \ \ 
    B =
    \begin{pmatrix}
        c & -1 & 0 & 0 & 0 \\
        1 & 3 & 0 & 0 & 0 \\
        0 & 0 & a & b & 1 \\
        0 & 0 & -2 & 0 & 1 \\
        0 & 0 & 1 & 1 & 1
    \end{pmatrix}
    $$
    Calcular $det(A)$ y $det(B)$ en función de $a,b$ y $c$.
\end{problem}

\begin{problem}{3}
    (\textit{20pts.}) Decidir si existe y si es así darlo, un vector ortogonal a $(1,2,2)$, $(0,1,0)$ y $(0,0,1)$.  
\end{problem}

\begin{problem}{4}
    (\textit{10pts.}) Sea $T: M_{2\times 3} \to \mathbb{R}^7$ una transformación lineal tal que su núcleo es generado por las matrices
    $$
    \begin{pmatrix}
        1 & 0 & 2  \\ 0 & 3 & 0 
    \end{pmatrix}, \ \ \ \ 
    \begin{pmatrix}
        0 & 1 & 0  \\ 2 & 0 & 3 
    \end{pmatrix}, \ \ \ \
    \begin{pmatrix}
        1 & 1 & 2  \\ 2 & 3 & 3 
    \end{pmatrix}
    $$
    \begin{itemize}
        \item[(a)] (\textit{5pts.}) Determinar la dimensión de la imagen de $T$.  
    \end{itemize}
\end{problem}

\begin{problem}{5}
    Sea $T: \mathbb{R}^2 \to \mathbb{R}^2$ la transformación lineal definida como
    $$
    T(x,y) = (4x-y,3x+2y)
    $$
    \begin{itemize}
        \item[(a)] (\textit{10pts.}) Dar $[T]_{\mathcal{BB'}}$ con $\mathcal{B}\mathcal{B'}= \{ (-1,1),(4,3) \}$.
        \item[(b)] (\textit{5pts.}) Determinar el núcleo e imagen de $T$.
    \end{itemize}
\end{problem}

\newpage
\begin{problem}{6}
    Considere la transformaciónlineal $T: \mathbb{R}_2[x] \to \mathbb{R}_3[x]$ definida como
    $$
    T(p(x)) = x \cdot p(x) + p(0)
    $$
    \begin{itemize}
        \item[(a)] (\textit{10pts.}) Determinar el nuceo e imagen de $T$.
        \item[(b)] (\textit{5pts.}) Decidir si $T$ es inyectiva y/o sobreyectiva.
        \item[(c)] (\textit{10pts.}) Decidir si $T$ es diagonizable. 
    \end{itemize}
\end{problem}
%%%%%%%%%%%%%%%%%%%%%%%%%%%%%%%%%%%%%%%%%%%%%


\end{document}