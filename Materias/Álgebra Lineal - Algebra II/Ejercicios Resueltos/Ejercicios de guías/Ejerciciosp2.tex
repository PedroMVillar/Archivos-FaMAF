\documentclass[a4paper,12pt]{article}
\author{}
\date{}
\usepackage[papersize={216mm,330mm},tmargin=20mm,bmargin=20mm,lmargin=20mm,rmargin=20mm]{geometry}
\usepackage[english]{babel}
\usepackage[utf8]{inputenc}
\usepackage{amsmath,amssymb,mathabx}%\for eqref
\usepackage{lscape}
\usepackage{graphicx}
\usepackage[colorinlistoftodos]{todonotes}
\usepackage{fancyhdr}
\usepackage{tcolorbox}
\newtcolorbox{defbox}[1]{colback=red!5!white,colframe=red!75!black,fonttitle=\bfseries,title=#1}
\newtcolorbox{ejbox}[1]{colback=blue!5!white,colframe=blue!75!black,fonttitle=\bfseries,title=#1}
\newtcolorbox{teobox}[1]{colback=gray!5!white,colframe=black!75!black,fonttitle=\bfseries,title=#1}
\newtcolorbox{obbox}[1]{colback=gray!5!yellow,colframe=gray!75!black,fonttitle=\bfseries,title=#1}
\newtcolorbox{bbox}{colback=gray!5!white,colframe=black!75!black}
\usepackage{dsfont}
\usepackage{hyperref}
\pagestyle{fancy}
\fancyhf{}
\chead{\thepage}

\title{\Large \textbf{Ejercicios Resueltos - Segunda Parte} \\ Álgebra Lineal \\ FaMAF, segundo cuatrimestre 2023 }

\begin{document}
\maketitle
\tableofcontents
\newpage
% ---------- PRÁCTICO 7 ---------- %
\section{Práctico 7: Espacios vectoriales}
\subsection{Ejercicio 1}
Decidir si los siguientes conjuntos son R-espacios vectoriales, con las operaciones abajo definidas.
\begin{enumerate}
    \item $\mathds{R}^n$, con $v \oplus w= v - w$, y el producto por escalares usual.
    \item $\mathds{R}^2$, con $(x,y)\oplus(x_1,y_2)=(x+x_1,0)$, $c\odot(x,y)=(cx,0)$.
\end{enumerate}
\subsubsection{Conceptos}
\begin{teobox}{Teorema}
    Las propiedades del espacio vectorial son:
 	\begin{itemize}
 		\item[S1:] $v+w=w+v$
 		\item[S2:] $(v+w)+u=v+(w+u)$ 
 		\item[S3:] $0+v=v+0=v$
 		\item[S4:] $v+(-v)=0$
 		\item[P1:] $1\cdot v=v$
 		\item[P2:] $\lambda_1(\lambda_2v)=(\lambda_1\lambda_2)v$
 		\item[D1:] $\lambda(x+w)=\lambda x + \lambda w$
 		\item[D2:] $(\lambda_1+\lambda_2)x=\lambda_1x+\lambda_2x$
 	\end{itemize}
\end{teobox}
\subsubsection{Prueba 1}
Hay que probar que se cumplen todas las propiedades: \newline
Sea $\mathds{R}^n$ un $\mathds{R}$-espacio, $v$ una n-tupla $(x_1,x_2,..,x_n)$ y w otra n-tupla $(y_1,y_2,...,y_n)$, se definió la suma como:
$$v+w=(x_1-y_1,x_2-y_2,...,x_n-y_n)$$
Se puede ver que \textbf{S1} no se cumple:
$$v+w=(x_1-y_1,x_2-y_2,...,x_n-y_n)\neq w+v=(y_1-x_1,y_2-x_2,...,y_n-x_n)$$
\begin{center}
 	\textit{ya que la resta no es asociativa en $\mathds{R}$.}
\end{center}
Por lo tanto, \textbf{no es un $\mathds{R}$-espacio}.
\subsubsection{Prueba 2}
Se puede ver que no cumple \textbf{P1}, es decir no tiene neutro del producto escalar, cualquier número escalar multiplicado por un par ordenado con dichas propiedades devuelve otro producto cuya coordenada $y$ es 0, por lo tanto no tiene posibilidades de cumplir la propiedad. Entonces \textbf{no es un  $\mathds{R}$-espacio}.
\subsection{Ejercicio 2}
Demostrar que el conjunto de números reales positivos $\mathds{R}_{>0}=\{ x \in \mathds{R} : x>0 \} $ es un $\mathds{R}$-espacio vectorial con las operaciones $x\oplus y = x\cdot y $ y $\lambda \odot x = x^\lambda$
\newline Hay que probar que el conjunto cumple con las propiedades de un espacio vectorial:
\begin{itemize}
    \item[S1:] 
    $$
    x\oplus y = x\cdot y = y \cdot x = y \oplus x
    $$
    \item[S2:]
    $$
    (x \oplus y) \oplus u = (x \oplus y)\cdot u = (x\cdot y) \cdot u = x\cdot (y \cdot u) = x \oplus (y \cdot u) = x \oplus (y \oplus u)
    $$
    \item[S3:] 
    $$
    1 \oplus x = 1\cdot x = x
    $$
    \begin{center}
        \textit{Existe $1\in\mathds{R}_{>0}$ elemento neutro de $\oplus$}
    \end{center}
    \item[S4:]
    $$
    \frac{1}{x} \oplus x = \frac{1}{x} \cdot x = 1
    $$
    \begin{center}
        \textit{Existe $\frac{1}{x}\in\mathds{R}_{>0}$ elemento opuesto de $\oplus$}
    \end{center}
    \item[P1:]
    $$
    1 \odot x = x^1= x
    $$
    \begin{center}
        \textit{Existe $1\in\mathds{R}_{>0}$ elemento neutro de $\odot$}
    \end{center}
    \item[P2:]
    $$
    \lambda_1\odot(\lambda_2\odot x) = \lambda_1 \odot x^{\lambda_2}= x^{\lambda_1 \cdot \lambda_2}= (\lambda_1 \cdot \lambda_2) \odot x
    $$
    \item[D1:]
    $$
    \lambda \odot (x \oplus y) = \lambda \odot (x\cdot y) = (x\cdot y)^{\lambda} = x^{\lambda} \cdot y^{\lambda} = (\lambda \odot  x) \oplus (\lambda \odot y)
    $$
    \item[D2:]
    $$
    (\lambda_1 \oplus \lambda_2) \odot x = x^{\lambda_1 \oplus \lambda_2} = x^{\lambda_1}\oplus x^{\lambda_2} = (\lambda_1 \odot x) \oplus (\lambda_2 \odot x)
    $$
\end{itemize}
Cumple con todas las propiedades, entonces \textbf{es un espacio vectorial.}
\subsection{Ejercicio 3}
Sea $\mathbb{K}$ un cuerpo. Si $(V,\oplus, \odot)$ es un $\mathbb{K}$-espacio vectorial y $S$ un conjunto cualquiera entonces:
$$
V^S=\{ f: S \rightarrow V : f\text{ es una función}\},
$$
denota al conjunto de todas las funciones de $S$ en $V$. Definimos en $V^S$ la suma y el producto por escalares de la siguiente manera: Si $f,g \in V^S$ y $c\in \mathbb{K}$ entonces $f+g:S \rightarrow V$ y $c \cdot f : S \rightarrowtail V$ están dadas por:
$$
(f+g)(x)=f(x) \oplus g(x), \ \ \ \ \ \ \ (c \cdot f)(x) = c \odot f(x), \ \ \ \ \forall x \in S.
$$
Probar que $(V^S, +, \cdot)$ es un $\mathbb{K}$-espacio vectorial. En el caso de que $V=\mathbb{K}$, este espacio vectorial se denotará $F(S)$.
\newline Hay que probar que cumple con las propiedades de un espacio vectorial:
    \begin{itemize}
	\item[S1:]
	$$
	(f+g)(x)=f(x)\oplus g(x)=g(x) \oplus f(x) = (g+f)(x)
	$$
	\begin{center}
		\textit{Se cumple por conmutatividad de $V$ por ser un espacio vectorial}.
	\end{center}
	\item[S2:]
	$$
	(f+(g+h))(x) = f(x) \oplus (g + h)(x) = f(x) \oplus (g(x) \oplus h(x)) = (f(x) \oplus g(x)) \oplus h(x) 
	$$
	$$
	= (f+g) (x) \oplus h(x) = ((f+g)+h)(x)
	$$
	\begin{center}
		\textit{Se cumple por asociatividad de $V$ por ser un espacio vectorial}.
	\end{center}
	\item[S3:] El único vector nulo es la función cero, que asigna a cada elemento de $S$ el escalar 0 de $F$.
	$$
	(0 + f) (x) = 0(x) \oplus f(x) = 0 \oplus f(x) = f(x)
	$$
	\item[S4:] Dada $f \in V^S$, sea $-f(x)$ el opuesto de $f(x)$, que existe debido a que $V$ es un espacio vectorial, entonces:
	$$
	(-f+f)(x) = -f(x) \oplus f(x) = 0(x).
	$$ 
	\item[P1:] Existe un neutro $1$ ya que existe neutro para $\odot$ en $V$, tal que:
	$$
	(1 \cdot f)(x) = 1 \odot f(x) = f(x)
	$$
	\item[P2:]
	$$
	(\lambda_1 \cdot ( \lambda_2 \cdot f ))(x) = \lambda_1 \odot (\lambda_2 \cdot f)(x) = \lambda_1 \odot (\lambda_2 \odot f(x)) 
	$$
	$$
	= (\lambda_1 \cdot \lambda_2) \odot f(x) = ((\lambda_1 \cdot \lambda_2) \cdot f)(x)
	$$
	\item[D1:] 
	$$
	(\lambda \cdot (f + g))(x) = \lambda \odot (f + g)(x) = (\lambda \odot f(x)) \oplus (\lambda \odot g(x)) 
	$$
	$$
	= (\lambda \cdot f)(x) \oplus (\lambda \cdot g)(x) = (\lambda \cdot f + \lambda \cdot g)(x)
	$$
	\item[D2:]
	$$
	((\lambda_1 + \lambda_2)\cdot f)(x) = (\lambda_1 + \lambda_2) \odot f(x) = (\lambda_1 \odot f(x)) \oplus (\lambda_2 \odot f(x))
	$$
	$$
	= (\lambda_1 \cdot f + \lambda_2 \cdot f)(x)
	$$
\end{itemize}
Como cumple con las 8 propiedades del espacio vectorial, entonces $(V^S, +, \cdot)$ \textbf{es un $\mathbb{K}$-espacio vectorial.}
\subsection{Ejercicio 4}
Sea $\mathbb{K}$ un cuerpo y $m,n \in \mathbb{N}$. Dar estructura de $\mathbb{K}$-espacio vectorial de matrices $M_{m\times n}(\mathbb{K})$.
\newline El \textbf{espacio de matrices} $m \times n$, $F^{m \times n}$. Sea $F$ cualquier cuerpo y sean $m$ y $n$ enteros positivos. Sea $F^{m \times n}$ el conjunto de todas las matrices $m \times n$ sobre el cuerpo $F$. La suma de dos vectores $A$ y $B$ en $F^{m \times n}$ se define por:
$$
(A + B)_{ij}= A_{ij} + B_{ij}.
$$
El producto de un escalar $c$, y de la matriz $A$ se define por:
$$
(cA)_{ij}= cA_{ij}
$$
\subsection{Ejercicio 5}
Decidir si los siguientes subconjuntos de $\mathds{R}^n$ son subespacios vectoriales.
\begin{itemize}
    \item[(a)] $\{ (x_1,...,x_n) \in  \mathds{R}^n : x_1=x_n\}$
    \item[(b)] $\{ (x_1,...,x_n) \in  \mathds{R}^n : x_1+...+x_n=1 \}$
    \item[(c)] $\{ (x_1,...,x_n) \in  \mathds{R}^n : x_1+...+x_n=0 \}$
    \item[(d)] $\{ (x_1,...,x_n) \in  \mathds{R}^n : x_1 \leq x_2 \}$
    \item[(e)] $\{ (x_1,...,x_n) \in  \mathds{R}^n : x_n=1 \}$
    \item[(f)] $\{ (x_1,...,x_n) \in  \mathds{R}^n : x_n=0 \}$
\end{itemize}
\subsubsection{Conceptos}
\begin{defbox}{Definición}
    Sea $V$ un espacio vectorial sobre $\mathds{K}$. diremos que $W \subset V$ es \textit{subespacio de $V$} si $W \neq \emptyset$ y
        \begin{enumerate}
            \item\label{def-sub-a} si para cualesquiera $w_1,w_2 \in W$, se cumple que $w_1+w_2 \in W$ y
            \item\label{def-sub-b} si $\lambda \in \mathds{K}$ y  $w \in W$, entonces $\lambda w \in W$.
        \end{enumerate}
\end{defbox}
\begin{teobox}{Teorema}
    Sea $V$ un espacio vectorial sobre $\mathds{K}$ y $W$ un subespacio de $V$. Entonces $W$ con las operaciones suma y producto por escalares de $V$ es un espacio vectorial.
\end{teobox}
\subsubsection{Prueba A}
Hay que probar que la adición y el producto esten contenidos en el subespacio de $\mathds{R}^n$:
\newline Sea $x=(x_1,...,x_1)\in\mathds{R}^n$ e $y=(y_1,...,y_1)\in\mathds{R}^n$, pruebo la suma y el producto:
$$
x + \lambda\cdot y = (x_1,...,x_1) + \lambda \cdot (y_1,...,y_1) = (x_1,...,x_1) + (\lambda y_1 + ... + \lambda y_1)
$$
$$
= (x_1+\lambda y_1 , ... , x_1 + \lambda y_1)
$$
Sigue perteneciendo al espacio vectorial, por lo tanto, $\{ (x_1,...,x_n) \in  \mathds{R}^n : x_1=x_n\}$ \textbf{es un subespacio} de de $\mathds{R}^n$.
\subsubsection{Prueba B}
Hay que probar que la adición y el producto esten contenidos en el subespacio de $\mathds{R}^n$:
\newline Sea $x=(1,0,0,...,0)\in\mathds{R}^n$ tal que $x_1+...+x_n=1$ e $y=(1,0,0,...,0)\in\mathds{R}^n$ tal que $y_1+...+y_n=1$, pruebo la suma y el producto:
$$
x + y = (1,0,0,...,0) + (1,0,0,...,0) = (2,0,0,...,0) 
$$
Ya no pertenece al supuesto subespacio vectorial, por lo tanto $\{ (x_1,...,x_n) \in  \mathds{R}^n : x_1+...+x_n=1 \}$ \textbf{no es un espacio}
de $\mathds{R}^n$.
\subsubsection{Prueba C}
Hay que probar que la adición y el producto esten contenidos en el subespacio de $\mathds{R}^n$:
\newline Sea $x=(0,0,0,...,0)\in\mathds{R}^n$ tal que $x_1+...+x_n=0$ e $y=(0,0,0,...,0)\in\mathds{R}^n$ tal que $y_1+...+y_n=0$, pruebo la suma y el producto:
$$
x + y = (0,0,0,...,0) + (0,0,0,...,0) = (0,0,0,...,0) 
$$
$$
\lambda x = \lambda (0,0,0,...,0) = (0,0,0,...,0) 
$$
La adición y el producto devuelven resultados dentro del subespacio, por lo tanto, $\{ (x_1,...,x_n) \in  \mathds{R}^n : x_1+...+x_n=0 \}$ \textbf{es un subespacio} de $\mathds{R}^n$.
\subsubsection{Prueba D}
Hay que probar que la adición y el producto esten contenidos en el subespacio de $\mathds{R}^n$: \newline
Sea $x=(3,4,...,x_n)$ donde $x_1=3$ y $x_2=4$, $x_1\leq x_2$, al probar el producto con un escalar:
$$
(-1)\cdot x = (-1) \cdot (3,4,...,x_n) = (-3,-4,...,-x_n)
$$
Da como resultado un vector tal que $x_1>x_2$, el cual no pertenece al supuesto subespacio, entonces, $\{ (x_1,...,x_n) \in  \mathds{R}^n : x_1 \leq x_2 \}$ \textbf{no es un subespacio} de $\mathds{R}^n$
\subsubsection{Prueba E}
Hay que probar que la adición y el producto esten contenidos en el subespacio de $\mathds{R}^n$: \newline
Sea $x=(x_1,x_2,...,1)$ e $y=(y_1,y_2,...,1)$ la suma de los vectores:
$$
x + y = (x_1,x_2,...,1) + (y_1,y_2,...,1) = (x_1+y_1,x_2+y_2,...,2)
$$
da como resultado un vector cuyo último componente es distinto de $1$, por lo tanto no pertenece al supuesto subespacio, por lo tanto, $\{ (x_1,...,x_n) \in  \mathds{R}^n : x_n=1 \}$ \textbf{no es un subespacio} de $\mathds{R}^n$.
\subsubsection{Prueba F}
Hay que probar que la adición y el producto esten contenidos en el subespacio de $\mathds{R}^n$: \newline
Sea $x=(x_1,x_2,...,0)$ e $y=(y_1,y_2,...,0)$ la suma y el producto por escalar de dichos vectores:
$$
x + \lambda y = (x_1,x_2,...,0) + \lambda (y_1,y_2,...,0) = (x_1,x_2,...,0) + (\lambda y_1, \lambda y_2,...,0) = (x_1 + \lambda y_1, x_2 + \lambda y_2,...,0)
$$
Da como resultado un vector cuyo ultimo componente sigue siendo $0$, entonces $\{ (x_1,...,x_n) \in  \mathds{R}^n : x_n=0 \}$ \textbf{es un subespacio} de $\mathds{R}^n$.
\subsection{Ejercicio 8}
Sean $W_1$, $W_2$ subespacios de un espacio vectorial $V$. Probar que $W_1\cup W_2$ es subespacio de $V$, si y sólo si $W_1\subseteq W_2$ o bien $W_2 \subseteq W_1$. \newline
Hay que probar lo siguiente:
\begin{center}
    $W_1\cup W_2$ es subespacio de $V$ $\Leftrightarrow$ $W_1\subseteq W_2 \vee W_2 \subseteq W_1$
\end{center}
\begin{itemize}
    \item[($\Leftarrow$)] Al tener una disjunción el problema se separa en dos casos:
    \begin{itemize}
        \item Si $W_1\subseteq W_2 \Rightarrow W_1\cup W_2 = W_2$, entonces \textbf{sí es subespacio} de $V$ ya que $W_2$ es subespacio por hipótesis.
        \item Si $W_2\subseteq W_1 \Rightarrow W_1\cup W_2 = W_1$, entonces \textbf{sí es subespacio} de $V$ ya que $W_1$ es subespacio por hipótesis.
    \end{itemize}
    \item[($\Rightarrow$)] Se probará por absurdo, supongamos que $W_1\cup W_2$  es subespacio de $V$ $\Rightarrow$ $W_1 \nsubseteq W_2 \wedge W_2 \nsubseteq W_1$. Si $W_1 \cup W_2$ es subespacio de $V$, entonces la suma de dos vectores $w_1$ y $w_2$ deben estar en la unión. $$w_1+w_2\in W_1 \cup W_2$$ pero esto implica que $w_1\in W_2 \vee w_2 \in W_1$ lo cual es absurdo. Por lo tanto, $W_1\cup W_2$  es subespacio de $V$ $\Rightarrow$ $W_1\subseteq W_2 \vee W_2 \subseteq W_1$ 
\end{itemize}
% ---------- PRÁCTICO 8 ---------- %
\section{Práctico 8: Independencia lineal y generadores. Bases y dimensión}
\subsection{Ejercicio 1}
Sea $\mathds{K}$ un cuerpo y $V$ un $\mathds{K}$-espacio vectorial. Demostrar las siguientes afirmaciones:
\begin{enumerate}
    \item Sean $v_1,v_2\in V$ y $\lambda\in\mathds{K}$. Demostrar que $\left \langle v_1,v_2 \right \rangle =\left \langle v_1,v_2 + \lambda v_1 \right \rangle$
    \item Si $\lambda,\mu\in\mathds{K}$ y $0\neq v\in V$ son tales que $\lambda v= \mu v$. Demostrar que $\lambda=\mu$.
    \item Sean $w,v_1,...,v_n \in V$. Demostrar que $\left \langle w,v_1,...,v_n \right \rangle = \left \langle v_1,...,v_n \right \rangle$ si y sólo si $w\in \left \langle v_1,...,v_n \right \rangle$.
\end{enumerate}
\subsubsection{Conceptos}
\begin{teobox}{Teorema}
    Sea $V$ un $\mathds{K}$-espacio vectorial, y $v_1,...,v_n$ vectores en $V$. Dado $v\in V$, diremos que $v$ es combinación lineal de los $v_1,...,v_n$ si existen escalares $\lambda_1,...,\lambda_n$ en $\mathds{K}$ tal que: \\ $v=\lambda_1v_1+...+\lambda_nv_n$.
\end{teobox}
\begin{defbox}{Definición}
    Sea $V$ un $\mathds{K}$-espacio vectorial y sean $v_1,...,v_k$ vectores en $V$. Al subespacio vectorial \\ $W=\{ \lambda_1v_1+...+\lambda_kv_k \ : \lambda_1,...,\lambda_k \in \mathds{K} \}$ de las combinaciones lineales de $v_1,...,v_n$ se lo denomina \textit{subespacio generado por $v_1,...,v_k$} y se lo puede denotar de las siguientes formas:
    \begin{enumerate}
        \item $W = \left \langle v_1,...,v_k \right \rangle$,
        \item $W = gen\{ v_1,...,v_k \}$,
        \item $W = span\{ v_1,...,v_k\}$
    \end{enumerate}
    Además, en este caso, diremos que el conjunto $S = \{ v_1,...,v_k\}$ genera al
    subespacio $W$ o que los vectores $v_1,...,v_k$ generan $W$.
\end{defbox}

\subsubsection{Demostración 1}
Hay que probar lo siguiente:
$$
\left \langle v_1,v_2 \right \rangle =\left \langle v_1,v_2 + \lambda v_1 \right \rangle
$$
Por un lado se tiene que existen escalares $\lambda_1,\lambda_2\in\mathds{K}$ tal que $v=\lambda_1v_1+\lambda_2v_2$ y por otro se tiene que existen escalares $\lambda'_1,\lambda'_2\in\mathds{K}$ tal que $v'=\lambda'_1v_1+\lambda'_2\cdot(v_2 + \lambda v_1)$ reescribiendo la ecuación:
$$
v'=\lambda'_1v_1+\lambda'_2\cdot(v_2 + \lambda v_1) = \lambda'_1v_1+\lambda'_2\cdot v_2 + \lambda'_2\cdot\lambda v_1 = (\lambda'_1+\lambda'_2)\cdot v_1+\lambda'_2\cdot v_2
$$
Por lo tanto, se prueba que un elemento de $\left \langle v_1,v_2 + \lambda v_1 \right \rangle$ pertenece a $\left \langle v_1,v_2 \right \rangle$. Analogamente se prueba que un elemento de $\left \langle v_1,v_2 \right \rangle$ pertenece a $\left \langle v_1,v_2 + \lambda v_1 \right \rangle$.
\newline Ambos se pueden escribir como combinación lineal de $v_1$ y $v_2$, por lo tanto queda demostrada la afirmación.
\subsubsection{Demostración 2}
Hay que probar lo siguiente:
$$
\lambda v= \mu v \Rightarrow \lambda=\mu
$$
Partiendo de la ecuación original, al restarle $\mu v$ de ambos lados:
$$
\lambda v - \mu v = \mu v - \mu v
$$
$$
\lambda v - \mu v = 0
$$
$$
(\lambda - \mu ) \cdot v = 0
$$
Como $v\neq 0$, entonces $(\lambda - \mu )=0$ (Por propiedades $P_1$ y $P_2$ demostradas abajo), entonces $\lambda = \mu$. Queda probada la demostrada la afirmación.
\begin{itemize}
    \item[($P_1$)] \textbf{$k\cdot 0 = 0$}
    $$
    k \cdot 0 = k \cdot (0+0)
    $$
    $$
    k \cdot 0 = k \cdot 0 + k \cdot 0
    $$
    $$
    k\cdot 0 - k\cdot 0 = k \cdot 0 + k \cdot 0 - k \cdot 0
    $$
    $$
    0 = k\cdot 0
    $$
    \item[($P_2$)] \textbf{$\lambda \cdot v = 0 \Rightarrow \lambda = 0 \vee v = 0$} \newline
    Supongamos que $\lambda \neq 0 \Rightarrow$ existe $\lambda^{-1}$ tal que $\lambda \cdot \lambda^{-1} = 1$.
    $$
    \lambda \cdot v = 0
    $$
    $$
    \lambda \cdot v \cdot \lambda^{-1} = 0 \cdot \lambda^{-1}
    $$
    $$
    v = 0
    $$
    Análogamente se cumple si $v\neq 0$.
    
\end{itemize}
\subsubsection{Demostración 3}
Hay que probar lo siguiente:
$$
\left \langle w,v_1,...,v_n \right \rangle = \left \langle v_1,...,v_n \right \rangle \Leftrightarrow w\in \left \langle v_1,...,v_n \right \rangle
$$
\begin{itemize}
    \item[($\Rightarrow$)] Se tiene que $\left \langle w,v_1,...,v_n \right \rangle = \left \langle v_1,...,v_n \right \rangle$. Entonces, esto quiere decir que el subespacio generado por el conjunto $\{ v_1,...,v_n \}$ es el mismo que el subespacio generado por $\{ w,v_1,...,v_n \}$. \newline
    Sea $x$ un vector genérico en $\left \langle w,v_1,...,v_n \right \rangle$ e $y$ un vector en $\left \langle v_1,...,v_n \right \rangle$:
    $$
    x = \lambda_ww+\lambda_1v_1+...+\lambda_nv_n
    $$
    $$
    y = \lambda'_1v_1+...+\lambda'_nv_n
    $$
    Por hipótesis, se pueden igualar y luego despejar $w$:
    $$
    \lambda_ww+\lambda_1v_1+...+\lambda_nv_n = \lambda'_1v_1+...+\lambda'_nv_n
    $$
    Al despejar $w$
    $$
    w = \frac{\lambda'_1-\lambda_1}{\lambda_w}\cdot v_1 + ... + \frac{\lambda'_n-\lambda_n}{\lambda_w}\cdot v_n
    $$
    Esto quiere decir que $w$ puede escribirse como combinación lineal de $v_1,...,v_n$, en otras palabras, pertenece al subespacio generado por el conjunto $\{ v_1,...,v_n \}$, \textbf{$w\in \left \langle v_1,...,v_n \right \rangle$}
    \item[($\Leftarrow$)] Se tiene que $w \in \left \langle v_1,...,v_n \right \rangle$ entonces $w$ puede escribirse como $w=\lambda_1v_1+...+\lambda_nv_n$.
    \newline Por lo tanto, un vector genérico del subespacio generado por $\{ w,v_1,...,v_n \}$ es de la forma
    $$
    x = \lambda'_ww+\lambda'_1v_1+...+\lambda'_nv_n
    $$
    Sustituyendo $w$ por $\lambda_1v_1+...+\lambda_nv_n$
    $$
    x = \lambda'_w\cdot (\lambda_1v_1+...+\lambda_nv_n) + \lambda'_1v_1+...+\lambda'_nv_n = \lambda'_w\cdot\lambda_1v_1 + ... + \lambda'_w\cdot\lambda_nv_n+ \lambda'_1v_1+...+\lambda'_nv_n
    $$
    $$
    = \lambda''_1v_1+...+\lambda''_nv_n
    $$
    Coincide con el subespacio generado $\left \langle v_1,...,v_n \right \rangle$. Queda demostrada la afirmación.
\end{itemize}
\subsection{Ejercicio 2}
En cada caso, determinar si el vector $w$ es combinación lineal de los vectores del conjunto $S$ en el espacio vectorial $V$.
\begin{itemize}
    \item[(a)] $V = \mathds{R}^4, w=(1,-1,0,0)$ y $S=\{ (1,0,0,2), (1,1,2,2), (0,1,0,1)\}$
    \item[(b)] $V = \mathds{R}[x], w=1+x+x^2+x^3+x^4+x^5$ y $S=\{2,x+x^2,x^2+x^3,x^3+x^4,x^4+x^5 \}$
\end{itemize}
\subsubsection{Conceptos}
\begin{obbox}{Observación}
    La pregunta de si un vector $v =(b_1,\ldots,b_m) \in \mathds{K}^m$ es combinación lineal de vectores $v_1,\ldots,v_n \in \mathds{K}^m$ se resuelve con un sistema de ecuaciones lineales: si 
    $$
    v_i = (a_{1i},\ldots,a_{mi}), \quad \text{para $1 \le i \le n$,}
    $$
    entonces $v = \lambda_1v_1 + \cdots +\lambda_nv_n$ se traduce, en coordenadas, a
    \begin{align*}
        (b_1,\ldots,b_m) &= \lambda_1(a_{11},\ldots,a_{m1}) + \cdots +\lambda_n(a_{1n},\ldots,a_{mn}) \\
        &= (\lambda_1a_{11} + \cdots+ \lambda_na_{1n}, \ldots, \lambda_1a_{m1} + \cdots+ \lambda_na_{mn}).
    \end{align*}
    Luego, $v$  es combinación lineal de los vectores $v_1,\ldots,v_n \in \mathds{K}^m$ si y sólo si  el sistema de ecuaciones:
    \begin{equation*}
    \begin{matrix}
    a_{11}\lambda_1& + &a_{12}\lambda_2& + &\cdots& + &a_{1n}\lambda_n &= &b_1\\
    \vdots&  &\vdots& &&  &\vdots \\
    a_{m1}\lambda_1& + &a_{m2}\lambda_2& + &\cdots& + &a_{mn}\lambda_n &=&b_m,
    \end{matrix}
    \end{equation*}
    \noindent con incógnitas $\lambda_1,\ldots, \lambda_n$ tiene solución.
\end{obbox}
\subsubsection{Punto A}
$V = \mathds{R}^4, w=(1,-1,0,0)$ y $S=\{ (1,0,0,2), (1,1,2,2), (0,1,0,1)\}$
\newline Se tiene que verificar si se cumple que:
$$
(1,-1,0,0) = \lambda_1(1,0,0,2)+\lambda_2(1,1,2,2)+\lambda_3(0,1,0,1)
$$
$$
= (\lambda_1,0,0,2\lambda_1) + (\lambda_2,\lambda_2,2\lambda_2, 2\lambda_2) + (0,\lambda_3,0,\lambda_3) = (\lambda_1+\lambda_2,\lambda_2+\lambda_3,2\lambda_2,2\lambda_1+2\lambda_2+\lambda_3)
$$
Se forma el sistema de ecuaciones:
$$
\begin{cases}
\lambda_1+\lambda_2 = 1 & \\ 
\lambda_2+\lambda_3 = -1 & \\ 
2\lambda_2 = 0& \\ 
2\lambda_1+2\lambda_2+\lambda_3 = 0 & 
\end{cases}
$$
Al resolverlo:
$$
\left ( \left.\begin{matrix}
1 & 1 & 0 \ \\  
0 & 1 & 1 \ \\ 
0 & 2 & 0 \ \\ 
2 & 2 & 1 \ \\
\end{matrix}\right| \begin{matrix}
1 \\ 
-1\\ 
0 \\
0 \\ 
\end{matrix}\right )
\xrightarrow[]{f_1-f_2}
\left ( \left.\begin{matrix}
1 & 0 & -1 \ \\  
0 & 1 & 1 \ \\ 
0 & 2 & 0 \ \\ 
2 & 2 & 1 \ \\
\end{matrix}\right| \begin{matrix}
2 \\ 
-1\\ 
0 \\
0 \\ 
\end{matrix}\right )
\xrightarrow[]{f_4-2f_1}
\left ( \left.\begin{matrix}
1 & 0 & -1 \ \\  
0 & 1 & 1 \ \\ 
0 & 2 & 0 \ \\ 
0 & 2 & 3 \ \\
\end{matrix}\right| \begin{matrix}
2 \\ 
-1\\ 
0 \\
-4 \\ 
\end{matrix}\right )
\xrightarrow[]{f_4-f_3}
\left ( \left.\begin{matrix}
1 & 0 & -1 \ \\  
0 & 1 & 1 \ \\ 
0 & 2 & 0 \ \\ 
0 & 0 & 3 \ \\
\end{matrix}\right| \begin{matrix}
2 \\ 
-1\\ 
0 \\
-4 \\ 
\end{matrix}\right )
$$
$$
\xrightarrow[]{f_3-2f_2}
\left ( \left.\begin{matrix}
1 & 0 & -1 \ \\  
0 & 1 & 1 \ \\ 
0 & 0 & -2 \ \\ 
0 & 0 & 3 \ \\
\end{matrix}\right| \begin{matrix}
2 \\ 
-1\\ 
2 \\
-4 \\ 
\end{matrix}\right )
\xrightarrow[]{(-1/2)f_3}
\left ( \left.\begin{matrix}
1 & 0 & -1 \ \\  
0 & 1 & 1 \ \\ 
0 & 0 & 1 \ \\ 
0 & 0 & 3 \ \\
\end{matrix}\right| \begin{matrix}
2 \\ 
-1\\ 
-1 \\
-4 \\ 
\end{matrix}\right )
\xrightarrow[]{f_4-3f_3}
\left ( \left.\begin{matrix}
1 & 0 & -1 \ \\  
0 & 1 & 1 \ \\ 
0 & 0 & 1 \ \\ 
0 & 0 & 0 \ \\
\end{matrix}\right| \begin{matrix}
2 \\ 
-1\\ 
-1 \\
-1 \\ 
\end{matrix}\right )
\xrightarrow[]{(-1)f_4)}
\left ( \left.\begin{matrix}
1 & 0 & -1 \ \\  
0 & 1 & 1 \ \\ 
0 & 0 & 1 \ \\ 
0 & 0 & 0 \ \\
\end{matrix}\right| \begin{matrix}
2 \\ 
-1\\ 
-1 \\
1 \\ 
\end{matrix}\right )
$$
$$
\xrightarrow[]{f_3+f_4}
\left ( \left.\begin{matrix}
1 & 0 & -1 \ \\  
0 & 1 & 1 \ \\ 
0 & 0 & 1 \ \\ 
0 & 0 & 0 \ \\
\end{matrix}\right| \begin{matrix}
2 \\ 
-1\\ 
0 \\
1 \\ 
\end{matrix}\right )
\xrightarrow[f_1+f_3]{f_2+f_4}
\left ( \left.\begin{matrix}
1 & 0 & 0 \ \\  
0 & 1 & 1 \ \\ 
0 & 0 & 1 \ \\ 
0 & 0 & 0 \ \\
\end{matrix}\right| \begin{matrix}
2 \\ 
0 \\ 
0 \\
1 \\ 
\end{matrix}\right )
\xrightarrow[f_1-2f_4]{f_2-f_3}
\left ( \left.\begin{matrix}
1 & 0 & 0 \ \\  
0 & 1 & 0 \ \\ 
0 & 0 & 1 \ \\ 
0 & 0 & 0 \ \\
\end{matrix}\right| \begin{matrix}
0 \\ 
0 \\ 
0 \\
1 \\ 
\end{matrix}\right )
$$
Se llega a $0=1$, entonces se concluye que $w$ \textbf{no se puede expresar como combinación lineal} de los vectores de $S$.
\subsubsection{Punto B}
Un polinomio de grado 5 se puede escribir como vector $v=(x^5,x^4,x^3,x^2,x^1,x^0)$. Entonces: \newline
$w=(1,1,1,1,1,1)$ y $S=\{ (0,0,0,0,0,2),(0,0,0,1,1,0),(0,0,1,1,0,0),(0,1,1,0,0,0),(1,1,0,0,0,0)\}$ \newline
Se tiene que verificar si se cumple que:
$$
(1,1,1,1,1,1) = \lambda_1(0,0,0,0,0,2)+\lambda_2(0,0,0,1,1,0)+\lambda_3(0,0,1,1,0,0)+\lambda_4(0,1,1,0,0,0)+\lambda_5(1,1,0,0,0,0)
$$ 
$$
= (0,0,0,0,0,2\lambda_1) + (0,0,0,\lambda_2,\lambda_2,0) + (0,0,\lambda_3,\lambda_3,0,0) + (0,\lambda_4,\lambda_4,0,0,0) + (\lambda_5,\lambda_5,0,0,0,0)
$$
$$
(\lambda_5,\lambda_4+\lambda_5,\lambda_3+\lambda_4,\lambda_2+\lambda_3,\lambda_2,2\lambda_1)
$$
Se forma el sistema de ecuaciones:
$$
\begin{cases}
\lambda_5 = 1 & \\ 
\lambda_4+\lambda_5 = 1 & \\ 
\lambda_3+\lambda_4 = 1 & \\ 
\lambda_2+\lambda_3 = 1 & \\
\lambda_2 = 1 & \\
2\lambda_1 = 1 \\
\end{cases}
$$
De la ecuación 1 se obtiene $\lambda_5=1$, de la 3 $\lambda_4=1$, de la 4 $\lambda_3=0$, de la 5 $\lambda_2=1$ y de la 6 $\lambda_1=1/2$. Pero al reemplazar en la segunda ecuación $\lambda_4+\lambda_5=1+1\neq 1$, por lo tanto el sistema no tiene soluciones y por consecuencia, $w$ no se puede escribir como combinación lineal de los vectores de $S$.
\subsection{Ejercicio 3}
Decidir si los siguientes subconjuntos de $\mathds{R}^3$ son linealmente independientes. Cuando un conjunto no lo sea, mostrar una relación lineal no trivial entre sus elementos.
\begin{itemize}
    \item[(a)] $\{ (1,0,-1),(1,2,1), (0,-3,2) \}$
    \item[(b)] $\{ (1,3,-3),(2,3,-4), (1,-3,1) \}$
\end{itemize}
\subsubsection{Conceptos}
\begin{defbox}{Definición}
    Sea $V$ un espacio vectorial sobre $\mathds{K}$. Un subconjunto $S$ de $V$ se dice \textit{linealmente dependiente} (o simplemente, \textit{LD} o \textit{dependiente}) si existen vectores distintos $v_1,\ldots,v_n \in S$  y escalares $\lambda_1,\ldots,\lambda_n$ de $\mathds{K}$, no todos nulos, tales que 	
        \begin{equation*}
            \lambda_1v_1+\cdots+\lambda_nv_n=0.
        \end{equation*}
        Un conjunto que no es linealmente dependiente se dice \textit{linealmente independiente} (o simplemente, \textit{LI} o \textit{independiente}).
        
        Si el conjunto $S$ tiene solo un número finito de vectores $v_1,\ldots,v_n$, diremos, para simplificar, que los $v_1,\ldots,v_n$ son LD (o LI), en vez de decir
        que $S$ es LD (o LI, respectivamente).
\end{defbox}
\begin{obbox}{Observación}
    En  general,  en $\mathds{K}^m$, si queremos determinar si  $v_1,\ldots,v_n$ es LI, planteamos la ecuación  
    \begin{equation*}
    \lambda_1v_1+\cdots+\lambda_nv_n=(0,\ldots,0),
    \end{equation*}
    que, viéndola coordenada a coordenada, es equivalente a un sistema de $m$ ecuaciones lineales con  $n$ incógnitas (que son $\lambda_1,\ldots,\lambda_n$). Si  la única solución es la trivial entonces $v_1,\ldots,v_n$ es LI. Si hay alguna solución no trivial, entonces $v_1,\ldots,v_n$ es LD. 
\end{obbox}
\subsubsection{Punto A}
De la obsevación se deduce que: Para $A\in\mathds{K}^{m\times n}$, las filas $\{ 
f1,...,f_m \}$ de $A$ son vectores de $\mathds{K}^n$ y si $A$ es una matriz escalonada, los vectores son LI, si se anula alguna fila son LD.
\newline Entonces planteo la matriz:
$$
\begin{bmatrix}
1 & 0 & -1\\ 
1 & 2 & 1\\ 
0 & -3 & 2 
\end{bmatrix}
\xrightarrow[]{f_2-f_1}
\begin{bmatrix}
1 & 0 & -1\\ 
0 & 2 & 2\\ 
0 & -3 & 2 
\end{bmatrix}
\xrightarrow[]{f_3-(3/2)f_2}
\begin{bmatrix}
1 & 0 & -1\\ 
0 & 2 & 2\\ 
0 & 0 & 5 
\end{bmatrix}
$$
Como se llega a una matriz escalonada, \textbf{los vectores son Linealmente independientes}.
\subsubsection{Punto B}
Planteo la matriz con los vectores dados:
$$
\begin{bmatrix}
1 & 3 & -3\\ 
2 & 3 & -4\\ 
1 & -3 & 1 
\end{bmatrix}
\xrightarrow[]{f_2-2f_1}
\begin{bmatrix}
1 & 3 & -3\\ 
0 & -3 & 2\\ 
1 & -3 & 1 
\end{bmatrix}
\xrightarrow[]{f_3-f_1}
\begin{bmatrix}
1 & 3 & -3\\ 
0 & -3 & 2\\ 
0 & -6 & 4 
\end{bmatrix}
\xrightarrow[]{f_3+2f_2}
\begin{bmatrix}
1 & 3 & -3\\ 
0 & -3 & 2\\ 
0 & 0 & 8 
\end{bmatrix}
$$
Como se llega a una matriz escalonada, \textbf{los vectores son Linealmente independientes}.
\subsection{Ejercicio 4}
Sea $A\in M_{3\times 4}(\mathds{R})$ la matriz dada por:
$$
A=\begin{bmatrix}
2 & 1 & 0 & 3\\ 
-1 & 0 & 1 & 0\\ 
0 & -1 & 0 & 1
\end{bmatrix}
$$
\begin{itemize}
    \item[(a)] Calcular la dimensión del subespacio vectorial de $\mathds{R}^4$ generado por las filas de $A$.
    \item[(b)] Calcular la dimensión del subespacio vectorial de $\mathds{R}^3$ generado por las columnas de $A$.
\end{itemize}
\subsubsection{Conceptos}
\begin{defbox}{Definición}
    Sea $V$ un espacio vectorial. Una \textit{base} de $V$ es un conjunto $\mathcal{B} \subseteq V$ tal que
     \begin{enumerate}
         \item\label{it.genera} $\mathcal{B}$ genera a $V$, y
         \item\label{it.li} $\mathcal{B}$ es LI.
     \end{enumerate}
      El espacio $V$ es de \textit{dimensión finita} si tiene una base finita,  es decir con  un número finito de elementos.
\end{defbox}
\begin{defbox}{Definición}
    Sea $A = [a_{ij}] \in M_{m \times n}(\mathds{K})$. El \textit{vector fila $i$} es el vector  $(a_{i1},\ldots,a_{in}) \in \mathds{K}^n$. El \textit{espacio fila} de $A$ es el subespacio de $\mathds{K}^n$ generado por los $m$ vectores fila de $A$.  De forma análoga, se define  el vector columna $j$ al vector $(a_{1j},\ldots,a_{mj}) \in \mathds{K}^m$ y  el \textit{espacio columna} de $A$ es el subespacio de $\mathds{K}^m$ generado por los $n$ vectores columna de $A$.
\end{defbox}
\begin{teobox}{Colorario}
    Sean $A$ matriz $m \times n$ y $R$ la MRF equivalente por filas a $A$. Entonces, el espacio fila de $A$ es igual al espacio fila de $R$ y las filas no nulas de $R$ forman una base del espacio fila de $A$. 
\end{teobox}
\begin{obbox}{Método}
    El  corolario  nos provee un método para encontrar una base de un  subespacio de $\mathds{K}^n$ generado por $m$ vectores: si $v_1,\ldots,v_m \in \mathds{K}^n$ y $W = \langle v_1,\ldots,v_m\rangle$, consideramos la matriz 
    $$
    A = \begin{bmatrix}
    v_1 \\ v_2 \\ \vdots \\ v_m
    \end{bmatrix}
    $$
    donde las filas son los vectores $v_1,\ldots,v_m$. Luego calculamos $R$, una MRF equivalente por filas a $A$, y si $R$ tiene $r$ filas no nulas, las $r$ filas no nulas son una base de $W$ y, por consiguiente, $\dim W = r$. 
\end{obbox}
\subsubsection{Punto A}
Sea $W$ el subespacio vectorial de $\mathds{R}^4$ generado por las filas de $A$:
$$W=\left \langle (2,1,0,3), (-1,0,1,0), (0,-1,0,1) \right \rangle$$
Hay que reducir la matriz a MRF:
$$
\begin{bmatrix}
2 & 1 & 0 & 3\\ 
-1 & 0 & 1 & 0\\ 
0 & -1 & 0 & 1
\end{bmatrix}
\xrightarrow[]{(1/2)f_1}
\begin{bmatrix}
1 & 1/2 & 0 & 3/2\\ 
-1 & 0 & 1 & 0\\ 
0 & -1 & 0 & 1
\end{bmatrix}
\xrightarrow[]{f_2+f_1}
\begin{bmatrix}
1 & 1/2 & 0 & 3/2\\ 
0 & 1/2 & 1 & 3/2\\ 
0 & -1 & 0 & 1
\end{bmatrix}
\xrightarrow[]{2f_2}
\begin{bmatrix}
1 & 1/2 & 0 & 3/2\\ 
0 & 1 & 2 & 3\\ 
0 & -1 & 0 & 1
\end{bmatrix}
$$
$$
\xrightarrow[]{f_3+f_2}
\begin{bmatrix}
1 & 1/2 & 0 & 3/2\\ 
0 & 1 & 2 & 3\\ 
0 & 0 & 2 & 4
\end{bmatrix}
\xrightarrow[]{(1/2)f_3}
\begin{bmatrix}
1 & 1/2 & 0 & 3/2\\ 
0 & 1 & 2 & 3\\ 
0 & 0 & 1 & 2
\end{bmatrix}
$$
Sea $R$ la matriz reducida por filas obtenida de $A$, la cantidad de filas no nulas $r$ en este caso $3$, es una base de $W$, y por consiguiente, $dim \ W = r$, es decir $dim \ W = 3$.
\subsubsection{Punto B}
Sea $V$ el subespacio vectorial de $\mathds{R}^3$ generado por las columnas de $A$:
$$
V=\left \langle (2,-1,0), (1,0,-1), (0,1,0), (3,0,1) \right \rangle
$$
Hay que colocarlos en una matriz, y reducirla:
$$
\begin{bmatrix}
2 & -1 & 0\\ 
1 & 0 & -1\\ 
0 & 1 & 0 \\
3 & 0 & 1
\end{bmatrix}
\xrightarrow[]{(1/2)f_1}
\begin{bmatrix}
1 & -1/2 & 0\\ 
1 & 0 & -1\\ 
0 & 1 & 0 \\
3 & 0 & 1
\end{bmatrix}
\xrightarrow[]{(f_2-f_1}
\begin{bmatrix}
1 & -1/2 & 0\\ 
0 & 1/2 & -1\\ 
0 & 1 & 0 \\
3 & 0 & 1
\end{bmatrix}
\xrightarrow[]{(f_4-3f_1}
\begin{bmatrix}
1 & -1/2 & 0\\ 
0 & 1/2 & -1\\ 
0 & 1 & 0 \\
0 & 3/2 & 1
\end{bmatrix}
$$
$$
\xrightarrow[]{(f_4-(3/2)f_3}
\begin{bmatrix}
1 & -1/2 & 0\\ 
0 & 1/2 & -1\\ 
0 & 1 & 0 \\
0 & 0 & 1
\end{bmatrix}
\xrightarrow[f_2-(1/2)f_3]{f_1+(1/2)f_3}
\begin{bmatrix}
1 & 0 & 0\\ 
0 & 0 & -1\\ 
0 & 1 & 0 \\
0 & 0 & 1
\end{bmatrix}
\xrightarrow[]{f_2+f_4}
\begin{bmatrix}
1 & 0 & 0\\ 
0 & 0 & 0\\ 
0 & 1 & 0 \\
0 & 0 & 1
\end{bmatrix}
$$
Sea $R$ la matriz reducida por filas obtenida de las columnas de $A$, la cantidad de filas no nulas $r$ en este caso $3$, es una base de $W$, y por consiguiente, $dim \ V = r$, es decir $dim \ V = 3$.
\subsection{Ejercicio 5}
Sea $A\in M_{3\times 4}$ la matriz dada por
$$
A = \begin{bmatrix}
2 & 1 & 0 & 0\\ 
0 & 0 & 1 & 0 \\ 
0 & -1 & 0 & 1
\end{bmatrix}
$$
Sea $W\subseteq \mathds{R}^4 $ el subespacio vectorial que consta de aquellos $X\in\mathds{R}^4$ tales que $AX=0$. Encontrar un conjunto de vectores que generen a $W$. \newline
Sea $AX=0$ la ecuación para la cual se deben buscar soluciones, se forma el siguiente sistema de ecuaciones:
$$
\begin{cases}
    2x+y = 0 & \\
    z = 0 & \\
    -y + w = 0 
\end{cases}
$$
Reduzco la matriz:
$$
\begin{bmatrix}
2 & 1 & 0 & 0\\ 
0 & 0 & 1 & 0 \\ 
0 & -1 & 0 & 1
\end{bmatrix}
\xrightarrow[]{(-1)f_3}
\begin{bmatrix}
2 & 1 & 0 & 0\\ 
0 & 0 & 1 & 0 \\ 
0 & 1 & 0 & -1
\end{bmatrix}
\xrightarrow[]{f_1-f_3}
\begin{bmatrix}
2 & 0 & 0 & 1\\ 
0 & 0 & 1 & 0 \\ 
0 & 1 & 0 & -1
\end{bmatrix}
\xrightarrow[]{(1/2)f_1}
\begin{bmatrix}
1 & 0 & 0 & 1/2\\ 
0 & 0 & 1 & 0 \\ 
0 & 1 & 0 & -1
\end{bmatrix}
$$
El sistema queda:
$$
\begin{cases}
    x = -1/2w & \\
    z = 0 & \\
    y = w 
\end{cases}
$$
Por lo tanto las soluciones son de la forma $\{ w(-1/2,1,0,1) \}$ y por definición de espacio generado, el conjunto de vectores que generan a $W$ es:
$$
W = \left \langle (-1/2,1,0,1) \right \rangle
$$
\subsection{Ejercicio 6}
Demostrar las siguientes afirmaciones:
\begin{itemize}
    \item[(a)] Todo subconjunto de un conjunto LI es LI.
    \item[(b)] Todo conjunto que contiene a un conjunto LD es también LD.
    \item[(c)] Todo conjunto que contiene al vector 0 es LD.
    \item[(d)] Un conjunto es LI si y sólo si todos sus subconjuntos \textit{finitos} son LI
    \item[(e)] Probar que un conjunto de vectores $\{v_1,..,v_n\}$ es LD si y sólo si alguno de los vectores está en el generado por los otros, esto es: existe $i: \ 1\leq i\leq n $ tal que $v_i\in\left \langle v_1,...,v_{i-1},v_{i+1},...,v_n \right \rangle $.
\end{itemize}
\subsubsection{Demostración A}
\textsc{Todo subconjunto de un conjunto LI es LI} \\
Sea $L$ un subconjunto \textit{linealmente dependiente}, es decir existen vectores $l_1,l_2,...,l_n\in L$ y escalares $\lambda_1,\lambda_2,...,\lambda_n$ pertenecientes a un cuerpo $\mathds{K}$, no todos nulos, tales que: $\lambda_1l1+\lambda_2l_2+...+\lambda_nl_n=0$. Entonces, si se le agregan vectores con coeficientes 0, el conjunto seguiría siendo LD ya que la combinación lineal seguiría dando 0, con sus escalares no todos nulos. \\
Supongamos que se tiene un conjunto $V$ LI y un subconjunto $W$ linealmente LD, entonces agregando todos los vectores que faltan para formar el conjunto inicial seguiría siendo LD, y eso es una contradicción, por lo tanto, los subconjuntos que se tomen de un conjunto LI son necesariamente LI. $\square$
\subsubsection{Demostración B}
\textsc{Todo conjunto que contiene a un conjunto LD es tambien LD}. \\
Sea $L$ un subconjunto \textit{linealmente dependiente}, es decir existen vectores $l_1,l_2,...,l_n\in L$ y escalares $\lambda_1,\lambda_2,...,\lambda_n$ pertenecientes a un cuerpo $\mathds{K}$, no todos nulos, tales que: $\lambda_1l1+\lambda_2l_2+...+\lambda_nl_n=0$. Entonces, si se le agregan vectores con coeficientes 0, el conjunto seguiría siendo LD ya que la combinación lineal seguiría dando 0, con sus escalares no todos nulos. \\
Entonces cualquier subconjunto LD de un conjunto mas grande arbitrario, al agregarle los elementos restantes para el conjunto original, seguirá siendo LD, por consecuencia, todo conjunto que contiene a un conjunto LD también será LD. $\square$
\subsubsection{Demostración C}
\textsc{Todo conjunto que contiene al vector 0 es LD} \\
Sea $V$ un conjunto con vectores $v_1,v_2,...,v_n,0_v$ con el vector nulo. Va a existir una combinación lineal 
$$
0v_1+0v_2+...+0v_n+\lambda0_v=0
$$
Es decir, con cualquier $\lambda\in\mathds{K}$ va a existir una combinación lineal que da 0 con no todos los escalares nulos, por lo tanto, el conjunto será LD. $\square$
\subsubsection{Demostración D}
\textsc{Un conjunto es LI si y sólo si todos sus subconjuntos finitos son LI}
\begin{itemize}
    \item[($\Rightarrow$)] Supongamos que se tiene un conjunto $V$ LI y un subconjunto finito $W$ linealmente LD, entonces agregando todos los vectores que faltan para formar el conjunto inicial seguiría siendo LD, y eso es una contradicción, por lo tanto, los subconjuntos finitos que se tomen de un conjunto LI son necesariamente LI.
    \item[($\Leftarrow$)] Supongamos que se tiene un subconjunto finito $W$ LI de un conjunto $V$. Es decir la ecuación $\lambda_1w_1+\lambda_2w_2+...+\lambda_nw_n=0$ tiene solución únicamente con los escalares $\lambda_1,...,\lambda_n=0$. Al agregarle los vectores faltantes para el conjunto $V$, (sean $w_{m1},w_{m2},...,w_{mn}$ dichos vectores) la ecuación queda de la forma: $\lambda_1w_1+\lambda_2w_2+...+\lambda_nw_n + \lambda´_1w_{m1}+\lambda'_2w_{m2}+...+\lambda'_nw_{mn}=0$. Por hipótesis, todos los subconjuntos finitos son LI, por lo tanto la ecuación seguirá teniendo solución cuando los escalares $\lambda_1,...,\lambda_n,\lambda'_1,...,\lambda'_n$ son 0, por consecuencia, el conjunto también es LI. $\square$
\end{itemize}
\subsubsection{Demostración E}
\textsc{Probar que un conjunto de vectores $\{v_1,..,v_n\}$ es LD si y sólo si alguno de los vectores está en el generado por los otros, esto es: existe $i: \ 1\leq i\leq n $ tal que $v_i\in\left \langle v_1,...,v_{i-1},v_{i+1},...,v_n \right \rangle $.} \\
Se puede afirmar que si $v_i\in \left \langle v_1,...,v_{i-1},v_{i+1},...,v_n \right \rangle$, entonces \\ $\left \langle v_1,...,v_i,...,v_n \right \rangle = \left \langle v_1,...,v_{i-1},v_{i+1},...,v_n \right \rangle$.
\begin{bbox}
    Se tiene que $v_i \in \left \langle v_1,...,v_{i-1},v_{i+1},...,v_n \right \rangle$ entonces $v_i$ puede escribirse como $v_i=\lambda_1v_1+...+\lambda_{i-1}v_{i-1}+\lambda_{i+1}v_{i+1}+...+\lambda_nv_n$.
    \newline Un vector genérico del subespacio generado por $\{v_1,...,v_i,...,v_n \}$ es de la forma
    $$
    x = \lambda'_iv_i+\lambda'_1v_1+...+\lambda'_nv_n
    $$
    Sustituyendo $v_i$ por $v_i=\lambda_1v_1+...+\lambda_{i-1}v_{i-1}+\lambda_{i+1}v_{i+1}+...+\lambda_nv_n$
    $$
    x = \lambda'_i\cdot (\lambda_1v_1+...+\lambda_{i-1}v_{i-1}+\lambda_{i+1}v_{i+1}+...+\lambda_nv_n) + \lambda'_1v_1+...+\lambda'_nv_n 
    $$
    $$
    = \lambda''_1v_1+...+\lambda''_{i-1}v_{i-1}+\lambda''_{i+1}v_{i+1}+...+\lambda''_nv_n
    $$
    Coincide con el subespacio generado $\left \langle v_1,...,v_{i-1},v_{i+1},...,v_n \right \rangle$. Queda demostrada la afirmación.
\end{bbox}

Con esto el problema puede reescribirse como la prueba de que: un conjunto de vectores $\{v_1,v_2,...,v_n\}$ es LD $\Leftrightarrow$ $\left \langle v_1,...,v_i,...v_n \right \rangle = \left \langle v_1,...,v_{i-1},v_{i+1},...,v_n \right \rangle$ con $1\leq i\leq n$.
\begin{itemize}
    \item[($\Leftarrow$)] Se tiene que $\left \langle v_1,...,v_i,...v_n \right \rangle = \left \langle v_1,...,v_{i-1},v_{i+1},...,v_n \right \rangle$ con $1\leq i\leq n$ por hipótesis. \\
    Es decir existen escalares $\lambda_j$ ($j\neq i$), tal que
    $
    v_i= \sum_{j=1, j\neq i}^{n} \lambda_jv_j$. 
    Entonces:
    $$
    0 = \sum_{j=1}^{i-1}\lambda_jv_j+(-1)v_i+\sum_{j= i+1}^{n}\lambda_jv_j
    $$
    Por lo tanto, el conjunto es \textbf{linealmente dependiente}.
    \item[($\Rightarrow$)] Si $\{v_1,v_2,...,v_n\}$ es linealmente dependiente, existen $\lambda_1,\lambda_2,...,\lambda_n$ no todos nulos, tales que $\sum_{j=1}^{n}\lambda_jv_j=0$. Sin pérdida de generalidad, suponemos que $\lambda_i\neq 0$. Entonces:
    $$
    v_i=\sum_{j=1, j\neq i}^{n}\frac{\lambda_j}{\lambda_i}v_j \in \left \langle v_1,...,v_{i-1},v_{i+1},...,v_n \right \rangle  \ \ \ \ \square
    $$
\end{itemize}
\subsection{Ejercicio 8}
Dar 3 vectores en $\mathds{R}^3$ que sean LD y tales que dos cualesquiera de ellos sean LI. \\
Para que el conjunto sea LD, tomo primero el vector $(0,0,0)$ y luego tomo dos vectores que a simple vista sean LI, el conjunto queda: 
$$
\{ (0,0,0),(1,1,0), (2,0,1) \}
$$

\subsection{Ejercicio 9}
Sea $V$ el $\mathbb{Q}$-espacio vectorial de sucesiones con valores racionales, es decir funciones $a:\mathbb{N}\rightarrow\mathbb{Q}$. Para todo $n\in\mathbb{N}$ encontrar un subconjunto de $V$ de cardinal $n$ que sea $LI$. \\
En este caso el $\mathbb{Q}$-espacio vectorial está definido como: 
$$
V = \{  a:\mathds{N}\rightarrow\mathbb{Q} \}= \{ \{ a_n \} \}
$$
Sea $a,b:\mathds{N}\rightarrow\mathbb{Q}$, la suma y el producto por escalar del espacio vectorial están definidos como: $(a+b)(n)=a(n)+b(n)$ y $(\lambda \cdot a) = \lambda \cdot a(n)$ con $\lambda\in\mathbb{Q}$. \\
Si se tiene $a_i$ se puede escribir como $a_i:\mathds{N}\rightarrow\mathbb{Q}=(a_1,a_2,...)$. Por lo tanto, tomando sucesiones de la forma:
$$
e_j:\mathds{N}\rightarrow\mathbb{Q}=(0,0,...,1,...,0,...)
$$
Donde
$$
e_j(n)=\begin{cases}
1 & \text{ si } j=n \\ 
0 & \text{ si } j\neq n 
\end{cases}
$$
El conjunto $\{ e_1,e_2,...,e_m \}$ es LI, para probarlo tomo una combinación lineal
$$
\sum_{i=1}^{m}\lambda_ie_i=0
$$
Notar que el $0$ es la función 0 tal que $0(n)=0$. Luego tomando la función de ambos lados:
$$
\left ( \sum_{i=1}^{m}\lambda_ie_i\right )(n)=0 (n)
$$
$$
= \sum_{i=1}^{m}\lambda_ie_i(n)=0
$$
Ahora bien, por definición de $e_i(n)=0$ para todo $i\neq n$, entonces en la sumatoria el único término que va a sobrevivir es donde $i=n$ ya que $e_i(n)=1$ con $i=n$. La ecuación queda:
$$
= \lambda_n \cdot 1 = 0
$$
Con esto queda probado que $\lambda_n=0$ y por consecuencia, el conjunto es LI.
\subsection{Ejercicio 11}
Sea $\mathbb{K}$ un cuerpo. Decidir si las siguientes afirmaciones son verdaderas o falsas. Justificar.
\begin{itemize}
    \item[(a)] Si $W_1$ y $W_2$ son subespacios vectoriales de $\mathbb{K}^8$ de dimensión 5, entonces $W_1\cap W_2=0$.
\end{itemize}
\subsubsection{Demostración A}
\textsc{Si $W_1$ y $W_2$ son subespacios vectoriales de $\mathbb{K}^8$ de dimensión 5, entonces $W_1\cap W_2=0$.} \\
Sea $W_1$ y $W_2$ subespacios vectoriales de $\mathbb{K}^8$ de dimensión 5, entonces $dim(W_1+W_2)\leq 8$ y por teorema, $dim (W_1+W_2)= dim(W_1)+ dim (W_2) - dim(W_1 \cup W_2)$
y $dim(W_1)+dim(W_2)=10$ por hipótesis y $dim(W_1 \cup W_2)$ si fuese 0, $dim(W_1+W_2)=10$ lo que es absurdo. Entonces la afirmación es falsa. 
\subsection{Ejercicio 13}
Dar una base y dimensión de los siguientes subespacios vectoriales:
\begin{itemize}
    \item[(a)] $\{ (x,y,z)\in\mathds{R}^3 : z=x+y \}$,
    \item[(b)] $\{ (x,y,z,w,u)\in\mathds{R}^5:w=x+z, y=x-z, u=2x-3z \}$,
    \item[(c)] $\{ a_0+a_1x+a_2x^2+a_3x^3 \in\mathds{R}_3[x]:a_0+a_3=a_1+a_2 \}$
\end{itemize}
\subsubsection{Punto A}
Sea $W$ el subespacio vectorial de $\mathds{R}^3$ dado por:
$$
W=\{ (x,y,z)\in\mathds{R}^3 : z=x+y \}
$$
Reemplazando $z$ en el vector se obtiene: $(x,y,x+y)$, por lo tanto, el conjunto de vectores que generan a $W$ es:
$$
W=\left \langle (1,0,1), (0,1,1) \right \rangle
$$
Entonces una base de $W$ es $\{(1,0,1), (0,1,1)\}$ y su dimensión es 2.
\subsubsection{Punto B}
Sea $W$ el subespacio vectorial de $\mathds{R}^5$ dado por:
$$
W=\{ (x,y,z,w,u)\in\mathds{R}^5:w=x+z, y=x-z, u=2x-3z \}
$$
Reemplazando $w,y,u$ en el vector se obtiene: $(x,x-z,z,x+z,2x-3z)$, por lo tanto, el conjunto de vectores que generan a $W$ es:
$$
W=\left \langle (1,1,0,1,2), (0,-1,1,1,-3) \right \rangle
$$
Entonces una base de $W$ es $\{(1,1,0,1,2), (0,-1,1,1,-3)\}$ y su dimensión es 2.
\subsubsection{Punto C}
Sea $W$ el subespacio vectorial de $\mathds{R}_3[x]$ dado por:
$$
W=\{ a_0+a_1x+a_2x^2+a_3x^3 \in\mathds{R}_3[x]:a_0+a_3=a_1+a_2 \}
$$
Primero despejo $a_0$ de la ecuación, $a_0=-a_3+a_1+a_2$ y reemplazo en el polinomio dado:
$$
W=\{ (-a_3+a_1+a_2)+a_1x+a_2x^2+a_3x^3 \in\mathds{R}_3[x] \}
$$
$$
=a_1(x+1)+a_2(x^2+1)+a_3(x^3-1)
$$
Por lo tanto, el conjunto de vectores que generan a $W$ es:
$$
W=\left \langle (x+1),(x^2+1),(x^3-1) \right \rangle
$$
Entonces una base de $W$ es $\{(x+1),(x^2+1),(x^3-1)\}$ y su dimensión es 3.
\subsection{Ejercicio 16}
Probar que los vectores $v_1=(1,0,-i)$, $v_2=(1+i,1-i,1)$, $v_3=(i,i,i)$ forman una base de $\mathbb{C}^3$. Dar las coordenadas de un vector $(x,y,z)$ en esta base. 
\subsubsection{Conceptos}
\begin{defbox}{Definición}
    Sea $V$ un espacio vectorial. Una \textit{base} de $V$ es un conjunto $\mathcal{B} \subseteq V$ tal que
     \begin{enumerate}
         \item $\mathcal{B}$ genera a $V$, y
         \item $\mathcal{B}$ es LI.
     \end{enumerate}
      El espacio $V$ es de \textit{dimensión finita} si tiene una base finita,  es decir con  un número finito de elementos.
\end{defbox}
\begin{defbox}{Definición}
    Sea $V$ espacio vectorial de dimensión finita. Diremos que $n$  es \textit{la dimensión de $V$} y  denotaremos $\dim V =n$,  si existe una base de $V$  de $n$  vectores. Si $V = \{0\}$,  entonces definimos $\dim V =0$.
\end{defbox}
\subsubsection{Prueba}
Para verificar que forman una base de $\mathbb{C}^3$ hay que probar que son LI y generan. Se puede notar que $dim(\mathbb{C}^3)=3$ por lo tanto la cantidad de vectores coincide. \\
Para que sean LI, no deben existir soluciones no triviales a la ecuación:
$$
\lambda_1v_1+\lambda_2v_2+\lambda_3v_3=0
$$
Sea $A$ una matriz, podemos plantear $v_1,v_2,v_3$ como las respectivas columnas:
$$
\begin{pmatrix}
1 & 1+i & i\\ 
0 & 1-i & i\\ 
-i & 1 & i
\end{pmatrix}
$$
Y al reducirla se llega a:
$$
\begin{pmatrix}
1 & 0 & 0\\ 
0 & 1 & 0\\ 
0 & 0 & 1
\end{pmatrix}
$$
Esto quiere decir que ningun vector es combinación lineal del otro, que la ecuación $AX=0$ donde $A$ es la matriz planteada y $X$ es la columna $\lambda_1, \lambda_2, \lambda_3$. Tendrá solución solo cuando $\lambda_1, \lambda_2, \lambda_3=0$, entonces es LI. y por consecuencia \textbf{sí forman una base de $\mathbb{C}^3$}. \\
Para dar las coordenadas de un vector $(x,y,z)$ en esta base, hay que resolver la ecuación:
$$
\lambda_1v_1+\lambda_2v_2+\lambda_3v_3=(x,y,z)
$$
Esto es equivalente a resolver el sistema de ecuaciones:
$$
\begin{cases}
    \lambda_1+\lambda_2+\lambda_3=x & \\
    (1+i)\lambda_1+(1-i)\lambda_2+i\lambda_3=y & \\
    -i\lambda_1+i\lambda_2+i\lambda_3=z
\end{cases}
$$
Que en forma matricial es:
$$
\begin{pmatrix}
1 & 1+i & i\\
1 & 1-i & i\\
-i & 1 & i
\end{pmatrix}
\begin{pmatrix}
\lambda_1\\
\lambda_2\\
\lambda_3
\end{pmatrix}
=
\begin{pmatrix}
x\\
y\\
z
\end{pmatrix}
$$
Resolviendo el sistema se obtiene:
$$
\begin{pmatrix}
\lambda_1\\
\lambda_2\\
\lambda_3
\end{pmatrix}
=
\begin{pmatrix}
\frac{1}{2}x-\frac{1}{2}y+\frac{1}{2}z\\
\frac{1}{2}x+\frac{1}{2}y-\frac{1}{2}z\\
\frac{1}{2}x-\frac{1}{2}y-\frac{1}{2}z
\end{pmatrix}
$$
Entonces las coordenadas del vector $(x,y,z)$ en la base $\{v_1,v_2,v_3\}$ son:
\subsection{Ejercicio 17}
Sean $\lambda_1,...,\lambda_n$ todos distintos y recordemos el $\mathds{R}$-espacio vectorial $F(\mathds{R})$. Demostrar que el conjunto $\{e^{\lambda_1x},...,e^{\lambda_nx}\} \subset F(\mathds{R})$ es LI. Concluir que $\dim_{\mathds{R}} F(\mathds{R})$ es infinita. \\
Para probar que el conjunto es LI tomo una combinacion lineal:
$$
\sum_{i=1}^{n}\lambda_ie^{\lambda_ix}=0
$$
Derivo ambos lados:
$$
\sum_{i=1}^{n}\lambda_i\lambda_ie^{\lambda_ix}=0
$$
$$
\sum_{i=1}^{n}\lambda_i^2e^{\lambda_ix}=0
$$
Derivo nuevamente:
$$
\sum_{i=1}^{n}\lambda_i^3e^{\lambda_ix}=0
$$
Y así sucesivamente, se puede ver que la derivada de la combinación lineal es la combinación lineal de las derivadas, por lo tanto, si se evalúa en $x=0$ se obtiene:
$$
\sum_{i=1}^{n}\lambda_i^ke^{\lambda_i0}=0
$$
$$
\sum_{i=1}^{n}\lambda_i^k=0
$$
Esto es una contradicción ya que los $\lambda_i$ son todos distintos, por lo tanto, la combinación lineal es 0 solo cuando los escalares son 0, es decir, el conjunto es LI. \\
Por definición de dimensión, si un conjunto es LI y tiene infinitos elementos, entonces el espacio vectorial es de dimensión infinita. Por lo tanto, $\dim_{\mathds{R}} F(\mathds{R})$ es infinita.
\subsection{Ejercicio 20}
Decidir si es posible extender los siguientes conjuntos a una base de los respectivos espacios vectoriales.
En caso afirmativo, extender a una base.
\begin{itemize}
    \item[(a)] $ \{ (1, 2, 1, 1),(1, 0, 1, 1),(3, 2, 3, 3) \}\subseteq \mathds{R}^4$
    \item[(b)] $ \{ (1, 2, 0, 0),(1, 0, 1, 0) \} \subseteq \mathds{R}^4$
    \item[(c)] $\left \{ 
\begin{bmatrix}
1 & 0\\ 
0 & -1
\end{bmatrix}, 
\begin{bmatrix}
1 & 0\\ 
-2 & 1
\end{bmatrix}, 
\begin{bmatrix}
1 & 2\\ 
3 & 2
\end{bmatrix} \right \} \subseteq M_{2\times 2}(\mathds{R})$
\end{itemize}
\subsubsection{Conceptos}
\begin{teobox}{Teorema}
    Sea $V$ espacio vectorial de dimensión finita $n$ y $S_0$ un subconjunto LI de $V$. Entonces $S_0$  es finito  y existen $w_1,\ldots,w_m$ vectores en  $V$ tal que  $S_0 \cup \{w_1,\ldots,w_m\}$ es una base de $V$. 
\end{teobox}
\begin{teobox}{Colorario}
    Sea $W$ es un subespacio de un espacio vectorial de dimensión finita $n$ y $S_0$ un subconjunto LI de $W$. Entonces, $S_0$ se puede completar a una base de $W$. 
\end{teobox}
\subsubsection{Punto A}
Para ver si es posible exteneder a una base lo primero es probar si el conjunto es LI, ya que de no ser así no se puede extender a base. Para ello coloco los vectores como filas de una matriz y la reduzco:
$$
\begin{bmatrix}
1 & 2 & 1 & 1\\ 
1 & 0 & 1 & 1\\ 
3 & 2 & 3 & 3
\end{bmatrix}
\xrightarrow[f_3-3f_1]{f_2-f_1}
\begin{bmatrix}
1 & 2 & 1 & 1\\ 
1 & -2 & 0 & 0\\ 
0 & -4 & 0 & 0
\end{bmatrix}
\xrightarrow[]{(1/2)f_2}
\begin{bmatrix}
1 & 2 & 1 & 1\\ 
1 & 1 & 0 & 0\\ 
0 & -4 & 0 & 0
\end{bmatrix}
\xrightarrow[f_3+4f_2]{f_1-2f_2}
\begin{bmatrix}
1 & 0 & 1 & 1\\ 
1 & 1 & 0 & 0\\ 
0 & 0 & 0 & 0
\end{bmatrix}
$$
Esto quiere decir que el tercer vector es combinación lineal de los restantes, por lo tanto es un conjunto Linealmente Dependiente y por consecuencia \textbf{no se puede extender a base}.
\subsubsection{Punto B}
Para ver si es posible exteneder a una base lo primero es probar si el conjunto es LI, ya que de no ser así no se puede extender a base. Para ello coloco los vectores como filas de una matriz y la reduzco:
$$
\begin{bmatrix}
1 & 2 & 0 & 0\\ 
1 & 0 & 1 & 0
\end{bmatrix}
\xrightarrow[]{f_2-f_1}
\begin{bmatrix}
1 & 2 & 0 & 0\\ 
0 & -2 & 1 & 0
\end{bmatrix}
\xrightarrow[]{(1/2)f_2}
\begin{bmatrix}
1 & 2 & 0 & 0\\ 
0 & 1 & -1/2 & 0
\end{bmatrix}
\xrightarrow[]{f_1-2f_2}
\begin{bmatrix}
1 & 0 & 1 & 0\\ 
0 & 1 & -1/2 & 0
\end{bmatrix}
$$
Esto quiere decir que el conjunto es LI, para que sea una base de $\mathds{R}^4$ debe tener dimensión $4$, por lo tanto se deben agregar dos vectores mas, que pueden ser los vectores pertenecientes a la base canónica de $\mathds{R}^4$ tal que en conjunto con los actuales, sean LI. \\
Tomando $\{ (1, 2, 0, 0),(1, 0, 1, 0), (0,0,1,0), (0,0,0,1) \}$:
$$
\begin{bmatrix}
1 & 2 & 0 & 0\\ 
1 & 0 & 1 & 0\\ 
0 & 0 & 1 & 0\\ 
0 & 0 & 0 & 1
\end{bmatrix}
\xrightarrow[]{}
\begin{bmatrix}
1 & 0 & 0 & 0\\ 
0 & 1 & 0 & 0\\ 
0 & 0 & 1 & 0\\ 
0 & 0 & 0 & 1
\end{bmatrix}
$$
Esto quiere decir que ningun vector es combinación lineal del otro, es LI y por lo tanto es base de $\mathds{R}^4$.
\subsubsection{Punto C}
Para ver si el conjunto es LI, hay que probar que la única solución a esta ecuación es cuando los coeficientes son 0.
$$
\lambda_1\cdot 
\begin{bmatrix}
1 & 0\\ 
0 & -1
\end{bmatrix}
+ \lambda_2 \cdot
\begin{bmatrix}
1 & 0\\ 
-2 & 1
\end{bmatrix}
+ \lambda_3 \cdot
\begin{bmatrix}
1 & 2\\ 
3 & 2
\end{bmatrix}
= 
\begin{bmatrix}
0 & 0\\ 
0 & 0
\end{bmatrix}
$$
De ahí se forma el siguiente sistema de ecuaciones ya que la igualdad de matrices es coordenada a coordenada:
$$
\left\{\begin{matrix}
\lambda_1 + \lambda_2 + \lambda_3 = 0\\ 
2\lambda_2 = 0\\ 
-2\lambda_2 + 3 \lambda_3 = 0\\ 
-\lambda_1+\lambda_2+2\lambda_3 = 0
\end{matrix}\right.
$$
De la segunda ecuación sale que $\lambda_2=0$ luego usando en la tercera, sale que $\lambda_3=0$ con esos resultados en la primera ecuación sala que $\lambda_1$ también es 0, y queda probado lo que se buscaba. Luego para convertirlo en una base agrego una matriz canónica de la siguiente manera:
$$
\left \{ 
\begin{bmatrix}
1 & 0\\ 
0 & -1
\end{bmatrix}, 
\begin{bmatrix}
1 & 0\\ 
-2 & 1
\end{bmatrix}, 
\begin{bmatrix}
1 & 2\\ 
3 & 2
\end{bmatrix},
\begin{bmatrix}
0 & 1\\ 
0 & 0
\end{bmatrix}
\right \}
$$
Es LI y forma una base de dimensión 2 de $ M_{2\times 2}(\mathds{R})$.
\newpage



\section{Práctico 9: Transformaciones lineales. Núcleo e Imágen.}
\subsection{Ejercicio 1}
Sean $V$,$W$,$U$ $\mathds{K}$-espacios vectoriales y supongamos que $T:V\rightarrow W, \ S: W \rightarrow U$ son transformaciones lineales. Demostrar que $S\circ T$ y $\lambda T$ son transformaciones lineales. \\
Para probar que $S\circ T$ es transformación lineal, hay que probar que cumple las dos propiedades de transformación lineal:
\begin{itemize}
    \item[(a)] $(S\circ T)(v+v') = (S\circ T)(v)+(S\circ T)(v')$
    \item[(b)] $(S\circ T)(\lambda v) = \lambda (S\circ T)(v)$
\end{itemize}
\subsubsection{Prueba A}
$$
(S\circ T)(v+v') = S(T(v+v'))
$$
$$
= S(T(v)+T(v')) = S(T(v))+S(T(v')) = (S\circ T)(v)+(S\circ T)(v')
$$
\subsubsection{Prueba B}
$$
(S\circ T)(\lambda v) = S(T(\lambda v))
$$
$$
= S(\lambda T(v)) = \lambda S(T(v)) = \lambda (S\circ T)(v)
$$
Con esto queda probado que $S\circ T$ \textbf{es transformación lineal}. \\


\subsection{Ejercicio 2} 
¿Cuáles de las siguientes funciones de $\mathds{R}^n\rightarrow\mathds{R}^m$ son transformaciones lineales?
\begin{itemize}
    \item[(a)] $T(x,y)=(1+x,y)$.
    \item[(b)] $T(x,y)=(y,x,x-2y)$.
    \item[(c)] $T(x,y) = xy$
    \item[(d)] $T(x,y,z)=3x-2y+7z$
\end{itemize}
\subsubsection{Conceptos}
\begin{defbox}{Definición}
    Sean $V$ y $W$ dos espacios vectoriales sobre el cuerpo $\mathds{K}$. Una \textit{transformación lineal} de $V$ en $W$ es una función $T:V \to W$  tal que
    \begin{enumerate}
        \item $T(v+v') = T(v)+ T(v')$, para $v,v' \in V$,
        \item $T(\lambda v) = \lambda T(v)$, para $v \in V$, $\lambda \in \mathds{K}$.
    \end{enumerate}
\end{defbox}
\begin{obbox}{Observación}
    $T:V \to W$ es transformación lineal si y sólo si
    \begin{enumerate}
        \item $T(\lambda v+v') = \lambda T(v)+ T(v')$, para $v,v' \in V$, $\lambda \in \mathds{K}$.
    \end{enumerate}
    Algunas veces usaremos esto último para comprobar si una aplicación de $V$ en $W$ es una transformación lineal.
\end{obbox}

\subsubsection{Prueba A}
$$
T((x,y)+\lambda(x',y'))=T((x,y)+(\lambda x', \lambda y')) = T (x+\lambda x', y + \lambda y')
$$
$$
= (1+(x+\lambda x'),y + \lambda y') = (1+x, y) + (\lambda x', \lambda y') = T(x,y) + (\lambda x', \lambda y')
$$
Como no se llega a $T((x,y)+\lambda(x',y')) = T(x,y)+\lambda T(x,y)$, entonces \textbf{no es una transformación lineal}.
\subsubsection{Prueba B}
$$
T((x,y)+\lambda(x',y'))=T((x,y)+(\lambda x', \lambda y')) = T (x+\lambda x', y + \lambda y')
$$
$$
= (y + \lambda y', x+\lambda x', x+\lambda x'-2(y + \lambda y')) = (y,x,x-2y) + (\lambda y', \lambda x', \lambda x'-2\lambda y')
$$
Se llega a $T((x,y)+\lambda(x',y')) = T(x,y)+\lambda T(x,y)$, entonces \textbf{es una transformación lineal}.
\subsubsection{Prueba C}
$$
T((x,y)+\lambda(x',y'))=T((x,y)+(\lambda x', \lambda y')) = T (x+\lambda x', y + \lambda y')
$$
$$
= (x+\lambda x')(y + \lambda y') = xy + \lambda xy' + \lambda x'y + \lambda^2 x'y' = xy + \lambda (xy' + x'y + \lambda xy')
$$
Como no se llega a $T((x,y)+\lambda(x',y')) = T(x,y)+\lambda T(x,y)$, entonces \textbf{no es una transformación lineal}.
\subsubsection{Prueba D}
$$
T(x,y,z)+\lambda(x',y',z')=T((x,y,z)+(\lambda x', \lambda y', \lambda z')) = T (x+\lambda x', y + \lambda y', z + \lambda z')
$$
$$
= 3(x+\lambda x')-2(y + \lambda y')+7(z + \lambda z') = 3x-2y+7z + \lambda (3x'-2y'+7z')
$$
Se llega a $T((x,y,z)+\lambda(x',y',z')) = T(x,y,z)+\lambda T(x,y,z)$, entonces \textbf{es una transformación lineal}.
\subsection{Ejercicio 3}
Consideremos el $\mathds{R}$-espacio vectorial $\mathds{R}_{>0}= \{ x\in\mathds{R} : x>0 \}$ con las operaciones $x \oplus y = x \cdot y$ y $\lambda \odot x = x^{\lambda}$. Demostrar que la función: $$exp:\mathds{R}\rightarrow\mathds{R}_{>0}, \ \ \ \ exp(r) = e^r$$ es una transformación lineal. Donde a $\mathds{R}$ se lo considera con la estructura usuar de $\mathds{R}$-espacio vectorial. \\
Para que sea una transformación lineal debe cumplir:
$$
exp(r+\lambda r')=exp(r)\oplus\lambda exp(r')
$$
$$
= e^{r+\lambda r'} = e^r \cdot e^{\lambda r'}
$$
$$
= e^r \cdot (e^{r'})^{\lambda} = exp(r) \odot \lambda exp(r')
$$
Queda probado que \textbf{es una transformación lineal}.
\subsection{Ejercicio 4}
Sea $V$ un espacio vectorial y $T:V\rightarrow V$ una transformación lineal \textit{idempotente}, es decir que cumple que $T\circ T=T$. Demostrar que $Im(T)\oplus Nu(T)=V$. \\
Hay que demostrar las siguientes afirmaciones:
\begin{itemize}
    \item $Im(T)\cap Nu(T)=\{0\}$: \\
    Tomando $v\in Nu(t) \cap Im(T)$, entonces $v\in Nu(T)$ y $v\in Im(T)$. Por definición de núcleo, $T(v)=0$ y por definición de imagen, $T(v)=w$ con $w\in V$. Entonces $T(v)=0=w$, por lo tanto $v=0$ y queda demostrado que $Im(T)\cap Nu(T)=\{0\}$.
    \item $Im(T)+Nu(T)=V$: \\
    Tomando $v\in V$ y $v\in Im(T)$, entonces $v\in V$ y $v\in Im(T)$. Por definición de imagen, $T(v)=w$ con $w\in V$. Entonces $T(v)=w$, por lo tanto $v=w$ y queda demostrado que $Im(T)+Nu(T)=V$.
\end{itemize}
\subsection{Ejercicio 5}
Para cada una de las siguientes funciones de $\mathbb{C}$ en $\mathbb{C}$, decidir si son $\mathds{R}$-lineales o $\mathbb{C}$-lineales.
\begin{itemize}
    \item[(a)] $T(z)=iz$
    \item[(b)] $R(z)=\overline{z}$
    \item[(c)] $S(z)= Re(z)+Im(z)$.  
\end{itemize}
\subsubsection{Prueba A}
$$
T(z+z')=i(z+z')=iz+iz'=T(z)+T(z')
$$
$$
T(\lambda z)=i(\lambda z)=\lambda iz=\lambda T(z)
$$
Con esto queda probado que \textbf{es $\mathbb{C}$-lineal}.
\subsubsection{Prueba B}
$$
R(z+z')=\overline{z+z'}=\overline{z}+\overline{z'}=R(z)+R(z')
$$
$$
R(\lambda z)=\overline{\lambda z}=\overline{\lambda}\overline{z}=\lambda\overline{z}=\lambda R(z)
$$
Con esto queda probado que \textbf{es $\mathds{R}$-lineal}.
\subsubsection{Prueba C}
$$
S(z+z')=Re(z+z')+Im(z+z')=Re(z)+Re(z')+Im(z)+Im(z')=S(z)+S(z')
$$
$$
S(\lambda z)=Re(\lambda z)+Im(\lambda z)=\lambda Re(z)+\lambda Im(z)=\lambda (Re(z)+Im(z))=\lambda S(z)
$$
Con esto queda probado que \textbf{es $\mathds{R}$-lineal}.
\subsection{Ejercicio 6}
En cada caso, si es posible, dar una transformación lineal $T:\mathds{R}^n\to \mathds{R}^m$ que satisfaga las condiciones exigidas. Si existe, estudiar la unicidad, si no existe, explicar porqué no es posible definirla.
\begin{itemize}
    \item[(a)] $T(0,1)=(1,2,0,0)$, $T(1,0)= (1,1,0,0)$.
    \item[(b)] $T(1,1,1)=(0,1,3)$, $T(1,2,1)=(1,1,3)$, $T(2,1,1)=(3,1,0)$.
\end{itemize}
\subsubsection{Prueba A}
Como $\{ (0,1), (1,0) \}$ forman la base canonica de $\mathds{R}^2$, entonces, para hallar $T(x,y)$, planeto lo siguiente:
$$
T(x,y) = x\cdot T(e_1) + y\cdot T(e_2)
$$
$$
T(x,y)= x\cdot T(1,0) + y\cdot T(0,1) = x\cdot (1,1,0,0) + y\cdot (1,2,0,0) = (x+y, x+2y, 0, 0)
$$
\subsubsection{Prueba B}
Primero hay que probar que $\{ (1,1,1), (1,2,1), (2,1,1) \}$ sea una base de $\mathds{R}^3$:
$$
\begin{bmatrix}
    1 & 1 & 1 \\
    1 & 2 & 1 \\
    2 & 1 & 1
\end{bmatrix} 
\xrightarrow[]{\ldots}
\begin{bmatrix}
    1 & 0 & 0 \\
    0 & 1 & 0 \\
    0 & 0 & 1
\end{bmatrix}
$$
Son LI y por lo tanto generan y son base de $\mathds{R}^3$. Entonces, para hallar $T(x,y,z)$, planeto lo siguiente:
$$
T(x,y,z)= x\cdot T(e_1) + y\cdot T(e_2) + z\cdot T(e_3)
$$
Ahora debo obtener $T(e_1)$, $T(e_2)$ y $T(e_3)$ que cumplan las condiciones:
\begin{itemize}
    \item[($T(e_1)$)] Sea $e_1=(1,0,0)$, entonces $e_1$ se puede escribir como:
    $$
        e_1 = a\cdot (1,1,1) + b\cdot (1,2,1) + c\cdot (2,1,1)
    $$
    De ahi se forma el sistema de ecuaciones:
    $$
    \begin{cases}
        a+b+2c=1 \\
        a+2b+c=0 \\
        a+b+c=0
    \end{cases}
    $$
    Que al reducirlo se llega a:
    $$
    \begin{cases}
        a=-1 \\
        b=0 \\
        c=1
    \end{cases}
    $$
    Entonces $e_1$ se puede escribir como:
    $$
        e_1 = (-1)\cdot (1,1,1) + 0\cdot (1,2,1) + 1\cdot (2,1,1)
    $$
    Entonces:
    $$
        T(e_1) = (-1)\cdot T(1,1,1) + 0\cdot T(1,2,1) + 1\cdot T(2,1,1) = (3,0,-3)
    $$
    \item[($T(e_2)$)] Sea $e_2=(0,1,0)$, entonces $e_2$ se puede escribir como:
    $$
        e_2 = a\cdot (1,1,1) + b\cdot (1,2,1) + c\cdot (2,1,1)
    $$
    De ahi se forma el sistema de ecuaciones:
    $$
    \begin{cases}
        a+b+2c=0 \\
        a+2b+c=1 \\
        a+b+c=0
    \end{cases}
    $$
    Que al reducirlo se llega a:
    $$
    \begin{cases}
        a=-1 \\
        b=1 \\
        c=0
    \end{cases}
    $$
    Entonces $e_2$ se puede escribir como:
    $$
        e_2 = (-1)\cdot (1,1,1) + 1\cdot (1,2,1) + 0\cdot (2,1,1)
    $$
    Entonces:
    $$
        T(e_2) = (-1)\cdot T(1,1,1) + 1\cdot T(1,2,1) + 0\cdot T(2,1,1) = (1,0,0)
    $$
    \item[($T(e_3)$)]
    Sea $e_3=(0,0,1)$, entonces $e_3$ se puede escribir como:
    $$
        e_3 = a\cdot (1,1,1) + b\cdot (1,2,1) + c\cdot (2,1,1)
    $$
    De ahi se forma el sistema de ecuaciones:
    $$
    \begin{cases}
        a+b+2c=0 \\
        a+2b+c=0 \\
        a+b+c=1
    \end{cases}
    $$
    Que al reducirlo se llega a:
    $$
    \begin{cases}
        a=3 \\
        b=-1 \\
        c=-1
    \end{cases}
    $$
    Entonces $e_3$ se puede escribir como:
    $$
        e_3 = 3\cdot (1,1,1) + (-1)\cdot (1,2,1) + (-1)\cdot (2,1,1)
    $$
    Entonces:
    $$
        T(e_3) = 3\cdot T(1,1,1) + (-1)\cdot T(1,2,1) + (-1)\cdot T(2,1,1) = (-4,1,6)
    $$
\end{itemize}
Por lo tanto:
$$
T(x,y,z)= x\cdot T(e_1) + y\cdot T(e_2) + z\cdot T(e_3)
$$
$$
= x\cdot (3,0,-3) + y\cdot (1,0,0) + z\cdot (-4,1,6) = (3x+y-4z,z,-3x+6z)
$$  

\subsection{Ejercicio 7}
Sea $T:\mathds{R}^6\rightarrow\mathds{R}^2$ una transformación lineal suryectiva y $W\subseteq \mathds{R}^6$ un subespacio de dimensión 3. Demostrar que existe un $w\in W$ con $w\neq 0$ tal que $T(w)=0$. \\
Por ser $T$ suryectiva, $Im(T)=\mathds{R}^2$, por lo tanto, $dim(Im(T))=2$. Por el teorema de la dimensión, $dim(Nu(T))=dim(\mathds{R}^6)-dim(Im(T))=6-2=4$. \\
Por otro lado, $dim(W)=3$, entonces $W=\left \langle w_1,w_2,w_3 \right \rangle$. Entonces $w_1,w_2,w_3$ son LI, por lo tanto, $dim(\left \langle w_1,w_2,w_3 \right \rangle)=3$ y $T(w)$ es un subespacio de $ImT$ generado por $\{ T(w_1), T(w_2), T(w_3) \}$, pero $dimImT=2$, entonces, el conjunto $\{ T(w_1), T(w_2), T(w_3) \}$ es LD. Esto quiere decir que: \\
$$
\exists \lambda_1,\lambda_2,\lambda_3 \in \mathds{R} \text{ tal que } \lambda_1T(w_1)+\lambda_2T(w_2)+\lambda_3T(w_3)=0
$$
Pero como $T$ es transformación lineal, se puede sacar factor común:
$$
T(\lambda_1w_1+\lambda_2w_2+\lambda_3w_3)=0
$$
Entonces, $\lambda_1w_1+\lambda_2w_2+\lambda_3w_3$ es un vector de $W$ tal que $T(w)=0$ y $w\neq 0$.
\subsection{Ejercicio 8}
Dar una transformación lineal $T:\mathds{R}^3\rightarrow\mathds{R}^3$ tal que su imagen sea el subespacio generado por $(1,0,-1)$ y $(1,2,2)$. Hallar $T(x,y,z)$. \\
Dada $\beta = \{ e_1,e_2,e_3 \}$ base canónica de $\mathds{R}^3$, se puede escribir $T$ como:
$$
T(e_1)=(1,0,-1), \ T(e_2)=(1,2,2), \ T(e_3)=(0,0,0)
$$
Entonces, $T(x,y,z)$ se puede escribir como:
$$
T(x,y,z)=T(xe_1+ye_2+ze_3)=xT(e_1)+yT(e_2)+zT(e_3)
$$
$$
=x(1,0,-1)+y(1,2,2)+z(0,0,0)=(x+y,2y,-x+2y)
$$
\subsection{Ejercicio 10}
Definí una transformación $T:\mathds{R}^3\to\mathds{R}^{2\times 2}$ tal que $Nu(T)= \{ (x,y,z) : z=2x=u \}$ e $Im(T)= \left \{  \begin{bmatrix} a & b \\ c & d \end{bmatrix} : \ b =a-c,  b-d=a+c \right \} $. Hallar $T(x,y,z)$. \\
La imagen está definida por dos ecuaciones, al despejar $d$ de la segunda ecuación y reemplazando b en la segunda, se obtiene que $d=-2c$. Entonces, la imagen se puede escribir como:
$$
Im(T)= \left \{  \begin{bmatrix} a & a-c \\ c & -2c \end{bmatrix} : \ a,c \in \mathds{R} \right \}
$$
$$
= \left \{ a\begin{bmatrix} 1 & 1 \\ 0 & 0 \end{bmatrix} + c\begin{bmatrix} 0 & -1 \\ 1 & -2 \end{bmatrix} : \ a,c \in \mathds{R} \right \}
$$
$$
= \left \langle \begin{bmatrix} 1 & 1 \\ 0 & 0 \end{bmatrix}, \begin{bmatrix} 0 & -1 \\ 1 & -2 \end{bmatrix} \right \rangle
$$
El núcleo está definido por dos ecuaciones, reemplazando $y$ y $z$ por dos x se obtiene:
$$
Nu(T)= \{ (x,y,z) : z=2x=u \} = \{ (x,2x,2x) : x \in \mathds{R} \}
$$
$$
= \left \langle (1,2,2) \right \rangle
$$
Hay que verificar que se cumpla el teorema de las dimensiones, es decir que $dim\mathds{R}^3=dim(NuT)+dim(ImT)$, se cumple ya que $3=1+2$. Entonces, $Nu(T)$ y $Im(T)$ son subespacios de $\mathds{R}^3$ y $\mathds{R}^{2\times 2}$ respectivamente. \\
Como $NuT = \left \langle (1,2,2) \right \rangle$, hay que tomar una base que contenga al vector $(1,2,2)$. Completo la base con los vectores de la base canónica:
$$
\mathcal{B}= \{ (1,2,2), (1,0,0), (0,0,1) \}
$$
La transformación $T$ se puede escribir como:
$$
T(1,2,2)= \begin{bmatrix} 0 & 0 \\ 0 & 0 \end{bmatrix}, \ T(1,0,0)= \begin{bmatrix} 1 & 1 \\ 0 & 0 \end{bmatrix}, \ T(0,0,1)= \begin{bmatrix} 0 & -1 \\ 1 & -2 \end{bmatrix}
$$
Ahora para obtener $T(x,y,z)$ hay que escribir las coordenadas en la base utilizada:
$$
(x,y,z)=\lambda_1 (1,0,0) +\lambda_2 (0,0,1) + \lambda_3 (1,2,2)
$$
$$
= (\lambda_1, 0,0) + (0,0,\lambda_2) + (\lambda_3, 2\lambda_3, 2\lambda_3)
$$
$$
= (\lambda_1+\lambda_3, 2\lambda_3, \lambda_2+2\lambda_3)
$$
Se forma el siguiente sistema de ecuaciones:
$$
\begin{cases}
    \lambda_1+\lambda_3=x \\
    2\lambda_3=y \\
    \lambda_2+2\lambda_3=z
\end{cases}
$$
Que al reducirlo se llega a:
$$
\begin{cases}
    \lambda_1=x-y/2 \\
    \lambda_2=z-y \\
    \lambda_3=y/2
\end{cases}
$$
Entonces:
$$
T(x,y,z)=T(\lambda_1,\lambda_2,\lambda_3)=T(x-y/2,z-y,y/2)
$$
$$
= (x-y/2)\begin{bmatrix} 1 & 1 \\ 0 & 0 \end{bmatrix} + (z-y)\begin{bmatrix} 0 & -1 \\ 1 & -2 \end{bmatrix} + (y/2)\begin{bmatrix} 0 & 0 \\ 0 & 0 \end{bmatrix}
$$
$$
= \begin{bmatrix} x-y/2 & x-y/2-(z-y) \\ z-y & -2(z-y)  \end{bmatrix}.
$$
\subsection{Ejercicio 11}
Probar que no existe transformació lineal $T:\mathds{R}^3\rightarrow R^{2\times 2}$ tal que $Nu(T)= \{ (x,y,z) : z = 2x = y \}$ e $Im(T)= \left \{ \begin{bmatrix} a & b \\ c & d \end{bmatrix} : \ b-d=a+c \right \}$. \\
Por el teorema de las dimensiones: $dim\mathds{R}^3=dim(NuT)+dim(ImT)$, en este caso la imagen y el núcleo estan definidos como: 
$$
Nu(T)= \{ (x,y,z) : z = 2x = y \} = \{ (x,2x,2x) : x \in \mathds{R} \}
$$
$$
= \left \langle (1,2,2) \right \rangle
$$
$$
Im(T)= \left \{ \begin{bmatrix} a & b \\ c & d \end{bmatrix} : \ b-d=a+c \right \}
$$
$$
\left \langle \begin{bmatrix} 1 & 1 \\ 0 & 0 \end{bmatrix}, \begin{bmatrix} 0 & 1 \\ 1 & 0 \end{bmatrix}, \begin{bmatrix} 0 & 1 \\ 0 & 1 \end{bmatrix} \right \rangle
$$
Entonces, $dim(NuT)=1$ y $dim(ImT)=3$, pero $dim\mathds{R}^3=3\neq1+3$, por lo tanto, \textbf{no existe transformación lineal} que cumpla las condiciones.
\subsection{Ejercicio 12}
Sea $T:M_2(\mathds{R})\to\mathds{R}_3[x]$ la función dada por:
$$
T \left ( \begin{bmatrix} a & b \\ c & d \end{bmatrix} \right ) = (a-d)x^2+cz+(a+b+c+d).
$$
\begin{itemize}
    \item[(a)] Probar que $T$ es una transformación lineal.
    \item[(b)] Calcular la dimensión del núcleo de T.
\end{itemize}
\subsubsection{Prueba A}
Para demostrar que es una transformación lineal debo probar la suma y el producto por escalar:
$$
T \left ( \begin{bmatrix} a & b \\ c & d \end{bmatrix} + \begin{bmatrix} a' & b' \\ c' & d' \end{bmatrix} \right ) = T \left ( \begin{bmatrix} a+a' & b+b' \\ c+c' & d+d' \end{bmatrix} \right )
$$
$$
= (a+a'-(d+d'))x^2+(c+c')z+(a+a'+b+b'+c+c'+d+d')
$$
$$
= (a-d)x^2+cz+(a+b+c+d)+(a'-d')x^2+c'z+(a'+b'+c'+d')
$$
$$
= T \left ( \begin{bmatrix} a & b \\ c & d \end{bmatrix} \right ) + T \left ( \begin{bmatrix} a' & b' \\ c' & d' \end{bmatrix} \right )
$$
$$
T \left ( \lambda \begin{bmatrix} a & b \\ c & d \end{bmatrix} \right ) = T \left ( \begin{bmatrix} \lambda a & \lambda b \\ \lambda c & \lambda d \end{bmatrix} \right )
$$
$$
= (\lambda a-\lambda d)x^2+(\lambda c)z+(\lambda a+\lambda b+\lambda c+\lambda d)
$$
$$
= \lambda (a-d)x^2+\lambda cz+\lambda (a+b+c+d)
$$
$$
= \lambda T \left ( \begin{bmatrix} a & b \\ c & d \end{bmatrix} \right )
$$
Con esto queda probado que \textbf{es una transformación lineal}.
\subsubsection{Prueba B}
Para calcular la dimensión del núcleo, voy a tomar la imagen y calcular una base que la genere, para luego utilizar el teorema de las dimensiones. \\
Tomando la base canonica que genera a $M_2(\mathds{R})$:
$$
\mathcal{B}=\left \{ \begin{bmatrix} 1 & 0 \\ 0 & 0 \end{bmatrix}, \begin{bmatrix} 0 & 1 \\ 0 & 0 \end{bmatrix}, \begin{bmatrix} 0 & 0 \\ 1 & 0 \end{bmatrix}, \begin{bmatrix} 0 & 0 \\ 0 & 1 \end{bmatrix} \right \}
$$
Entonces, aplico la transformación para obtener la salida:
$$
T \left ( \begin{bmatrix} 1 & 0 \\ 0 & 0 \end{bmatrix} \right ) = (1-0)x^2+0x+(1+0+0+0)=x^2+1
$$
$$
T \left ( \begin{bmatrix} 0 & 1 \\ 0 & 0 \end{bmatrix} \right ) = (0-0)x^2+0x+(0+1+0+0)=1
$$
$$
T \left ( \begin{bmatrix} 0 & 0 \\ 1 & 0 \end{bmatrix} \right ) = (0-0)x^2+1x+(0+0+1+0)=x+1
$$
$$
T \left ( \begin{bmatrix} 0 & 0 \\ 0 & 1 \end{bmatrix} \right ) = (0-1)x^2+0x+(0+0+0+1)=-x^2+1
$$
Planteando la combinacion lineal:
$$
\lambda_1(x^2+1)+\lambda_2(1)+\lambda_3(x+1)+\lambda_4(-x^2+1)=0
$$
$$
(\lambda_1-\lambda_4)x^2+\lambda_3x+(\lambda_1+\lambda_2+\lambda_3+\lambda_4)=0
$$
Se forma el sistema de ecuaciones:
$$
\begin{cases}
    \lambda_1-\lambda_4=0 \\
    \lambda_3=0 \\
    \lambda_1+\lambda_2+\lambda_3+\lambda_4=0
\end{cases}
$$
Que da como resultado:
$$
\begin{cases}
    \lambda_1=\lambda_4 \\
    \lambda_3=0 \\
    \lambda_2=-2\lambda_4 \\
    \lambda_4=\lambda_4
\end{cases}
$$
Esto quiere decir que el conjunto es linealmente dependiente. Se puede notar que $-x^2+1=(-1)(x^2+1)+2\cdot 1$, por lo tanto, el conjunto se puede reescribir sin el vector que es combinacion lineal de los otros dos:
$$
\mathcal{B}=\left \{ \begin{bmatrix} 1 & 0 \\ 0 & 0 \end{bmatrix}, \begin{bmatrix} 0 & 1 \\ 0 & 0 \end{bmatrix}, \begin{bmatrix} 0 & 0 \\ 1 & 0 \end{bmatrix} \right \}
$$
Luego vuelvo a probar que los vectores resultantes sean linealmente independientes:
$$
\lambda_1 (x^2+1) + \lambda_2 (1) + \lambda_3 (x+1) = 0
$$
$$
(\lambda_1)x^2+(\lambda_3)x+(\lambda_1+\lambda_2+\lambda_3)=0
$$
Se forma el sistema de ecuaciones:
$$
\begin{cases}
    \lambda_1=0 \\
    \lambda_3=0 \\
    \lambda_1+\lambda_2+\lambda_3=0
\end{cases}
$$
Que da como resultado:
$$
\begin{cases}
    \lambda_1=0 \\
    \lambda_3=0 \\
    \lambda_2=0
\end{cases}
$$
Entonces, el conjunto es linealmente independiente y por lo tanto, es una base de $Im(T)$. Entonces, $dim(Im(T))=3$. \\
Ahora bien, aplicando el teorema de las dimensiones, se llega a que:
$$
dim(M_2(\mathds{R}))=dim(NuT)+dim(ImT)
$$
$$
dim(M_2(\mathds{R}))=4=dim(NuT)+3 \Rightarrow dim(NuT)=1
$$
\textbf{la dimensión del núcleo es 1}.
\subsection{Ejercicio 13}
Consideremos $C^2([0,1])$ el $\mathds{R}$-espacio vectorial de funciones $f:[0,1]\to\mathds{R}$ que son dos veces diferenciables con $f^{(2)}$ continua y $C^1([0,1])$ el $\mathds{R}$-espacio vectorial de funciones $f:[0,1]\to\mathds{R}$ que son diferenciables tal que $f'$ continua.
\begin{itemize}
    \item[(a)] Demostrar que las funciones $D:C^2([0,1])\to C^1([0,1])$, $I:C^1([0,1])\to C^2([0,1])$ dadas por
    $$
    D(f)(x)=f'(x) \ , \ \ \ \ I(f)(x)=\int_0^x f(t)dt
    $$
    son transformaciones lineales.
    \item[(b)] Demostrar que $D\circ I = Id$ pero que $I\circ D \neq Id$.
\end{itemize}
\subsubsection{Prueba A}
Para demostrar que son transformaciones lineales, debo probar la suma y el producto por escalar:
$$
D(f+g)(x)=(f+g)'(x)=f'(x)+g'(x)=D(f)(x)+D(g)(x)
$$
$$
D(\lambda f)(x)=(\lambda f)'(x)=\lambda f'(x)=\lambda D(f)(x)
$$
$$
I(f+g)(x)=\int_0^x (f+g)(t)dt=\int_0^x f(t)dt+\int_0^x g(t)dt=I(f)(x)+I(g)(x)
$$
$$
I(\lambda f)(x)=\int_0^x (\lambda f)(t)dt=\lambda \int_0^x f(t)dt=\lambda I(f)(x)
$$
Con esto queda probado que \textbf{son transformaciones lineales}.
\subsubsection{Prueba B}
$$
D(I(f))(x)=D(\int_0^x f(t)dt)=\frac{d}{dx} \int_0^x f(t)dt=f(x)
$$
$$
I(D(f))(x)=I(f')(x)=\int_0^x f'(t)dt=f(x)-f(0)
$$
Entonces, $D\circ I = Id$ y $I \circ D \neq Id$ \\
\subsection{Ejercicio 19}
Dar una base del núcleo y caracterizar por ecuaciones la imagen de las siguientes transformaciones lineales:
\begin{itemize}
    \item[(a)] $T:\mathds{K}^3\to\mathds{K}^3$, $T(x,y,z)=(x+2y+3z,y-z,0)$. 
    \item[(b)] $T:\mathds{R}_3[x]\to \mathds{R}^2$, $T(p(x))= (p(1), p(2))$.
\end{itemize}
\subsubsection{Punto A}
Para hallar el núcleo, hay que plantear la definición:
$$
NuT = \{ (x,y,z) \in \mathds{K}^3 : T(x,y,z)=(0,0,0) \}
$$
$$
= \{ (x,y,z) \in \mathds{K}^3 : (x+2y+3z,y-z,0)=(0,0,0) \}
$$
$$
= \{ (x,y,z) \in \mathds{K}^3 : x+2y+3z=0 \text{ y } y-z=0 \}
$$
$$
= \{ (x,y,z) \in \mathds{K}^3 : x+2y+3z=0 \text{ y } y=z \}
$$
Despejando $x$ en la primera ecuación y luego reemplazando en el vector:
$$
NuT = \{ (-2y-3z,y,z) \in \mathds{K}^3 : y = z \text { y } y,z \in \mathds{K} \}
$$
$$
= \{ z(-2-3,1,1) \in \mathds{K}^3 : z \in \mathds{K} \}
$$
$$
NuT = \left \langle (-5,1,1) \right \rangle
$$
Para hallar la imagen, hay que plantear la definición:
$$
ImT = \{ y \in \mathds{K}^3 : \exists x \in \mathds{K}^3 \text{ tal que } T(x)=y \}
$$
$$
= \{ y \in \mathds{K}^3 : \exists x \in \mathds{K}^3 \text{ tal que } (x+2y+3z,y-z,0)=y \}
$$
Entonces, un elemento de $y$ pertenece a la imagen si y solo si:
$$
y = (x+2y+3z,y-z,0) = x(1,0,0) + y(2,1,0) + z(3,-1,0)
$$
$$
ImT = \left \langle (1,0,0), (2,1,0), (3,-1,0) \right \rangle
$$
Como $dim(NuT)=1$, entonces el conjunto de arriba es LD:
$$
\begin{bmatrix}
    1 & 0 & 0 \\
    2 & 1 & 0 \\
    3 & -1 & 0
\end{bmatrix}
\xrightarrow[]{f_2-2f_1}
\begin{bmatrix}
    1 & 0 & 0 \\
    0 & 1 & 0 \\
    3 & -1 & 0
\end{bmatrix}
\xrightarrow[]{f_3-3f_1}
\begin{bmatrix}
    1 & 0 & 0 \\
    0 & 1 & 0 \\
    0 & -1 & 0
\end{bmatrix}
\xrightarrow[]{f_3+f_2}
\begin{bmatrix}
    1 & 0 & 0 \\
    0 & 1 & 0 \\
    0 & 0 & 0
\end{bmatrix}
$$
Por lo tanto, la imagen queda definida como:
$$
ImT = \left \langle (1,0,0), (2,1,0) \right \rangle
$$
Para caracterizar la imagen con ecuaciones, coloco cada vector como columna de una matriz y reduzco:
$$
\begin{bmatrix}
    1 & 2 & | \ x \\
    0 & 1 & | \ y \\
    0 & 0 & | \ z
\end{bmatrix}
\xrightarrow[]{f_1-2f_2}
\begin{bmatrix}
    1 & 0 & | \ x-2y \\
    0 & 1 & | \ y \\
    0 & 0 & | \ z
\end{bmatrix}
$$
Entonces, la imagen queda definida como:
$$
ImT = \left \{ (x,y,z) \in \mathds{K}^3 : z = 0 \right \}
$$
\subsubsection{Punto B}
Para hallar el nucleo planteo la definición:
$$
NuT = \{ p(x) \in \mathds{R}_3[x] : T(p(x))=(0,0) \}
$$
$$
= \{ p(x) \in \mathds{R}_3[x] : (p(1),p(2))=(0,0) \}
$$
$$
= \{ p(x) \in \mathds{R}_3[x] : p(1)=0 \text{ y } p(2)=0 \}
$$
$$
= \{ a_0+a_1x+a_2x^2: a_0+a_1+a_2=0 \text{ y } a_0+2a_1+4a_2=0 \}
$$
Con esto, se forma el sistema de ecuaciones homogeneo:
$$
\begin{cases}
    a_0+a_1+a_2=0 \\
    a_0+2a_1+4a_2=0
\end{cases}
$$
Que al reducirlo se llega a:
$$
\begin{cases}
    a_0-2a_2 = 0 \\
    a_1+3a_2 = 0
\end{cases}
$$
Entonces el nucleo pasa a ser:
$$
NuT = \{ a_0+a_1x+a_2x^2: a_0=2a_2 \text{ y } a_1=-3a_2 \}
$$
Reemplazando en el polinomio:
$$
NuT = \{ 2a_2-3a_2x+a_2x^2: a_2\in\mathds{R} \}
$$
Saco $a_2$ como factor común:
$$
NuT = \{ a_2(2-3x+x^2): a_2\in\mathds{R} \}
$$
$$
NuT = \left \langle 2-3x+x^2 \right \rangle
$$
Ahora para encontrar la imagen, planteo la definición nuevamente:
$$
ImT = \{ (x,y) \in \mathds{R}^2 : \exists p(x) \in \mathds{R}_3[x], T(p(x))=(x,y) \}
$$
$$
= \{ (x,y) \in \mathds{R}^2 : \exists p(x) \in \mathds{R}_3[x], (p(1),p(2))=(x,y) \}
$$
$$
= \{ (a_0+a_1+a_2,a_0+2a_1+4a_2) : a_0,a_1,a_2\in\mathds{R} \}
$$
$$
= \{ a_0(1,1) + a_1(1,2) + a_2(1,4) : a_0,a_1,a_2\in\mathds{R} \}
$$
Pero la dimensión de la imagen es 2 por teorema de las dimensiones, entonces el conjunto es LD, para sacar el vector que es comnbinación lineal del resto coloco cada uno como fila de una matriz y reduzco:
$$
\begin{bmatrix}
    1 & 1  \\
    1 & 2  \\
    1 & 4
\end{bmatrix}
\xrightarrow[]{f_2-f_1}
\begin{bmatrix}
    1 & 1  \\
    0 & 1  \\
    1 & 4
\end{bmatrix}
\xrightarrow[]{f_3-f_1}
\begin{bmatrix}
    1 & 1  \\
    0 & 1  \\
    0 & 3
\end{bmatrix}
\xrightarrow[]{f_3-3f_2}
\begin{bmatrix}
    1 & 1  \\
    0 & 1  \\
    0 & 0
\end{bmatrix}
$$
Entonces, la imagen queda definida por:
$$
ImT = \left \langle (1,1), (1,2) \right \rangle
$$
Entonces, la imagen queda definida por todos los pares pertenecientes $\mathds{R}^2$:
$$
ImT = \{ (x,y) \in \mathds{R}^2 : x,y\in\mathds{R} \}
$$
\newpage
\section{Práctico 10: Matriz de una transformación lineal. Coordenadas.}
\subsection{Ejercicio 1}
Los vectores $v_1=(1,0,-i), \ v_2=(1+i,1-i,1), \ v_3=(i,i,i)$ forman una base de $\mathds{C}^3$. Dar las coordenadas de un vector $(x,y,z)$ en esta base. \\    
Se tiene la base:
$$
\mathcal{B}=\{ (1,0,-i), \ (1+i,1-i,1), \ (i,i,i) \}
$$
Para plantear las coordenadas, se debe escribir el vector como combinación lineal de los vectores de la base:
$$
(x,y,z)=x(1,0,-i)+y(1+i,1-i,1)+z(i,i,i)
$$
$$
= (x+y(1+i)+iz, y(i-1)+z, -ix+y+iz)
$$
Entonces, se forma el sistema de ecuaciones:
$$
\begin{cases}
    x+y(1+i)+iz=x \\
    y(i-1)+z=y \\
    -ix+y+iz=z
\end{cases}
$$
Al pasarlo a una matriz ampliada queda:
$$
\left ( \left.\begin{matrix}
    1 & 1+i & i \\ 
    0 & 1-i & i \\ 
    -i & 1 & i
    \end{matrix}\right| \begin{matrix}
    x \\ 
    y \\ 
    z
    \end{matrix}\right )
\xrightarrow[]{f_3+if_1}
\left ( \left.\begin{matrix}
    1 & 1+i & i \\ 
    0 & 1-i & i \\ 
    0 & i & -1+i
    \end{matrix}\right| \begin{matrix}
    x \\ 
    y \\ 
    z+ix
\end{matrix}\right )
\xrightarrow[]{f_2/1-i}
\left ( \left.\begin{matrix}
    1 & 1+i & i \\ 
    0 & 1 & \frac{-1+i}{2} \\ 
    0 & i & -1+i
    \end{matrix}\right| \begin{matrix}
    x \\ 
    \frac{(1+i)y}{2} \\ 
    z+ix
\end{matrix}\right )
$$
$$
\xrightarrow[]{f_3-if_2}
\left ( \left.\begin{matrix}
    1 & 1+i & i \\ 
    0 & 1 & \frac{-1+i}{2} \\ 
    0 & 0 & \frac{-1+3i}{2}
    \end{matrix}\right| \begin{matrix}
    x \\ 
    \frac{(1+i)y}{2} \\ 
    \frac{2ix+(1-i)y+2z}{2}
\end{matrix}\right )
\xrightarrow[]{f_3/(1-3i)/2}
\left ( \left.\begin{matrix}
    1 & 1+i & i \\ 
    0 & 1 & \frac{-1+i}{2} \\ 
    0 & 0 & 1
    \end{matrix}\right| \begin{matrix}
    x \\ 
    \frac{(1+i)y}{2} \\ 
    \frac{(3-i)x-(2+i)y-(1+3i)z}{5}
\end{matrix}\right )
$$
$$
\xrightarrow[]{f_2-\frac{-1+i}{2}f_3}
\left ( \left.\begin{matrix}
    1 & 1+i & i \\ 
    0 & 1 & 0 \\ 
    0 & 0 & 1
    \end{matrix}\right| \begin{matrix}
    x \\ 
    \frac{(1-2i)x+(1+3i)y-(2+i)z}{5} \\ 
    \frac{(3-i)x-(2+i)y-(1+3i)z}{5}
\end{matrix}\right)
\xrightarrow[]{f_1-if_3}
\left ( \left.\begin{matrix}
    1 & 1+i & 0 \\ 
    0 & 1 & 0 \\ 
    0 & 0 & 1
    \end{matrix}\right| \begin{matrix}
    \frac{(4-3i)x-(1-2i)y-(3-i)z}{5} \\ 
    \frac{(1-2i)x+(1+3i)y-(2+i)z}{5} \\ 
    \frac{(3-i)x-(2+i)y-(1+3i)z}{5}
\end{matrix}\right)
$$
$$
\xrightarrow[]{f_1-(1+i)f_2}
\left ( \left.\begin{matrix}
    1 & 0 & 0 \\ 
    0 & 1 & 0 \\ 
    0 & 0 & 1
    \end{matrix}\right| \begin{matrix}
    \frac{(1-2i)x-(1-2i)y-(2-4i)z}{5} \\ 
    \frac{(1-2i)x+(1+3i)y-(2+i)z}{5} \\ 
    \frac{(3-i)x-(2+i)y-(1+3i)z}{5}
\end{matrix}\right)
$$
El resultado es el vector coordenada $(x,y,z)_{\mathcal{B}}$:
$$
\left ( \frac{(1-2i)x-(1-2i)y-(2-4i)z}{5}, \frac{(1-2i)x+(1+3i)y-(2+i)z}{5}, \frac{(3-i)x-(2+i)y-(1+3i)z}{5} \right )
$$
\subsection{Ejercicio 2}
Sea $v=(1,0,-1)\in\mathds{R}^3$ un vector. Hallar $\mathcal{B}$ una base de $\mathds{R}^3$ tal que $[v]_{\mathcal{B}}=(0,1,0)$. \\
Se tiene el vector $v=(1,0,-1)$ y la base ordenada $\mathcal{B}$:
$$
\mathcal{B} = \{ (1,0,0), (1,0,-1), (0,1,0) \}
$$
Pruebo que es una base de $\mathds{R}^3$:
$$
\begin{bmatrix}
    1 & 0 & 0 \\
    1 & 0 & -1 \\
    0 & 1 & 0
\end{bmatrix}
\xrightarrow[]{f_2-f_1}
\begin{bmatrix}
    1 & 0 & 0 \\
    0 & 0 & -1 \\
    0 & 1 & 0
\end{bmatrix}
\xrightarrow[]{f_2 \leftrightarrow f_3}
\begin{bmatrix}
    1 & 0 & 0 \\
    0 & 1 & 0 \\
    0 & 0 & -1
\end{bmatrix}
\xrightarrow[]{(-1)f_3}
\begin{bmatrix}
    1 & 0 & 0 \\
    0 & 1 & 0 \\
    0 & 0 & 1
\end{bmatrix}
$$
Es Linealmente Independiente, genera y $dim\mathds{R}^3=dim\mathcal{B}$ por lo tanto es base de $\mathds{R}^3$. \\
El vector $v$ se puede escribir como combinación lineal de los vectores de la base:
$$
v = 0\cdot(1,0,0)+1\cdot(1,0,-1)+0\cdot(0,1,0)
$$
Entonces, el vector coordenada de $v$ en la base $\mathcal{B}$ es $[v]_{\mathcal{B}}=(0,1,0)$.
\subsection{Ejercicio 3}
Sea $\mathcal{B}= \{ (1,-2,1),(2,-3,3), (-2,2,-3) \}$ un subconjunto del $\mathds{R}$-espacio vectorial $\mathds{R}^3$.
\begin{itemize}
    \item[(a)] Probar que $\mathcal{B}$ es una base de $\mathds{R}^3$.
    \item[(b)] Hallar la matriz de cambio de base de la base canónica $\mathcal{C}$ a $\mathcal{B}$.
    \item[(c)] Hallar las coordenadas, respecto de $\mathcal{B}$, de los vectores $(1,0,1)$ y $(-1,2,1)$.
    \item[(d)] Mas aún, describir $(x,y,z)$ en términos de la base $\mathcal{B}$. 
\end{itemize}
\subsubsection{Punto A}
Para demostrar que $\mathcal{B}$ es una base de $\mathds{R}^3$ debo probar que:
\begin{itemize}
    \item Es Linealmente Independiente y por lo tanto genera: \\
    Para la prueba, coloco cada vector como fila de una matriz y reduzco:
    $$
    \begin{bmatrix}
        1 & -2 & 1 \\
        2 & -3 & 3 \\
        -2 & 2 & -3
    \end{bmatrix}
    \xrightarrow[]{f_2-2f_1}
    \begin{bmatrix}
        1 & -2 & 1 \\
        0 & 1 & 1 \\
        -2 & 2 & -3
    \end{bmatrix}
    \xrightarrow[]{f_3+2f_1}
    \begin{bmatrix}
        1 & -2 & 1 \\
        0 & 1 & 1 \\
        0 & -2 & -1
    \end{bmatrix}
    \xrightarrow[]{f_3+2f_2}
    \begin{bmatrix}
        1 & -2 & 1 \\
        0 & 1 & 1 \\
        0 & 0 & 1
    \end{bmatrix}
    $$
    $$
    \xrightarrow[]{f_2-f_3}
    \begin{bmatrix}
        1 & -2 & 1 \\
        0 & 1 & 0 \\
        0 & 0 & 1
    \end{bmatrix}
    \xrightarrow[]{f_1+2f_2-f_3}
    \begin{bmatrix}
        1 & 0 & 0 \\
        0 & 1 & 0 \\
        0 & 0 & 1
    \end{bmatrix}
    $$
    Entonces, es Linealmente Independiente y genera.
    \item $dim\mathds{R}^3=dim\mathcal{B}$: \\
    $dim\mathds{R}^3=3$ y $dim\mathcal{B}=3$, por lo tanto, es base de $\mathds{R}^3$.
\end{itemize}
\subsubsection{Punto B}
Para hallar la matriz de cambio de base de la base canónica $\mathcal{C}$ a $\mathcal{B}$ primero obtengo la matriz de cambio de base de $\mathcal{B}$ a $\mathcal{C}$:
Se tiene que $[v]_{\mathcal{C}} = P_{\mathcal{B} \mathcal{C}} \cdot [v]_{\mathcal{B}}$. \\
Primero debo expresar los vectores de $\mathcal{B}$ como combinacion lineal de los vectores de la base canónica para obtener las coordenadas:
$$
(1,-2,1) = 1(1,0,0)+(-2)(0,1,0)+1(0,0,1)
$$
$$
(2,-3,3) = 2(1,0,0)+(-3)(0,1,0)+3(0,0,1)
$$
$$
(-2,2,-3) = -2(1,0,0)+2(0,1,0)+(-3)(0,0,1)
$$
Entonces, la matriz de cambio de base de $\mathcal{B}$ a $\mathcal{C}$ es:
$$
P_{\mathcal{B} \mathcal{C}} = \begin{bmatrix}
    1 & 2 & -2 \\
    -2 & -3 & 2 \\
    1 & 3 & -3
\end{bmatrix}
$$
Ahora, para obtener la matriz de cambio de base de $\mathcal{C}$ a $\mathcal{B}$, debo invertir la matriz anterior, ya que lo que se busca es $[v]_{\mathcal{B}}$:
planteo la matriz ampliada:
$$
\left ( \left.\begin{matrix}
    1 & 2 & -2 \\ 
    -2 & -3 & 2 \\ 
    1 & 3 & -3
    \end{matrix}\right| \begin{matrix}
    1 & 0 & 0 \\ 
    0 & 1 & 0 \\ 
    0 & 0 & 1
    \end{matrix}\right )
\xrightarrow[]{f_2+2f_1}
\left ( \left.\begin{matrix}
    1 & 2 & -2 \\ 
    0 & 1 & -2 \\ 
    1 & 3 & -3
    \end{matrix}\right| \begin{matrix}
    1 & 0 & 0 \\ 
    2 & 1 & 0 \\ 
    0 & 0 & 1
    \end{matrix}\right )
\xrightarrow[]{f_3-f_1}
\left ( \left.\begin{matrix}
    1 & 2 & -2 \\ 
    0 & 1 & -2 \\ 
    0 & 1 & -1
    \end{matrix}\right| \begin{matrix}
    1 & 0 & 0 \\ 
    2 & 1 & 0 \\ 
    -1 & 0 & 1
    \end{matrix}\right )
$$
$$
\xrightarrow[]{f_3-f_2}
\left ( \left.\begin{matrix}
    1 & 2 & -2 \\ 
    0 & 1 & -2 \\ 
    0 & 0 & 1
    \end{matrix}\right| \begin{matrix}
    1 & 0 & 0 \\ 
    2 & 1 & 0 \\ 
    -3 & -1 & 1
    \end{matrix}\right )
\xrightarrow[]{f_2+2f_3}
\left ( \left.\begin{matrix}
    1 & 2 & -2 \\ 
    0 & 1 & 0 \\ 
    0 & 0 & 1
    \end{matrix}\right| \begin{matrix}
    1 & 0 & 0 \\ 
    -4 & -1 & 2 \\ 
    -3 & -1 & 1
    \end{matrix}\right )
$$
$$
\xrightarrow[]{f_1+2f_3}
\left ( \left.\begin{matrix}
    1 & 2 & 0 \\ 
    0 & 1 & 0 \\ 
    0 & 0 & 1
    \end{matrix}\right| \begin{matrix}
    -5 & -2 & 2 \\ 
    -4 & -1 & 2 \\ 
    -3 & -1 & 1
    \end{matrix}\right )
\xrightarrow[]{f_1-2f_2}
\left ( \left.\begin{matrix}
    1 & 0 & 0 \\ 
    0 & 1 & 0 \\ 
    0 & 0 & 1
    \end{matrix}\right| \begin{matrix}
    3 & 0 & -2 \\ 
    -4 & -1 & 2 \\ 
    -3 & -1 & 1
    \end{matrix}\right )
$$
Entonces, la matriz de cambio de base de $\mathcal{C}$ a $\mathcal{B}$ es:
$$
P_{\mathcal{C} \mathcal{B}} = \begin{bmatrix}
    3 & 0 & -2 \\ 
    -4 & -1 & 2 \\ 
    -3 & -1 & 1
\end{bmatrix}
$$
\subsubsection{Punto C}
Para hallar las coordenadas de los vectores $(1,0,1)$ y $(-1,2,1)$ en la base $\mathcal{B}$, debo plantear la matriz de cambio de base de $\mathcal{C}$ a $\mathcal{B}$ y multiplicarla por el vector:
$$
[(1,0,1)]_{\mathcal{B}} = \begin{bmatrix}
    3 & 0 & -2 \\ 
    -4 & -1 & 2 \\ 
    -3 & -1 & 1
\end{bmatrix}
\begin{bmatrix}
    1 \\ 
    0 \\ 
    1
\end{bmatrix}
=
\begin{bmatrix}
    1 \\ 
    -2 \\ 
    -2
\end{bmatrix}
$$
$$
[(-1,2,1)]_{\mathcal{B}} = \begin{bmatrix}
    3 & 0 & -2 \\ 
    -4 & -1 & 2 \\ 
    -3 & -1 & 1
\end{bmatrix}
\begin{bmatrix}
    -1 \\ 
    2 \\ 
    1
\end{bmatrix}
=
\begin{bmatrix}
    -5 \\ 
    4 \\ 
    2
\end{bmatrix}
$$
\subsubsection{Punto D}
Para describir $(x,y,z)$ en términos de la base $\mathcal{B}$, debo plantear la matriz de cambio de base de $\mathcal{C}$ a $\mathcal{B}$ y multiplicarla por el vector:
$$
[(x,y,z)]_{\mathcal{B}} = \begin{bmatrix}
    3 & 0 & -2 \\ 
    -4 & -1 & 2 \\ 
    -3 & -1 & 1
\end{bmatrix}
\begin{bmatrix}
    x \\ 
    y \\ 
    z
\end{bmatrix}
=
\begin{bmatrix}
    3x-2z \\ 
    -4x-y+2z \\ 
    -3x-y+z
\end{bmatrix}
$$
\subsection{Ejercicio 4}
Sea $\mathcal{B} = \left \{ \begin{bmatrix} 0 & 2 \\ -3 & 3 \end{bmatrix}, \begin{bmatrix} 0 & -1 \\ 2 & 0 \end{bmatrix}, \begin{bmatrix} 2 & 1 \\ -2 & 1 \end{bmatrix}, \begin{bmatrix} 4 & -2 \\ 2 & -3 \end{bmatrix} \right \}$.
\begin{itemize}
    \item[(a)] Probar que $\mathcal{B}$ es una base de $M_2(\mathds{R})$.
    \item[(b)] Sea $\mathcal{C} = \left \{ \begin{bmatrix} 1 & 0 \\ 0 & 0 \end{bmatrix}, \begin{bmatrix} 0 & 1 \\ 0 & 0 \end{bmatrix}, \begin{bmatrix} 0 & 0 \\ 1 & 0 \end{bmatrix}, \begin{bmatrix} 0 & 0 \\ 0 & 1 \end{bmatrix} \right \}$. Hallar la matriz de cambio de base de $\mathcal{B}$ a $\mathcal{C}$ y la matriz de cambio de base de $\mathcal{C}$ a $\mathcal{B}$.
    \item[(c)] Hallar las coordenadas respecto de $\mathcal{B}$ de las matrices $\begin{bmatrix} 1 & 0 \\ 0 & 1 \end{bmatrix}, \begin{bmatrix} 2 & 5 \\ -7 & 3 \end{bmatrix}, \begin{bmatrix} 10 & -1 \\ -4 & 2 \end{bmatrix}$. 
\end{itemize}
\subsubsection{Punto A}
Para demostrar que $\mathcal{B}$ es una base de $M_2(\mathds{R})$ debo probar que:
\begin{itemize}
    \item Es Linealmente Independiente y por lo tanto genera: \\
    Para la prueba, planteo la definición de un conjunto linealmente independiente:
    $$
    \{ v_1, v_2, v_3, v_4 \} \text{ es Linealmente Independiente } \Leftrightarrow \alpha_1v_1+\alpha_2v_2+\alpha_3v_3+\alpha_4v_4=0 \Rightarrow \alpha_1=\alpha_2=\alpha_3=\alpha_4=0
    $$
    Entonces, coloco cada vector de la matriz como columna de una matriz y reduzco:
    $$
    \begin{bmatrix}
        0 & 0 & 2 & 4 \\
        -3 & -1 & 1 & -2 \\
        2 & 2 & -2 & 2 \\
        3 & 0 & 1 & -3
    \end{bmatrix}
    \xrightarrow[]{f_2\leftrightarrow f_1}
    \begin{bmatrix}
        -3 & -1 & 1 & -2 \\
        0 & 0 & 2 & 4 \\
        2 & 2 & -2 & 2 \\
        3 & 0 & 1 & -3
    \end{bmatrix}
    \xrightarrow[]{f1/(-3)}
    \begin{bmatrix}
        1 & \frac{1}{3} & -\frac{1}{3} & \frac{2}{3} \\
        0 & 0 & 2 & 4 \\
        2 & 2 & -2 & 2 \\
        3 & 0 & 1 & -3
    \end{bmatrix}
    $$
    $$
    \xrightarrow[]{f_3-2f_1}
    \begin{bmatrix}
        1 & \frac{1}{3} & -\frac{1}{3} & \frac{2}{3} \\
        0 & 0 & 2 & 4 \\
        0 & \frac{4}{3} & -\frac{4}{3} & \frac{2}{3} \\
        3 & 0 & 1 & -3
    \end{bmatrix}
    \xrightarrow[]{f_4-3f_1}
    \begin{bmatrix}
        1 & \frac{1}{3} & -\frac{1}{3} & \frac{2}{3} \\
        0 & 0 & 2 & 4 \\
        0 & \frac{4}{3} & -\frac{4}{3} & \frac{2}{3} \\
        0 & -1 & 2 & -5
    \end{bmatrix}
    \xrightarrow[]{f_3 \leftrightarrow f_2}
    \begin{bmatrix}
        1 & \frac{1}{3} & -\frac{1}{3} & \frac{2}{3} \\
        0 & \frac{4}{3} & -\frac{4}{3} & \frac{2}{3} \\
        0 & 0 & 2 & 4 \\
        0 & -1 & 2 & -5
    \end{bmatrix}
    $$
    $$
    \xrightarrow[]{f_2/(4/3)}
    \begin{bmatrix}
        1 & \frac{1}{3} & -\frac{1}{3} & \frac{2}{3} \\
        0 & 1 & -1 & \frac{1}{2} \\
        0 & 0 & 2 & 4 \\
        0 & -1 & 2 & -5
    \end{bmatrix}
    \xrightarrow[]{f_4+f_2}
    \begin{bmatrix}
        1 & \frac{1}{3} & -\frac{1}{3} & \frac{2}{3} \\
        0 & 1 & -1 & \frac{1}{2} \\
        0 & 0 & 2 & 4 \\
        0 & 0 & 1 & -\frac{9}{2}
    \end{bmatrix}
    \xrightarrow[]{f_3/2}
    \begin{bmatrix}
        1 & \frac{1}{3} & -\frac{1}{3} & \frac{2}{3} \\
        0 & 1 & -1 & \frac{1}{2} \\
        0 & 0 & 1 & 2 \\
        0 & 0 & 1 & -\frac{9}{2}
    \end{bmatrix}
    $$
    $$
    \xrightarrow[]{f_4-f_3}
    \begin{bmatrix}
        1 & \frac{1}{3} & -\frac{1}{3} & \frac{2}{3} \\
        0 & 1 & -1 & \frac{1}{2} \\
        0 & 0 & 1 & 2 \\
        0 & 0 & 0 & -\frac{13}{2}
    \end{bmatrix}
    \xrightarrow[]{f_4/(-13/2)}
    \begin{bmatrix}
        1 & \frac{1}{3} & -\frac{1}{3} & \frac{2}{3} \\
        0 & 1 & -1 & \frac{1}{2} \\
        0 & 0 & 1 & 2 \\
        0 & 0 & 0 & 1
    \end{bmatrix}
    \xrightarrow[]{f_3-2f_4}
    \begin{bmatrix}
        1 & \frac{1}{3} & -\frac{1}{3} & \frac{2}{3} \\
        0 & 1 & -1 & \frac{1}{2} \\
        0 & 0 & 1 & 0 \\
        0 & 0 & 0 & 1
    \end{bmatrix}
    $$
    $$
    \xrightarrow[]{f_2-(1/2)f_4}
    \begin{bmatrix}
        1 & \frac{1}{3} & -\frac{1}{3} & \frac{2}{3} \\
        0 & 1 & -1 & 0 \\
        0 & 0 & 1 & 0 \\
        0 & 0 & 0 & 1
    \end{bmatrix}
    \xrightarrow[]{f_1-(2/3)f_4}
    \begin{bmatrix}
        1 & \frac{1}{3} & -\frac{1}{3} & 0 \\
        0 & 1 & -1 & 0 \\
        0 & 0 & 1 & 0 \\
        0 & 0 & 0 & 1
    \end{bmatrix}
    \xrightarrow[]{f_2+f_3}
    \begin{bmatrix}
        1 & \frac{1}{3} & -\frac{1}{3} & 0 \\
        0 & 1 & 0 & 0 \\
        0 & 0 & 1 & 0 \\
        0 & 0 & 0 & 1
    \end{bmatrix}
    $$
    $$
    \xrightarrow[]{f_1+(1/3)f_2}
    \begin{bmatrix}
        1 & 0 & -\frac{1}{3} & 0 \\
        0 & 1 & 0 & 0 \\
        0 & 0 & 1 & 0 \\
        0 & 0 & 0 & 1
    \end{bmatrix}
    \xrightarrow[]{f_1+(1/3)f_3}
    \begin{bmatrix}
        1 & 0 & 0 & 0 \\
        0 & 1 & 0 & 0 \\
        0 & 0 & 1 & 0 \\
        0 & 0 & 0 & 1
    \end{bmatrix}
    $$
    Entonces, es Linealmente Independiente y genera.
    \item $dimM_2(\mathds{R})=dim\mathcal{B}$: \\
    $dimM_2(\mathds{R})=4$ y $dim\mathcal{B}=4$, por lo tanto, es base de $M_2(\mathds{R})$.
\end{itemize}
\subsubsection{Punto B}
Para hallar la matriz de cambio de base de $\mathcal{C}$ a $\mathcal{B}$ primero obtengo la matriz de cambio de base de $\mathcal{B}$ a $\mathcal{C}$:
Se tiene que $[v]_{\mathcal{C}} = P_{\mathcal{B} \mathcal{C}} \cdot [v]_{\mathcal{B}}$. \\
Primero debo expresar los vectores de $\mathcal{B}$ como combinacion lineal de los vectores de la base canónica para obtener las coordenadas:
$$
\begin{bmatrix}
    0 & 2 \\
    -3 & 3
\end{bmatrix}
= 0\begin{bmatrix}
    1 & 0 \\
    0 & 0
\end{bmatrix}
+2\begin{bmatrix}
    0 & 1 \\
    0 & 0
\end{bmatrix}
+(-3)\begin{bmatrix}
    0 & 0 \\
    1 & 0
\end{bmatrix}
+3\begin{bmatrix}
    0 & 0 \\
    0 & 1
\end{bmatrix}
$$
$$
\begin{bmatrix}
    0 & -1 \\
    2 & 0
\end{bmatrix}
= 0\begin{bmatrix}
    1 & 0 \\
    0 & 0
\end{bmatrix}
+(-1)\begin{bmatrix}
    0 & 1 \\
    0 & 0
\end{bmatrix}
+2\begin{bmatrix}
    0 & 0 \\
    1 & 0
\end{bmatrix}
+0\begin{bmatrix}
    0 & 0 \\
    0 & 1
\end{bmatrix}
$$
$$
\begin{bmatrix}
    2 & 1 \\
    -2 & 1
\end{bmatrix}
= 2\begin{bmatrix}
    1 & 0 \\
    0 & 0
\end{bmatrix}
+1\begin{bmatrix}
    0 & 1 \\
    0 & 0
\end{bmatrix}
+(-2)\begin{bmatrix}
    0 & 0 \\
    1 & 0
\end{bmatrix}
+1\begin{bmatrix}
    0 & 0 \\
    0 & 1
\end{bmatrix}
$$
$$
\begin{bmatrix}
    4 & -2 \\
    2 & -3
\end{bmatrix}
= 4\begin{bmatrix}
    1 & 0 \\
    0 & 0
\end{bmatrix}
+(-2)\begin{bmatrix}
    0 & 1 \\
    0 & 0
\end{bmatrix}
+2\begin{bmatrix}
    0 & 0 \\
    1 & 0
\end{bmatrix}
+(-3)\begin{bmatrix}
    0 & 0 \\
    0 & 1
\end{bmatrix}
$$
Entonces, la matriz de cambio de base de $\mathcal{B}$ a $\mathcal{C}$ es:
$$
P_{\mathcal{B} \mathcal{C}} = \begin{bmatrix}
    0 & 0 & 2 & 4 \\
    2 & -1 & 1 & -2 \\
    -3 & 2 & -2 & 2 \\
    3 & 0 & 1 & -3
\end{bmatrix}
$$
Ahora, para obtener la matriz de cambio de base de $\mathcal{C}$ a $\mathcal{B}$, debo invertir la matriz anterior, ya que lo que se busca es $[v]_{\mathcal{B}}$:
planteo la matriz ampliada:
$$
\left ( \left.\begin{matrix}
    0 & 0 & 2 & 4 \\ 
    2 & -1 & 1 & -2 \\ 
    -3 & 2 & -2 & 2 \\
    3 & 0 & 1 & -3
    \end{matrix}\right| \begin{matrix}
    1 & 0 & 0 & 0 \\ 
    0 & 1 & 0 & 0 \\ 
    0 & 0 & 1 & 0 \\
    0 & 0 & 0 & 1
    \end{matrix}\right )
\xrightarrow[]{f_1 \leftrightarrow f_2}
\left ( \left.\begin{matrix}
    2 & -1 & 1 & -2 \\ 
    0 & 0 & 2 & 4 \\ 
    -3 & 2 & -2 & 2 \\
    3 & 0 & 1 & -3
    \end{matrix}\right| \begin{matrix}
    0 & 1 & 0 & 0 \\ 
    1 & 0 & 0 & 0 \\ 
    0 & 0 & 1 & 0 \\
    0 & 0 & 0 & 1
    \end{matrix}\right )
$$
$$
\xrightarrow[]{f_1/2}
\left ( \left.\begin{matrix}
    1 & -\frac{1}{2} & \frac{1}{2} & -1 \\ 
    0 & 0 & 2 & 4 \\ 
    -3 & 2 & -2 & 2 \\
    3 & 0 & 1 & -3
    \end{matrix}\right| \begin{matrix}
    0 & \frac{1}{2} & 0 & 0 \\ 
    1 & 0 & 0 & 0 \\ 
    0 & 0 & 1 & 0 \\
    0 & 0 & 0 & 1
    \end{matrix}\right )
\xrightarrow[]{f_3+3f_1}
\left ( \left.\begin{matrix}
    1 & -\frac{1}{2} & \frac{1}{2} & -1 \\ 
    0 & 0 & 2 & 4 \\ 
    0 & \frac{1}{2} & -\frac{1}{2} & -1 \\
    3 & 0 & 1 & -3
    \end{matrix}\right| \begin{matrix}
    0 & \frac{1}{2} & 0 & 0 \\ 
    1 & 0 & 0 & 0 \\ 
    0 & \frac{3}{2} & 1 & 0 \\
    0 & 0 & 0 & 1
    \end{matrix}\right )
$$
$$
\xrightarrow[]{f_4-3f_1}
\left ( \left.\begin{matrix}
    1 & -\frac{1}{2} & \frac{1}{2} & -1 \\ 
    0 & 0 & 2 & 4 \\ 
    0 & \frac{1}{2} & -\frac{1}{2} & -1 \\
    0 & \frac{3}{2} & -\frac{1}{2} & 0
    \end{matrix}\right| \begin{matrix}
    0 & \frac{1}{2} & 0 & 0 \\ 
    1 & 0 & 0 & 0 \\ 
    0 & \frac{3}{2} & 1 & 0 \\
    0 & -\frac{3}{2} & 0 & 1
    \end{matrix}\right )
\xrightarrow[]{f_3 \leftrightarrow f_2}
\left ( \left.\begin{matrix}
    1 & -\frac{1}{2} & \frac{1}{2} & -1 \\ 
    0 & \frac{1}{2} & -\frac{1}{2} & -1 \\
    0 & 0 & 2 & 4 \\ 
    0 & \frac{3}{2} & -\frac{1}{2} & 0
    \end{matrix}\right| \begin{matrix}
    0 & \frac{1}{2} & 0 & 0 \\ 
    0 & \frac{3}{2} & 1 & 0 \\
    1 & 0 & 0 & 0 \\ 
    0 & -\frac{3}{2} & 0 & 1
    \end{matrix}\right )
$$
$$
\xrightarrow[]{f_2/(1/2)}
\left ( \left.\begin{matrix}
    1 & -\frac{1}{2} & \frac{1}{2} & -1 \\ 
    0 & 1 & -1 & -2 \\
    0 & 0 & 2 & 4 \\ 
    0 & \frac{3}{2} & -\frac{1}{2} & 0
    \end{matrix}\right| \begin{matrix}
    0 & \frac{1}{2} & 0 & 0 \\ 
    0 & 3 & 2 & 0 \\
    1 & 0 & 0 & 0 \\ 
    0 & -\frac{3}{2} & 0 & 1
    \end{matrix}\right )
\xrightarrow[]{f_4-(3/2)f_2}
\left ( \left.\begin{matrix}
    1 & -\frac{1}{2} & \frac{1}{2} & -1 \\ 
    0 & 1 & -1 & -2 \\
    0 & 0 & 2 & 4 \\ 
    0 & 0 & 1 & 3
    \end{matrix}\right| \begin{matrix}
    0 & \frac{1}{2} & 0 & 0 \\ 
    0 & 3 & 2 & 0 \\
    1 & 0 & 0 & 0 \\ 
    0 & -6 & -3 & 1
    \end{matrix}\right )
$$
$$
\xrightarrow[]{f_3/2}
\left ( \left.\begin{matrix}
    1 & -\frac{1}{2} & \frac{1}{2} & -1 \\ 
    0 & 1 & -1 & -2 \\
    0 & 0 & 1 & 2 \\ 
    0 & 0 & 1 & 3
    \end{matrix}\right| \begin{matrix}
    0 & \frac{1}{2} & 0 & 0 \\ 
    0 & 3 & 2 & 0 \\
    \frac{1}{2} & 0 & 0 & 0 \\ 
    0 & -6 & -3 & 1
    \end{matrix}\right )
\xrightarrow[]{f_4-f_3}
\left ( \left.\begin{matrix}
    1 & -\frac{1}{2} & \frac{1}{2} & -1 \\ 
    0 & 1 & -1 & -2 \\
    0 & 0 & 1 & 2 \\ 
    0 & 0 & 0 & 1
    \end{matrix}\right| \begin{matrix}
    0 & \frac{1}{2} & 0 & 0 \\ 
    0 & 3 & 2 & 0 \\
    \frac{1}{2} & 0 & 0 & 0 \\ 
    -\frac{1}{2} & -6 & -3 & 1
    \end{matrix}\right )
$$
$$
\xrightarrow[]{f_3-2f_4}
\left ( \left.\begin{matrix}
    1 & -\frac{1}{2} & \frac{1}{2} & -1 \\ 
    0 & 1 & -1 & -2 \\
    0 & 0 & 1 & 0 \\ 
    0 & 0 & 0 & 1
    \end{matrix}\right| \begin{matrix}
    0 & \frac{1}{2} & 1 & 0 \\ 
    0 & 3 & 2 & 0 \\
    \frac{3}{2} & 12 & 6 & -2 \\ 
    -\frac{1}{2} & -6 & -3 & 1
    \end{matrix}\right )
\xrightarrow[]{f_2+2f_4}
\left ( \left.\begin{matrix}
    1 & -\frac{1}{2} & \frac{1}{2} & -1 \\ 
    0 & 1 & -1 & 0 \\
    0 & 0 & 1 & 0 \\ 
    0 & 0 & 0 & 1
    \end{matrix}\right| \begin{matrix}
    0 & \frac{1}{2} & 1 & 0 \\ 
    -1 & -9 & -4 & 2 \\
    \frac{3}{2} & 12 & 6 & -2 \\ 
    -\frac{1}{2} & -6 & -3 & 1
    \end{matrix}\right )
$$
$$
\xrightarrow[]{f_1+f_4}
\left ( \left.\begin{matrix}
    1 & -\frac{1}{2} & \frac{1}{2} & 0 \\ 
    0 & 1 & -1 & 0 \\
    0 & 0 & 1 & 0 \\ 
    0 & 0 & 0 & 1
    \end{matrix}\right| \begin{matrix}
    -\frac{1}{2} & -\frac{11}{2} & -3 & 1 \\ 
    -1 & -9 & -4 & 2 \\
    \frac{3}{2} & 12 & 6 & -2 \\ 
    -\frac{1}{2} & -6 & -3 & 1
    \end{matrix}\right )
\xrightarrow[]{f_2+f_3}
\left ( \left.\begin{matrix}
    1 & -\frac{1}{2} & 0 & 0 \\ 
    0 & 1 & 0 & 0 \\
    0 & 0 & 1 & 0 \\ 
    0 & 0 & 0 & 1
    \end{matrix}\right| \begin{matrix}
    -\frac{1}{2} & -\frac{11}{2} & -3 & 1 \\ 
    \frac{1}{2} & 3 & 2 & 0 \\
    \frac{3}{2} & 12 & 6 & -2 \\ 
    -\frac{1}{2} & -6 & -3 & 1
    \end{matrix}\right )
$$
$$
\xrightarrow[]{f_1-\frac{1}{2}f_3}
\left ( \left.\begin{matrix}
    1 & -\frac{1}{2} & 0 & 0 \\ 
    0 & 1 & 0 & 0 \\
    0 & 0 & 1 & 0 \\ 
    0 & 0 & 0 & 1
    \end{matrix}\right| \begin{matrix}
    -\frac{5}{4} & -\frac{23}{2} & -3 & 1 \\ 
    \frac{1}{2} & 3 & 2 & 0 \\
    \frac{3}{2} & 12 & 6 & -2 \\ 
    -\frac{1}{2} & -6 & -3 & 1
    \end{matrix}\right )
\xrightarrow[]{f_1+\frac{1}{2}f_2}
\left ( \left.\begin{matrix}
    1 & 0 & 0 & 0 \\ 
    0 & 1 & 0 & 0 \\
    0 & 0 & 1 & 0 \\ 
    0 & 0 & 0 & 1
    \end{matrix}\right| \begin{matrix}
    -1 & -10 & -5 & 2 \\ 
    \frac{1}{2} & 3 & 2 & 0 \\
    \frac{3}{2} & 12 & 6 & -2 \\ 
    -\frac{1}{2} & -6 & -3 & 1
    \end{matrix}\right )
$$
Entonces, la matriz de cambio de base de $\mathcal{C}$ a $\mathcal{B}$ es:
$$
P_{\mathcal{B} \mathcal{C}} = \begin{bmatrix}
    -1 & -10 & -5 & 2 \\ 
    \frac{1}{2} & 3 & 2 & 0 \\
    \frac{3}{2} & 12 & 6 & -2 \\ 
    -\frac{1}{2} & -6 & -3 & 1
\end{bmatrix}
$$
\subsubsection{Punto C}
Para hallar las coordenadas de las matrices $\begin{bmatrix} 1 & 0 \\ 0 & 1 \end{bmatrix}, \begin{bmatrix} 2 & 5 \\ -7 & 3 \end{bmatrix}, \begin{bmatrix} 10 & -1 \\ -4 & 2 \end{bmatrix}$ en la base $\mathcal{B}$, debo plantear la matriz de cambio de base de $\mathcal{C}$ a $\mathcal{B}$ y multiplicarla por el vector: \\
Los vectores coordenada de cada matriz con respecto a la base canonica son:
$$
\begin{bmatrix} 1 & 0 \\ 0 & 1 \end{bmatrix} = \begin{bmatrix} 1 \\ 0 \\ 0 \\ 1 \end{bmatrix}
$$
$$
\begin{bmatrix} 2 & 5 \\ -7 & 3 \end{bmatrix} = \begin{bmatrix} 2 \\ 5 \\ -7 \\ 3 \end{bmatrix}
$$
$$
\begin{bmatrix} 10 & -1 \\ -4 & 2 \end{bmatrix} = \begin{bmatrix} 10 \\ -1 \\ -4 \\ 2 \end{bmatrix}
$$
Ahora multiplicado por la matriz de cambio de base de $\mathcal{C}$ a $\mathcal{B}$:
$$
\left [  \begin{bmatrix} 1 & 0 \\ 0 & 1 \end{bmatrix} \right ]_{\mathcal{B}} =
\begin{bmatrix}
    -1 & -10 & -5 & 2 \\ 
    \frac{1}{2} & 3 & 2 & 0 \\
    \frac{3}{2} & 12 & 6 & -2 \\ 
    -\frac{1}{2} & -6 & -3 & 1
\end{bmatrix}
\begin{bmatrix} 1 \\ 0 \\ 0 \\ 1 \end{bmatrix}
=
\begin{bmatrix} 1 \\ \frac{1}{2} \\ -\frac{1}{2} \\ \frac{1}{2} \end{bmatrix}
$$
$$
\left [  \begin{bmatrix} 2 & 5 \\ -7 & 3 \end{bmatrix} \right ]_{\mathcal{B}} =
\begin{bmatrix}
    -1 & -10 & -5 & 2 \\ 
    \frac{1}{2} & 3 & 2 & 0 \\
    \frac{3}{2} & 12 & 6 & -2 \\ 
    -\frac{1}{2} & -6 & -3 & 1
\end{bmatrix}
\begin{bmatrix} 2 \\ 5 \\ -7 \\ 3 \end{bmatrix}
=
\begin{bmatrix} -11 \\ 2 \\ 15 \\ -7 \end{bmatrix}
$$
$$
\left [  \begin{bmatrix} 10 & -1 \\ -4 & 2 \end{bmatrix} \right ]_{\mathcal{B}} =
\begin{bmatrix}
    -1 & -10 & -5 & 2 \\ 
    \frac{1}{2} & 3 & 2 & 0 \\
    \frac{3}{2} & 12 & 6 & -2 \\ 
    -\frac{1}{2} & -6 & -3 & 1
\end{bmatrix}
\begin{bmatrix} 10 \\ -1 \\ -4 \\ 2 \end{bmatrix}
=
\begin{bmatrix} 24 \\ -6 \\ -25 \\ 15 \end{bmatrix}
$$
    
\end{document}