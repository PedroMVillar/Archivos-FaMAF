%%%%%%%%%%%%%%%%%%%%%%%%%%%%%%%%%%%%%%%%%%%%%%%%%%%%%%%%%%%%%%%%%%%%%%%%%%%%%%%%%%%%
% Configuración de Paquetes
\documentclass{article}
\usepackage[margin=1in]{geometry} 
\usepackage{amsmath,amsthm,amssymb,amsfonts, fancyhdr, color, comment, graphicx, environ}
\usepackage{xcolor}
\usepackage{mdframed}
\usepackage[shortlabels]{enumitem}
\usepackage{indentfirst}
\usepackage{hyperref}
\usepackage{listings}
\usepackage{tikz}
\usepackage[framemethod=TikZ]{mdframed}
\usepackage{mathptmx}
\usepackage{cfr-lm}
\hypersetup{
    colorlinks=true,
    linkcolor=blue,
    filecolor=magenta,      
    urlcolor=blue,
}
\setlength{\headheight}{1.5cm}
\renewcommand{\qed}{\quad\qedsymbol}
 % Incluir configuración de paquetes y encabezado
%%%%%%%%%%%%%%%%%%%%%%%%%%%%%%%%%%%%%%%%%%%%%%%%%%%%%%%%%%%%%%%%%%%%%%%%%%%%%%%%%%%%
% Configuración de listings
\lstdefinelanguage{Haskell}{
  basicstyle=\small\ttfamily,
  keywordstyle=\color{blue},         % Estilo para palabras clave (como if, else, etc.)
  commentstyle=\color{gray},        % Estilo para comentarios
  stringstyle=\color{purple},       % Estilo para cadenas de texto
  literate=
    {->}{{$\rightarrow$}}2
    {λ}{{$\lambda$}}1
    {\\\\}{{\char`\\\char`\\}}1
    {>>}{{>>}}2
    {>>=}{{>>=}}2
    {=<<}{{=<<}}2
    {|}{{$\mid$}}1,
  morekeywords={class, instance, deriving, data, where, let, in, case, of, do},
  sensitive=true,
  morecomment=[l]{--},
  morecomment=[s]{\{-}{-\}},
  morestring=[b]'',
  numbers=left,
  numberstyle=\tiny\color{gray!70}, % Color y tamaño de los números de línea
  numbersep=5pt,                    % Separación entre números de línea y código
  numberstyle=\tiny,
  frame=tb,
  breaklines=true,
  showstringspaces=false,
  backgroundcolor=\color{blue!10},
  emph={::},                        % Resaltar ::
  emphstyle=\color{purple},            % Color para ::
  emph={[2]String, Int, Bool},      % Resaltar nombres de tipos
  emphstyle={[2]\color{red}},     % Color para nombres de tipos
  emph={[3]foldr, map, filter},     % Resaltar nombres de funciones
  emphstyle={[3]\color{red}}     % Color para nombres de funciones
}
\lstnewenvironment{haskell}[1][]
{
    \lstset{language=Haskell,#1}
}
{}

\lstdefinelanguage{C}{
  basicstyle=\small\ttfamily,
  keywordstyle=\color{blue},         % Estilo para palabras clave (como if, else, etc.)
  commentstyle=\color{gray},        % Estilo para comentarios
  stringstyle=\color{purple},       % Estilo para cadenas de texto
  morekeywords={int, char, void, if, else, while, for, return}, % Añade más palabras clave si es necesario
  sensitive=true,
  morecomment=[l]{//},
  morecomment=[s]{/*}{*/},
  numbers=left,
  numberstyle=\tiny,
  frame=single,
  breaklines=true,
  showstringspaces=false,
  backgroundcolor=\color{yellow!20},
}
\lstnewenvironment{c_code}[1][]
{
    \lstset{language=C,#1}
}
{}

\lstdefinelanguage{Python}{
  basicstyle=\small\ttfamily,
  keywordstyle=\color{blue},         % Estilo para palabras clave (como if, else, etc.)
  commentstyle=\color{gray},        % Estilo para comentarios
  stringstyle=\color{purple},       % Estilo para cadenas de texto
  morekeywords={if, elif, else, while, for, in, def, return, True, False, None, import}, % Añade más palabras clave si es necesario
  sensitive=true,
  morecomment=[l]{\#},
  numbers=left,
  numberstyle=\tiny,
  frame=single,
  breaklines=true,
  showstringspaces=false,
  backgroundcolor=\color{yellow!20},
}
\lstnewenvironment{python_code}[1][]
{
    \lstset{language=Python,#1}
}
{}

\lstdefinelanguage{Lean4}{
  basicstyle=\small\ttfamily,
  keywordstyle=\color{blue},         
  commentstyle=\color{gray},        
  stringstyle=\color{purple},       
  morekeywords={def, theorem, lemma, example, axioms, constant, variable, inductive, structure, namespace, section, end, import, namespace, export, private, open},
  sensitive=true,
  morecomment=[l]{--},
  morecomment=[s]{\{-}{-\}},
  morestring=[b]'',
  numbers=left,
  numberstyle=\tiny,
  frame=single,
  breaklines=true,
  showstringspaces=false,
  backgroundcolor=\color{lightgray},
}
\lstnewenvironment{lean_code}[1][]
{
    \lstset{language=Lean4,#1}
}
{}
%%%%%%%%%%%%%%%%%%%%%%%%%%%%%%%%%%%%%%%%%%%%%%%%%%%%%%%%%%%%%%%%%%%%%%%%%%%%%%%%%%%%%%%%%% % Incluir configuración de listings
%%%%%%%%%%%%%%%%%%%%%%%%%%%%%%%%%%%%%%%%%%%%%%%%%%%%%%%%%%%%%%%%%%%%%%%%%%%%%%%%%%%%%%%%%%
%Configuración de Enviroments para cuadros de texto
\newenvironment{problem}[2][Ejercicio]
    { \begin{mdframed}[backgroundcolor=gray!20] \textbf{#1 #2} \\}
    {  \end{mdframed}}

\newenvironment{definition}[2][Definición]
    { \begin{mdframed}[backgroundcolor=red!20] \textbf{#1 #2} \\}
    {  \end{mdframed}}

\newenvironment{theorem}[2][Teorema]
    { \begin{mdframed}[backgroundcolor=green!20] \textbf{#1 #2} \\}
    {  \end{mdframed}}

\newenvironment{lemma}[2][Lema]
    { \begin{mdframed}[backgroundcolor=green!20] \textbf{#1 #2} \\}
    {  \end{mdframed}}

\newenvironment{proposition}[2][Proposición]
    { \begin{mdframed}[backgroundcolor=green!20] \textbf{#1 #2} \\}
    {  \end{mdframed}}

\newenvironment{corollary}[2][Corolario]
    { \begin{mdframed}[backgroundcolor=green!20] \textbf{#1 #2} \\}
    {  \end{mdframed}}

\newenvironment{example}[2][Ejemplo]
    { \begin{mdframed}[backgroundcolor=yellow!20] \textbf{#1 #2} \\}
    {  \end{mdframed}}

\newenvironment{remark}[2][Observación]
    { \begin{mdframed}[backgroundcolor=yellow!20] \textbf{#1 #2} \\}
    {  \end{mdframed}}

\newenvironment{note}
    {\textit{Nota:}}
    {}

\newenvironment{conclusion}
    {\textit{Conclusión:}}
    {}

\newenvironment{notation}
    {\textit{Notación:}}
    {}

\newenvironment{question}
    {\textit{Pregunta:}}
    {}

\newenvironment{answer}
    {\textit{Respuesta:}}
    {}
%%%%%%%%%%%%%%%%%%%%%%%%%%%%%%%%%%%%%%%%%%%%%%%%%%%%%%%%%%%%%%%%%%%%%%%%%%%%%%%%%%%%%%%%%% % Incluir configuración de mdframed
%%%%%%%%%%%%%%%%%%%%%%%%%%%%%%%%%%%%%%%%%%%%%%%%%%%%%%%%%%%%%%%%%%%%%%%%%%%%%%%%%%%%

\pagestyle{fancy}

%Configuraciones adicionales
\binoppenalty=\maxdimen 
\relpenalty=\maxdimen 

%%%%%%%%%%%%%%%%%%%%%%%%%%%%%%%%%%%%%%%%%%%%%
%Condiguracion de encabezado y pie de página
\lhead{Ejercicios de práctica para final}
\rhead{Pedro Villar} %Sección 
%%%%%%%%%%%%%%%%%%%%%%%%%%%%%%%%%%%%%%%%%%%%%

\begin{document}

\section{Programación Funcional}

\subsection{Ejercicios sobre proyectos} 

\subsubsection{Ejercicio 1}
Con los siguientes datos, definidos en el \texttt{Ejercicio 4} del \texttt{Proyecto 2}
\begin{haskell}
-- Sinonimos de tipo
type Altura = Int
type NumCamiseta = Int
-- Tipos algebraicos sin parametros (aka enumerados)
data Zona = Arco | Defensa | Mediocampo | Delantera
data TipoReves = DosManos | UnaMano
data Modalidad = Carretera | Pista | Monte | BMX
data PiernaHabil = Izquierda | Derecha
-- Sinonimo
type ManoHabil = PiernaHabil
-- Deportista es un tipo algebraico con constructores parametricos
data Deportista = Ajedrecista                        -- Constructor sin argumentos
    | Ciclista Modalidad                             -- Constructor con un argumento
    | Velocista Altura                               --  Constructor con un argumento
    | Tenista TipoReves ManoHabil Altura             -- Constructor con tres argumentos
    | Futbolista Zona NumCamiseta PiernaHabil Altura -- Constructor con ...
\end{haskell}
\begin{enumerate}
    \item Definir las siguientes funciones:
        \begin{itemize}
            \item \texttt{esDeportista :: Deportista -> Bool} que dado un deportista, nos dice si es un deportista.
            \item \texttt{esCiclista :: Deportista -> Bool} que dado un deportista, nos dice si es un ciclista.
            \item \texttt{esVelocista :: Deportista -> Bool} que dado un deportista, nos dice si es un velocista.
            \item \texttt{esTenista :: Deportista -> Bool} que dado un deportista, nos dice si es un tenista.
            \item \texttt{esFutbolista :: Deportista -> Bool} que dado un deportista, nos dice si es un futbolista.
        \end{itemize}
    \item Definir las siguientes funciones:
        \begin{itemize}
            \item \texttt{contarCiclistas :: [Deportista] -> Int} que dado una lista de deportistas, nos dice cuantos ciclistas hay.
            \item \texttt{cuantosVelocistasAltos :: [Deportista] -> Int -> Int} que dado una lista de deportistas y una altura, nos dice cuantos velocistas hay de esa altura o mayor.
            \item \texttt{contarTenistasDiestros :: [Deportista] -> Int} que dado una lista de deportistas, nos dice cuantos tenistas diestros hay.
            \item \texttt{cuantosZurdosDefensores :: [Deportista] -> Int} que dado una lista de deportistas, nos dice cuantos zurdos defensores hay.
        \end{itemize}
\end{enumerate}

\begin{note}
    Para resolver este ejercicio, utilizar pattern matching y no utilizar igualdad de tipos.
\end{note}

\newpage

\subsubsection{Ejercicio 2}
En el ejercicio 6 del \texttt{Proyecto 2}, definimos a la función \texttt{primerElemento} como:
\begin{haskell}
primerElemento :: [a] -> Maybe a
primerElemento [] = Nothing
primerElemento (x:_) = Just x
\end{haskell}
Definir ahora las siguientes funciones:
\begin{itemize}
    \item \texttt{tercerElemento :: [a] -> Maybe a} que dado una lista, nos devuelve el tercer elemento de la lista si es que existe.
    \item \texttt{ultimoElemento :: [a] -> Maybe a} que dado una lista, nos devuelve el último elemento de la lista si es que existe.
    \item \texttt{elementoEn :: Int -> [a] -> Maybe a} que tome una lista y un índice, y devuelva \texttt{Just x} donde \texttt{x} es el elemento en la posición indicada, o Nothing si el índice está fuera de rango.
    \item \texttt{maximo :: Ord a => [a] -> Maybe} que devuelva el máximo elemento de una lista no vacía, o \texttt{Nothing} si la lista es vacía.
    \item \texttt{filtrarYTomar :: (a -> Bool) -> [a] -> Maybe a} que tome una función de predicado y una lista, y devuelva \texttt{Just x} donde \texttt{x} es el primer elemento que cumple el predicado, o \texttt{Nothing} si ningún elemento lo cumple.
\end{itemize}

\subsubsection{Ejercicio 3}
Con el tipo definido 
\begin{haskell}
data Cola = VaciaC | Encolada Deportista Cola
\end{haskell}
Definir las siguientes funciones:
\begin{itemize}
    \item \texttt{esVacia :: Cola -> Bool} que dado una cola, nos dice si está vacía.
    \item \texttt{longitud :: Cola -> Int} que dado una cola, nos devuelve la cantidad de deportistas encolados.
    \item \texttt{filtrarCola :: (Deportista -> Bool) -> Cola -> Cola} que dado una función de predicado y una cola, nos devuelve una cola con los deportistas que cumplen el predicado.
\end{itemize}

\subsubsection{Ejercicio 4}
Con el tipo definido como
\begin{haskell}
data Arbol a = Hoja | Rama (Arbol a) a (Arbol a)
\end{haskell}
y teniendo en cuenta el ejercicio 10 del \texttt{Proyecto 2++} sobre árboles binarios, definir las siguientes funciones:
\begin{itemize}
    \item \texttt{a\_eliminar :: Ord a => a -> Arbol a -> Arbol a} que tome un elemento y un árbol de búsqueda binaria y devuelva un nuevo árbol que no contenga el elemento.
    \item \texttt{a\_espejo :: Arbol a -> Arbol a} que tome un árbol y devuelva su espejo, es decir, un árbol con todas las ramas reflejadas.
    \item \texttt{a\_hojaMasProfunda :: Arbol a -> Maybe a} que devuelva el valor almacenado en la hoja más profunda del árbol.
\end{itemize}

\newpage
\subsection{Ejercicios extra}

\subsubsection{Ejercicio 1}
Programar la función
\begin{haskell}
estaEnDNI :: Int -> Bool
\end{haskell}
que dado un número, devuelve \texttt{True} si es una de las cifras de tu DNI.

\subsubsection{Ejercicio 2}
Programa mediante composición de funciones la función 
\begin{haskell}
sumaDNI :: [Int] -> Int
\end{haskell}
que dada una lista de enteros \texttt{xs} suma solo los elementos que son cifras de tu DNI.

\begin{note}
    Para los ejercicios \texttt{1} y \texttt{2}, el DNI no se toma como entrada de datos, sino que se utiliza directamente.
    \newline Para el ejercicio \texttt{2}, se puede utilizar la función \texttt{sumatoria'} definida en el \texttt{Proyecto 1}.
\end{note}

\subsubsection{Ejercicio 3}
Programar la función
\begin{haskell}
    suma_multiplos :: [Int] -> Int -> Int
\end{haskell}
que dada una lista de enteros \texttt{xs} y un entero \texttt{n}, devuelve la suma de los elementos de \texttt{xs} que son múltiplos de \texttt{n}.
La función se debe programar por composición, \textbf{sin utilizar recursión}, eligiendo una de estas alternativas:
\begin{itemize}
    \item Usar \texttt{filter} y \texttt{sumatoria} definida en el \texttt{Proyecto 1}.
    \begin{haskell}
        sumatoria :: [Int] -> Int
    \end{haskell}
    \item Usar \texttt{sumatoria'} definida en el \texttt{Proyecto 1}.
    \begin{haskell}
        sumatoria' :: [a] -> (a -> Int) -> Int
    \end{haskell}
\end{itemize}
En ambos casos se pueden definir las funciones auxiliares que consideren necesarias.

\subsubsection{Ejercicio 4}
Se va a construir una base de datos de estudiantes de la facultad. Para ello se deben definir los tipos de datos:
\begin{itemize}
    \item Tipo \texttt{Carrera}: es un tipo que tiene cuatro constructores sin parámetros que son \texttt{Matematica}, \texttt{Astronomia}, \texttt{Fisica} y \texttt{Computacion}.
    \item Tipo \texttt{Nombre}: es un sinónimo de \texttt{String}.
    \item Tipo \texttt{Legajo}: es un sinónimo de \texttt{Int}.
\end{itemize}
\begin{note}
    El tipo \texttt{Carrera} no debe pertenecer a la clase \texttt{Eq}.
\end{note}

Programar utilizando recursión y pattern matching las siguientes funciones:
\begin{itemize}
    \item \texttt{esDeComputacion :: Carrera -> Bool} que dado una carrera, nos dice si es de computación.
    \item \texttt{esDeFisica :: Carrera -> Bool} que dado una carrera, nos dice si es de física.
    \item \texttt{esDeAstronomia :: Carrera -> Bool} que dado una carrera, nos dice si es de astronomía.
    \item \texttt{esDeMatematica :: Carrera -> Bool} que dado una carrera, nos dice si es de matemática.
\end{itemize}

Y luego la función
\begin{haskell}
    buscar :: [Estudiante] -> Carrera -> [Nombre]
\end{haskell}
que dada una lista de estudiantes \texttt{xs} y una carrera \texttt{c}, devuelve los nombres de los estudiantes en \texttt{xs} que estudian la carrera \texttt{c}.

\section{Programación Imperativa}

\subsection{Ejercicios sobre proyectos}

\subsubsection{Ejercicio 1}
En el ejercicio 6 del \texttt{Proyecto 3}, definimos la función \texttt{int pedir\_entero (void)} como:
\begin{c_code}
int pedir_entero(char name){
    int x;
    printf("Ingrese el valor para la variable %c: ", name);
    scanf("%d", &x);
    return x;
}
\end{c_code}
Defina ahora las siguientes funciones:
\begin{itemize}
    \item \texttt{float pedir\_flotante (char name)} que pida un número flotante.
    \item \texttt{char pedir\_caracter (char name)} que pida un caracter.
    \item \texttt{void imprimir\_mensaje (char* mensaje)} que imprima un mensaje.
\end{itemize}

\subsubsection{Ejercicio 2}
En el ejercicio 3 del \texttt{Proyecto 4} definimos a la función \texttt{int suma\_hasta(int n)} como
\begin{c_code}
int suma_hasta(int n){
    assert(n > 0);
    int suma = 0;
    while(n > 0){
        suma = n + suma;
        n--;
    }
    printf("La suma es: %d", suma);
    return suma;
}
\end{c_code}
Defina ahora las siguientes funciones:
\begin{itemize}
    \item \texttt{int suma\_hasta\_par(int n)} que sume los números pares hasta \texttt{n}.
    \item \texttt{int suma\_hasta\_impar(int n)} que sume los números impares hasta \texttt{n}.
    \item \texttt{int suma\_hasta\_multiplo(int n, int m)} que sume los múltiplos de \texttt{m} hasta \texttt{n}.
\end{itemize}

\newpage
\subsubsection{Ejercicio 3}
En el ejercicio 7 del \texttt{Proyecto 4}, definimos la función \texttt{int sumatoria(int tam, int a[])} como
\begin{c_code}
int sumatoria(int tam, int a[]){
    int suma = 0 ;
    int i = 0;
    while(i < tam){
    suma = a[i] + suma;
    i = i + 1;
  }
  return suma;
}
\end{c_code}
Defina ahora las siguientes funciones:
\begin{itemize}
    \item \texttt{int sumatoria\_pares(int tam, int a[])} que sume los números pares de un arreglo.
    \item \texttt{int sumatoria\_impares(int tam, int a[])} que sume los números impares de un arreglo.
    \item \texttt{int sumatoria\_multiplos(int tam, int a[], int m)} que sume los múltiplos de \texttt{m} de un arreglo.
\end{itemize}
Y luego la función
\begin{c_code}
    productoria(int tam, int a[])
\end{c_code}
que devuelva el producto de los elementos de un arreglo.

\subsection{Ejercicios extra}

\subsubsection{Ejercicio 1}
Definir la función
\begin{c_code}
    int factorial(int n)
\end{c_code}
que devuelva el factorial de un número \texttt{n}.

\subsubsection{Ejercicio 2}
Definir la función
\begin{c_code}
    int fibonacci(int n)
\end{c_code}
que devuelva el n-ésimo número de la sucesión de Fibonacci.

\subsubsection{Ejercicio 3}
Hacer un programa que verifique si el elemento ubicado en el índice \texttt{k} de un arreglo \texttt{a} es máximo o si es el mínimo. Para ello programar la siguiente función:
\begin{c_code}
struct s_minmax_t verificar_minmax(int a[], int tam, int k);
\end{c_code}
donde la estructura \texttt{struct s\_minmax\_t} es
\begin{c_code}
struct s_minmax_t {
    bool es_maximo;
    bool es_minimo;
};
\end{c_code}

La función toma un arreglo \texttt{a[]}, su tamaño \texttt{tam} y un índice \texttt{k} y debe devolver una estructura con los dos booleanos que respectivamente indican:
\begin{itemize}
    \item Todos los elementos de \texttt{a[]} son menores o iguales que \texttt{a[k]}. (se guarda en \texttt{es\_maximo})
    \item Todos los elementos de \texttt{a[]} son mayores o iguales que \texttt{a[k]}. (se guarda en \texttt{es\_minimo})
\end{itemize}
La función debe implementarse con un solo ciclo y \textbf{no debe mostrar mensajes por pantalla ni pedir valores al usuario}.

En la función main se debe solicitar al usuario ingresar un arreglo de longitud \texttt{N}, donde \texttt{N} debe definirse como una constante. \textbf{El usuario no debe elegir el tamaño del arreglo.} Luego se debe pedir el índice \texttt{k} y verificar con \texttt{assert} que \texttt{k} es un número mayor que \texttt{0} y menor que \texttt{N}. Finalmente mostrar el resultado de la función \texttt{verificar\_minmax} por pantalla (los dos valores de la estructura).

\end{document}